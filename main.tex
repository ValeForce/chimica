\documentclass[openany,12pt]{book}%openany toglie la pagine bianche tra i capitoli

% Language setting
% Replace `english' with e.g. `spanish' to change the document language
\usepackage[italian]{babel}
\usepackage{import}
% Set page size and margins
% Replace `letterpaper' with `a4paper' for UK/EU standard size
\usepackage[letterpaper,top=2cm,bottom=2cm,left=3cm,right=3cm,marginparwidth=1.75cm]{geometry}

% Useful packages
\usepackage[utf8]{inputenc}
\usepackage{amsmath}
\usepackage{graphicx}
\usepackage{hyperref}
\usepackage{array}% only needed for injecting commands at the beginning of columns in the tabular below
\usepackage{electrons}
\usepackage[version=4]{mhchem}
\newcommand{\comment}[1]{}

\usepackage{mathtools}
\usepackage{tikz}
\usepackage{lipsum}
\usepackage{chemfig}
\usepackage{amsmath}
\usepackage{array}
\setlength\parindent{0pt}%e si gode, toglie lo spostmento a destra di una nuova riga


\makeatletter
\newcommand\mathcircled[1]{%
  \mathpalette\@mathcircled{#1}%
}
\newcommand\@mathcircled[2]{%
  \tikz[baseline=(math.base)] \node[draw,ellipse,inner sep=1pt] (math) {$\m@th#1#2$};%
}

\newcommand\circleanswer[1]{%%
  \begin{tikzpicture}[baseline=(word.base)]
    \node[inner sep=0pt]  (word) {#1};
    \begin{pgfinterruptboundingbox}
    \node[draw] at (word) {\phantom{#1}};
    \end{pgfinterruptboundingbox}
  \end{tikzpicture}}

\makeatother

\title{Chimica Nino}
\author{V. Favitta \& S. Arena}

\begin{document}
\maketitle
\tableofcontents

\chapter{Composizione materia e reazioni chimiche}

\section{Prime nozioni base}%roba a cazzo
\import{./Chapter/1-Sections/}{1-primo}

\section{Nomenclatura}
\import{./Chapter/1-Sections/}{2-secondo}

\section{Differenza tra acido e base}
\import{./Chapter/1-Sections/}{3-terzo}

\section{Base (o dovrei dire idrossido perché si)}
\import{./Chapter/1-Sections/}{4-quarto}

\section{Ossido acido (o dovrei dire anidride)}
\import{./Chapter/1-Sections/}{5-quinto}

\section{Idracidi (no ossigeno)}
\import{./Chapter/1-Sections/}{6-sesta}

\section{Reazioni di salificazione}
\import{./Chapter/1-Sections/}{7-settima}

\section{Reazioni varie}
\import{./Chapter/1-Sections/}{8-ottava}

\section{Ossidoriduzioni}
\import{./Chapter/1-Sections/}{9-nona}

\end{document}