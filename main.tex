\documentclass[openany,12pt]{book}%openany toglie la pagine bianche tra i capitoli

% Language setting
% Replace `english' with e.g. `spanish' to change the document language
\usepackage[italian]{babel}
\newcommand{\E}{È \hspace{0.1mm}}
\usepackage{import}
% Set page size and margins
% Replace `letterpaper' with `a4paper' for UK/EU standard size
\usepackage[letterpaper,top=2cm,bottom=2cm,left=3cm,right=3cm,marginparwidth=1.75cm]{geometry}

\usepackage{amssymb} % for harpoons
\newcommand{\electron}[2]{{%
        \newcommand*\up{\fbox{$\mathord\upharpoonleft\phantom{\downharpoonright}$}}%
        \newcommand*\dwn{\fbox{$\mathord\downharpoonleft\phantom{\upharpoonright}$}}%
        \newcommand*\updwn{\fbox{$\upharpoonleft\downharpoonright$}}%
        \newcommand*\emp{\fbox{$\phantom{\downharpoonright}\phantom{\downharpoonright}$}}%
        \setlength\tabcolsep{0pt}% remove extra horizontal space from tabular
        \begin{tabular}{c}#2\\[2pt]#1\end{tabular}%
}}
% Useful packages
\usepackage[utf8]{inputenc}
\usepackage{afterpage}
\newcommand\blankpage{%
    \null
    \thispagestyle{empty}%
    \newpage}
\usepackage{amsmath}
\usepackage{graphicx}
\usepackage{hyperref}
\usepackage{array}% only needed for injecting commands at the beginning of columns in the tabular below
\usepackage{electrons}
\usepackage[version=4]{mhchem}
\newcommand{\comment}[1]{}

\newcommand\myline[1][]{%
  \,\tikz[baseline]\draw[very thick,#1](0,-\dp\strutbox)--(0,\ht\strutbox);\,%
}

\usepackage{float}
\usepackage{mathtools}
\usepackage{tikz}
\usetikzlibrary {shapes.geometric}
\usetikzlibrary{decorations.markings}
\usepackage{lipsum}
\usepackage{chemfig}
\usepackage{amsmath}
\usepackage{array}
\setlength\parindent{0pt}%e si gode, toglie lo spostmento a destra di una nuova riga
\usepackage{caption}
\usepackage{float}
\pgfdeclaredecoration{ddbond}{initial}{
  \state{initial}[width=4pt]{
    \pgfpathlineto{\pgfpoint{4pt}{0pt}}
    \pgfpathmoveto{\pgfpoint{2pt}{2pt}}
    \pgfpathlineto{\pgfpoint{4pt}{2pt}}
    \pgfpathmoveto{\pgfpoint{4pt}{0pt}}
  }
  \state{final}{
    \pgfpathlineto{\pgfpointdecoratedpathlast}
  }
}
\tikzset{lddbond/.style={decorate, decoration=ddbond}}
\tikzset{rddbond/.style={decorate, decoration={ddbond, mirror}}}

% #ilchinonlaavràvintaesonoioadecideredovecazzodevestare
\DeclareRobustCommand{\rchi}{{\mathpalette\irchi\relax}}
\newcommand{\irchi}[2]{\raisebox{\depth}{$#1\chi$}} % inner command, used by \rchi

\makeatletter
\newcommand\mathcircled[1]{%
  \mathpalette\@mathcircled{#1}%
}
\newcommand\@mathcircled[2]{%
  \tikz[baseline=(math.base)] \node[draw,ellipse,inner sep=1pt] (math) {$\m@th#1#2$};%
}
\makeatother

\begin{document}

\thispagestyle{empty}
\begin{center}

\begin{minipage}[c]{0.45\textwidth}
\begin{flushleft}
\includegraphics[width=0.8\textwidth]{logo-unict-orizzontale-grigio.png}
\end{flushleft}
\end{minipage}
\hfill
\begin{minipage}[c]{0.45\textwidth}
\begin{flushright}
\includegraphics[width=\textwidth]{logo_dfa_orizzontale}
\end{flushright}
\end{minipage}\\
\medskip
\hbox to \textwidth{\hrulefill}

\vfill
\vfill

\uppercase{\sc{ \Large{\textbf{Chimica Nino}}}}\\

\vfill
\large{A cura di V. Favitta \& S. Arena }

\vfill
\vfill
\hbox to \textwidth{\hrulefill}
{\sc anno 2023}
\end{center}

\afterpage{\blankpage}
\newpage

\begin{figure}[H]
    \includegraphics[angle=90,origin=c]{immagini/tavola periodica.png}
\end{figure}

\afterpage{\blankpage}

\tableofcontents

\chapter*{Introduzione}

Con la seguente opera gli autori si sono proposti di trascrivere, in maniera quanto più

affidabile, le lezioni del corso di chimica per il Cdl in Fisica in formato \LaTeX. Pertanto

l'ordine degli argomenti affrontati è stato mantenuto quasi del tutto fedelmente, tranne

che per per il paragrafo 3.5.3, che è stato spostato per coerenza negli argomenti trattati:

opinione nostra è, infatti, che alcuni argomenti risultino non molto chiari ad una prima

lettura, in quanto si adoperano concetti che vengono sì spiegati esaustivamente, ma solo

in un secondo momento.

\vspace{0.2cm}
Questo testo, pertanto, potrebbe richiedere più di una lettura prima di riuscire ad avere

una padronanza dei concetti. Ciò ovviamente non implica affatto che sia sufficiente ad

affrontare la materia, in quanto non potrà mai sostituire le lezioni tenute dal professore,

nonché i libri di testo, ma può di certo rappresentare un punto di inizio per lo studio di

tale materia, che è stata e continua ad essere troppo spesso all'interno di tale corso un

ostacolo

\vspace{0.2cm}
Messo alla fine troverete un intero capitolo dedicato alla risoluzione degli esercizi

e

c

c

a

n

i

c

i

\afterpage{\blankpage}
\newpage

\chapter{Composizione materia e reazioni chimiche}

\section{Prime nozioni base}%roba a cazzo
\import{./Chapter/1-Sections/}{1-primo}

\newpage

\section{Nomenclatura}
\import{./Chapter/1-Sections/}{2-secondo}

\section{Differenza tra acido e base}
\import{./Chapter/1-Sections/}{3-terzo}

\section{Base (o dovrei dire idrossido perché si)}
\import{./Chapter/1-Sections/}{4-quarto}

\section{Ossido acido (o dovrei dire anidride)}
\import{./Chapter/1-Sections/}{5-quinto}

\section{Idracidi (no ossigeno)}
\import{./Chapter/1-Sections/}{6-sesta}

\section{Reazioni di salificazione}
\import{./Chapter/1-Sections/}{7-settima}

\section{Reazioni varie}
\import{./Chapter/1-Sections/}{8-ottava}

\section{Ossidoriduzioni}
\import{./Chapter/1-Sections/}{9-nona}

\chapter{Teoria atomica}

\section{Idee bizzarre di fisici bizzarri}
\import{./Chapter/2-Sections}{1-primo}

\section{L'equazione di Schrödinger}
\import{./Chapter/2-Sections}{2-secondo}

\newpage

\section{Proprietà periodiche}
\import{./Chapter/2-Sections}{3-terzo}

\chapter{Il legame chimico}

\section{Il formalismo di Lewis}
\import{./Chapter/3-Sections}{1-primo}

\newpage

\section{Geometrie molecolari}
\import{./Chapter/3-Sections}{2-secondo}

\newpage

\section{La teoria V.S.E.P.R.}
\import{./Chapter/3-Sections}{3-terzo}

\newpage

\section{Modelli di legame}
\import{./Chapter/3-Sections}{4-quarto}

\newpage

\section{Il legame ionico}
\import{./Chapter/3-Sections}{5-quinto}

\newpage

\section{Il legame covalente}
\import{./Chapter/3-Sections}{6-sesto}

\chapter{Elementi di termodinamica}

\section{Stato gassoso}
\import{./Chapter/4-Sections}{1-primo}

\section{Stato liquido e cambiamenti di stato}
\import{./Chapter/4-Sections}{2-secondo}

\chapter{Soluzioni acquose}

\section{Soluzioni a due componenti}
\import{./Chapter/5-Sections}{1-primo}

\section{La concentrazione}
\import{./Chapter/5-Sections}{2-secondo}

\section{La legge di Raoult}
\import{./Chapter/5-Sections}{3-terzo}

\vspace{-0.3cm}\section{Distillazione}
\import{./Chapter/5-Sections}{4-quarto}

\section{Proprietà colligative}
\import{./Chapter/5-Sections}{5-quinto}

\newpage

\section{Diagramma di stato o fase}
\import{./Chapter/5-Sections}{6-sesto}

\newpage

\section{Solubilità}
\import{./Chapter/5-Sections}{7-settimo}

\newpage

\section{Forze intermolecolari}
\import{./Chapter/5-Sections}{8-ottavo}

\section{Tensione superficiale}
\import{./Chapter/5-Sections}{9-nono}

\chapter{L'equilibrio chimico}

\section{Equilibrio chimico omogeneo}
\import{./Chapter/6-Sections}{1-primo}

\section{Fattori che influenzano l'equilibrio}
\import{./Chapter/6-Sections}{2-secondo}

\chapter{Acidi e basi}

\section{Definizioni}
\import{./Chapter/7-Sections}{1-primo}

\section{Coppie coniugate acido-base}
\import{./Chapter/7-Sections}{2-secondo}

\section{Reazioni acido-base}
\import{./Chapter/7-Sections}{3-terzo}

\newpage

\section{Titolazioni}
\import{./Chapter/7-Sections}{4-quarto}

\chapter{Elettrochimica}

\section{Celle galvaniche}
\import{./Chapter/8-Sections}{1-primo}

\section{Elettrodi}
\import{./Chapter/8-Sections}{2-secondo}

\section{Elettrolisi}
\import{./Chapter/8-Sections}{3-terzo}


\afterpage{\blankpage}
\newpage

\appendix

\chapter{Esercizi}

Nel seguente capitolo sono riportati sia esercizi svolti a lezione dal professore (indicati con $\bigstar$) che esercizi di compiti passati che abbiamo provato a risolvere. Di questi non garantiamo la corretta risoluzione.

\section{Reazioni}
\import{./Chapter/9-Sections}{1-primo}

\newpage

\section{Ossidoriduzioni}
\import{./Chapter/9-Sections}{2-secondo}

\newpage

\section{Gas e soluzioni}
\import{./Chapter/9-Sections}{3-terzo}

\newpage

\section{Calcolo pH}
\import{./Chapter/9-Sections}{4-quarto}

\newpage

\section{Elettrochimica}
\import{./Chapter/9-Sections}{5-quinto}

\end{document}