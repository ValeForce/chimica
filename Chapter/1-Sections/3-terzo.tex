In generale ossido/anidride + acqua:
\(\ce{AO + H_2O -> A(OH)_2}\)
Nella molecola vi sono due tipi di legami: \ce{ A\bond{-}O} e \ce{ O\bond{-}H}. Dipenderà dalla polarità relativa di questi due legami se trattasi di acido o di base. In acqua si romperà il legame più polare ed il composto si dissocerà in ioni diversi.

Se il legame \ce{ A\bond{-}O} è meno polare del legame \ce{ O\bond{-}H}, quest'ultimo in soluzione acquosa tenderà ad rompersi più facilmente, generando un acido \(\ce{(AO_2- + 2H+)}\), ed allora la molecola va scritta come segue: \(\ce{H_2AO_2}\).

Se il legame \ce{ A\bond{-}O} è più polare del legame \ce{ O\bond{-}H}, sarà esso che in soluzione acquosa tenderà a rompersi più facilmente, generando quindi una base \(\ce{(A^{2+} + 2OH^-)}\), conservando l’integrità del gruppo OH, ed allora il composto va scritto come: $\rm A(OH)_2$.
