Reazione che porta alla formazione di un sale, il quale si può pensare come un acido in cui gli atomi di idrogeno sono sostituiti da metalli.\\
\textbf{Nota} tutte le reazioni considerate a seguire avvengono in acqua, cioè i composti che reagiscono sono immersi in acqua.
%già mi sono seccato
\subsection{Basi e idracidi}
$$\ce{NaOH_{(aq)} + HCl_{(aq)} -> NaCl_{(aq)} + H_2O}$$
$$\text{idrossido di sodio + acido cloridrico} \ce{->} \text{cloruro di sodio + acqua}$$
$$\ce{Na+ + \mathcircled{\text{OH}^- + \text{H}^+} + Cl- ->Na+ + Cl- + H2O}$$

In queste reazioni, se il sale che si forma è solubile in  acqua, l'unica vera reazione che avviene è quella di formazione dell'acqua, cioè \ce{H+} e \ce{OH-} si associano.
$$\ce{Ca(OH)_2 + HCl -> CaCl_2 + H_2O}$$
Il metallo prende il posto dell'idrogeno e ne prende il suo pedice (in questi primi esempi è 1, dopo sarà $>$1) e il suo stato di ossidazione diventa il pedice dell'alogeno.
\\Bilancio
$$\ce{Ca(OH)_2 + 2 HCl -> CaCl_2 + 2 H_2O}$$
$$\text{idrossido di calcio + acido cloridrico} \ce{->} \text{cloruro di calcio + acqua}$$

$$\ce{Al(OH)_3 + HCl -> AlCl_3 + H_2O}$$
Bilancio
$$\ce{Al(OH)_3 + 3 HCl -> AlCl_3 + 3 H_2O}$$
$$\text{idrossido di alluminio + acido cloridrico} \ce{->} \text{cloruro di alluminio + acqua}$$

$$\ce{Fe(OH)_2 + HCl -> FeCl_2 + H_2O}$$
Bilancio
$$\ce{Fe(OH)_2 + 2 HCl -> AlCl_3 + 2 H_2O}$$
$$\text{idrossido ferroso + acido cloridrico} \ce{->} \text{cloruro ferroso + acqua}$$

$$\ce{Fe(OH)_3 + HCl -> FeCl_3 + H_2O}$$
Bilancio
$$\ce{Fe(OH)_3 + 3 HCl -> FeCl_3 + 3 H_2O}$$
$$\text{idrossido ferrico + acido cloridrico} \ce{->} \text{cloruro ferrico + acqua}$$
La regola per bilanciare è la seguente: si parte dal pedice dell'alogeno a destra e si moltiplica la molecola che a sinistra lo contiene per questo numero, analogamente si fa per il metallo. A questo punto per capire quante molecole d'acqua ci sono basta contare il numero di ioni \ce{OH-} e \ce{H+}.

In quest'ultimo caso il cloro ha pedice 3 a destra, per cui moltiplico l'\ce{HCl} che lo contiene a sinistra per 3. Se adesso contiamo gli ioni, abbiamo 3 ioni \ce{OH-} dall'idrossido e 3 ioni \ce{H+} dall'acido, che insieme danno luogo a 3 molecole di acqua.
\subsection{Basi e ossiacidi}

$$\ce{NaOH + HNO_3 -> NaNO_3 + H_2O}$$
È già bilanciata
$$\text{idrossido sodio + acido nitrico} \ce{->} \text{nitrato di sodio + acqua}$$
Anche qui il metallo prende il posto dell'idrogeno, e la sua valenza diventa il pedice di quello che resta dell'ossiacido. In questo caso il sodio ha valenza 1, quindi avremmo dovuto scrivere \ce{(NO3)_1}
Per quanto riguardo l'\ce{HNO_3}, esso è un acido monoprotico, cioè un acido avente un solo protone (uno ione \ce{H+}) da poter scambiare.
$$\ce{Ca(OH)_2 + HNO_3 -> Ca(NO_3)_2 + H_2O}$$Bilancio
$$\ce{Ca(OH)_2 + 2 HNO_3 -> Ca(NO_3)_2 + 2 H_2O}$$
$$\text{idrossido calcio + acido nitrico} \ce{->} \text{nitrato di calcio + acqua}$$

$$\ce{Al(OH)_3 + HNO_3 -> Al(NO_3)_3 + H_2O}$$Bilancio
\newpage
$$\ce{Al(OH)_3 + 3 HNO_3 -> Al(NO_3)_3 + 3 H_2O}$$
$$\text{idrossido alluminio + acido nitrico} \ce{->} \text{nitrato di alluminio + acqua}$$

$$\ce{Na(OH) + H_2SO_4 -> Na_2SO_4 + H_2O}$$Bilancio
$$\ce{2 Na(OH) + H_2SO_4 -> Na_2SO_4 + 2 H_2O}$$
$$\text{idrossido sodio + acido solforico} \ce{->} \text{solfato di sodio + acqua}$$

$$\ce{Ca(OH)_2 + H_2SO_4 -> Ca_2(SO_4)_2 + H_2O}$$Bilancio
$$\ce{2 Ca(OH) + 2H_2SO_4 -> Ca_2(SO_4)_2 +  2 H_2O}$$
$$\text{idrossido calcio + acido solforico} \ce{->} \text{solfato di calcio + acqua}$$

$$\ce{Al(OH)_3 + H_2SO_4 -> Al_2(SO_4)_3 + H_2O}$$Bilancio
$$\ce{2 Al(OH)_3 + 3H_2SO_4 -> Al_2(SO_4)_3 + 6 H_2O}$$
$$\text{idrossido calcio + acido solforico} \ce{->} \text{solfato di calcio + acqua}$$

$$\ce{Fe(OH)_2 + H_2SO_4 -> Fe_2(SO_4)_2 + H_2O \text{=} FeSO_4 + H_2O}$$Bilancio
$$\ce{Fe(OH)_2 + H_2SO_4 -> Fe_2(SO_4)_2 + 2 H_2O}$$
$$\text{idrossido ferroso + acido solforico} \ce{->} \text{solfato di ferro (II) o ferroso + acqua}$$

$$\ce{Fe(OH)_3 + H_2SO_4 -> Fe_2(SO_4)_3 + H_2O}$$Bilancio
$$\ce{2 Fe(OH)_3 + 3H_2SO_4 -> Fe_2(SO_4)_3 + 6 H_2O}$$
$$\text{idrossido ferrico + acido solforico} \ce{->} \text{solfato di ferro (III) o ferrico + acqua}$$

$$\ce{Na(OH) + H_3PO_4 -> Na_3PO_4 + H_2O}$$Bilancio
$$\ce{3 Na(OH) + H_3PO_4 -> Na_3PO_4 + 3 H_2O}$$
$$\text{idrossido di sodio + acido ortofosforico} \ce{->} \text{(orto)fosfato di sodio + acqua}$$

$$\ce{Ca(OH)_2 + H_3PO_4 -> Ca_3(PO_4)_2 + H_2O}$$Bilancio
$$\ce{3 Ca(OH)_2 + 2 H_3PO_4 -> Ca_3(PO_4)_2 + 6 H_2O}$$
$$\text{idrossido di calcio + acido ortofosforico} \ce{->} \text{(orto)fosfato di calcio + acqua}$$

$$\ce{Al(OH)_3 + H_3PO_4 -> Al_3(PO_4)_3 + H_2O = AlPO_4 + H_2O}$$È già bilanciata
$$\ce{Al(OH)_3 + H_3PO_4 -> AlPO_4 + H_2O}$$
$$\text{idrossido di alluminio + acido ortofosforico} \ce{->} \text{fosfato di alluminio + acqua}$$
In queste reazioni bisogna stare attenti al nome dell'acido che stiamo usando, ad esempio con lo zolfo abbiamo sia acido solforoso \ce{H_2SO_3} che deriva dall'anidride solforosa \ce{SO_2} dove lo zolfo ha valenza 4, sia acido solforico \ce{H_2SO_4} che deriva dall'anidride solforica \ce{SO_3} dove lo zolfo ha valenza 6. Se in una reazione usiamo il primo, il sale che si forma si chiama solfito, se si usa il secondo il sale si chiama solfato. Nel caso generale, in cui si possono avere anche 4 valenze per la stessa specie, useremo in ordine crescente di valenza la seguente nomenclatura:
\begin{center}
\textbf{ipo-}...\textbf{-oso} diventa \textbf{ipo-}...\textbf{-ito}\\
\textbf{-oso} diventa \textbf{-ito}\\
\textbf{-ico} diventa \textbf{-ato}\\
\textbf{per-}...\textbf{-ico} diventa \textbf{per-}...\textbf{-ato}
\end{center}

%N.d.r.: Un acido del tipo ipo-...-oso si ottiene per valenza 1/2, uno del tipo -oso per valenza 3/4, uno del tipo -ico per valenza 5/6 e uno del tipo per-...-ico si ha per valenza +7.

$$\ce{Al(OH)_3 + H_2SO_3 -> Al_2(SO_3)_3 + H_2O}$$Bilancio
$$\ce{2 Al(OH)_3 + 3 H_2SO_3 -> Al_2(SO_3)_3 + 6 H_2O}$$
$$\text{idrossido di alluminio + acido solforoso} \ce{->} \text{solfito di alluminio + acqua}$$

$$\ce{Fe(OH)_2 + H_2SO_3 -> Fe_2(SO_3)_2 + H_2O=Fe(SO_3) + H_2O}$$È già bilanciata
$$\ce{Fe(OH)_2 + H_2SO_3 -> Fe(SO_3) + H_2O}$$
$$\text{idrossido ferroso + acido solforoso} \ce{->} \text{solfito ferroso + acqua}$$

$$\ce{Fe(OH)_3 + H_2SO_3 -> Fe_2(SO_3)_3 + H_2O}$$Bilancio
$$\ce{2 Fe(OH)_3 + 3 H_2SO_3 -> Fe_2(SO_3)_3 + 6 H_2O}$$
$$\text{idrossido ferrico + acido solforoso} \ce{->} \text{solfito ferrico + acqua}$$

$$\ce{NaOH + HClO -> NaClO + H_2O}$$
$$\text{idrossido di sodio + acido ipocloroso}\ce{->}\text{ipoclorito di sodio + acqua}$$
È già bilanciata.

$$\ce{Ca(OH)_2 + HClO -> Ca(ClO)_2 + H_2O}$$Bilancio
\newpage
$$\ce{Ca(OH)_2 + 2 HClO -> Ca(ClO)_2 + 2 H_2O}$$
$$\text{idrossido di calcio + acido ipocloroso}\ce{->}\text{ipoclorito di calcio + acqua}$$

$$\ce{Al(OH)_3 + HClO -> Al(ClO)_3 + H_2O}$$Bilancio
$$\ce{Al(OH)_3 + 3HClO -> Al(ClO)_3 + 3H_2O}$$
$$\text{idrossido di alluminio + acido ipocloroso}\ce{->}\text{ipoclorito di alluminio + acqua}$$

$$\ce{Fe(OH)_2 + HClO -> Fe(ClO)_2 + H_2O}$$Bilancio
$$\ce{Fe(OH)_2 + 2 HClO -> Fe(ClO)_2 + 2 H_2O}$$
$$\text{idrossido ferroso + acido ipocloroso}\ce{->}\text{ipoclorito ferroso + acqua}$$

$$\ce{Fe(OH)_3 + HClO -> Fe(ClO)_3 + H_2O}$$Bilancio
$$\ce{Fe(OH)_3 + 3 HClO -> Fe(ClO)_3 + 3 H_2O}$$
$$\text{idrossido ferrico + acido ipocloroso}\ce{->}\text{ipoclorito ferrico + acqua}$$

$$\ce{NaOH + HClO_2 -> NaClO_2 + H_2O}$$
$$\text{idrossido di sodio + acido cloroso}\ce{->}\text{clorito di sodio + acqua}$$
È già bilanciata.

$$\ce{Ca(OH)_2 + HClO_2 -> Ca(ClO_2)_2 + H_2O}$$Bilancio
$$\ce{Ca(OH)_2 + 2 HClO_2 -> Ca(ClO_2)_2 + 2 H_2O}$$
$$\text{idrossido di calcio + acido cloroso}\ce{->}\text{clorito di calcio + acqua}$$

$$\ce{Al(OH)_3 + HClO_2 -> Al(ClO_2)_3 + H_2O}$$Bilancio
$$\ce{Al(OH)_3 + 3HClO_2 -> Al(ClO_2)_3 + 3H_2O}$$
$$\text{idrossido di alluminio + acido cloroso}\ce{->}\text{clorito di alluminio + acqua}$$

$$\ce{Fe(OH)_2 + HClO_2 -> Fe(ClO_2)_2 + H_2O}$$Bilancio
$$\ce{Fe(OH)_2 + 2 HClO_2 -> Fe(ClO_2)_2 + 2 H_2O}$$
$$\text{idrossido ferroso + acido cloroso}\ce{->}\text{clorito ferroso + acqua}$$

$$\ce{Fe(OH)_3 + HClO_2 -> Fe(ClO_2)_3 + H_2O}$$Bilancio
$$\ce{Fe(OH)_3 + 3 HClO_2 -> Fe(ClO_2)_3 + 3 H_2O}$$
$$\text{idrossido ferrico + acido cloroso}\ce{->}\text{clorito ferrico + acqua}$$

$$\ce{NaOH + HClO_3 -> NaClO_3 + H_2O}$$
$$\text{idrossido di sodio + acido clorico}\ce{->}\text{clorato di sodio + acqua}$$
È già bilanciata.

$$\ce{Ca(OH)_2 + HClO_3 -> Ca(ClO_3)_2 + H_2O}$$Bilancio
$$\ce{Ca(OH)_2 + 2 HClO_3 -> Ca(ClO_3)_2 + 2 H_2O}$$
$$\text{idrossido di calcio + acido clorico}\ce{->}\text{clorato di calcio + acqua}$$

$$\ce{Al(OH)_3 + HClO_3 -> Al(ClO_3)_3 + H_2O}$$Bilancio
$$\ce{Al(OH)_3 + 3 HClO_3 -> Al(ClO_3)_3 + 3 H_2O}$$
$$\text{idrossido di alluminio + acido clorico}\ce{->}\text{clorato di alluminio + acqua}$$

$$\ce{Fe(OH)_2 + HClO_3 -> Fe(ClO_3)_2 + H_2O}$$Bilancio
$$\ce{Fe(OH)_2 + 2HClO_3 -> Fe(ClO_2)_2 + 2H_2O}$$
$$\text{idrossido ferroso + acido clorico}\ce{->}\text{clorato ferroso + acqua}$$

$$\ce{Fe(OH)_3 + HClO_3 -> Fe(ClO_2)_3 + H_2O}$$Bilancio
$$\ce{Fe(OH)_3 + 3 HClO_3 -> Fe(ClO_2)_3 + 3 H_2O}$$
$$\text{idrossido ferrico + acido clorico}\ce{->}\text{clorato ferrico + acqua}$$

$$\ce{NaOH + HClO_4 -> NaClO_4 + H_2O}$$
$$\text{idrossido di sodio + acido perclorico}\ce{->}\text{perclorato di sodio + acqua}$$
È già bilanciata.

$$\ce{Ca(OH)_2 + HClO_4 -> Ca(ClO_4)_2 + H_2O}$$Bilancio
$$\ce{Ca(OH)_2 + 2 HClO_4 -> Ca(ClO_4)_2 + 2 H_2O}$$
$$\text{idrossido di calcio + acido perclorico}\ce{->}\text{perclorato di calcio + acqua}$$

$$\ce{Al(OH)_3 + HClO_4 -> Al(ClO_4)_3 + H_2O}$$Bilancio
$$\ce{Al(OH)_3 + 3 HClO_4 -> Al(ClO_4)_3 + 3 H_2O}$$
$$\text{idrossido di alluminio + acido perclorico}\ce{->}\text{perclorato di alluminio + acqua}$$

$$\ce{Fe(OH)_2 + HClO_4 -> Fe(ClO_4)_2 + H_2O}$$Bilancio
$$\ce{Fe(OH)_2 + 2 HClO_4 -> Fe(ClO_4)_2 + 2 H_2O}$$
$$\text{idrossido ferroso + acido perclorico}\ce{->}\text{perclorato ferroso + acqua}$$

$$\ce{Fe(OH)_3 + HClO_4 -> Fe(ClO_4)_3 + H_2O}$$Bilancio
$$\ce{Fe(OH)_3 + 3 HClO_4 -> Fe(ClO_4)_3 + 3 H_2O}$$
$$\text{idrossido ferrico + acido perclorico}\ce{->}\text{perclorato ferrico + acqua}$$

\subsection{Sali insolubili in acqua}
$$\ce{Ba(OH)_2 + H_2SO_4 -> Ba_2(SO_4)_2 + H_2O = BaSO_4 v + 2H_2O}$$
$$\text{idrossido di bario + acido solforico}\ce{->}\text{solfato di bario + acqua}$$

La freccia rivolta verso il basso dopo il $\rm BaSO_4$ ci dice che esso "precipita", cioè in questa reazione oltre alla formazione di acqua si ha anche la formazione di un precipitato (cioè quando mescoliamo i due composti nel contenitore si forma qualcosa che va a fondo). In altre parole, a inizio reazione si avevano ioni \ce{Ba^{2+}} e ioni \ce{SO_4^{2-}}, i quali a fine reazione non restano dissociati ma li troviamo uniti nel sale.

\vspace{0.2cm}Riassumendo:

\vspace{0.2cm}$\bullet$ Se il sale che si forma è insolubile in acqua si forma un sale più nuova acqua.

\vspace{0.2cm}$\bullet$ Se il sale che si forma è solubile, nei fatti è come se non si fosse mai formato, perché idrossido e acido erano già solubili prima, per cui catione e anione (rispettivamente ione positivo e negativo) erano già dissociati in essi e restano tali nel sale
\subsection{Sali acidi}
Gli acidi monoprotici possono dissociare solo un protone. Ad esempio l'acido nitrico, che è un acido forte (=totalmente dissociato in acqua) si dissocia in un protone e in uno ione nitrato:
$$\ce{HNO_3_{(aq)} -> H+ + NO_3^-}$$
Nel caso di acidi poliprotici, la dissociazione dei protoni avviene ad uno ad uno, cioè in step diversi.
Consideriamo ad esempio l'acido solforico $\rm H_2SO_4$. Avendo due protoni, la sua dissociazione avverrà in due step:
$$\ce{H_2SO_4_{(aq)} -> HSO_4- + H+}\; \text{(prima dissociazione)}$$
$$\ce{HSO_4^-_{(aq)} -> SO_4^{2-} + H+}\; \text{(seconda dissociazione)}$$
L'anione \ce{HSO_4-} é detto "solfato acido o p-solfato", mentre l'anione \ce{SO_4^{2-}} si chiama solfato.

Per semplicità finora abbiamo immaginato (e continueremo a fare così in futuro) che l'$\rm H_2SO_4$ si dissoci in un unico step:
$$\ce{H_2SO_4_{(aq)} -> SO_4^{2-} + 2  H+}$$
Del resto ciò avviene anche nella reazione
$$\ce{2 NaOH + H_2SO_4 -> Na_2SO_4 + 2 H_2O}$$
dove una molecola di acido reagisce con due molecole di idrossido, cioè si dice che il loro rapporto stechiometrico è 2:1.

Se invece avessimo imposto che una molecola di acido reagisca con solo una molecola di idrossido, cioè rapporto stechiometrico 1:1, sarebbe avvenuta una reazione incompleta che è questa:
$$\ce{NaOH + H_2SO_4 -> NaHSO_4 + H_2O}$$
$$\text{idrossido di sodio + acido solforico}\ce{->}\text{solfato acido di sodio + acqua}$$
Imponendo che l'NaOH sia stechiometricamente uguale all'$\rm HSO_4$ si riesce a neutralizzare/salificare solo uno dei due protoni dell'acido. Ciò avviene perché quest'ultimo si dissocia in due step diversi, per cui gli ioni $\rm OH^-$ della base neutralizzeranno solo la metà degli ioni $\rm H^+$ dell'acido, mentre l'altra metà resterà invariata.

$$\chemname{\chemfig{H-O-[7]S(-[1]O)(-[5]O-[4]H)(-[7]O)}}{acido solforico}  \qquad \chemname{\chemfig{Na-O-[7]S(-[1]O)(-[5]O-[4]H)(-[7]O)}}{solfato acido di sodio} \qquad \chemname{\chemfig{Na-O-[7]S(-[1]O)(-[5]O-[4]Na)(-[7]O)}}{solfato di sodio}$$

$$\ce{NaOH + H_2CO_3 -> Na_2CO_3 + H_2O}$$Bilancio
$$\ce{2 NaOH + H_2CO_3 -> Na_2CO_3 + 2 H_2O}$$
$$\text{idrossido di sodio + acido carbonico}\ce{->}\text{carbonato di sodio + acqua}$$
Anche in questo caso due molecole di base reagiscono con una di acido. Facciamone reagire solo una di base con una di acido:
$$\ce{NaOH + H_2CO_3 -> NaHCO_3 + H_2O}$$
$$\text{idrossido di sodio + acido carbonico}\ce{->}\text{carbonato acido (o bicarbonato) di sodio + acqua}$$
$$\ce{H_3PO_4_{(aq)} -> H_2PO_4- + H+}\; \text{(prima dissociazione)}$$
$$\text{acido fosforico}\ce{->}\text{ione fosfato monoacido + 1 protone}$$
$$\ce{H_2PO_4-_{(aq)} -> HPO_4^{2-} + H+}\; \text{(seconda dissociazione)}$$
$$\text{ione fosfato monoacido}\ce{->}\text{ione fosfato biacido + 1 protone}$$
$$\ce{HPO_4^{2-}_{(aq)} -> PO_4^{3-} + H+}\; \text{(terza dissociazione)}$$
$$\text{ione fosfato biacido}\ce{->}\text{ione fosfato + 1 protone}$$
In un unico step sarebbe
$$\ce{H_3PO_4_{(aq)} -> PO_4^{3-} + 3 H+}$$
$$\text{acido fosforico}\ce{->}\text{ione fosfato + 3 protoni}$$
\subsection{Sali basici}
Mentre i sali acidi si formano quando non è possibile neutralizzare tutti gli ioni \ce{H+}, quando non è possibile neutralizzare tutti gli ioni \ce{OH-} si ottengono dei sali basici
$$\ce{2 Al(OH)_3 + 3 H_2SO_4 -> Al_2(SO_4)_3 + 6 H_2O}$$
$$\text{idrossido di alluminio + acido solforico}\ce{->}\text{solfato di alluminio + acqua}$$

Se facciamo reagire una molecola di idrossido con una di acido, la reazione incompleta sarà:
$$\ce{Al(OH)_3 + H_2SO_4 -> AlOHSO_4 + 2 H_2O}$$
$$\text{idrossido di alluminio + acido solforico} \ce{->} \text{solfato monobasico di alluminio + acqua}$$
\subsection{Sali acidi che non possono esistere}
Osserviamo le formula di struttura degli acidi fosforico e fosforoso
$$\chemname{\chemfig{H-[7]O-[0.5]P(>[2]O)(-[6.5]O-H)(-[7.5,,,, dash pattern=on 2pt off 2pt]O-[0.5]H)}}{acido fosforico} \qquad \chemname{\chemfig{H-[7]O-[0.5]P(>[2]O)(-[6.5]O-H)(-[7.5]H)}}{acido fosforoso}$$
Nel primo i 3 idrogeni sono legati ognuno ad un atomo di ossigeno (cioè abbiamo tre gruppi OH), mentre nel secondo un idrogeno è legato direttamente al fosforo. Questo legame \ce{H-P} è covalente, per cui questo idrogeno non sarà salificabile (cioè non può essere sostituito da un metallo). Infatti per essere tali, gli idrogeni devono aver formato un legame polare. Quindi è inutile far reagire ad esempio una molecola di $\rm H_3PO_3$ con tre di NaOH, perché ci sono solo due gruppi OH che possono dar luogo a salificazione:
$$\ce{H_3PO_3 + 3 NaOH -> Na_3PO_3 + 3 H_2O}$$
Questa reazione NON può esistere!

Nota: non è la presenza del gruppo OH a permettere la salificazione, ma il fatto che tra ossigeno e idrogeno ci sia un legame polare, che negli ossiacidi si presenta nei gruppi OH.

\vspace{0.2cm}Le dissociazioni dell'acido fosforoso pertanto sono
$$\ce{H_3PO_3_{(aq)} -> H_2PO_3- + H+}\; \text{(prima dissociazione)}$$
$$\text{acido fosforoso}\ce{->}\text{ione fosfito monoacido + 1 protone}$$
$$\ce{H_2PO_4^-_{(aq)} -> HPO_3^{2-} + H+}\; \text{(seconda dissociazione)}$$
$$\text{ione fosfito monoacido}\ce{->}\text{ione fosfito + 1 protone}$$
E le reazioni che compie sono
$$\ce{H_3PO_3 + NaOH -> NaH_2PO_3 + H_2O}$$
$$\text{acido fosforoso + idrossido di sodio}\ce{->}\text{fosfito monoacido di sodio + acqua}$$

$$\ce{H_3PO_3 + 2 NaOH -> Na_2HPO_3 + 3 H_2O}$$
$$\text{acido fosforoso + idrossido di sodio}\ce{->}\text{fosfito di sodio + 1 protone}$$

Passiamo infine all'acido ipofosforoso:

$${\chemfig{H-O-[1]P(>[2]O)(-[6.5]H)(-[7.7]H)}}$$

In esso c'è solo un gruppo OH, quindi solo un idrogeno sarà salificabile:
$$\ce{H_3PO_2 + NaOH -> NaH_2PO_2 + H_2O}$$
\text{acido ipofosforoso + idrossido di sodio \ce{->} ipofosfito di sodio + acqua}
\subsection{Altri modi per ottenere sali}
Le reazioni di salificazioni osservate finora, hanno (eccetto quelle con gli idracidi) come reagenti un idrossido e un ossiacido, i quali si ottengono sommando, rispettivamente, un ossido con l'acqua e un'anidride con l'acqua:
$$\ce{Ca(OH)_2 + H_2CO_3 -> CaCO_3 + 2 H_2O}$$
$$\ce{Ca(OH)_2 -> CaO + H2O \qquad H_2CO_3 -> CO_2 + H_2O}$$
È tuttavia possibile usare come reagenti un ossido al posto dell'idrossido  o un'anidride al posto dell'acido. Quello che cambierà sarà che nei prodotti ci sarà una quantità minore di acqua.

\vspace{0.2cm}$\bullet$ \textbf{ idrossido + anidride \ce{->} sale + acqua}
$$\ce{Ca(OH)_2 + CO_2 -> CaCO_3 + H_2O}$$
$$\text{idrossido di calcio + anidride carbonica \ce{->} carbonato di calcio + acqua}$$
A differenza di prima, nei prodotti abbiamo una sola molecola d'acqua anziché due.

\vspace{0.2cm}$\bullet$ \textbf{ ossido + acido \ce{->} sale + acqua}
$$\ce{CaO + H_2CO_3 -> CaCO_3 + H_2O}$$
$$\text{ossido di calcio + acido carbonico \ce{->} carbonato di calcio + acqua}$$
Anche qui una sola molecola d'acqua nei prodotti.

\vspace{0.2cm}$\bullet$ \textbf{ ossido + anidride \ce{->} sale}
$$\ce{CaO + CO_2 -> CaCO_3}$$
$$\text{ossido di calcio + anidride carbonica \ce{->} carbonato di calcio}$$
Non essendoci acqua nei reagenti, sarà assente anche nei prodotti.

\vspace{0.2cm}$\bullet$ \textbf{ metallo + acido \ce{->} sale + idrogeno}

$$\ce{Ca + H_2SO_4 -> CaSO_4 + H_2 ^}$$
$$\text{calcio + acido solforico\ce{->}solfato di calcio + idrogeno}$$

In questo caso al posto dell'idrossido abbiamo un metallo. Il processo è lo stesso, solo che stavolta non si forma $\rm H_2O$ in quanto abbiamo solo gli ioni $\rm H^+$ dell'acido, bensì viene rilasciato idrogeno in forma gassosa (indicato dalla freccia $\uparrow$ nella reazione).