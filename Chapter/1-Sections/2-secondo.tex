Con ossigeno si intende \(\ce{ O2 }\).

ossido = composto tra un elemento ed ossigeno

ossido acido = composto tra non metallo ed ossigeno = anidride

ossido basico = composto tra metallo ed ossigeno = ossido

ossido acido + acqua = acido (ossiacido)

ossido basico + acqua = base (idrossido)

base + acido = sale (neutralizzazione)

Esempi:
$$\ce{ AO + H2O -> H2AO2 } \; \text{acido}$$ 
$$\ce{ AO + H2O -> A(OH)_2 } \; \text{base}$$
Infine c'è l'ossido anfotero che può essere base o acido:
$$\ce{ H_2AO_2 <- AO + H_2O -> A(OH)_2 }$$

N.d.r.: Nelle reazioni che seguono il ragionamento è il seguente: consideriamo un elemento A, che ha valenza n, e lo facciamo reagire con $\rm O_2$. A destra della reazione, come pedice di A metteremo il 2 dell'ossigeno e come pedice di O la valenza n di A:
$$\ce{A + O_2 -> A_2O_{\text{n}}}$$
Questa reazione ci dice che n atomi di ossigeno reagiscono con 2 atomi di A. Tuttavia la reazione non è bilanciata, cioè non ci sono lo stesso numero di atomi sia a destra che a sinistra. Per bilanciare mettiamo il pedice n dell'ossigeno a destra come coefficiente dell'ossigeno a sinistra, mentre il pedice 2 dell'elemento A a destra lo moltiplichiamo per 2 e lo mettiamo come coefficiente di A a sinistra (il motivo è che a sinistra abbiamo la specie $\rm O_2$, per cui ci serve un numero doppio di atomi). Infine moltiplichiamo per 2 il composto a destra:
$$\ce{4A + nO_2= 2A_2O_n}$$

\subsection{Primo gruppo}
I metalli del primo gruppo hanno un elettrone esterno che cedono:
$$\ce{ Li + O_2 -> Li_2O }$$
La massa non è scomparsa, l'equazione bilanciata è così:
$$\ce{ 4 Li + O_2 -> 2 (Li_2O) } \; \text{ossido di litio}$$
Uguale con sodio, potassio e altri elementi del primo gruppo!
\subsection{Secondo gruppo}
Questi hanno due elettroni di valenza.
$$\ce{ Be + O_2 -> Be_2O_2=BeO }$$
Bilanciamento:
$$\ce{ 2 Be + O_2 -> 2 (BeO) } \; \text{ossido di berillio}$$
\subsection{Terzo gruppo}
Tre elettroni di valenza.
$$\ce{ Al + O_2 -> Al_2O_3 }$$
Bilanciamento:
$$\ce{ 4 Al + 3O_2 -> 2 (Al_2O_3)} \; \text{ossido di alluminio}$$
\textbf{Nota} Gallio, indio e tallio hanno valenza 1 o valenza 3. La loro configurazione elettronica infatti ha due elettroni nell's e uno nel p, quindi o gliene strappiamo uno (quello del p) o tutti e tre.

$$\subshell{ns:2} \; \subshell{np:100}$$
Attenzione che se è 4s o 5s ecc dipende all'elemento (per gallio è 4s/4p).
$$\ce{ Ga + O_2 -> Ga2O }$$
Bilanciamo:
$$\ce{ 4 Ga + O_2 -> 2 (Ga2O) }$$

$$\ce{ Ga + O_2 -> Ga2O_3 }$$
Bilanciamo:
$$\ce{ 2 Ga + 3 O_2 -> 2 (Ga2O_3) }$$
Sono entrambe possibili: nella prima il gallio valenza 1 e nella seconda ha valenza 3.

Si distinguono in base all'abbondanza di ossigeno: più ce n'è, maggiore è la valenza, cioè se c'è O$_3$ allora la valenza è più alta.

Si usano i suffissi \textbf{-oso} per la valenza minore e \textbf{-ico} per la valenza maggiore.

$$\rm Ga_2O \; ossido \; galloso$$
$$\rm Ga_2O_3 \; ossido \; gallico$$
Innio
$$\ce{In + O_2 -> In_2O}$$
Bilanciamo
$$\ce{4In + O_2 -> 2(In_2O)} \; \text{ossido indioso}$$

$$\ce{In + O_2 -> In_2O_3}$$
Bilanciamo
$$\ce{4In + 3O_2 -> 2(In_2O_3)} \; \text{ossido indico}$$
Tallio
$$\ce{Tl + O_2 -> Tl_2O}$$
Bilanciamo
$$\ce{4Tl + O_2 -> 2(Tl_2O)} \; \text{ossido talloso}$$
$$\ce{Tl + O_2 -> Tl_2O_3}$$
Bilanciamo
$$\ce{4Tl + 3O_2 -> 2(Tl_2O_3)} \; \text{ossido tallico}$$
\subsection{Quarto gruppo}
Possono avere sia valenza 2 che valenza 4.
$$\subshell{ns:2} \; \subshell{np:110}$$

Silicio
$$\ce{ Si + O_2 -> Si_2O_4=SiO_2 } \; \text{(bi)ossido di silicio}$$

Germanio
$$\ce{ Ge + O_2 -> Ge_2O_2=GeO } \; \text{(valenza 2)}$$
Bilanciamo
$$\ce{ 2 Ge + O_2 -> 2 (GeO) } \; \text{ossido germanoso}$$

$$\ce{ Ge + O_2 -> Ge_2O_4=GeO_2 } \; \text{(valenza 4)}$$
Bilanciamo
$$\ce{ Ge + O_2 -> GeO_2 } \; \text{ossido germanico}$$

Stagno
$$\ce{Sn + O_2 -> Sn_2O_2=SnO} \; \text{(valenza 2)}$$
Bilanciamo
$$\ce{2Sn + O_2 -> 2Sn_O}$$
$$\ce{Sn + O_2 -> Sn_2O_4=SnO_2} \; \text{(valenza 4)}$$
Bilanciamo
$$\ce{Sn + 2O_2 -> 2SnO_2}$$

Piombo
$$\ce{Pb + O_2 -> Pb_2O_2=PbO} \; \text{(valenza 2)}$$
Bilanciamo
$$\ce{2Pb + O_2 -> 2PbO}$$
$$\ce{Pb + O_2 -> Pb_2O_4=PbO_2} \; \text{(valenza 4)}$$
Bilanciamo
$$\ce{Pb + 2O_2 -> 2PbO_2}$$
\subsection{Quinto gruppo}
Valenza 3 o 5 (3 elettroni spaiati nel p).
$$\subshell{ns:2} \; \subshell{np:111}$$
Se la valenza è 3 viene coinvolto solo il livello p, se è 5 anche quello s\\
Bismuto
$$\ce{ Bi + O_2 -> Bi_2O_3} \; \text{valenza 3}$$
Bilanciamo
$$\ce{ 4 Bi + 3 O_2 -> 2 (Bi_2O_3)} \; \text{ossido bismutoso}$$
$$\ce{ Bi + O_2 -> Bi_2O_5} \; \text{valenza 5}$$
Bilanciamo
$$\ce{ 4 Bi + 5 O_2 -> 2 (Bi_2O_5)} \; \text{ossido bismutico}$$
\subsection{Elementi di transizione}
\textbf{DEF Elemento di transizione}

Hanno orbitali \textbf{d} parzialmente occupati. Per questo motivo lo zinco non è considerato tale.

Scandio(valenza 3)
$$\ce{ Sc + O_2 -> Sc_2O_3}$$
Bilanciamo
$$\ce{4 Sc + 3O_2 -> 2 (Sc_2O_3)} \; \text{ossido di scandio}$$
Titanio
$$\ce{ Ti + O_2 -> Ti_2O_3} \; \text{(valenza 3)}$$
Bilanciamo
$$\ce{4 Ti + 3 O_2 -> 2 (Ti_2O_3)} \; \text{ossido di titanio (III)}$$

$$\ce{ Ti + O_2 -> Ti_2O_4=TiO_2} \; \text{(valenza 4)}$$
\E già bilanciata
$$\ce{ Ti + O_2 -> Ti_2O_4=TiO_2} \; \text{ossido di titanio (IV)}$$
Vanadio
$$\ce{ V + O_2 -> V_2O_3} \; \text{(valenza 3)}$$
Bilanciamo
$$\ce{ 4 V + 3 O_2 -> 2 (V_2O_3)} \; \text{(ossido di vanadio (III))}$$

$$\ce{V + O_2 -> V_2O_5} \; \text{(valenza 5)}$$
Bilanciamo
$$\ce{ 4 V + 5 O_2 -> 2 (V_2O_5)} \;  \text{ossido di vanadio (V) o anidride vanadica}$$

Il cromo ha molte valenze, le più importanti sono 3 e 6.

$$\ce{ Cr + O_2 -> Cr_2O_3} \; \text{(valenza 3)}$$
Bilanciamo:
$$ \ce{ 4 Cr + 3 O_2 -> 2 (Cr_2O_3)}\; \text{ossido di cromo (III)}$$

$$\ce{ Cr + O_2 -> Cr_2O_6=CrO_3} \; \text{(valenza 6)}$$
Bilanciamo:
$$\ce{ 2 Cr + 3 O_2 -> 2 (CrO_3)}\; \text{ossido di cromo (VI)}$$

Manganese:
$$\ce{ Mn + O_2 -> Mn_2O_2=MnO} \; \text{(valenza 2)}$$
Bilanciamo
$$\ce{ 2 Mn + O_2 -> 2 (MnO)} \; \text{monossido di manganese}$$

$$\ce{ Mn + O_2 -> Mn_2O_3} \; \text{(valenza 3)}$$
Bilanciamo
$$\ce{ 4 Mn + 3O_2 -> 2 (Mn_2O_3)} \; \text{ossido di manganese (III)}$$

$$\ce{ Mn + O_2 -> Mn_2O_4=MnO_2} \; \text{(valenza 4)}$$
È già bilanciata
$$\ce{ Mn + O_2 -> MnO_2} \; \text{biossido di manganese}$$

$$\ce{ Mn + O_2 -> Mn_2O_7} \; \text{(valenza 7)}$$
Bilanciamo
$$\ce{ 4 Mn + 7O_2 -> 2 (Mn_2O_7)} \; \text{anidride permanganica}$$

Ferro 
$$\ce{ Fe + O_2 -> Fe_2O_2=FeO} \; \text{valenza 2}$$
Bilanciamo
$$\ce{ 2 Fe + O_2 -> 2 (FeO)} \; \text{ossido ferroso}$$

$$\ce{ Fe + O_2 -> Fe_2O_3} \; \text{valenza 3}$$
Bilanciamo
$$\ce{ 2 Fe + 3O_2 -> 2 (Fe_2O_3)} \; \text{ossido ferrico}$$
Esiste poi l'ossido misto di ferro \ce{Fe_3O_4}, dato dalla somma algebrica di una molecola di \ce{FeO} più una di \ce{Fe_2O_3}, per cui due ferri saranno trivalenti e uno bivalente:
$$\ce{FeO + Fe_2O_3 -> Fe_3O_4} \; \text{ossido ferroso-ferrico o magnetite}$$

Cobalto
$$\ce{ Co + O_2 -> Co_2O_2=CoO} \; \text{(valenza 2)}$$
Bilanciamo
$$\ce{ 2 Co + O_2 -> 2 (CoO)} \; \text{ossido cobaltoso}$$

$$\ce{ Co + O_2 -> Co_2O_3} \; \text{(valenza 3)}$$ Bilanciamo
$$\ce{ 4 Co + 3 O_2 -> 2 (Co_2O_3)} \; \text{ossido cobaltico}$$

$$\ce{ Co_2O_2 + CoO -> Co_3O_4=} \; \text{ossido cobaltoso-cobaltico}$$

Nichel
$$\ce{ Ni + O_2 -> Ni_2O_2=NiO} \; \text{(valenza 2)}$$
Bilanciamo
$$\ce{ 2 Ni + O_2 -> 2 (NiO)} \; \text{ossido di nichel}$$

Rame
$$\ce{ Cu + O_2 -> Cu_2O} \; \text{(valenza 1)}$$
Bilanciamo
$$\ce{ 4 Cu + O_2 -> 2 (Cu_2O)} \; \text{ossido rameoso}$$

$$\ce{ Cu + O_2 -> Cu_2O} \; \text{(valenza 3)}$$
Bilanciamo
$$\ce{ 2 Cu + O_2 -> 2 (CuO)} \; \text{ossido rameico}$$