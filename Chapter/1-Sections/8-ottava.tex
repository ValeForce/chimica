\subsection{Reazioni di precipitazione}
In alcune reazioni avviene che, sebbene i reagenti siano solubili in acqua, i prodotti che si formano lo sono poco o per niente. In questo caso si forma un precipitato e la reazione è detta di precipitazione:
$$\ce{ AgNO_3_{(aq)} + NaCl_{(aq)} -> AgCl_{(s)} v  + NaNO_3_{(aq)} }$$
$$\text{nitrato d'argento + cloruro di sodio \ce{->} cloruro d'argento + nitrato di sodio}$$
Il simbolo (s) sta per "solido", la freccia verso il basso $\downarrow$ che precipita a fondo.
$$\ce{Pb(NO_3)_2_{(aq)} + K_2CrO_4_{(aq)} ->  PbCrO_4_{(s)} v + 2 K(NO_3){(aq)}}$$
$$\text{nitrato piomboso + cromato di potassio \ce{->} cromato di piombo + nitrato di potassio}$$
$$\ce{Pb(NO_3)_2_{(aq)} + 2 KI_{(aq)} ->  PbI_2_{(s)} v + 2 K(NO_3){(aq)}}$$
$$\text{nitrato piomboso + ioduro di potassio \ce{->} ioduro di piombo + nitrato di potassio}$$
\subsection{Reazioni di sintesi}
Si chiamano così soltanto quelle reazioni in cui i reagenti sono sostanze elementari.
$$\ce{2 H_2 + O_2 -> 2 H_2O}$$
$$\ce{3 H_2 + N_2 -> 2 NH_3} \; \text{ammoniaca}$$
$$\ce{H_2 + Cl_2 -> 2 HCl}$$
\subsection{Reazioni di decomposizione}
Sono reazioni in cui un composto si decompone in due o più composti. In alcuni casi per avvenire è necessario riscaldare.
$$\ce{CaCO_3 ->[{900^{\circ}C}] CaO + CO_2-{(gas)} }$$
Questa è la reazione inversa di quella in cui da ossido e anidride si forma il sale(nel senso che reagenti e prodotti sono invertiti). In essa il sale riscaldato si decompone in ossido e liberando in aria anidride carbonica 
$$\ce{PCl_5 <=>[{180^{\circ}C}] PCl_3 + Cl_2_{(gas)}}$$
$$\text{pentacloruro di fosforo \ce{<=>} tricloruro di fosforo + cloro}$$