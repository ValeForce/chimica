\subsection{Boro}
$$\ce{B + O_2 -> B_2O_3}$$
Bilancio
$$\ce{4 B + 3 O_2 -> 2 (B_2O_3)} \; \text{Anidride borica}$$
\subsection{Carbonio}
$$\ce{C + O_2 -> C_2O_2=CO} \; \text{(valenza 2)}$$ Bilancio
$$\ce{2 C + O_2 -> 2 (CO)} \; \text{anidride carboniosa (o monossido di carbonio)}$$
Idratiamo
$$\ce{CO + H_2O -> H_2CO_2} \; \text{acido carbonioso}$$

$$\ce{C + O_2 -> C_2O_4=CO_2} \; \text{(valenza 4)}$$ È già bilanciata
$$\ce{C + O_2 -> CO_2} \; \text{anidride carbonica (o biossido di carbonio)}$$
idratiamo:
$$\ce{CO_2 + H_2O -> H_2CO_3} \; \text{acido carbonico}$$
\subsection{Azoto}
In natura si trova in forma molecolare \ce{N_2}
$$\ce{N_2 + O_2 -> N_2O_3} \; \text{(valenza 3)}$$ Bilancio
$$\ce{2 N_2 + 3 O_2 -> 2 (N_2O_3)} \; \text{anidride nitrosa}$$
Idratiamo
$$\ce{N_2O_3 + H_2O -> H_2N_2O_4=HNO_2} \; \text{acido nitroso}$$

$$\ce{N_2 + O_2 -> N_2O_5} \; \text{(valenza 5)}$$ Bilancio
$$\ce{2 N_2 + 5 O_2 -> 2 (N_2O_5)} \; \text{anidride nitrica }$$
idratiamo:
$$\ce{N_2O_5 + H_2O -> H_2N_2O_6=HNO_3} \; \text{acido nitrico}$$
\subsection{Sesto gruppo}
%(non saprò mai perché non esiste la anidride ossigenosa). \\
$$\ce{S + O_2 -> S_2O_4=SO_2} \; \text{(valenza 4)}$$ È già bilanciata
$$\ce{S + O_2 -> SO_2} \; \text{anidride solforosa}$$
Idratiamo
$$\ce{SO_2 + H_2O -> H_2SO_3} \; \text{acido solforoso}$$

$$\ce{S + O_2 -> S_2O_6=SO_3} \; \text{(valenza 6)}$$ Bilancio
$$\ce{2 S + 3 O_2 -> 2 (SO_3)} \; \text{anidride solforica }$$
idratiamo:
$$\ce{SO_3 + H_2O -> H_2SO_4} \; \text{acido solforico}$$
\subsubsection{Settimo gruppo}
Il cloro ha valenza 1, 3, 5, 7. In natura si trova nella forma molecolare \ce{Cl_2}
$$\ce{Cl + O_2 -> Cl_2O} \; \text{(valenza 1)}$$ Bilancio
$$\ce{4 Cl + O_2 -> 2 (Cl_2O)} \; \text{anidride ipoclorosa}$$
Idratiamo:
$$\ce{Cl_2O + H_2O -> 2 (HClO)} \; \text{acido ipocloroso}$$

$$\ce{Cl_2 + O_2 -> Cl_2O_3} \; \text{(valenza 3)}$$ Bilancio
$$\ce{2 Cl_2 + 3 O_2 -> 2 (Cl_2O_3)} \; \text{anidride clorosa}$$
Idratiamo:
$$\ce{Cl_2O_3 + H_2O -> 2 (HClO_2) } \; \text{acido cloroso}$$

$$\ce{Cl_2 + O_2 -> Cl_2O_5} \; \text{(valenza 5)}$$ Bilancio
$$\ce{2 Cl_2 + 5 O_2 -> 2 (Cl_2O_5)} \; \text{anidride clorica}$$
Idratiamo:
$$\ce{Cl_2O_5 + H_2O -> 2 (HClO_3)} \; \text{acido clorico}$$

$$\ce{Cl_2 + O_2 -> Cl_2O_7} \; \text{(valenza 7)}$$ Bilancio
$$\ce{2 Cl_2 + 7 O_2 -> 2 Cl_2O_7} \; \text{anidride perclorica}$$
Idratiamo:
$$\ce{Cl_2O_7 + H_2O -> 2 (HClO_4)} \; \text{acido perclorico}$$

\subsection{Anidridi avide d'acqua}
Assorbono la quantità massima di molecole d'acqua, cioè 3.
% (assurdo pensavo 7 fosse la misura massima di ogni cosa).
$$\ce{B_2O_3 + \textbf{3} H_2O -> 2 (H_6B_2O_6)= 2 (H_3BO_3)} \; \text{acido \textbf{orto}borico}$$\hspace{+1.2cm}(prende tanta acqua sufficiente per un orto, cioè 3 molecole il massimo)
$$\ce{P_2O_3 + \textbf{3} H_2O -> 2 (H_6P_2O_6)= 2 (H_3PO_3)} \; \text{acido \textbf{orto}fosforoso}$$
$$\ce{P_2O_5 + \textbf{1} H_2O -> H_2P_2O_6= 2 (HPO_3)} \; \text{acido \textbf{meta}fosforico}$$
$$\ce{P_2O_5 + \textbf{2} H_2O -> H_4P_2O_7} \; \text{acido \textbf{piro}fosforico}$$
$$\ce{P_2O_5 + \textbf{3} H_2O -> H_6P_2O_8 = 2 (H_3PO_4)} \; \text{acido \textbf{orto}fosforico}$$
\subsection{Acidi ottenuti per condensazione}
Alcuni acidi si ottengono condensando due molecole di acido, cioè sottraendo ad esse una molecola d'acqua
$$\ce{CrO_3 + H_2O -> H_2CrO_4} \; \text{acido cromico}$$
$$\ce{2 (H_2CrO_4)-H_2O -> H_2Cr_2O_7} \; \text{acido bicromico}$$
$$\ce{MnO_3 + H_2O -> H_2MnO_4} \; \text{acido manganico}$$
$$\ce{Mn_2O_7 + H_2O -> 2 (HMnO_4)} \; \text{acido permanganico}$$

I due acidi di cromo hanno la stessa “valenza” (VI).

I due acidi di manganese hanno “valenza” diversa (VI e VII).

Anche gli acidi del fosforo possono essere ottenuti per condensazione:
$$\ce{2 (H_3PO_4) - H_2O -> H_4P_2O_7}$$
$$\ce{H_4PO_7 - H_2O -> H_2P_2O_6=HPO_3}$$
