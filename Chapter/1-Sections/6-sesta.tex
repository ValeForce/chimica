Sono acidi nati da reazioni tra idrogeno ed alogeni (elementi del settimo gruppo) e niente ossigeno.

$$\ce{HF} \; \text{acido fluoridrico}$$
$$\ce{HCl} \; \text{acido cloridrico}$$
$$\ce{HBr} \; \text{acido bromidrico}$$
$$\ce{HI} \; \text{acido iodidrico}$$
$$\ce{H_2S} \; \text{acido solfidrico}$$
\hspace{+4cm}(perché lo zolfo ha stato di ossidazione -2)
$$\ce{HCN (H-C+N)} \; \text{acido cianidrico}$$
$$\ce{HCNS (H-S-C+N)} \; \text{acido tiocianidrico}$$
Qui gli alogeni hanno numero di ossidazione -1, -2 (negativi). I sali ottenuti dagli idracidi hanno suffisso \textbf{-uro}. Ad esempio cloruro, fluoruro ecc...

