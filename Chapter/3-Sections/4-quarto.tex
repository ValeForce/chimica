Studiamo adesso in dettaglio il legame chimico tramite la teoria degli orbitali molecolari.

Ciò significa che studieremo il legame nelle molecole, sebbene sappiamo che ci sono molti sistemi che non esistono sotto forma molecolare, ossia ci sono composti fortemente ionici dei quali conosciamo i solidi ma non le molecole. Ne è un esempio l'NaCl: la specie molecolare del cloruro di sodio non esiste, ma esiste il solido.

Dobbiamo allora comprendere perché non esistano le molecole singole delle speci fortementi ioniche, per poi ragionare sul legame chimico, che ha una natura diversa nei sistemi molecolari.

Dobbiamo quindi rispondere a una serie di domande:
\begin{itemize}
    \item Perché si formano le molecole, dato che molti sistemi non esistono come tali?
    \item Quali condizioni bisogna soddisfare affinché il composto che si forma a seguito di una reazione, sia stabile?
    \item Perché esistono geometrie ben definite?
\end{itemize}
Prima di andare avanti va da ricordare che in linea generale se facciamo reagire una specie A con una specie B otterremo una specie C in modo spontaneo solo se l'energia del sistema diminuisce. 

Ricordiamo poi che qualunque stato legato avrà un'energia potenziale negativa: in un sistema AB dove A e B sono due atomi, essi sono legati se la loro energia potenziale è minore di zero. Se per qualche motivo l'energia dovesse aumentare, nell'istante in cui essa diventasse pari a zero gli atomi non sarebbero più legati.

Ragioniamo solo sull'energia potenziale per via dell'esistenza del teorema del viriale, il quale afferma che "\textit{La variazione dell'energia totale di un sistema ha lo stesso segno della variazione dell'energia potenziale}"
\subsection{Il legame polare}
\subsection{Il legame metallico}
\subsection{Il legame ionico}
\subsection{Il legame covalente}