Utilizzeremo il formalismo di Lewis per descrivere il legame in molecole semplici percHè i risultati spesso sono confrontabili con quelli delle teorie più sofisticate, cioè in molti casi il formalismo di Lewis descrive bene il legame chimico.
\subsection{Rappresentazione grafica}
Immaginiamo di avere due atomi A e B legati tra loro con un legame semplice. Questo legame si rappresenta con una lineetta che unisce A e B, e immaginiamo che sia dovuto a una coppia di elettroni. Gli elettroni esterni che invece non partecipano ai legami vengono raffigurati con dei puntini.

Es:

\setlength\extrarowheight{0.7cm}
\begin{tabular}{ m{2cm}m{6cm}m{2cm}m{6cm}}
    1° gruppo & $\chemfig{\charge{[circle]180=\.}{Li}} \quad \chemfig{\charge{[circle]180=\.}{Na}}$ & 5° gruppo & $\chemfig{\charge{[circle]0=\.,90=\.,180=\.,270=\:}{N}} \quad \chemfig{\charge{[circle]0=\.,90=\.,180=\.,270=\:}{P}}$\\
    2° gruppo & $\chemfig{\charge{[circle]0=\.,180=\.}{Be}} \quad \chemfig{\charge{[circle]0=\.,180=\.}{Mg}} \quad \chemfig{\charge{[circle]0=\.,180=\.}{Ca}} \quad \chemfig{\charge{[circle]0=\.,180=\.}{Sr}} \quad \chemfig{\charge{[circle]0=\.,180=\.}{Ba}}$ & 6° gruppo & $\chemfig{\charge{[circle]0=\.,90=\.,180=\.,270=\:}{O}} \quad \chemfig{\charge{[circle]0=\.,90=\.,180=\.,270=\:}{S}}$\\
    3° gruppo & $\chemfig{\charge{[circle]0=\.,90=\.,180=\.}{B}} \quad \chemfig{\charge{[circle]0=\.,90=\.,180=\.}{Al}} \quad \chemfig{\charge{[circle]0=\.,90=\.,180=\.}{Ga}} \quad \chemfig{\charge{[circle]0=\.,90=\.,180=\.}{In}} \quad \chemfig{\charge{[circle]0=\.,90=\.,180=\.}{Tl}}$ & 7° gruppo & $\chemfig{\charge{[circle]0=\:,90=\:,180=\.,270=\:}{F}} \quad \chemfig{\charge{[circle]0=\:,90=\:,180=\.,270=\:}{Cl}}$\\
    4° gruppo & $\chemfig{\charge{[circle]0=\.,90=\.,180=\.,270=\.}{C}} \quad \chemfig{\charge{[circle]0=\.,90=\.,180=\.,270=\.}{Si}}$ & 8° gruppo & $\chemfig{\charge{[circle]0=\:,90=\:,180=\:,270=\:}{Ne}} \quad \chemfig{\charge{[circle]0=\:,90=\:,180=\:,270=\:}{Ar}}$\\
\end{tabular}\\

Facciamo attenzione alla disposizione dei puntini successivamente al carbonio: il carbonio ha 4 elettroni esterni, quindi abbiamo 4 punti a 90° l'uno dall'altro. Nell'istante in cui il numero di elettroni è superiore a 4 gli elettroni iniziano ad accoppiarsi, quindi l'azoto avrà una coppia di elettroni e tre elettroni liberi, l'ossigeno due coppie di elettroni e due elettroni liberi, il fluoro tre coppie di elettroni e un elettrone libero, il neon quattro coppie di elettroni.

Stiamo quindi parlando soltanto degli elettroni di valenza, non di quelli interni.

Quindi, dopo aver messo un puntino su ogni "lato" a 90° l'uno dall'altro (e ci arriviamo col carbonio) iniziamo ad appaiare questi elettroni in modo tale da formare delle coppie che arrivano ad essere 4 col neon. Va poi da ricordare che il fondamento di questa teoria sta nel fatto che per gli elementi che stiamo considerando il numero di elettroni esterni rappresenta la configurazione elettronica esterna del gas nobile che segue questi elementi, ossia questi elementi tenderanno in qualche modo ad avere una configurazione elettronica esterna tipo gas nobile.

