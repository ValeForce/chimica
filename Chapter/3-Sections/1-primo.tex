Utilizzeremo il formalismo di Lewis per descrivere il legame in molecole semplici perché i risultati spesso sono confrontabili con quelli delle teorie più sofisticate, cioè in molti casi il formalismo di Lewis descrive bene il legame chimico.
\subsection{Rappresentazione grafica}
Immaginiamo di avere due atomi A e B legati tra loro con un legame semplice. Questo legame si rappresenta con una lineetta che unisce A e B, e immaginiamo che sia dovuto a una coppia di elettroni. Gli elettroni esterni che invece non partecipano ai legami vengono raffigurati con dei puntini.

Es:

\setlength\extrarowheight{0.7cm}
\begin{tabular}{ m{2cm}m{6cm}m{2cm}m{6cm}}
    1° gruppo & $\chemfig{\charge{[circle]180=\.}{Li}} \quad \chemfig{\charge{[circle]180=\.}{Na}}$ & 5° gruppo & $\chemfig{\charge{[circle]0=\.,90=\.,180=\.,270=\:}{N}} \quad \chemfig{\charge{[circle]0=\.,90=\.,180=\.,270=\:}{P}}$\\
    2° gruppo & $\chemfig{\charge{[circle]0=\.,180=\.}{Be}} \quad \chemfig{\charge{[circle]0=\.,180=\.}{Mg}} \quad \chemfig{\charge{[circle]0=\.,180=\.}{Ca}} \quad \chemfig{\charge{[circle]0=\.,180=\.}{Sr}} \quad \chemfig{\charge{[circle]0=\.,180=\.}{Ba}}$ & 6° gruppo & $\chemfig{\charge{[circle]0=\.,90=\.,180=\.,270=\:}{O}} \quad \chemfig{\charge{[circle]0=\.,90=\.,180=\.,270=\:}{S}}$\\
    3° gruppo & $\chemfig{\charge{[circle]0=\.,90=\.,180=\.}{B}} \quad \chemfig{\charge{[circle]0=\.,90=\.,180=\.}{Al}} \quad \chemfig{\charge{[circle]0=\.,90=\.,180=\.}{Ga}} \quad \chemfig{\charge{[circle]0=\.,90=\.,180=\.}{In}} \quad \chemfig{\charge{[circle]0=\.,90=\.,180=\.}{Tl}}$ & 7° gruppo & $\chemfig{\charge{[circle]0=\:,90=\:,180=\.,270=\:}{F}} \quad \chemfig{\charge{[circle]0=\:,90=\:,180=\.,270=\:}{Cl}}$\\
    4° gruppo & $\chemfig{\charge{[circle]0=\.,90=\.,180=\.,270=\.}{C}} \quad \chemfig{\charge{[circle]0=\.,90=\.,180=\.,270=\.}{Si}}$ & 8° gruppo & $\chemfig{\charge{[circle]0=\:,90=\:,180=\:,270=\:}{Ne}} \quad \chemfig{\charge{[circle]0=\:,90=\:,180=\:,270=\:}{Ar}}$\\
\end{tabular}
\setlength\extrarowheight{0cm}\\

Facciamo attenzione alla disposizione dei puntini successivamente al carbonio: il carbonio ha 4 elettroni esterni, quindi abbiamo 4 punti a 90° l'uno dall'altro. Nell'istante in cui il numero di elettroni è superiore a 4 gli elettroni iniziano ad accoppiarsi, quindi l'azoto avrà una coppia di elettroni e tre elettroni liberi, l'ossigeno due coppie di elettroni e due elettroni liberi, il fluoro tre coppie di elettroni e un elettrone libero, il neon quattro coppie di elettroni.

Stiamo quindi parlando soltanto degli elettroni di valenza, non di quelli interni.

Quindi, dopo aver messo un puntino su ogni "lato" a 90° l'uno dall'altro (e ci arriviamo col carbonio) iniziamo ad appaiare questi elettroni in modo tale da formare delle coppie che arrivano ad essere 4 col neon. Va poi da ricordare che il fondamento di questa teoria sta nel fatto che per gli elementi che stiamo considerando il numero di elettroni esterni rappresenta la configurazione elettronica esterna del gas nobile che segue questi elementi, ossia questi elementi tenderanno in qualche modo ad avere una configurazione elettronica esterna tipo gas nobile. In altre parole, gli elementi perdono o acquistano eletteroni in modo tale da raggiungere la configurazione esterna del gas nobile. Tale comportamento prende il nome di \textbf{regola dell'ottetto}:

"\textit{Gli atomi, nel legarsi tra loro, perdono, acquistano o cedono elettroni per raggiungere uno strato esterno di 8 elettroni. Gli atomi tendono a fare ciò per assumere la configurazione del gas nobile più vicino}". 

Tale regola vale per elementi del secondo e del terzo periodo e per una gran parte degli altri in composti semplici.

ES1

Consideriamo il fluoruro di litio LiF. Il litio ha un solo elettrone esterno di valenza, l'elettrone s2. Il fluoro appartiene al settimo gruppo e ha quindi 7 elettroni esterni, che sono 2 elettroni 2s e 5 elettroni 2p ($2s^22p^5$), quindi per entrambi gli elementi gli elettroni 1s non sono di valenza.

Immaginiamo che la molecola LiF sia rappresentabile come un sistema puramente ionico nel quale il litio abbia ceduto il suo unico elettronce esterno al fluoro:
\\


$$\schemestart
\chemfig{Li[He]@{x1}{2s^1}}
\+
\chemfig{F[He]2s^2@{x2}{2p^5}}
\arrow
\chemfig{Li+1s^2 \text{(o[He])}}
\+
\chemfig{F^{-}[He]2s^22p^6 \text{(o [Ne])}}
\schemestop
\chemmove[shorten <=2pt]{
\draw(x1)..controls +(60:1cm)and+(120:1cm)..(x2);}$$

$$\ce{Li \subshells{{1s:2}{2s:1}{2p:000}} + F \subshells{{1s:2}{2s:2}{2p:221}} -> Li^+ \subshells{{1s:2}{2s:0}{2p:000}} + F- \subshells{{1s:2}{2s:2}{2p:222}}}$$

$$\schemestart
\chemfig{@{x1}{\charge{[circle]180=\.}{Li}}}
\+
\chemfig{@{x2}{\charge{[circle]0=\:,90=\:,180=\.,270=\:}{F}}}
\arrow
\chemfig{Li^+}
\+
\chemfig{\charge{[circle]0=\:, 45=$\scriptstyle-$, 90=\:,180=\:,270=\:}{F}}
\schemestop
\chemmove[shorten >=4pt]{
\draw(x1.155)..controls +(60:0.6cm)and+(120:0.6cm)..(x2.190);}$$

Dopo la freccia il litio non ha più l'elettrone 2s (quindi gli orbitali 2s e 2p sono tutti vuoti). Il fluoro invece ha acquistato questo elettrone ceduto dal litio e adesso ha i livelli 2p pieni, quindi su di esso abbiamo 8 elettroni di valenza, cioè esterni: 2 nel livello 2s e 6 nei tre livelli 2p. In questo modo raggiunge quindi l'ottetto. Il litio, avendo perso l'unico suo elettrone esterno 2s, ha assunto la configurazione elettronica esterna del gas nobile che lo precede, cioè l'elio.

Quindi nel sistema LiF sia il litio, rappresentato in forma cationica, che il litio, rappresentato in forma anionica, si trovano ad avere la configurazione elettronica del gas nobile che lo precede (litio) o che che lo segue (fluoro). Utilizzando il formalismo di Lewis si vede come il litio abbia formalmente ceduto un elettrone al fluoro e su quest'ultimo si abbiano 4 coppie, ossia 8 elettroni esterni.

Questo esempio ci indica il modo di ragionare per ottenere le strutture di Lewis.

Attenzione! il formalismo di Lewis non ci permette di ragionare sulle geometrie delle molecole: non ci sono dati che ci permettono di dire nulla sulle geometrie, anche se scriveremo delle molecole tutte su un piano, ma questo approccio non rappresenta una disposizione spaziale della molecola che andremo a scrivere/disegnare.

Es2 \ce{MgCl_2}

Il magnesio è un elemento del secondo gruppo, ed ha 2 elettroni di tipo s. I suoi orbitali di tipo p sono totalmente vuoti. Il cloro invece è un alogeno, cioè appartiene al 7° gruppo e ha quindi sette elettroni esterni: due nel livello 2s e cinque nel livello 2p.

Immaginiamo che il sistema sia puramente ionico. Sotto questa ipotesi ciò che avviene è che i due elettroni del magnesio vengono ceduti a due diversi atomi di cloro, i quali hanno ciascuno sette elettroni esterni, e in questo modo raggiungono l'ottetto:
$$\ce{Mg \subshells{{1s:2}{2s:2}{2p:000}} + 2Cl \subshells{{1s:2}{2s:2}{2p:221}} -> Mg^{2+} \subshells{{1s:2}{2s:0}} + 2Cl- \subshells{{1s:2}{2s:2}{2p:222}}}$$

$$\schemestart
\chemfig{@{x1}{\charge{[circle]180=\:}{Mg \; +}}(-[1,,,,draw=none]@{x2}{\charge{[circle]0=\:,90=\:,180=\.,270=\:}{Cl}})(-[7,,,,draw=none]@{x3}{\charge{[circle]0=\:,90=\:,180=\.,270=\:}{Cl}})}
\arrow
\chemfig{Mg^{2+}}
\+
2 \chemleft(\chemfig{\charge{[circle]0=\:, 90=\:,180=\:,270=\:}{Cl}}\chemright{)^{-}}
\schemestop
\chemmove[shorten <=6pt, shorten >=4pt]{
\draw(x1.180)..controls +(110:0.8cm)and+(0:0cm)..(x2.200);
\draw(x1.180)..controls +(70:-0.8cm)and+(0:0cm)..(x3.150);
}$$

Ciò significa che stiamo immaginando che il magnesio perda totalmente due elettroni di tipo s, per cui avrà una carica due volte positiva, e il cloro acquisti un elettrone donato dal magnesio per completare il riempimento dei suoi livelli p. Ovviamente, se il magnesio cede due elettroni e il cloro può acquistarne solo uno, serviranno due atomi di cloro, i quali diventeranno due anioni cloruro.
\subsection{Eccezioni alla regola dell'ottetto}
La regola dell'ottetto parte dall'idea che riempire totalmente i livelli di valenza conferisce elevata stabilità al sistema. Quindi, ragionando in termini di blocco s e p, questo significa avere otto elettroni esterni al massimo, perché nei livelli s e p in totale riusciamo a mettere 8 elettroni. Tuttavia per alcuni composti essa non vale più:
\begin{itemize}
    \item Per gli elementi di transizione e i loro sistemi molecolari si parla di "\textbf{regola dei 18 elettroni}". Tali elementi presentano orbitali 4s, 3d e 4p. Ciò significa che possono ospitare 2 elettroni sui 4s, 10 elettroni sui 3d e 6 elettroni sui 4p, per un totale di 18 elettroni.
    \item Ci sono molecole che non hanno la struttura tale da raggiungere la somma degli elettroni pari a 8. Ad esempio la molecola \ce{H_2}, due atomi di idrogeno che danno luogo alla sua forma molecolare, avrà un totale di due elettroni, in quanto l'atomo di idrogeno ne possiede solo uno. Altri esempi sono gli idruri: \ce{LiH, \; BeH_2}.
    \item Ci sono molecole che presentano un numero di elettroni superiore a 8. In questi casi si parla di "\textbf{valenza espansa}".
    \item Ci sono infine vari casi in cui il numero di elettroni è dispari, per cui ovviamente non può essere 8.
\end{itemize}
\newpage
\subsection{Criteri per scrivere le strutture di Lewis}
\begin{itemize}
    \item L'atomo di idrogeno è sempre un atomo terminale, quindi in molecole con più atomi sappiamo per certo che l'atomo di idrogeno non andrà al centro;
    \item L'atomo che presenta la più bassa elettronegatività (esclusi H e ioni metallici) è quello centrale e gli altri vanno disposti attorno ad esso. Se quindi ad esempio in una molecola avessi carbonio e ossigeno al centro ci starebbe il carbonio, perché l'ossigeno è più elettronegativo del carbonio e quindi non si mette al centro. Se avessi zolfo e ossigeno, metterei lo zolfo al centro e l'ossigeno ai lati;
    \item Si considera il numero totale di elettroni di valenza, tenendo conto di eventuali cariche formali (ioni). Cioè: contiamo tutti gli elettroni di valenza di tutti gli atomi e li sommiamo, in modo tale da sapere quanti siamo gli elettroni totali di valenza della molecola. Se poi il sistema molecolare fosse carico ne dovremmo tenere conto, ossia se ha una carica n volte negativa dovremo aggiungere n elettroni al conteggio, se avesse carica m volte positiva dovremo sottrarre m elettroni;
    \item Si pone un legame singolo (detto legame $\sigma$), ovvero una coppia di elettroni tra ciascuna coppia di atomi
    
    Cioè: avendo messo un atomo al centro, metto gli altri attorno (tenendo presente che non c'è una regola su come metterli attorno, perché non sto ragionando sulla reale geometria, che questa teoria non mi permette di vedere) e li unisco a quello centrae con un legame singolo, che immagino essere la rappresentazione di una coppia di elettroni di legame. Se immagino che ogni legame consumi due elettroni, avendo già fatto la somma totale degli elettroni, sottraggo gli elettroni che ho usato per i legami semplici al conto totale. Se restano elettroni, questi vanno assegnati a coppie agli atomi esterni (che sono quelli più elettronegativi) tentando di far raggiungere ad ognuno di questi l'ottetto. Se continuano ad avanzarne anche dopo che tutti gli elettroni esterni hanno raggiunto l'ottetto, li assegno all'atomo centrale.
    \item Si sottrae dal numero di elettroni totali quelli usati per i legami $\sigma$;
    \item Le coppie di elettroni restanti vanno assegnate agli atomi esterni, cercando di soddisfare la regola dell'ottetto; gli elettroni che dovessero ancora risultare in eccesso vanno assegnati all'atomo centrale;
    \item Se dopo quest'ultima operazione l'atomo centrale non risulta ancora circondato da quattro coppie elettroniche, si trasformano alcuni doppietti degli atomi periferici in ulteriore legami con l'atomo centrale. Questi saranno detti legami $\pi$ (presenti in genere con atomi quali C, N, O e S)
\end{itemize}

A questo punto devo verificare se tutti gli atomi hanno raggiunto l'ottetto. Se si ho finito, se no inizio a trasformare alcuni doppietti cioè alcune coppie di elettroni degli atomi esterni in doppi legami, con l'obiettivo di far raggiungere anche all'atomo centrale l'ottetto elettronico. Secondo questa teoria, se i  legami iniziali erano etichettati legami semplici di tipo $\sigma$, i successivi (cioè laddove avremo per esempio un doppio legame) verranno etichettati legami $\pi$.

Per quanto riguarda gli orbitali: orbitale di natura $\sigma$ signidica che c'è densità di carica lungo l'asse di legame, cioè lungo la congiungente dei due nuclei; orbitale di natura $\pi$ signfica che lungo l'asse di legame ci sarà un piano nodale

Attenzione! La regola di Lewis per cui il secondo legame che si forma è di tipo $\pi$ non è sempre vera, infatti ci sono alcune molecole in cui sono presenti due legami, nma non è vero che sono uno $\sigma$ e uno $\pi$. Una di queste è la molecola \ce{C_2}, in cui i due atomi di carbonio sono legati da un doppio legame ed entrambi i legami hanno natura $\pi$.

Come abbiamo visto si lavora solo con gli elettroni di valenza. Va da ricordare:
\begin{itemize}
    \item Gli elementi del 1° gruppo hanno un solo elettrone di valenza;
    \item Quelli del 2° gruppo ne hanno due;
    \item Quelli del 3° gruppo ne hanno tre;
    \item Quelli del 4° gruppo ne hanno quattro;
    \item Così via
\end{itemize}

Se quindi volessi sapere ad esempio quanti elettroni di valenza ha il silicio, come scrivo la configurazione elettronica?

Il silicio ha pieni gli orbitali 1s, 2s e 2p. Ha poi come elettroni di valenza i 3s e i 3p. In particolare il silicio sta sotto l'ossigeno\footnote{La sequenza è: ossigeno, silicio, germanio, stagno e piombo}, quindi esso avrà due elettroni nei livelli 3s e due elettroni sui 3p (come l'ossigeno, che ha configurazione $2s^22p^2$). Inoltre la configurazione $1s^22s^22p^63s^23p^2$ può essere scritta come configurazione del neon (che è il gas nobile che precede il silicio) più i suoi elettroni di valenza: $\text{[Ne]}3s^23p^2$.
\subsection{Esempi}
\begin{itemize}
    \item ES1 \ce{N_2} \quad n.e.=10
    
    L'azoto appartiene al quinto gruppo (azoto, fosforo, arsenico, antimonio e bismuto), quindi i suoi elettroni di valenza sono 5: due di tipo s e tre di tipo p. Avendone due atomi, in totale avremo 10 elettroni.

    Avendo solo due atomi dello stesso tipo, non si pone il problema di chi porre al centro, si possono considerare come due punti uniti da una linea, quindi li mettiamo accanto e li uniamo con un segmento che rappresenta un legame, il quale è costituito da una coppia di elettroni:
    $$\schemestart
    \chemfig{N-N}
    \schemestop 
    \quad 
    \text{n.e. restanti=8}$$
    Ciò significa che adesso restano 8 elettroni, che dovrò distribuire a coppie sui due atomi. Riusciamo a mettere due coppie su un azoto e due coppie sull'altro.

    A questo punto contiamo gli elettroni su ogni atomo. Va da notare che i due elettroni impiegati nel legame contano come 2 elettroni per ciascun atomo, per cui per entrambi avremo 4 elettroni dati da due coppie più due elettroni di legame, per un totale di 6. Entrambi gli atomi di azoto quindi non raggiungono l'ottetto. Siccome la formula di Lewis prevede che ci siano 8 elettroni attorno a tutti gli atomi andiamo avanti: per ognuno dei due atomi, trasformiamo un doppietto in un doppio legame:
    $$\schemestart
    \chemfig{@{x1}{\charge{[circle]90=\:,270=\:}{N}}-[@{x2}]@{x3}{\charge{[circle]90=\:,270=\:}{N}}}
    \schemestop
    \chemmove[shorten <=4pt, shorten >=4pt]{
    \draw(x3.290)..controls +(60:-0.5cm)and+(90:-0.5cm)..(x2.90);
    \draw(x1.110)..controls +(60:0.5cm)and+(90:0.5cm)..(x2);
    }$$
    Se trasformassi un solo doppietto in doppio legame (ad esempio uno dell'azoto a sinistra), sull'atomo a sinistra avrei 2 elettroni dati dai doppietti più 4 elettroni di legame, quindi il totale è ancora 6; sull'atomo a destra invece avrei 
    $$\schemestart
    \chemfig{\charge{[circle]180=\:}{N}~\charge{[circle]0=\:}{N}}
    \schemestop$$
    \item ES2 \ce{O_2}
    $$\schemestart
    \chemfig{O-O}
    \schemestop 
    \quad
    \text{n.e. restanti=10}$$
    paperino
    $$\schemestart
    \chemfig{@{x1}{\charge{[circle]90=\:,180=\:,270=\:}{O}}-[@{x2}]\charge{[circle]90=\:,270=\:}{O}}
    \schemestop
    \chemmove[shorten <=4pt, shorten >=4pt]{
    \draw(x1.110)..controls +(60:0.5cm)and+(90:0.5cm)..(x2);}$$
    paperino
    $$\schemestart
    \chemfig{\charge{[circle]90=\:,270=\:}{O} = \charge{[circle]90=\:,270=\:}{O}}
    \schemestop$$
    \item ES3 CO
    $$\schemestart
    \chemfig{C-O}
    \schemestop 
    \quad
    \text{n.e. restanti=8}$$
    paperino
    $$\schemestart
    \chemfig{\charge{[circle]180=\:}{C}-[@{x1}]@{x2}{\charge{[circle]0=\:,90=\:,270=\:}{O}}}
    \schemestop
    \chemmove[shorten <=4pt, shorten >=4pt]{
    \draw(x2.290)..controls +(60:-0.5cm)and+(90:-0.5cm)..(x1.90);
    \draw(x2.70)..controls +(120:0.5cm)and+(90:0.5cm)..(x1.90);
    }$$
    paperino
    $$\schemestart
    \chemfig{\charge{[circle]180=\:}{C}~\charge{[circle]0=\:}{O}}
    \schemestop$$
    \item ES4 NO
    
    $$\schemestart
    \chemfig{N-O}
    \schemestop
    \quad
    \text{n.e. restanti=9}
    $$
    paperino
    $$\schemestart
    \chemfig{\charge{[circle]90=\.,180=\:}{N}-[@{x1}]@{x2}{\charge{[circle]0=\:,90=\:,270=\:}{O}}}
    \schemestop
    \chemmove[shorten <=4pt, shorten >=4pt]{
    \draw(x2.290)..controls +(60:-0.5cm)and+(90:-0.5cm)..(x1.90);}
    $$
    paperino
    $$\schemestart
    \chemfig{\charge{[circle]90=\.,180=\:}{N}=\charge{[circle]0=\:,90=\:}{O}}
    \schemestop$$
    \item ES5 NO$^+$
    $$\schemestart
    \chemleft[\chemfig{N-O}\chemright{]^{+}}
    \schemestop
    \quad
    \text{n.e. restanti=8}
    $$
    paperino
    $$\schemestart
    \chemleft[\chemfig{\charge{[circle]90=\:}{N}-[@{x1}]@{x2}{\charge{[circle]0=\:,90=\:,270=\:}{O}}} \,\chemright{]^{+}}
    \schemestop
    \chemmove[shorten <=4pt, shorten >=4pt]{
    \draw(x2.290)..controls +(60:-0.5cm)and+(90:-0.5cm)..(x1.90);
    \draw(x2.70)..controls +(120:0.5cm)and+(90:0.5cm)..(x1.90);
    }$$
    paperino
    $$\schemestart
    \chemleft[\chemfig{\charge{[circle]90=\:}{N}~\charge{[circle]90=\:}{O}}\chemright{]^{+}}
    \schemestop$$
    \item ES6 HCl
    $$\schemestart
    \chemfig{H-Cl}
    \schemestop
    \quad
    \text{n.e. restanti=6}
    $$
    paperinp
    $$\schemestart
    \chemfig{H-\charge{[circle]0=\:,90=\:,270=\:}{Cl}}
    \schemestop$$
    \item ES7 \ce{CO_2}
    $$\schemestart
    \chemfig{O-C-O}
    \schemestop
    \quad
    \text{n.e. restanti=12}
    $$
    paperino
    $$\schemestart
    \chemfig{@{x1}{\charge{[circle]90=\:,180=\:,270=\:}{O}}-[@{x2}]C-[@{x3}]@{x4}{\charge{[circle]0=\:,90=\:,270=\:}{O}}}
    \schemestop
    \chemmove[shorten <=4pt, shorten >=4pt]{
    \draw(x1.110)..controls +(60:0.5cm)and+(90:0.5cm)..(x2);
    \draw(x4.290)..controls +(60:-0.5cm)and+(90:-0.5cm)..(x3.90);
    }$$
    paperino
    $$\schemestart
    \chemfig{\charge{[circle]180=\:,270=\:}{O}=C=\charge{[circle]0=\:,270=\:}{O}}
    \schemestop
    $$
    \item ES8 \ce{NO_2}
    $$\schemestart
    \chemfig{O-N-O}
    \schemestop
    \quad
    \text{n.e. restanti=13}
    $$
    paperino
    $$\schemestart
    \chemfig{@{x1}{\charge{[circle]90=\:,180=\:,270=\:}{O}}-[@{x2}]\charge{[circle]90=\.}{N}-\charge{[circle]0=\:,90=\:,270=\:}{O}}
    \schemestop
    \chemmove[shorten <=4pt, shorten >=4pt]{
    \draw(x1.110)..controls +(60:0.5cm)and+(90:0.5cm)..(x2);
    }$$
    paperino
    $$\schemestart
    \chemfig{\charge{[circle]180=\:,270=\:}{O}=\charge{[circle]90=\.}{N}-\charge{[circle]0=\:,90=\:,270=\:}{O}}
    \schemestop
    $$
    \item ES9 \ce{O_3}
    $$\schemestart
    \chemfig{O-O-O}
    \schemestop
    \quad
    \text{n.e. restanti=14}
    $$
    paperino
    $$\schemestart
    \chemfig{@{x1}{\charge{[circle]90=\:,180=\:,270=\:}{O}}-[@{x2}]\charge{[circle]90=\:}{O}-\charge{[circle]0=\:,90=\:,270=\:}{O}}
    \schemestop
    \chemmove[shorten <=4pt, shorten >=4pt]{
    \draw(x1.110)..controls +(60:0.5cm)and+(90:0.5cm)..(x2);
    }$$
    paperino
    $$\schemestart
    \chemfig{\charge{[circle]180=\:,270=\:}{O}=\charge{[circle]90=\:}{O}-\charge{[circle]0=\:,90=\:,270=\:}{O}}
    \schemestop
    $$
    \item ES10 \ce{NO_2^-}
    $$\schemestart
    \chemleft[\chemfig{O-N-O}\chemright{]^{-}}
    \schemestop
    \quad
    \text{n.e. restanti=14}
    $$
    paperino
    $$\schemestart
    \chemleft[ \chemfig{@{x1}{\charge{[circle]90=\:,180=\:,270=\:}{O}}-[@{x2}]\charge{[circle]90=\:}{N}-\charge{[circle]0=\:,90=\:,270=\:}{O}} \chemright{]^{-}}
    \schemestop
    \chemmove[shorten <=4pt, shorten >=4pt]{
    \draw(x1.110)..controls +(60:0.5cm)and+(90:0.5cm)..(x2);
    }$$
    paperino
    $$\schemestart
    \chemleft[ \chemfig{\charge{[circle]180=\:,270=\:}{O}=\charge{[circle]90=\:}{N}-\charge{[circle]0=\:,90=\:,270=\:}{O}} \chemright{]^{-}}
    \schemestop
    $$
    \item ES11 HCN
    $$\schemestart
    \chemfig{H-C-N}
    \schemestop 
    \quad
    \text{n.e. restanti=6}$$
    paperino
    $$\schemestart
    \chemfig{H-C-[@{x1}]@{x2}{\charge{[circle]0=\:,90=\:,270=\:}{N}}}
    \schemestop
    \chemmove[shorten <=4pt, shorten >=4pt]{
    \draw(x2.290)..controls +(60:-0.5cm)and+(90:-0.5cm)..(x1.90);
    \draw(x2.70)..controls +(120:0.5cm)and+(90:0.5cm)..(x1.90);
    }$$
    paperino
    $$\schemestart
    \chemfig{H-C~\charge{[circle]90=\:}{N}}
    \schemestop$$
    \item ES12 \ce{CH_2O}
    $$
    \schemestart
    \chemfig{H-C(-[6]O)(-H)}
    \schemestop
    \quad
    \text{n.e. restanti=6}
    $$
    paperino
    $$
    \schemestart
    \chemfig{H-C([6]-[@{x1}]@{x2}{\charge{[circle]0=\:,180=\:,270=\:}{O}})(-H)}
    \schemestop
    \chemmove[shorten <=4pt, shorten >=4pt]{
    \draw(x2.180)..controls +(180:0.5cm)and+(180:0.8cm)..(x1.60);}
    $$
    paperino
    $$
    \schemestart
    \chemfig{H-C(=[6]\charge{[circle]0=\:,270=\:}{O})(-H)}
    \schemestop
    $$
    \item ES13 \ce{CO_3^{2-}}
    $$
    \schemestart
    \chemleft[\chemfig{O-C(-[6]O)(-O)}\chemright{]^{2-}}
    \schemestop
    \quad
    \text{n.e. restanti=6}
    $$
    paperino
    $$
    \schemestart
    \chemleft[ \chemfig{\charge{[circle]90=\:,180=\:,270=\:}{O}-C([6]-[@{x1}]@{x2}{\charge{[circle]0=\:,180=\:,270=\:}{O}})(-\charge{[circle]0=\:,90=\:,270=\:}{O})} \chemright{]^{2-}}
    \schemestop
    \chemmove[shorten <=4pt, shorten >=4pt]{
    \draw(x2.180)..controls +(180:0.5cm)and+(180:0.8cm)..(x1.60);}
    $$
    paperino
    $$
    \schemestart
    \chemleft[ \chemfig{\charge{[circle]90=\:,180=\:,270=\:}{O}-C(=[6]\charge{[circle]0=\:,270=\:}{O})(-\charge{[circle]0=\:,90=\:,270=\:}{O})} \chemright{]^{2-}}
    \schemestop
    $$
    \item ES14 \ce{NO_3^-}
    $$
    \schemestart
    \chemleft[\chemfig{O-N(-[6]O)(-O)}\chemright{]^{-}}
    \schemestop
    \quad
    \text{n.e. restanti=6}
    $$
    paperino
    $$
    \schemestart
    \chemleft[ \chemfig{\charge{[circle]90=\:,180=\:,270=\:}{O}-N([6]-[@{x1}]@{x2}{\charge{[circle]0=\:,180=\:,270=\:}{O}})(-\charge{[circle]0=\:,90=\:,270=\:}{O})} \chemright{]^{-}}
    \schemestop
    \chemmove[shorten <=4pt, shorten >=4pt]{
    \draw(x2.180)..controls +(180:0.5cm)and+(180:0.8cm)..(x1.60);}
    $$
    paperino
    $$
    \schemestart
    \chemleft[ \chemfig{\charge{[circle]90=\:,180=\:,270=\:}{O}-N(=[6]\charge{[circle]0=\:,270=\:}{O})(-\charge{[circle]0=\:,90=\:,270=\:}{O})} \chemright{]^{-}}
    \schemestop
    $$
\end{itemize}
\subsection{Strutture molecolari pi sbaddu}
\begin{itemize}
    \item Gruppo 4A
    
    
    \chemname{\footnotesize\chemfig{H-C(-[2]H)(-[6]H)(-H)}}{metano \ce{CH_4}}
    \qquad
    \chemname{\footnotesize\chemfig{H-C(-[2]H)(-[6]H)-C(-[2]H)(-[6]H)-H}}{etano \ce{C_2H_6}}
    \qquad
    \chemname{\footnotesize\chemfig{H-C(-[6]H)=C(-[6]H)-H}}{etilene \ce{C_2H_4}}
    \qquad
    \chemname{\footnotesize\chemfig{H-C~C-H}}{acetilene \ce{C_2H_2}}
    
    \item Gruppo 5A
    
    \chemname{\footnotesize\chemfig{H-\charge{[circle]90=\:}{N}(-[6]H)-H}}{ammoniaca \ce{NH_3}}
    \qquad
    \chemname{\footnotesize\chemfig{H-\charge{[circle]90=\:}{N}(-[6]H)-\charge{[circle]90=\:}{N}(-[6]H)-H}}{idrazina \ce{N_2H_4}}
    \qquad
    \chemname{\footnotesize\chemleft[ \chemfig{H-N(-[6]H)(-[2]H)-H} \chemright{]^{+}}}{ione ammonio \ce{NH_4^+}}
    \qquad 
    \chemname{\footnotesize\chemleft[ \chemfig{H-\charge{[circle]90=\:,270=\:}{N}-H} \chemright{]^{-}}}{ione ammide \ce{NH_2^-}} 
    \item Gruppo 6A
    
    \chemname{\footnotesize\chemfig{H-\charge{[circle]90=\:,270=\:}{O}-H}}{acqua \ce{H_2O}}
    \qquad
    \chemname{\footnotesize\chemfig{H-\charge{[circle]90=\:,270=\:}{O}-\charge{[circle]90=\:,270=\:}{O}-H}}{perossido di\\diidrogeno \ce{H_2O_2}}
    \qquad
    \chemname{\footnotesize\chemleft[ \chemfig{H-\charge{[circle]90=\:}{O}(-[6]H)-H} \chemright{]^{+}}}{ione idronio \ce{H_3O^+}} 
    \qquad \,
    \chemname{\footnotesize\chemleft[ \chemfig{\charge{[circle]90=\:,180=\:,270=\:}{O}-H} \chemright{]^{-}}}{ione idrossido \ce{OH^-}}
    \item Gruppo 7A
    
    \chemname{\footnotesize\chemfig{H-\charge{[circle]0=\:,90=\:,270=\:}{F}}}{fluoruro di idrogeno \ce{HF}}
    \item Ossiacidi comuni e i loro anioni
    
    \chemname{\footnotesize\chemfig{H-\charge{[circle]90=\:,270=\:}{O}-N(-[6]\charge{[circle]0=\:,180=\:,270=\:}{O})=\charge{[circle]0=\:,90=\:}{O}}}{acido nitrico \ce{HNO_3}}
    \qquad
    \chemname{\footnotesize\chemleft[ \chemfig{\charge{[circle]90=\:,180=\:,270=\:}{O}-N(-[6]\charge{[circle]0=\:,180=\:,270=\:}{O})=\charge{[circle]0=\:,90=\:}{O}} \chemright{]^{-}}}{ione nitrato \ce{NO_3^-}}
    \qquad
    \chemname{\footnotesize\chemfig{\charge{[circle]90=\:,180=\:,270=\:}{O}-Cl(-[2]\charge{[circle]90=\:,180=\:}{O}-H)(-[6]\charge{[circle]0=\:,180=\:,270=\:}{O})-\charge{[circle]0=\:,90=\:,270=\:}{O}}}{acido perclorico\\\ce{HClO_4}}
    \qquad
    \chemname{\footnotesize\chemleft[ \chemfig{\charge{[circle]90=\:,180=\:,270=\:}{O}-Cl(-[2]\charge{[circle]90=\:,180=\:}{O})(-[6]\charge{[circle]0=\:,180=\:,270=\:}{O})-\charge{[circle]0=\:,90=\:,270=\:}{O}} \chemright{]^{-}}}{ione perclorato\\\ce{ClO_4^-}}

    \chemname{\footnotesize\chemfig{\charge{[circle]90=\:,180=\:,270=\:}{O}-P(-[2]\charge{[circle]90=\:,180=\:}{O}-H)(-[6]\charge{[circle]0=\:,270=\:}{O}-[4]H)(-\charge{[circle]0=\:,90=\:}{O}-[6]H)}}{acido fosforico\\\ce{H_3PO_4}}
    \qquad
    \chemname{\footnotesize\chemleft[ \chemfig{\charge{[circle]90=\:,180=\:,270=\:}{O}-P(-[2]\charge{[circle]0=\:,90=\:,180=\:}{O})(-[6]\charge{[circle]0=\:,180=\:,270=\:}{O})-\charge{[circle]0=\:,90=\:,270=\:}{O}}\chemright{]^{3-}}}{ione fosfato\\\ce{PO_4^-}}
    \qquad
    \chemname{\footnotesize\chemfig{H-\charge{[circle]90=\:,270=\:}{O}-\charge{[circle]0=\:,90=\:,270=\:}{Cl}}}{acido ipocloroso\\\ce{HOCl}}
    %\chemname{\footnotesize\chemleft[ \chemfig{\charge{[circle]90=\:,270=\:}{O}-\charge{[circle]0=\:,90=\:,270=\:}{Cl}} \chemright{]^{-}}}}{ione ipoclorito \ce{OCl^-}}
    \qquad
    %\chemname{\footnotesize\chemleft[ \chemfig{\charge{[circle]90=\:,270=\:}{O}-\charge{[circle]0=\:,90=\:,270=\:}{Cl}}\chemright{]^{-}}{ione ipoclorito\\\ce{OCl^-}}
    \chemname{\footnotesize\chemleft[ \chemfig{\charge{[circle]90=\:,180=\:,270=\:}{O}-\charge{[circle]0=\:,90=\:,270=\:}{Cl}} \chemright{]^{-}}}{ione ipoclorito\\\ce{ClO^-}}

    \chemname{\footnotesize\chemfig{\charge{[circle]90=\:,180=\:,270=\:}{O}-S(-[2]\charge{[circle]90=\:,180=\:}{O}-H)(-[6]\charge{[circle]180=\:,270=\:}{O}-H)-\charge{[circle]0=\:,90=\:,270=\:}{O}}}{acido solforico\\\ce{H_2SO_4}}
\end{itemize}
\subsection{Il concetto di risonanza}