Utilizzeremo il formalismo di Lewis per descrivere il legame in molecole semplici perché i risultati spesso sono confrontabili con quelli delle teorie più sofisticate, cioè in molti casi il formalismo di Lewis descrive bene il legame chimico.
\subsection{Rappresentazione grafica}
Immaginiamo di avere due atomi A e B legati tra loro con un legame semplice. Questo legame si rappresenta con una lineetta che unisce A e B, e immaginiamo che sia dovuto a una coppia di elettroni. Gli elettroni esterni che invece non partecipano ai legami vengono raffigurati con dei puntini.

\textbf{ES}:

\setlength\extrarowheight{0.7cm}
\begin{tabular}{ m{2cm}m{6cm}m{2cm}m{6cm}}
    1° gruppo & $\chemfig{\charge{[circle]180=\.}{Li}} \quad \chemfig{\charge{[circle]180=\.}{Na}}$ & 5° gruppo & $\chemfig{\charge{[circle]0=\.,90=\.,180=\.,270=\:}{N}} \quad \chemfig{\charge{[circle]0=\.,90=\.,180=\.,270=\:}{P}}$\\
    2° gruppo & $\chemfig{\charge{[circle]0=\.,180=\.}{Be}} \quad \chemfig{\charge{[circle]0=\.,180=\.}{Mg}} \quad \chemfig{\charge{[circle]0=\.,180=\.}{Ca}} \quad \chemfig{\charge{[circle]0=\.,180=\.}{Sr}} \quad \chemfig{\charge{[circle]0=\.,180=\.}{Ba}}$ & 6° gruppo & $\chemfig{\charge{[circle]0=\.,90=\.,180=\.,270=\:}{O}} \quad \chemfig{\charge{[circle]0=\.,90=\.,180=\.,270=\:}{S}}$\\
    3° gruppo & $\chemfig{\charge{[circle]0=\.,90=\.,180=\.}{B}} \quad \chemfig{\charge{[circle]0=\.,90=\.,180=\.}{Al}} \quad \chemfig{\charge{[circle]0=\.,90=\.,180=\.}{Ga}} \quad \chemfig{\charge{[circle]0=\.,90=\.,180=\.}{In}} \quad \chemfig{\charge{[circle]0=\.,90=\.,180=\.}{Tl}}$ & 7° gruppo & $\chemfig{\charge{[circle]0=\:,90=\:,180=\.,270=\:}{F}} \quad \chemfig{\charge{[circle]0=\:,90=\:,180=\.,270=\:}{Cl}}$\\
    4° gruppo & $\chemfig{\charge{[circle]0=\.,90=\.,180=\.,270=\.}{C}} \quad \chemfig{\charge{[circle]0=\.,90=\.,180=\.,270=\.}{Si}}$ & 8° gruppo & $\chemfig{\charge{[circle]0=\:,90=\:,180=\:,270=\:}{Ne}} \quad \chemfig{\charge{[circle]0=\:,90=\:,180=\:,270=\:}{Ar}}$\\
\end{tabular}
\setlength\extrarowheight{0cm}\\

Facciamo attenzione alla disposizione dei puntini successivamente al carbonio: il carbonio ha 4 elettroni esterni, quindi abbiamo 4 punti a 90° l'uno dall'altro. Nell'istante in cui il numero di elettroni è superiore a 4 gli elettroni iniziano ad accoppiarsi, quindi l'azoto avrà una coppia di elettroni e tre elettroni spaiati, l'ossigeno due coppie di elettroni e due elettroni spaiati, il fluoro tre coppie di elettroni e un elettrone spaiato, il neon quattro coppie di elettroni.

Stiamo quindi parlando soltanto degli elettroni di valenza, non di quelli interni.

Quindi, dopo aver messo un puntino su ogni "lato" a 90° l'uno dall'altro (e ci arriviamo col carbonio) iniziamo ad appaiare questi elettroni in modo tale da formare delle coppie che arrivano ad essere 4 col neon. Va poi da ricordare che il fondamento di questa teoria sta nel fatto che per gli elementi che stiamo considerando il numero di elettroni esterni rappresenta la configurazione elettronica esterna del gas nobile che segue questi elementi, ossia questi elementi tenderanno in qualche modo ad avere una configurazione elettronica esterna tipo gas nobile. In altre parole, gli elementi perdono o acquistano elettroni in modo tale da raggiungere la configurazione esterna del gas nobile. Tale comportamento prende il nome di \textbf{regola dell'ottetto}:

"\textit{Gli atomi, nel legarsi tra loro, perdono, acquistano o cedono elettroni per raggiungere uno strato esterno di 8 elettroni. Gli atomi tendono a fare ciò per assumere la configurazione del gas nobile più vicino}". 

Tale regola vale per elementi del secondo e del terzo periodo e per una gran parte degli altri in composti semplici.

\textbf{ES 1}

Consideriamo il fluoruro di litio LiF. Il litio ha un solo elettrone esterno di valenza, l'elettrone 2s. Il fluoro appartiene al settimo gruppo e ha quindi 7 elettroni esterni, che sono 2 elettroni 2s e 5 elettroni 2p ($2s^22p^5$), quindi per entrambi gli elementi gli elettroni 1s non sono di valenza.

Immaginiamo che la molecola LiF sia rappresentabile come un sistema puramente ionico nel quale il litio abbia ceduto il suo unico elettronce esterno al fluoro:
\\


$$\schemestart
\chemfig{Li[He]@{x1}{2s^1}}
\+
\chemfig{F[He]2s^2@{x2}{2p^5}}
\arrow
\chemfig{Li^+1s^2 \text{(=[He])}}
\+
\chemfig{F^{-}[He]2s^22p^6 \text{(=[Ne])}}
\schemestop
\chemmove[shorten <=2pt]{
\draw(x1)..controls +(60:1cm)and+(120:1cm)..(x2);}$$

$$\ce{Li \subshells{{1s:2}{2s:1}{2p:000}} + F \subshells{{1s:2}{2s:2}{2p:221}} -> Li^+ \subshells{{1s:2}{2s:0}{2p:000}} + F- \subshells{{1s:2}{2s:2}{2p:222}}}$$

$$\schemestart
\chemfig{@{x1}{\charge{[circle]180=\.}{Li}}}
\+
\chemfig{@{x2}{\charge{[circle]0=\:,90=\:,180=\.,270=\:}{F}}}
\arrow
\chemfig{Li^+}
\+
\chemfig{\charge{[circle]0=\:, 45=$\scriptstyle-$, 90=\:,180=\:,270=\:}{F}}
\schemestop
\chemmove[shorten >=4pt]{
\draw(x1.155)..controls +(60:0.6cm)and+(120:0.6cm)..(x2.190);}$$

Dopo la freccia il litio non ha più l'elettrone 2s (quindi gli orbitali 2s e 2p sono tutti vuoti). Il fluoro invece ha acquistato questo elettrone ceduto dal litio e adesso ha i livelli 2p pieni, quindi su di esso abbiamo 8 elettroni di valenza, cioè esterni: 2 nel livello 2s e 6 nei tre livelli 2p. In questo modo raggiunge quindi l'ottetto. Il litio, avendo perso l'unico suo elettrone esterno 2s, ha assunto la configurazione elettronica esterna del gas nobile che lo precede, cioè l'elio.

Quindi nel sistema LiF sia il litio, rappresentato in forma cationica, che il litio, rappresentato in forma anionica, si trovano ad avere la configurazione elettronica del gas nobile che lo precede (litio) o che che lo segue (fluoro). Utilizzando il formalismo di Lewis si vede come il litio abbia formalmente ceduto un elettrone al fluoro e su quest'ultimo si abbiano 4 coppie, ossia 8 elettroni esterni.

Questo esempio ci indica il modo di ragionare per ottenere le strutture di Lewis.

Attenzione! Il formalismo di Lewis non ci permette di ragionare sulle geometrie delle molecole: non ci sono dati che ci permettono di dire nulla sulle geometrie, anche se scriveremo delle molecole tutte su un piano, ma questo approccio non rappresenta una disposizione spaziale della molecola che andremo a scrivere/disegnare.

\textbf{ES 2} \ce{MgCl_2}

Il magnesio è un elemento del secondo gruppo, ed ha 2 elettroni di tipo s. I suoi orbitali di tipo p sono totalmente vuoti. Il cloro invece è un alogeno, cioè appartiene al 7° gruppo ed ha quindi sette elettroni esterni: due nel livello 2s e cinque nel livello 2p.

Immaginiamo che il sistema sia puramente ionico. Sotto questa ipotesi ciò che avviene è che i due elettroni del magnesio vengono ceduti a due diversi atomi di cloro, i quali hanno ciascuno sette elettroni esterni, e in questo modo raggiungono l'ottetto:
$$\ce{Mg \subshells{{1s:2}{2s:2}{2p:000}} + 2Cl \subshells{{1s:2}{2s:2}{2p:221}} -> Mg^{2+} \subshells{{1s:2}{2s:0}} + 2Cl- \subshells{{1s:2}{2s:2}{2p:222}}}$$

$$\schemestart
\chemfig{@{x1}{\charge{[circle]180=\:}{Mg \; +}}(-[1,,,,draw=none]@{x2}{\charge{[circle]0=\:,90=\:,180=\.,270=\:}{Cl}})(-[7,,,,draw=none]@{x3}{\charge{[circle]0=\:,90=\:,180=\.,270=\:}{Cl}})}
\arrow
\chemfig{Mg^{2+}}
\+
2 \chemleft(\chemfig{\charge{[circle]0=\:, 90=\:,180=\:,270=\:}{Cl}}\chemright{)^{-}}
\schemestop
\chemmove[shorten <=6pt, shorten >=4pt]{
\draw(x1.180)..controls +(110:0.8cm)and+(0:0cm)..(x2.200);
\draw(x1.180)..controls +(70:-0.8cm)and+(0:0cm)..(x3.150);
}$$

Ciò significa che stiamo immaginando che il magnesio perda totalmente due elettroni di tipo s, per cui avrà una carica due volte positiva, e il cloro acquisti un elettrone donato dal magnesio per completare il riempimento dei suoi livelli p. Ovviamente, se il magnesio cede due elettroni e il cloro può acquistarne solo uno, serviranno due atomi di cloro, i quali diventeranno due anioni cloruro.
\subsection{Eccezioni alla regola dell'ottetto}
La regola dell'ottetto parte dall'idea che riempire totalmente i livelli di valenza conferisce elevata stabilità al sistema. Quindi, ragionando in termini di blocco s e p, questo significa avere otto elettroni esterni al massimo, perché nei livelli s e p in totale riusciamo a mettere 8 elettroni. Tuttavia per alcuni composti essa non vale più:
\begin{itemize}
    \item Per gli elementi di transizione e i loro sistemi molecolari si parla di "\textbf{regola dei 18 elettroni}". Tali elementi presentano orbitali 4s, 3d e 4p. Ciò significa che possono ospitare 2 elettroni sui 4s, 10 elettroni sui 3d e 6 elettroni sui 4p, per un totale di 18 elettroni.
    \item Ci sono molecole che non hanno la struttura tale da raggiungere la somma degli elettroni pari a 8. Ad esempio la molecola \ce{H_2}, due atomi di idrogeno che danno luogo alla sua forma molecolare, avrà un totale di due elettroni, in quanto l'atomo di idrogeno ne possiede solo uno. Altri esempi sono gli idruri: \ce{LiH, \; BeH_2}.
    \item Ci sono molecole che presentano un numero di elettroni superiore a 8. In questi casi si parla di "\textbf{valenza espansa}".
    \item Ci sono infine vari casi in cui il numero di elettroni è dispari, per cui ovviamente non può essere 8.
\end{itemize}
\newpage
\subsection{Criteri per scrivere le strutture di Lewis}
\begin{itemize}
    \item L'atomo di idrogeno è sempre un atomo terminale, quindi in molecole con più atomi sappiamo per certo che l'atomo di idrogeno non andrà al centro;
    \item L'atomo che presenta la più bassa elettronegatività (esclusi H e ioni metallici) è quello centrale e gli altri vanno disposti attorno ad esso. Se quindi ad esempio in una molecola avessi carbonio e ossigeno al centro ci starebbe il carbonio, perché l'ossigeno è più elettronegativo del carbonio e quindi non si mette al centro. Se avessi zolfo e ossigeno, metterei lo zolfo al centro e l'ossigeno ai lati;
    \item Si dispongono gli atomi restanti attorno (tenendo presente che non c'è una regola su come metterli attorno, perché non sto ragionando sulla reale geometria, che questa teoria non mi permette di vedere)
    \item Si considera il numero totale di elettroni di valenza, tenendo conto di eventuali cariche formali (ioni). Cioè: contiamo tutti gli elettroni di valenza di tutti gli atomi e li sommiamo, in modo tale da sapere quanti siano gli elettroni totali di valenza della molecola. Se poi il sistema molecolare fosse carico ne dovremmo tenere conto, ossia se ha una carica n volte negativa dovremo aggiungere n elettroni al conteggio, se avesse carica m volte positiva dovremo sottrarre m elettroni;
    \item Si uniscono gli atomi attorno a quello centrale con un legame singolo (detto legame $\sigma$), formato da una coppia di elettroni, che immagino essere la rappresentazione di una coppia di elettroni di legame. 
    
    \item Si sottrae dal numero di elettroni totali quelli usati per i legami semplici $\sigma$;
    \item Le coppie di elettroni restanti vanno assegnate agli atomi esterni, tentando di far raggiungere ad ognuno di questi l'ottetto elettronico. Se continuano ad avanzarne anche dopo che tutti gli elettroni esterni hanno raggiunto l'ottetto, li assegno all'atomo centrale.
    \item Se dopo quest'ultima operazione l'atomo centrale non risulta ancora circondato da quattro coppie elettroniche, si trasformano alcuni doppietti degli atomi periferici, cioè alcune coppie di elettroni degli atomi esterni, in doppi legami, con l'obiettivo di far raggiungere anche all'atomo centrale l'ottetto elettronico. Secondo questa teoria, se i  legami iniziali erano etichettati legami semplici di tipo $\sigma$, i successivi (cioè laddove avremo per esempio un doppio legame) verranno etichettati legami $\pi$ (presenti in genere con atomi quali C, N, O ed S).
\end{itemize}

Per quanto riguarda gli orbitali: orbitale di natura $\sigma$ significa che c'è densità di carica lungo l'asse di legame, cioè lungo la congiungente dei due nuclei; orbitale di natura $\pi$ significa che lungo l'asse di legame ci sarà un piano nodale.

Attenzione! La regola di Lewis per cui il secondo legame che si forma è di tipo $\pi$ non è sempre vera, infatti ci sono alcune molecole in cui sono presenti due legami, ma non è vero che sono uno $\sigma$ e uno $\pi$. Una di queste è la molecola \ce{C_2}, in cui i due atomi di carbonio sono legati da un doppio legame ed entrambi i legami hanno natura $\pi$.

Come abbiamo visto si lavora solo con gli elettroni di valenza. Va da ricordare:
\begin{itemize}
    \item Gli elementi del 1° gruppo hanno un solo elettrone di valenza;
    \item Quelli del 2° gruppo ne hanno due;
    \item Quelli del 3° gruppo ne hanno tre;
    \item Quelli del 4° gruppo ne hanno quattro;
    \item Così via
\end{itemize}

Se quindi volessi sapere ad esempio quanti elettroni di valenza ha il silicio, come scrivo la configurazione elettronica?

Il silicio ha pieni gli orbitali 1s, 2s e 2p. Ha poi come elettroni di valenza i 3s e i 3p. In particolare il silicio sta sotto l'ossigeno\footnote{La sequenza è: ossigeno, silicio, germanio, stagno e piombo}, quindi esso avrà due elettroni nei livelli 3s e due elettroni sui 3p (come l'ossigeno, che ha configurazione $2s^22p^2$). Inoltre la configurazione $1s^22s^22p^63s^23p^2$ può essere scritta come configurazione del neon (che è il gas nobile che precede il silicio) più i suoi elettroni di valenza: $\text{[Ne]}3s^23p^2$.
\subsection{Esempi}
\begin{itemize}
    \item \textbf{ES1} \ce{N_2} \quad n.e.=10 (numero di elettroni di valenza)
    
    L'azoto appartiene al quinto gruppo (azoto, fosforo, arsenico, antimonio e bismuto), quindi i suoi elettroni di valenza sono 5: due di tipo s e tre di tipo p. Avendone due atomi, in totale avremo 10 elettroni.

    Avendo solo due atomi dello stesso tipo, non si pone il problema di chi porre al centro, si possono considerare come due punti uniti da una linea, quindi li mettiamo accanto e li uniamo con un segmento che rappresenta un legame, il quale è costituito da una coppia di elettroni:
    $$
    \chemfig{N-N}
    $$
    Ciò significa che adesso restano 8 elettroni, che dovrò distribuire a coppie sui due atomi. Riusciamo a mettere due coppie su un azoto e due coppie sull'altro.

    A questo punto contiamo gli elettroni su ogni atomo. Da notare che i due elettroni impiegati nel legame contano come 2 elettroni per ciascun atomo, per cui per entrambi avremo 4 elettroni dati da due coppie più due elettroni di legame, per un totale di 6. Entrambi gli atomi di azoto quindi non raggiungono l'ottetto. Siccome la formula di Lewis prevede che ci siano 8 elettroni attorno a tutti gli atomi andiamo avanti: per ognuno dei due atomi, trasformiamo un doppietto in un doppio legame:
    
    $$
    \chemfig{@{x1}{\charge{[circle]90=\:,270=\:}{N}}-[@{x2}]@{x3}{\charge{[circle]90=\:,270=\:}{N}}}
    \chemmove[shorten <=4pt, shorten >=4pt]{
    \draw(x3.290)..controls +(60:-0.5cm)and+(90:-0.5cm)..(x2.90);
    \draw(x1.110)..controls +(60:0.5cm)and+(90:0.5cm)..(x2);
    }$$

    Se trasformassi un solo doppietto in doppio legame (ad esempio uno dell'azoto a sinistra), sull'atomo a sinistra avrei 2 elettroni dati dai doppietti più 4 elettroni di legame, quindi il totale è ancora 6 e l'ottetto non è stato raggiunto; sull'atomo a destra invece avremmo 4 elettroni dai doppietti più 4 di legame, dunque totale è 8: l'ottetto è stato raggiunto.
    
    Per far raggiungere l'ottetto anche all'azoto a sinistra, trasformo un doppietto dell'atomo a destra in un legame. Così facendo avremo un triplo legame tra i due atomi di azoto e un solo doppietto su ciascuno di esso:
    $$
    \chemfig{\charge{[circle]180=\:}{N}~\charge{[circle]0=\:}{N}}
    $$

    A questo punto entrambi gli atomi hanno raggiunto l'ottetto, perché ambedue avranno 6 elettroni dai legami e 2 dai doppietti, per un totale di 8.
    \item \textbf{ES2} \ce{O_2} n.e.=12
    
    L'ossigeno è al sesto gruppo quindi ha 6 elettroni esterni, avendone due atomi in totale ci sono 12 elettroni. Mettiamo accanto i due atomi e li uniamo con un legame che consuma 2 elettroni:
    $$
    \chemfig{O-O}
    $$
    I 10 elettroni restanti vanno distribuiti a coppie tra i due atomi. Mettiamo una coppia su un ossigeno e una sull'altro. Restano 6 elettroni. Se diamo ancora una coppia ciascuno resta una coppia, che assegniamo in modo arbitrario ad uno dei due, ad esempio a quello sinistro. In questo modo un atomo di ossigeno avrà 8 elettroni (3 coppie più 2 di legame) e raggiunge l'ottetto, l'altro ha 6 elettroni (2 coppie più 2 di legame) e non lo raggiunge.
    
    $$
    \chemfig{@{x1}{\charge{[circle]90=\:,180=\:,270=\:}{O}}-[@{x2}]\charge{[circle]90=\:,270=\:}{O}}
    \chemmove[shorten <=4pt, shorten >=4pt]{
    \draw(x1.110)..controls +(60:0.5cm)and+(90:0.5cm)..(x2);}$$
    Trasformiamo allora uno dei doppietti dell'ossigeno che ne ha 3 in doppio legame:
    $$
    \chemfig{\charge{[circle]90=\:,270=\:}{O} = \charge{[circle]90=\:,270=\:}{O}}
    $$
    Contando adesso gli elettroni, sia a sinistra che a destra avremo attorno 8 elettroni (2 coppie più 4 di legame)\\

    Va da notare che non potevamo prevedere che un sistema a 10 elettroni quale \ce{N_2}, propone un triplo legame, mentre un sistema a 12 elettroni quale \ce{O_2}  è formato da un doppio legame. Questa regola quindi ci permette di scrivere le formule di struttura delle molecole
    \item ES3 CO n.e.=10
    
    Il monossido di carbonio è formato da un atomo di carbonio, che è del quarto gruppo e ha 4 elettroni di valenza, e da un atomo di ossigeno, che è del sesto gruppo e ha 6 elettroni di valenza, per un totale di 10 come nel caso di \ce{N_2}, quindi ci troveremo in una situazione analoga.

    Per prima cosa scriviamo il carbonio e l'ossigeno e li leghiamo con un legame semplice
    $$
    \chemfig{C-O}
    $$
    Abbiamo consumato due elettroni, quindi ne restano 8.
    Essendoci atomi diversi, si inizia ad assegnare doppietti all'atomo più elettronegativo fin quando non raggiunge l'ottetto. In questo caso l'ossigeno è più elettronegativo, quindi ad esso assegno 3 coppie e solo dopo la coppia che resta la metto sul carbonio
    
    $$
    \chemfig{\charge{[circle]180=\:}{C}-[@{x1}]@{x2}{\charge{[circle]0=\:,90=\:,270=\:}{O}}}
    \chemmove[shorten <=4pt, shorten >=4pt]{
    \draw(x2.290)..controls +(60:-0.5cm)and+(90:-0.5cm)..(x1.90);
    \draw(x2.70)..controls +(120:0.5cm)and+(90:0.5cm)..(x1.90);
    }
    $$

    L'ossigeno qui ha già raggiunto l'ottetto (3 coppie +un legame semplice), mentre il carbonio ha solo 4 elettroni (un doppietto e un legame semplice).

    Allora trasformiamo prima un doppietto dell'ossigeno in legame, ma così facendo il carbonio non raggiunge ancora l'ottetto, dunque trasformiamo anche un secondo doppietto in legame. In questo modo otterremo un triplo legame e ci sarà un doppietto sul carbonio e uno sull'ossigeno:
    $$
    \chemfig{\charge{[circle]180=\:}{C}~\charge{[circle]0=\:}{O}}
    $$
    L'ottetto è così raggiunto per entrambi gli atomi (un doppietto e 6 elettroni di legame)
    \item ES4 NO n.e.=11
    
    L'azoto ha 5 elettroni di valenza, mentre l'ossigeno ne ha 6. In totale abbiamo 11 elettroni, ed è il primo esempio con numero di elettroni dispari.
    Mettiamo i due atomi accanto e li leghiamo con un legame semplice:
    $$
    \chemfig{N-O}
    $$
    Restano 9 elettroni. Essendo dispari non possiamo ottenere solo doppietti: avremo 4 doppietti e un elettrone.

    Iniziamo ad assegnare a coppie questi elettroni all'atomo più elettronegativo fino a fargli raggiungere l'ottetto. In questo caso è l'ossigeno, a cui diamo 3 doppietti. Restano 3 elettroni, che diventano un doppietto e un elettrone dispari posti sull'azoto.

    Mentre l'ossigeno ha raggiunto l'ottetto, l'azoto no ed è anche lontano dal raggiungerlo, perché intorno ha 5 elettroni (due dal legame, una coppia e un elettrone spaiato). Trasformiamo allora un doppietto dell'ossigeno in doppio legame:
    
    $$
    \chemfig{\charge{[circle]90=\.,180=\:}{N}-[@{x1}]@{x2}{\charge{[circle]0=\:,90=\:,270=\:}{O}}}
    \chemmove[shorten <=4pt, shorten >=4pt]{
    \draw(x2.290)..controls +(60:-0.5cm)and+(90:-0.5cm)..(x1.90);}
    $$

    L'ossigeno continua ad aver raggiunto l'ottetto (4 dai legami più 4 dai doppietti), l'azoto avrà 7 elettroni (4 dai legami più 2 dai doppietti più 1 spaiato). Dobbiamo fermarci, non abbiamo cos'altro fare: l'ottetto, per l'azoto, non può essere raggiunto. Se trasformassimo un ulteriore doppietto dell'ossigeno in legame, sull'azoto ci sarebbero 9 elettroni, cosa impossibile perché al massimo attorno ad un atomo possono stare 8 elettroni.

    Secondo la descrizione di Lewis avremo un elettrone spaiato su questa molecola, cosa che comporta l'avere proprietà paramagnetiche. Infatti i fenomeni magnetici sono dovuti all'esistenza di elettroni spaiati sulla molecola 
    $$
    \chemfig{\charge{[circle]90=\.,180=\:}{N}=\charge{[circle]0=\:,90=\:}{O}}
    $$
    \item ES5 NO$^+$
    
    Consideriamo il catione della molecola precedente. Abbiamo strappato un elettrone, per cui nel conteggio finale dovremo togliere un elettrone. L'azoto dà 5 elettroni, l'ossigeno 6, meno uno in totale 10.

    Uniamo i due atomi con un legame:
    $$
    \chemleft[\chemfig{N-O}\chemright{]^{+}}
    $$
    Consumiamo due elettroni, ne restano 8. Inizio a completare l'ottetto dell'elemento più elettronegativo che è l'ossigeno: ad esso diamo 3 doppietti. Resta un doppietto che diamo all'azoto.

    L'ossigeno ha raggiunto l'ottetto, l'azoto ha solo 4 elettroni (un doppietto e due dal legame). Trasformiamo allora due diversi doppietti dell'ossigeno in ulteriori legami:
    $$
    \chemleft[\chemfig{\charge{[circle]90=\:}{N}-[@{x1}]@{x2}{\charge{[circle]0=\:,90=\:,270=\:}{O}}} \,\chemright{]^{+}}
    \chemmove[shorten <=4pt, shorten >=4pt]{
    \draw(x2.290)..controls +(60:-0.5cm)and+(90:-0.5cm)..(x1.90);
    \draw(x2.70)..controls +(120:0.5cm)and+(90:0.5cm)..(x1.90);
    }$$
    In questo modo si ottiene un triplo legame che contiamo come 6 elettroni, in più su ciascun atomo c'è un doppietto per cui entrambi hanno 8 elettroni intorno.
    Quindi quello che succede togliendo un elettrone a questa molecola è che il numero di legami aumenta: da 2 nella molecola \ce{NO} a 3 nella molecola \ce{NO^+}
    $$
    \chemleft[\chemfig{\charge{[circle]90=\:}{N}~\charge{[circle]90=\:}{O}}\chemright{]^{+}}
    $$
    \item ES6 HCl n.e.=8
    
    L'acido cloridrico è formato da idrogeno, che ha un solo elettrone e cloro, che ne ha 7, per un totale di 8.

    Leghiamo i due atomi usando due elettroni per formare un legame semplice:
    $$
    \chemfig{H-Cl}
    $$
    Restano 6 elettroni da assegnare all'atomo più elettronegativo, ossia al cloro:
    $$
    \chemfig{H-\charge{[circle]0=\:,90=\:,270=\:}{Cl}}
    $$
    Il cloro ha raggiunto l'ottetto. Ovviamente non ci poniamo il problema per l'idrogeno perché esso ha solo un orbitale 1s e al massimo può avere attorno a sé 2 elettroni, che sono quelli del legame.
    \item ES7 \ce{CO_2}
    
    Nell'anidride carbonica l'elemento meno elettronegativo è il carbonio, che quindi va al centro, mentre ai lati opposti vanno i due atomi di ossigeno:
    $$
    \chemfig{O-C-O}
    $$
    I due atomi di ossigeno danno 12 elettroni, il carbonio ne dà 4. In totale sono 16, ma ne abbiamo usati 4 per formare i legami e quindi ne restano 12, che distribuiamo agli atomi esterni più elettronegativi. Essendoci 2 ossigeni, assegnamo a ciascuno una coppia alla volta:
    $$
    \chemfig{@{x1}{\charge{[circle]90=\:,180=\:,270=\:}{O}}-[@{x2}]C-[@{x3}]@{x4}{\charge{[circle]0=\:,90=\:,270=\:}{O}}}
    \chemmove[shorten <=4pt, shorten >=4pt]{
    \draw(x1.110)..controls +(60:0.5cm)and+(90:0.5cm)..(x2);
    \draw(x4.290)..controls +(60:-0.5cm)and+(90:-0.5cm)..(x3.90);
    }$$
    Riusciamo a distribuire tutti gli elettroni mettendo 3 coppie su ciascun ossigeno, senza che ne rimangano.
    Questi avranno così raggiunto l'ottetto (6 elettroni dalle coppie e due dal legame), a differenza del carbonio che attorno a sé solo 4 elettroni dovuti ai due legami semplici. Pertanto trasformiamo un doppietto dell'ossigeno a sinistra in doppio legame e altrettanto facciamo con l'ossigeno a destra:
    $$
    \chemfig{\charge{[circle]180=\:,270=\:}{O}=C=\charge{[circle]0=\:,270=\:}{O}}
    $$
    In questo modo sia gli ossigeni (con due doppietti e 4 elettroni di legame) che il carbonio (con 8 elettroni di legame) hanno completato l'ottetto.

    Il formalismo di Lewis quindi ci dice che l'anidride carbonica va scritta con doppi legami carbonio-ossigeno. Ribadiamp poi che tale formalismo non entra nel merito della vera geometria della molecola, cionostante l'abitudine è quella di scrivere gli atomi che stanno attorno a quello centrale in modo tale che siano più distanti possibile.
    \item ES8 \ce{NO_2} n.e.=17
    
    Nel diossido di azoto i due ossigeni danno 12 elettroni, l'azoto 5, per un totale di 17 elettroni.
    L'azoto è l'atomo più elettronegativo tra i due, quindi lo mettiamo al centro e ai suoi lati poniamo gli ossigeni, che leghiamo all'azoto con due legami semplici:
    $$
    \chemfig{O-N-O}
    $$
    Abbiamo consumato 4 elettroni, ne restano 13 che distribuiamo attorno agli atomi esterni, perché più elettronegativi. Assegnando una coppia per volta alternando gli ossigeni fino a che questi raggiungano l'ottetto, consumiamo 12 elettroni (3 coppie per atomo). L'elettrone restante viene assegnato all'azoto centrale:
    $$
    \chemfig{@{x1}{\charge{[circle]90=\:,180=\:,270=\:}{O}}-[@{x2}]\charge{[circle]90=\.}{N}-\charge{[circle]0=\:,90=\:,270=\:}{O}}
    \chemmove[shorten <=4pt, shorten >=4pt]{
    \draw(x1.110)..controls +(60:0.5cm)and+(90:0.5cm)..(x2);
    }$$
    In questo modo l'azoto ha 5 elettroni. Avendo attorno a sé un numero dispari di elettroni possiamo già prevedere che non raggiungerà l'ottetto. L'unica cosa che si può fare è prendere un doppietto qualunque di uno dei due ossigeni e trasformalo in doppio legame:
    $$
    \chemfig{\charge{[circle]180=\:,270=\:}{O}=\charge{[circle]90=\.}{N}-\charge{[circle]0=\:,90=\:,270=\:}{O}}
    $$
    In questo modo l'ossigeno di sinistra ha raggiunto l'ottetto (4 dai doppietti più 4 dai legami), quello di destra pure (6 dai doppietti e 2 dal legame), ma l'azoto ha solo 7 elettroni (4 dal doppio legame, 2 dal legame singolo e 1 spaiato). Non possiamo fare altro.

    Tuttavia questo formalismo ci suggerisce che questa molecola, avendo un elettrone spaiato, mostrerà proprietà paramagnetiche.
    \newpage
    \item ES9 \ce{O_3}\footnote{L'ozono è un allotropo dell'ossigeno. Se ne sente l'odore quando durante un temporale  vi sono forti fulimini e l'ossigeno dell'aria viene in parte trasformato in ozono. L'odore è simile a quello della pioggia. Inoltre esso è molto ossidante, tant'è che in alcuni posti viene utilizzato per rendere potabili le acque} n.e.=18
    
    Essendo composto da 3 atomi di ossigeno, ne mettiamo uno qualunque al centro e gli altri due ai lati e leghiamo i tre atomi con due legami semplici
    $$
    \chemfig{O-O-O}
    $$
    In totale abbiamo 18 elettroni, meno 4 usati per i legami ne restano 14, che dobbiamo distribuire col criterio finora usato. Infatti, sebbene abbiano la stessa elettronegatività perché sono tutti identici, gli elettroni vanno comunque assegnati prima agli atomi periferici e se restano a quello centrale.
    
    14 elettroni significa 7 coppie. Assegnamo 3 coppie all'atomo a sinistra e 3 coppie all'atomo di destra così che raggiungano l'ottetto, e la coppia che resta la assegnamo all'atomo centrale:
    $$
    \chemfig{@{x1}{\charge{[circle]90=\:,180=\:,270=\:}{O}}-[@{x2}]\charge{[circle]90=\:}{O}-\charge{[circle]0=\:,90=\:,270=\:}{O}}
    \chemmove[shorten <=4pt, shorten >=4pt]{
    \draw(x1.110)..controls +(60:0.5cm)and+(90:0.5cm)..(x2);
    }$$
    A differenza di quelli periferici, l'ossigeno centrale non ha raggiunto l'ottetto. Prendiamo allora uno qualunque dei 6 doppietti degli atomi periferici e lo trasformiamo in un ulteriore legame:
    $$
    \chemfig{\charge{[circle]180=\:,270=\:}{O}=\charge{[circle]90=\:}{O}-\charge{[circle]0=\:,90=\:,270=\:}{O}}
    $$
    In questo modo l'atomo di sinistra (con 4 elettroni dai doppietti e 4 dai legami), che quello centrale (con 2 elettroni dai doppietti e 6 dai legami), che quello di destra (con 6 elettroni dai doppietti e 2 dai legami) hanno raggiunto l'ottetto  
    \item ES10 \ce{NO_2^-} n.e.=18
    
    Nell'anione nitrato i due ossigeni forniscono 12 elettroni, l'azoto 5 per un totale di 17, più una carica negativa si arriva a 18. Mettiamo al centro l'azoto perché meno elettronegativo, ai lati i due ossigeni e li leghiamo all'azoto con due legami seplici:
    $$
    \chemleft[\chemfig{O-N-O}\chemright{]^{-}}
    $$
    Consumiamo 4 elettroni coi legami, ne restano 14, ovvero 7 coppie. Assegnamo 6 di queste ai due ossigeni laterali che così raggiungono l'ottetto, e quella restante all'azoto:
    $$
    \chemleft[ \chemfig{@{x1}{\charge{[circle]90=\:,180=\:,270=\:}{O}}-[@{x2}]\charge{[circle]90=\:}{N}-\charge{[circle]0=\:,90=\:,270=\:}{O}} \chemright{]^{-}}
    \chemmove[shorten <=4pt, shorten >=4pt]{
    \draw(x1.110)..controls +(60:0.5cm)and+(90:0.5cm)..(x2);
    }$$
    L'azoto avrà solo 6 elettroni introno, quindi prendiamo un doppietto qualunque da uno dei due atomi e lo trasformiamo in un ulteriore legame: 
    $$
    \chemleft[ \chemfig{\charge{[circle]180=\:,270=\:}{O}=\charge{[circle]90=\:}{N}-\charge{[circle]0=\:,90=\:,270=\:}{O}} \chemright{]^{-}}
    $$
    Si ottiene una formula in cui l'ossigeno a sinistra (con 4 elettroni dai doppietti e 4 dai legami), quello a destra (con 6 elettroni dai doppietti e 2 dal legame) e l'azoto (con 2 elettroni dal doppietto e 6 dai legami) hanno raggiunto l'ottetto.

    Non avendo elettroni spaiati, questo anione non presenta proprietà paramagnetiche ma solo diamagnetiche, che sono una proprietà intrinseca di tutti i sistemi (anche di quelli che hanno elettroni spaiati, in quel caso si sommano proprietà paramagnetiche e diamagnetiche), dovuta a livelli elettronici pieni, e gli elementi, anche quando hanno elettroni spaiati, hanno i livelli interni pieni. Fa eccezione solo l'idrogeno, perché non ha livelli interni.
    \item ES11 HCN n.e.=10
    
    Per disegnare l'acido cianidrico ricordiamo che l'idrogeno è sempre un atomo terminale, per cui per quale atomo mettere al centro confrontiamo solo l'elettronegatività di carbonio e azoto. Il carbonio è meno elettronegativo, quindi va al centro:
    $$
    \chemfig{H-C-N}
    $$
    L'idrogeno contribuisce con un elettrone, il carbonio con 4 e l'azoto con 5 per un totale di 10. Legando idrogeno e azoto consumiamo 4 elettroni, quindi ne restano 6 ossia 3 doppietti che assegnamo all'atomo esterno più elettronegativo, cioè l'azoto
    
    $$
    \chemfig{H-C-[@{x1}]@{x2}{\charge{[circle]0=\:,90=\:,270=\:}{N}}}
    \chemmove[shorten <=4pt, shorten >=4pt]{
    \draw(x2.290)..controls +(60:-0.5cm)and+(90:-0.5cm)..(x1.90);
    \draw(x2.70)..controls +(120:0.5cm)and+(90:0.5cm)..(x1.90);}
    $$

    L'azoto raggiunge l'ottetto (6 dai doppietti e 2 dal legame), per l'idrogeno non si pone il problema perché può ospitare solo due elettroni.
    Il carbonio invece ha solo 4 elettroni dai legami, quindi trasformiamo due doppietti dell'azoto in ulteriori legami (se ne trasformassimo solo uno attorno al carbonio ci sarebbero 6 elettroni):
    $$
    \chemfig{H-C~\charge{[circle]90=\:}{N}}
    $$
    Carbonio e azoto saranno quindi legati da un triplo legame. In questo modo sia carbonio (con 8 elettroni di legame) e l'azoto (con 6 elettroni di legame e 2 dal doppietto) raggiungono l'ottetto
    \item ES12 \ce{CH_2O} n.e.=12
    
    La formaldeide va scritta con carbonio al centro (è il meno elettronegativo) e i due idrogeni e l'ossigeno intorno legati ad esso con un legame semplice:
    $$
    \chemfig{H-C(-[6]O)(-H)}
    $$
    N.B.: sebbene in questa formula i legati siano posti a 90° l'uno dall'altro, ciò non è un'indicazione sulla vera geometria della molecola.
    Due atomi di idrogeno danno 2 elettroni, l'ossigeno 6 e il carbonio 4, per un totale di 12. Consumiamo però 6 elettroni nei legami, quindi ne restano 6, che assegnamo all'ossigeno in quanto più elettronegativo e che raggiunge così l'ottetto (6 elettroni dai doppietti più 2 dal legame). Il carbonio centrale invece non lo ha raggiunto, quindi trasformiamo un doppietto dell'ossigeno in doppio legame:
    $$
    \chemfig{H-C([6]-[@{x1}]@{x2}{\charge{[circle]0=\:,180=\:,270=\:}{O}})(-H)}
    \chemmove[shorten <=4pt, shorten >=4pt]{
    \draw(x2.180)..controls +(180:0.5cm)and+(180:0.8cm)..(x1.60);}
    $$
    Si forma quindi un doppio legame tra carbonio e ossigeno. In questo modo il carbonio (con 8 elettroni di legame) e l'ossigeno (con 4 elettroni dai doppietti e 4 dai legami) raggiungono l'ottetto.
    $$
    \chemfig{H-C(=[6]\charge{[circle]0=\:,270=\:}{O})(-H)}
    $$
    \item ES13 \ce{CO_3^{2-}} n.e.=24
    
    Nell'anione carbonato il carbonio fornisce 4 elettroni, i tre ossigeni in totale 18, più le due cariche in totale abbiamo 24 elettroni. Al centro mettiamo il carbonio che è meno elettronegativo e mettiamo i tre ossigeni attorno, legandoli ad esso con legami semplici:
    $$
    \chemleft[\chemfig{O-C(-[6]O)(-O)}\chemright{]^{2-}}
    $$
    Con i legami consumiamo 6 elettroni, ne restano 18 cioè 9 coppie, che distribuiamo ai 3 ossigeni, esaurendoli tutti:
    $$
    \chemleft[ \chemfig{\charge{[circle]90=\:,180=\:,270=\:}{O}-C([6]-[@{x1}]@{x2}{\charge{[circle]0=\:,180=\:,270=\:}{O}})(-\charge{[circle]0=\:,90=\:,270=\:}{O})} \chemright{]^{2-}}
    \chemmove[shorten <=4pt, shorten >=4pt]{
    \draw(x2.180)..controls +(180:0.5cm)and+(180:0.8cm)..(x1.60);}
    $$
    Così facendo i 3 ossigeni hanno raggiunto l'ottetto, mentre il carbonio ha 6 elettroni dati da tre legami. Scegliamo allora un qualunque doppietto da uno qualsiasi dei tre ossigeni e lo trasformiamo in doppio legame
    $$
    \chemleft[ \chemfig{\charge{[circle]90=\:,180=\:,270=\:}{O}-C(=[6]\charge{[circle]0=\:,270=\:}{O})(-\charge{[circle]0=\:,90=\:,270=\:}{O})} \chemright{]^{2-}}
    $$
    In questo modo il carbonio raggiunge l'ottetto (8 elettroni di legame).

    Con quale criterio abbiamo scelto di usare un doppietto di uno dei tre atomi di ossigeno e non degli altri?

    Nessun criterio, in modo arbitrario. Siamo noi a scegliere quale dei tre ossigeni sia legato al carbonio con un doppio legame. Quindi quando abbiamo due o più atomi esterni identici e scegliamo quale debba essere legato all'atomo centrale con un doppio legame quello che stiamo facendo è ipotizzare che ci sia una "formula limite" tra le varie possibili, perché avremmo potuto benissimo scegliere che fosse l'atomo di sinistra a essere legato con un doppio legame, o quello a destra. Tutte e tre le formule sono possibili.
    
    Verrà chiarito meglio dopo, per il momento diciamo che stiamo scrivendo una delle possibili formule-limite.
    \item ES14 \ce{NO_3^-} n.e.=24
    
    Nell'anione nitrato i tre ossigeni forniscono 18 elettroni, l'azoto fornisce 5 elettroni e siamo a 23, avendo una carica negativa abbiamo ul elettrone extra da considerare per cui in totale 24, come nel caso precedente.

    L'azoto va al centro in quanto meno elettronegativo,mentre gli ossigeni vanno messi attorno. Per legarli all'azoto consumiamo 6 elettroni:
    $$
    \chemleft[\chemfig{O-N(-[6]O)(-O)}\chemright{]^{-}}
    $$
    Restano 18 elettroni, cioè 9 coppie: ne assegnamo 3 a ciascun ossigeno, esaurendole tutte:
    $$
    \chemleft[ \chemfig{\charge{[circle]90=\:,180=\:,270=\:}{O}-N([6]-[@{x1}]@{x2}{\charge{[circle]0=\:,180=\:,270=\:}{O}})(-\charge{[circle]0=\:,90=\:,270=\:}{O})} \chemright{]^{-}}
    \chemmove[shorten <=4pt, shorten >=4pt]{
    \draw(x2.180)..controls +(180:0.5cm)and+(180:0.8cm)..(x1.60);}
    $$
    Gli ossigeni hanno raggiunto l'ottetto, l'azoto no. Allora ancora una volta, in modo del tutto arbitrario, trasformiamo un doppietto di uno qualunque tra i tre ossigeni in doppio legame:
    $$
    \chemleft[ \chemfig{\charge{[circle]90=\:,180=\:,270=\:}{O}-N(=[6]\charge{[circle]0=\:,270=\:}{O})(-\charge{[circle]0=\:,90=\:,270=\:}{O})} \chemright{]^{-}}
    $$
    Adesso anche l'azoto ha raggiunto l'ottetto (8 elettroni di legame).

    Anche in questo caso abbiamo scelto una delle tre possibili formule limite, che corrispondono ad avere un doppio legame con l'ossigeno a sinistra, in basso o a destra. Quello che si dovrebbe fare è scriverle tutte e tre e dire che le formule sono ugualmente probabili.
\end{itemize}
\subsection{Strutture molecolari pi sbaddu}
\begin{itemize}
    \item Gruppo 4A
    
    
    \chemname{\footnotesize\chemfig{H-C(-[2]H)(-[6]H)(-H)}}{metano\\\ce{CH_4}}
    \qquad
    \chemname{\footnotesize\chemfig{H-C(-[2]H)(-[6]H)-C(-[2]H)(-[6]H)-H}}{etano\\\ce{C_2H_6}}
    \qquad
    \chemname{\footnotesize\chemfig{H-C(-[6]H)=C(-[6]H)-H}}{etilene\\\ce{C_2H_4}}
    \qquad
    \chemname{\footnotesize\chemfig{H-C~C-H}}{acetilene\\\ce{C_2H_2}}
    
    Nota: il metano è il gas che arriva nelle tubature. Considerando le regole di Lewis, l'idrogeno è periferico, quindi il carbonio va al centro. In totale la molecola ha 8 elettroni (4 dal carbonio e 4 dai quattro idrogeni), che consumiamo tutti nel legare gli atomo di idrogeno al carbonio, il quale così raggiunge l'ottetto. Analogamente per le altre molecola, che sono simili.
    \item Gruppo 5A
    
    \chemname{\footnotesize\chemfig{H-\charge{[circle]90=\:}{N}(-[6]H)-H}}{ammoniaca\\\ce{NH_3}}
    \qquad
    \chemname{\footnotesize\chemfig{H-\charge{[circle]90=\:}{N}(-[6]H)-\charge{[circle]90=\:}{N}(-[6]H)-H}}{idrazina\\\ce{N_2H_4}}
    \qquad
    \chemname{\footnotesize\chemleft[ \chemfig{H-N(-[6]H)(-[2]H)-H} \chemright{]^{+}}}{ione ammonio\\\ce{NH_4^+}}
    \qquad 
    \chemname{\footnotesize\chemleft[ \chemfig{H-\charge{[circle]90=\:,270=\:}{N}-H} \chemright{]^{-}}}{ione ammide\\\ce{NH_2^-}} 
    \item Gruppo 6A
    
    \chemname{\footnotesize\chemfig{H-\charge{[circle]90=\:,270=\:}{O}-H}}{acqua\\\ce{H_2O}}
    \qquad
    \chemname{\footnotesize\chemfig{H-\charge{[circle]90=\:,270=\:}{O}-\charge{[circle]90=\:,270=\:}{O}-H}}{perossido di\\diidrogeno\\\ce{H_2O_2}}
    \qquad
    \chemname{\footnotesize\chemleft[ \chemfig{H-\charge{[circle]90=\:}{O}(-[6]H)-H} \chemright{]^{+}}}{ione idronio\\\ce{H_3O^+}} 
    \qquad \,
    \chemname{\footnotesize\chemleft[ \chemfig{\charge{[circle]90=\:,180=\:,270=\:}{O}-H} \chemright{]^{-}}}{ione idrossido\\\ce{OH^-}}
    \item Gruppo 7A
    
    \footnotesize\chemfig{H-\charge{[circle]0=\:,90=\:,270=\:}{F}} \quad fluoruro di idrogeno \ce{HF}
    \item Ossiacidi comuni e i loro anioni
    
    \chemname{\footnotesize\chemfig{H-\charge{[circle]90=\:,270=\:}{O}-N(-[6]\charge{[circle]0=\:,180=\:,270=\:}{O})=\charge{[circle]0=\:,90=\:}{O}}}{acido nitrico\\\ce{HNO_3}}
    \qquad
    \chemname{\footnotesize\chemleft[ \chemfig{\charge{[circle]90=\:,180=\:,270=\:}{O}-N(-[6]\charge{[circle]0=\:,180=\:,270=\:}{O})=\charge{[circle]0=\:,90=\:}{O}} \chemright{]^{-}}}{ione nitrato\\\ce{NO_3^-}}
    \qquad
    \chemname{\footnotesize\chemfig{\charge{[circle]90=\:,180=\:,270=\:}{O}-Cl(-[2]\charge{[circle]90=\:,180=\:}{O}-H)(-[6]\charge{[circle]0=\:,180=\:,270=\:}{O})-\charge{[circle]0=\:,90=\:,270=\:}{O}}}{acido perclorico\\\ce{HClO_4}}
    \qquad
    \chemname{\footnotesize\chemleft[ \chemfig{\charge{[circle]90=\:,180=\:,270=\:}{O}-Cl(-[2]\charge{[circle]90=\:,180=\:}{O})(-[6]\charge{[circle]0=\:,180=\:,270=\:}{O})-\charge{[circle]0=\:,90=\:,270=\:}{O}} \chemright{]^{-}}}{ione perclorato\\\ce{ClO_4^-}}

    \chemname{\footnotesize\chemfig{\charge{[circle]90=\:,180=\:,270=\:}{O}-P(-[2]\charge{[circle]90=\:,180=\:}{O}-H)(-[6]\charge{[circle]0=\:,270=\:}{O}-[4]H)(-\charge{[circle]0=\:,90=\:}{O}-[6]H)}}{acido fosforico\\\ce{H_3PO_4}}
    \qquad
    \chemname{\footnotesize\chemleft[ \chemfig{\charge{[circle]90=\:,180=\:,270=\:}{O}-P(-[2]\charge{[circle]0=\:,90=\:,180=\:}{O})(-[6]\charge{[circle]0=\:,180=\:,270=\:}{O})-\charge{[circle]0=\:,90=\:,270=\:}{O}}\chemright{]^{3-}}}{ione fosfato\\\ce{PO_4^-}}
    \qquad
    \chemname{\footnotesize\chemfig{H-\charge{[circle]90=\:,270=\:}{O}-\charge{[circle]0=\:,90=\:,270=\:}{Cl}}}{acido ipocloroso\\\ce{HOCl}}
    \qquad
    \chemname{\footnotesize\chemleft[ \chemfig{\charge{[circle]90=\:,180=\:,270=\:}{O}-\charge{[circle]0=\:,90=\:,270=\:}{Cl}} \chemright{]^{-}}}{ione ipoclorito\\\ce{ClO^-}}
\begin{center}
    \chemname{\footnotesize\chemfig{\charge{[circle]90=\:,180=\:,270=\:}{O}-S(-[2]\charge{[circle]90=\:,180=\:}{O}-H)(-[6]\charge{[circle]180=\:,270=\:}{O}-H)-\charge{[circle]0=\:,90=\:,270=\:}{O}}}{acido solforico\\\ce{H_2SO_4}}
    \qquad
    \chemname{\footnotesize\chemleft[ \chemfig{\charge{[circle]90=\:,180=\:,270=\:}{O}-S(-[2]\charge{[circle]90=\:,180=\:}{O}-H)(-[6]\charge{[circle]0=\:,180=\:,270=\:}{O})-\charge{[circle]0=\:,90=\:,270=\:}{O}} \chemright{]^{-}}}{ione idrogeno-\\solfato \ce{HSO_4^-}}
    \qquad
    \chemname{\footnotesize\chemleft[ \chemfig{\charge{[circle]90=\:,180=\:,270=\:}{O}-S(-[2]\charge{[circle]0=\:,90=\:,180=\:}{O})(-[6]\charge{[circle]0=\:,180=\:,270=\:}{O})-\charge{[circle]0=\:,90=\:,270=\:}{O}} \chemright{]^{2-}}}{ione solfato\\\ce{HSO_4^{2-}}}
\end{center}
    Nota: a volte la formula dell'acido solforico viene scritta con dei doppi legami zolfo ossigeno. Il motivo è che inizialmente, partendo dal concetto di valenza, si facevano partire dall'atomo tanti legami quant'era la sua valenza. Nell'acido solforico lo zolfo ha valenza 6, quindi per legarlo a 4 ossigeni due di questi dovranno avere un doppio legame. In questo modo però attorno allo zolfo ci sarebbero 12 elettroni, quindi la regola dell'ottetto sarebbe violata. È quindi più corretto rappresentarlo come abbiamo fatto noi. Analogamente per i due ioni solfati.
\end{itemize}
\subsection{Il concetto di risonanza}
Entriamo nel merito di quei composti che possono presentare più formule-limite, come ad esempio gli ioni nitrato \ce{NO_3-} e carbonato \ce{CO_3^{2-}}. Ci chiediamo quindi perché scriviamo solo una delle varie formule possibili e cosa implica che ci sia più di una possibilità

Per rispondere a queste domanda, dobbiamo introdurre una definizione impropria dell'\textbf{ordine di legame}(o bond order): per il momento questa sarà sufficiente, in seguito ne daremo una più precisa. 

Per il momento consideriamo l'ordine di legame come il numero di legame. Quindi un legame semplice ha ordine di legame 1, un legame doppio ordine di legame 2, un triplo ordine 3 ecc.\\

Dobbiamo capire quali formule diverse possono essere scritte e come queste contribuiscano alla formula reale, perché è chiaro che se possiamo scrivere più formule queste saranno o tutte vere o tutte false, oppure se è vera solo una di questa dovremmo il perché, dato che l'abbiamo scelta arbitrariamente.

Riprendiamo quindi alcuni esempi delle formule di struttura:
\begin{itemize}
    \item ES1 \ce{O_3}
    $$
    \chemfig{\charge{[circle]90=\:,180=\:,270=\:}{O}-\charge{[circle]90=\:}{O}-\charge{[circle]0=\:,90=\:,270=\:}{O}}
    $$
    Osservando gli elettroni esterni degli ossigeni notavamo che quello centrale non raggiungeva l'ottetto, per cui sceglievamo un doppietto dell'ossigeno o di sinistra o di destra e lo trasformavamo in doppio legame:
    $$\schemestart
    \chemfig{\charge{[circle]180=\:,270=\:}{O}=\charge{[circle]90=\:}{O}-\charge{[circle]0=\:,90=\:,270=\:}{O}}
    \;
    \arrow{<->}
    \;
    \chemfig{\charge{[circle]90=\:,180=\:,270=\:}{O}-\charge{[circle]90=\:}{O}=\charge{[circle]0=\:,270=\:}{O}}
    \schemestop
    $$
    Entrambe le scelte sono valide, perché in entrambi i casi tutti e tre gli atomi hanno raggiunto l'ottetto. Qual è allora quella vera?

    Se non potessimo distinguere i due atomi di ossigeno, potremmo pensare che la formula a destra non sia altro che quella di sinistra ruotata di 180°, quindi sarebbero la stessa formula. Nella realtà invece possiamo distinguere i due ossigeni, ad esempio imponendo che siano due isotopi diversi, per cui le formule-limite sono diverse. Tuttavia esse non sono vere, perché nella realtà l'ozono può essere descritto al 50\% da una formula e al 50\%, dall'altra, ossia la funzione d'onda molecolare sarà data da una combinazione lineare della funzione d'onda che descrive la struttura a sinistra e di quella che descrive la struttura a destra.

    In altre parole, nella molecola dell'ozono non abbiamo un doppio legame e uno semplice: entrambi i legami saranno una via di mezzo tra un legame semplice e uno doppio. Ne segue che l'ordine di legame non è 1 per un legame e 2 per l'altro, ma vale 1.5  per entrambi. Ciò significa che essi mostrano solo in parte carattere di doppio legame.
    Ciò si può anche verificare sperimentalmente, perché conosciamo il valore delle distanze di legame sia nel caso di legame semplice che nel caso di legame doppio. Se ci fosse un legame semplice e uno doppio, uno dovrebbe essere più lungo dell'altro\footnote{maggiore è il numero del legame, minore sarà la distanza di legame, perché i due atomi risulteranno più legati, ovvero il legame che li unisce sarà più forte.}, ma il valore della distanza di legame che si misura per questi due legami è tra quello di un legame semplice e quello di un legame doppio.

    In questi casi si ha l'abitudine di tratteggiare un parziale carattere di doppio legame su tutti i legami semplici, per indicare che questi legami sono più forti dei legami semplici:
    $$
    \chemfig{O-[,,,,,rddbond]O-[,,,,,rddbond]O}
    $$
    Quindi nella realtà le formule limite non sono vere, ma sono vere le loro combinazioni lineari, sebbbene nella pratica poi le scriviamo, avendo però coscienza che nel dire che uno dei legami è semplice e l'altro è doppio stiamo ragionando sul concetto di formula-limite, che non è reale.
    \item ES2 \ce{CO_3^{2-}}
    $$
    \chemleft[ \footnotesize\chemfig{\charge{[circle]90=\:,180=\:,270=\:}{O}-[:-30]C([:270]-\charge{[circle]0=\:,180=\:,270=\:}{O})(-[:30]\charge{[circle]0=\:,90=\:,270=\:}{O})} \chemright{]^{2-}}
    $$
    (Poc'anzi scrivevamo lo ione carbonato con angoli di 90°, ora con angoli di 120°. Anche stavolta la scelta è arbitraria perché il modello di Lewis non ci permette di ipotizzare geometrie)
    In questa struttura il carbonio non raggiungeva l'ottetto, per cui trasformavamo un doppietto di uno qualunque degli ossigeni in doppi legame. Potendo scegliere tra tre ossigeni, riusciamo a scrivere tre formule limite:
    $$
    \schemestart
    \chemleft[ \footnotesize\chemfig{\charge{[circle]90=\:,180=\:}{O}=[:-30]C(-[270]\charge{[circle]0=\:,180=\:,270=\:}{O})(-[:30]\charge{[circle]0=\:,90=\:,270=\:}{O})} \chemright{]^{2-}}
    \arrow{<->}
    \chemleft[ \footnotesize\chemfig{\charge{[circle]90=\:,180=\:,270=\:}{O}-[:-30]C(-[270]\charge{[circle]0=\:,180=\:,270=\:}{O})(=[:30]\charge{[circle]90=\:,270=\:}{O})} \chemright{]^{2-}}
    \arrow{<->}
    \chemleft[ \footnotesize\chemfig{\charge{[circle]90=\:,180=\:,270=\:}{O}-[:-30]C(=[270]\charge{[circle]180=\:,270=\:}{O})(-[:30]\charge{[circle]0=\:,90=\:,270=\:}{O})} \chemright{]^{2-}}
    \schemestop
    $$
    A differenza dell'esempio precedente in cui l'ordine di legame era proprio 1.5 perché il legame si ripartisce tra duei, qui si ha un doppio legame che può essere distribuito in tre connessioni diverse, quindi ipotizziamo che l'ordine di legame sia 1.33, cioè ogni legame non è una via di mezzo, è più un legame semplice che un legame doppio, ma è comunque più forte di un legame semplice. 
    $$\chemleft[ \footnotesize\chemfig{
    O-[:-30,,,,,lddbond]C(-[:30,,,,,lddbond]O)-[:270,,,,rddbond]O} \chemright{]^{2-}}
    $$
    Osserviamo i valori sperimentali della distanza di legame carbonio-ossigeno:
    
    \begin{center}
    \begin{tabular}{|m{2cm}|m{2cm}|}
        \hline
        legame & distanza \\
        \hline
        \ce{C-O} & 143 pm \\
        \hline
        \ce{C=O} & 122 pm \\
        \hline
        \ce{C#O} & 113 pm \\
        \hline
    \end{tabular}
\end{center}
Nel carbonato si misura una distanza di legame carbonio-ossigeno di 129 pm: è più corto di un legame semplice, ma più lungo di un legame doppio, cioè è compreso tra i due valori. Pertanto è corretto aver ipotizzato che tutti e 3 i legami godano di un parziale doppio legame.

Dunque, quando scriviamo lo ione carbonato, usiamo una delle tre formule-limite, ma con la coscienza che dovremmo scriverlo con le linee tratteggiate:
    \item ES3 \ce{NO_2^-}
    $$
    \chemleft[ \chemfig{\charge{[circle]90=\:,180=\:,270=\:}{O}-[:30]\charge{[circle]90=\:}{N}-[:-30]\charge{[circle]0=\:,90=\:,270=\:}{O}} \chemright{]^{-}}
    $$
    Studiando lo ione nitrito ci accorgevamo che l'azoto non raggiungeva l'ottetto, per cui trasformavamo un doppietto in doppio legame:
    $$
    \schemestart
    \chemleft[ \chemfig{\charge{[circle]90=\:,270=\:}{O}=[:30]\charge{[circle]90=\:}{N}-[:-30]\charge{[circle]0=\:,90=\:,270=\:}{O}} \chemright{]^{-}}
    \arrow{<->}
    \chemleft[ \chemfig{\charge{[circle]90=\:,180=\:,270=\:}{O}-[:30]\charge{[circle]90=\:}{N}=[:-30]\charge{[circle]90=\:,270=\:}{O}} \chemright{]^{-}}
    \schemestop
    $$
    Anche stavolta scegliere quale sia l'ossigeno che forma il doppio legame è un arbitrio: entrambe sono scelte valide, sono le cosiddette formule limite che usiamo per scrivere questo ione.

    Il doppio legame dovrà distribuirsi in entrambi i legami azoto-ossigeno e quindi non ci sarà un legame con ordine 2 e uno con ordine 1, ma entrambi avranno ordine 1.5.

    La molecola andrà quindi scritta così:
    $$
    \chemleft[\chemfig{O-[:30,,,,,rddbond]N-[:-30,,,,,rddbond]O}\chemright{]^{-}}
    $$
Analizziamo i dati sperimentali:
    \begin{center}
        \begin{tabular}{|m{2cm}|m{2cm}|}
            \hline
            legame & distanza \\
            \hline
            \ce{N-O} & 136 pm \\
            \hline
            \ce{N=O} & 115 pm \\
            \hline
            \ce{N#O} & 108 pm \\
            \hline
        \end{tabular}
    \end{center}
Nell'\ce{NO_2^-} si misura una distanza di 122 pm, che sta tra un legame semplice e uno doppio, e questa si misura per entrambi i legami: sono uguali, non sono un più lungo e uno più corto. È quindi ragionevole ipotizzare che l'ordine di legame sia 1.5.
   \item ES4 \ce{NO_3^-}
   $$
    \chemleft[ \footnotesize\chemfig{\charge{[circle]90=\:,180=\:,270=\:}{O}-[:-30]N([:270]-\charge{[circle]0=\:,180=\:,270=\:}{O})(-[:30]\charge{[circle]0=\:,90=\:,270=\:}{O})} \chemright{]^{2-}}
    $$
    Stavolta possiamo scegliere tra tre ossigeni, per cui avremo tre formule limite:
    $$
    \schemestart
    \chemleft[ \footnotesize\chemfig{\charge{[circle]90=\:,180=\:}{O}=[:-30]N(-[270]\charge{[circle]0=\:,180=\:,270=\:}{O})(-[:30]\charge{[circle]0=\:,90=\:,270=\:}{O})} \chemright{]^{2-}}
    \arrow{<->}
    \chemleft[ \footnotesize\chemfig{\charge{[circle]90=\:,180=\:,270=\:}{O}-[:-30]N(-[270]\charge{[circle]0=\:,180=\:,270=\:}{O})(=[:30]\charge{[circle]90=\:,270=\:}{O})} \chemright{]^{2-}}
    \arrow{<->}
    \chemleft[ \footnotesize\chemfig{\charge{[circle]90=\:,180=\:,270=\:}{O}-[:-30]N(=[270]\charge{[circle]180=\:,270=\:}{O})(-[:30]\charge{[circle]0=\:,90=\:,270=\:}{O})} \chemright{]^{2-}}
    \schemestop
    $$
    Se abbiamo un solo doppio legame da distribuire in tre diverse connessioni, ognuno di questi legami dovrebbe avere un ordine di legame pari a 1.33. Infatti nello ione nitrato si misura una distanza di legame di 124 pm, compresa tra un legame semplice e uno doppio e più lunga di quella osservata nel nitrito. Ciò è corretto perché il legame osservato nel nitrito è 1.5 dato che il doppio legame si ripartisce su due legami anziché tre.
    $$\chemleft[ \footnotesize\chemfig{
    O-[:-30,,,,,lddbond]N(-[:30,,,,,lddbond]O)-[:270,,,,rddbond]O} \chemright{]^{2-}}
    $$
    \item ES4 \ce{CNO^-} n.e.=14
    
    Nello ione cianato l'atomo meno elettronegativo è il carbonio che quindi mettiamo al centro, mentre azoto e ossigeno vanno ai suoi lati. 
    
    Il carbonio fornisce 4 elettroni, l'azoto 5 e l'ossigeno 6 per un totale di 15, più una carica negativa dato che è un anione si arriva a 16. Legando i due atomi laterali a quello centrale ne consumiamo 4 e ce ne restano 12
    $$
    \chemleft[\chemfig{N-C-O}\chemright{]^{-}}
    $$
    Questi 12 elettroni vanno assegnati prima all'atomo esterno più elettronegativo, cioè l'ossigeno, e quelli che restano all'azoto.
    $$
    \chemleft[ \chemfig{@{x1}{\charge{[circle]90=\:,180=\:,270=\:}{N}}-[@{x2}]\charge{[circle]90=\:}{C}-[@{x3}]@{x4}{\charge{[circle]0=\:,90=\:,270=\:}{O}}} \chemright{]^{-}}
    \chemmove[shorten <=4pt, shorten >=4pt]{
    \draw(x1.110)..controls +(60:0.5cm)and+(90:0.5cm)..(x2);
    \draw(x4.70)..controls +(120:0.5cm)and+(90:0.5cm)..(x3);
    }$$
    Riusciamo a dare 3 coppie ad entrambi e in questo modo raggiungono l'ottetto. Per farlo raggiungere anche al carbonio, trasformiamo un doppietto dell'azoto in doppio legame e analogamente facciamo per l'ossigeno:
    $$
    \chemleft[ \chemfig{\charge{[circle]180=\:,270=\:}{N}=C=\charge{[circle]0=\:,270=\:}{O}} \chemright{]^{-}}
    $$
    In essa tutti e 3 gli atomi hanno raggiunto l'ottetto.
    Tuttavia possiamo scrivere altre due formule:
    $$
    \schemestart
    \chemleft[ \chemfig{\charge{[circle]180=\:,270=\:}{N}-C~\charge{[circle]0=\:,270=\:}{O}} \chemright{]^{-}}
    \arrow{<->}
    \chemleft[ \chemfig{\charge{[circle]180=\:,270=\:}{N}~C-\charge{[circle]0=\:,270=\:}{O}} \chemright{]^{-}}
    \schemestop
    $$
    Anche queste sono possibili, perché anche in esse tutti e 3 gli atomi hanno completato l'ottetto. Pertanto anche queste sono formule limite.
    \item ES5 \ce{F_2O} n.e.=20
    
    Nel difluoruro di ossigeno abbiamo 14 elettroni dai due atomi di fluoro e 6 dall'ossigeno, per un totale di 20. Il fluoro è più elettronegativo, quindi metto l'ossigeno al centro e i due fluoro ai lati:
    $$
    \chemfig{F-O-F}
    $$
    Abbiamo consumato 4 elettroni, ne restano 16: 6 li diamo ad un fluoro, 6 all'altro e i 4 restanti all'ossigeno:
    $$
    \chemfig{\charge{[circle]90=\:,180=\:,270=\:}{F}-\charge{[circle]90=\:,270=\:}{O}-\charge{[circle]0=\:,90=\:,270=\:}{F}}
    $$
    In questo modo tutti gli atomo hanno già raggiunto l'ottetto, per cui non è necessario trasformare doppietti in ulteriori legami.
\end{itemize}
\subsection{Esempio di ordine di legame (B.O.)}
\begin{itemize}
    \item B.O.=1
    \begin{itemize}
        \item \ce{H_2}
        
        $$
        \chemfig{H-H}
        $$

        Due atomi di idrogeno legati da un legame semplice.
        
        Non si può avere altro perché ogni atomo di idrogeno fornisce un solo elettrone e nel legarli li abbiamo consumati entrambi;
        \item \ce{F_2}
        
        $$
        \chemfig{\charge{[circle]90=\:,180=\:,270=\:}{F}-\charge{[circle]0=\:,90=\:,270=\:}{F}}
        $$

        Il fluoro è un alogeno, ha 7 elettroni per atomo, per un totale di 14.
        
        Li mettiamo accanto, consumiamo 2 elettroni legandoli, ne restano 12 cioè 6 doppietti, 3 per ciascun atomo di fluoro.
        \item \ce{NH_3}
        
        $$
        \chemfig{\charge{[circle]90=\:}{N}(-[:210]H)(-[:260]H)(-[:-40]H)}
        $$

        Vedi sopra
    \end{itemize}
    \item B.O.=2
    \begin{itemize}
        \item \ce{CO_2}
        
        $$
        \chemfig{\charge{[circle]180=\:,270=\:}{O}=C=\charge{[circle]0=\:,270=\:}{O}}
        $$

        Nell'anidride carbonica ogni legame è doppio, quindi l'ordine di legame è 2 sia a destra che a sinistra;
        \item \ce{C_2H_4}
        
        $$
        \chemfig{C(-[:130]H)(-[:230]H)=C(-[:50]H)(-[:-50]H)}
        $$

        Nell'etilene il legame carbonio-carbonio è di ordine 2;
    \end{itemize}
    \item B.0.=3
    \begin{itemize}
        \item \ce{CO}
        
        $$
        \chemfig{\charge{180=\:}{C}~\charge{90=\:}{O}}
        $$

        Nell'anidride carboniosa carbonio e ossigeno sono legati da un triplo legame, per cui l'ordine di legame è 3;
        \item \ce{C_2H_2}
        
        $$
        \chemfig{H-C~C-H}
        $$

        Nell'acetilene il legame carbonio-carbonio è di ordine 3;
    \end{itemize}
    \item B.O.=$\frac{3}{2}=1.5$
    \begin{itemize}
        \item \ce{O_3}
        
        $$
        \chemfig{\charge{180=\:,270=\:}{O}=\charge{90=\:}{O}-\charge{0=\:,90=\:,270=\:}{O}}
        $$

        L'ozono ha ordine di legame 1.5 perché il doppio legame si ripartisce in due legami diversi;
        \item \ce{NO_2^-}
        
        $$
        \chemleft[ \chemfig{\charge{[circle]90=\:,180=\:,270=\:}{O}-[:30]\charge{[circle]90=\:}{N}=[:-30]\charge{[circle]90=\:,270=\:}{O}} \chemright{]^{-}}
        $$

        Anche per lo ione nitrito abbiamo un doppio legame ripartito in due connessioni
    \end{itemize}
    \item B.O.=$\frac{4}{3}=1.33$
    \begin{itemize}
        \item \ce{NO_3^-}
        
        $$
        \chemleft[ \footnotesize\chemfig{N(=[:90]\charge{[circle]90=\:,180=\:}{O})(-[:-30]\charge{[circle]0=\:,90=\:,270=\:}{O})(-[:210]\charge{[circle]90=\:,180=\:,270=\:}{O})} \chemright{]^{2-}}
        $$
        Nello ione nitrato, un doppio legame si ripartisce in tre connessioni;
        \item \ce{CO_3^{2-}}
        
        $$
        \chemleft[ \footnotesize\chemfig{\charge{[circle]90=\:,180=\:,270=\:}{O}-[:-30]C(=[270]\charge{[circle]180=\:,270=\:}{O})(-[:30]\charge{[circle]0=\:,90=\:,270=\:}{O})} \chemright{]^{2-}}
        $$
        Nello ione carbonato la situazione è analoga al caso precedente.
    \end{itemize}
\end{itemize}
\subsection{Valenza espansa \\{\small (gusci di valenza espanza)}}
Esistono orbitali vuoti a bassa energia, ossia oltre agli orbitali nei quali allocare fino ad 8 elettroni, ce ne sono degli altri. Osserviamo due casi:
\begin{itemize}
    \item ES1 \ce{XeF_2} n.e.=22
    I gas nobili vengono etichettati tali perché non si combinano con gli altri elementi. Ciò è vero solo in parte: alcuni di questi, infatti, si riesce a farli reagire. E il caso dello xeno, ossia è possibile ottenere suoi composti, come il fluoruro di xeno.

    Se andiamo a contare gli elettroni di questo composto ci accorgiamo che sono tanti, perché il fluoro con due atomi dà 14 elettroni e lo xeno in quanto gas nobile ne dà 8, per un totale di 22. L'atomo più elettronegativo è il fluoro, quindi lo xeno va al centro e i due atomi di fluoro ai lati. 
 
    $$
    \chemfig{F-Xe-F}
    $$
    Legando i due atomi consumiamo 4 elettroni e ne restano 18, dei quali 6 vanno a un fluoro e 6 all'altro, che così raggiungono l'ottetto. Restano 6 elettroni che dobbiamo assegnare all'atomo centrale:
    $$
    \chemfig{\charge{[circle]90=\:,180=\:,270=\:}{F}-\charge{[circle]60=\:,120=\:,270=\:}{Xe}-\charge{[circle]0=\:,90=\:,270=\:}{F}}
    $$
    A questo punto lo xeno avrà apparentemente troppi elettroni, perché attorno gliene contiamo 10. Com'è possibile?
    \item ES2 \ce{PCl_5} n.e.=40
    
    Nel pentacloruro di fosforo, il fosforo fornisce 5 elettroni, mentre i 5 atomi di cloro forniscono 7 elettroni ciascuno, per un totale di 35, il che significa in definitiva 40 elettroni.

    Il cloro è più elettronegativo, quindi lo mettiamo come atomo terminale e mettiamo il fosforo al centro
    $$
    \chemfig{P(-[:210]Cl)(-[:270]Cl)(-[:-30]Cl)(-[:30]Cl)(-[:150]Cl)}
    $$
    Dovendo legare al fosforo tutti e cinque gli atomi di cloro consumiamo 10 elettroni e ne restano 30, che suddividiamo in 6 elettroni su ciascun atomo di cloro. Così facendo ogni cloro ha raggiunto l'ottetto, mentre il fosforo ne avrà 10 attorno a sé
\end{itemize}
La configurazione elettronica del fosforo è $1s^22s^22p^63s^23p^3$, quindi i 3p non sono totalmente pieni, e inoltre dopo i 3p ci sono i 3d di valenza, ma vuoti, tant'è che non abbiamo riempito neanche i p.

Quello che vogliamo dire è che per alcuni atomi si hanno orbitali di valenza vuoti, quindi virtuali, allo stesso numero quantico (dunque sono a bassa energia) di alcuni orbitali di valenza pieni o parzialmente riempiti. Essi sono disponibili per ospitare elettroni, per cui per i loro atomi si parlerà di valenza espanza, cioè essi potranno ospitare attorno a sè più di 8 elettroni. Ad esempio, se si hanno a disposizione gli orbitali d, si potranno allocare fino a 18 elettroni.