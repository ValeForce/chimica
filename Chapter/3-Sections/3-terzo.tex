Cerchiamo ora di ragionare su sistemi più complessi.

Finora non abbiamo studiato molecole in cui fossero presenti doppietti sull'atomo centrale. Per spiegare le geometrie di queste è necessario introdurre la teoria \textbf{V.S.E.P.R} (Valence Shell Elctron Pair Repulsion), che tiene conto della repulsione esercitata dalle coppie di elettroni. Infatti essendo cariche dello stesso segno gli elettroni tenderanno a respingersi fra di loro, per cui le molecole assumeranno delle geometrie tali che questa repulsione sia minima, ovvero quelle che tengono le coppie di elettroni il più lontano possibile tra loro.
\subsection{Coppie singole e coppie di legame}
Per spiegare alcune variazioni su certi angoli di legame si è pensato, ed è realistico, che le coppie di elettroni di legame non siano localizzate su un solo atomo, ma su entrambi quelli legati da tale coppia, e che quindi abbiano più spazio di quanto ne abbia a disposizione le coppie solitarie, localizzate su un solo atomo. Ne segue che ci sarà diversa repulsione tra coppie di legame e coppie di legame, tra coppie di legame e coppie solitarie e tra coppie solitarie e coppie solitarie.

È 
