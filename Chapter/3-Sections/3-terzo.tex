Cerchiamo ora di ragionare su sistemi più complessi.

Finora non abbiamo studiato molecole in cui fossero presenti doppietti sull'atomo centrale. Per spiegare le geometrie di queste è necessario introdurre la teoria \textbf{V.S.E.P.R} (Valence Shell Elctron Pair Repulsion), che tiene conto della repulsione esercitata dalle coppie di elettroni. Infatti, essendo cariche dello stesso segno, gli elettroni tenderanno a respingersi fra di loro, per cui le molecole assumeranno delle geometrie tali che questa repulsione sia minima, ovvero quelle che tengono le coppie di elettroni il più lontano possibile tra loro.
\subsection{Coppie singole e coppie di legame}
Per spiegare alcune variazioni su certi angoli di legame si è pensato, ed è realistico, che le coppie di elettroni di legame non siano localizzate su un solo atomo, ma su entrambi quelli legati da tale coppia, e che quindi abbiano più spazio di quanto ne abbiano a disposizione le coppie solitarie, localizzate su un solo atomo. Ne segue che ci sarà diversa repulsione tra coppie di legame e coppie di legame, tra coppie di legame e coppie solitarie e tra coppie solitarie e coppie solitarie.

È ragionevole che due coppie di elettroni non coinvolte nel legame e che si trovano nello stesso atomo si respingono di più di una coppia di elettroni di legame con una solitaria, le quali a loro volta si respingono di più di due coppie di legame.

Di volta in volta dovremo quindi andare a vedere come si dispongono nello spazio tutti gli elettroni, con l'obiettivo di vedere quale possa essere la geometria che permette una minore repulsione.\\

ES1 \ce{H_2O} e \ce{NH_3}

Secondo il formalismo di Lewis, nell'acqua l'ossigeno aveva due coppie solitarie. L'ammoniaca invece vedeva una sola coppia solitaria sull'azoto.

Se non ci fossero queste repulsioni, la teoria dell'ibridizzazione ci dice che si devono formare angoli di 109.5°, mentre se gli orbitali non si fossero ibridati dovremmo trovare angoli di legame di 90°, in quanto i legami sarebbero formati da due degli orbitali p che interagiscono con due atomi di idrogeno. Analogamente per l'ammoniaca.

Nei fatti non si osserva né una geometria né l'altra: si osserva una struttura simile a quella tetraedrica, ma con angoli diversi.

Grazie alla teoria V.S.E.P.R.
