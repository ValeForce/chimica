Per lo studio delle geometrie molecolari utilizzeremo due approcci diversi: prima tratteremo delle molecole semplici, dopodiché vedremo come sia possibie generalizzare alcune informazioni per studiare molecole più complesse.
\subsection{Lo stato di promozione del carbonio}
Consideriamo la molecola del metano \ce{CH_4}. In essa il carbonio è legato a quattro atomi di idrogeno, per cui la sua valenza è pari a 4.

La configurazione elettronica del carbonio nello stato fondamentale però è
$$\text{C} \quad 1s^22s^22p_x^12p_y^12p_z^0 \quad \subshell{1s:2} \; \subshell{2s:2} \; \subshell{2p:110}$$
Nota: l'avere scritto  un elettrone nel 2p$_x$, uno nel 2p$_y$ e zero nel 2p$_z$ è stata una nostra scelta: potevamo scegliere che fosse vuoto il 2p$_x$ e occupati 2p$_y$ e 2p$_z$, oppure che fosse vuoto il 2p$_y$ e occupati 2p$_x$ e 2p$_z$

Avendo solo due orbitali parzialmente occupati, questa configurazione ci porta a pensare che il carbonio abbia valenza 2, in contrasto in quasi tutti i suoi composti il carbonio è tetravalente.

Per spiegare la valenza 4, si è pensato che uno degli elettroni degli orbitali 2s (che col 2p sono i livelli di valenza) possa essere promosso al restante orbitale p vuoto in modo tale che la confogurazione elettronica del carbonio in questo nuovo stato, detto di promozione, sia

$$\text{C} \quad 1s^22s^12p_x^12p_y^12p_z^1 \quad \subshell{1s:2} \; \subshell{2s:1} \; \subshell{2p:111}$$

Lo stato di promozione tuttavia riesce a giustificare solo la valenza 4 del carbonio, ma non la geometria del metano: sperimentalmente per esso si misurano quattro legami tutti uguali, ma se abbiamo pensato di avere parzialmente occupati un orbitale s e tre orbitali p non ce li aspetteremmo tali, in quanto gli orbitali p sono direzionali (in particolar modo sono diretti lungo gli assi cartesiani), mentre l'orbitale s è a simmetria sferica. Inoltre i legami dovuti ai tre orbitali p dovrebbero stare a 90° l'uno dall'altro, ma nei fatti gli angoli sono di 109.5°.
\subsection{La teoria dell'ibridizzazione sp}
A questo punto interviene la teoria dell'ibridizzazione, ossia si parla di \textbf{orbitali ibridi} sp: si pensa che i tre orbitali p e l'orbitale s si mescolino fra loro per dare luogo a quattro orbitali identici, a meno della diversa orientazione, i quali si chiama ibridi proprio perché hanno sia carattere s che carattere p.

Gli orbitali sp possiedono un lobo molto più grande dell'altro, ciò è dovuto al fatto che le funzioni d'onda degli orbitali p hanno un segno che segue quello dell'asse lungo cui sono orientati