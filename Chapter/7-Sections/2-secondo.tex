\subsection{Elettrodo di prima specie}
Gli elettrodi visti finora appartengono a questa categoria. Essi sono metalli immersi in soluzioni dei loro ioni. Si fa ciò per raggiungere molto più velocemente l'equilibrio tra soluzione ed elettrodo, in quanto in acqua l'equilibrio è molto lento perché turbato dalla presenza degli ioni $\rm H_3O^+$ che competono assieme agli ioni liberati dal metallo nell'acqua per ritornare sull'elettrodo stesso.

L'elettrodo partecipa attivamente alla reazione, ossia è il metallo che libera ioni in soluzione, i quali lasciano elettroni sul metallo:

$$\ce{M <--> M^{n+} + ne^-}$$
\subsection{Elettrodo di seconda specie}
In questo caso il metallo è a contatto con un suo sale poco solubile, ad esempio argento Ag a contatto con cloruro di argento AgCl che è un sale insolubile. A sua volta il sale è a contatto con una soluzione di un sale (ovviamente solubile) che abbia l'anione in comune col sale poco solubile, quindi nel nostro esempio dovrà essere un sale che ha come anione ha lo ione cloruro, come il cloruro di potassio KCl. Un altro esempio è il mercurio Hg messo in contatto con un suo sale poco solubile che è il cloruro mercuroso $\rm Hg_2Cl_2$ (nota: i pedici non possono essere semplificati, tra poco vedremo perché), che a sua volta è a contatto con cloruro di potassio KCl.

Nello specifico, ciò che avviene nel primo esempio è che l'AgCl in minima parte si dissocia in $\rm Ag^+$ e $\rm Cl^-$

$$\ce{AgCl(s) <--> Ag^+ + Cl^-}$$

Se poi cediamo un elettrone ad $\rm Ag^+$, questo diventa $\rm Ag^0$, cioè argento metallico:

$$\ce{Ag^+ e^- <--> Ag}$$

In questo modo abbiamo ridotto l'argento.

La reazione totale allora sarà

$$\ce{AgCl(s) + e^- <--> Ag + Cl^-}$$
\subsection{Elettrodo di terza specie}
In esso un metallo nobile (cioè che non prende parte ai processi elettrolitici, funge solo da trasportatore di cariche) è immerso in una soluzione che contiene già stati ossidati e stati ridotti di un qualunque sistema redox.

Un esempio è un elettrodo di platino Pt che è un metallo nobile, immerso in una soluzione contenente un sale di ferro (II) e uno di ferro (III). Il platino fungerà solo da trasportatore di carica e lo stato ossidato e quello ridotto saranno. rispettivamente lo stato ferrico e lo stato ferroso.

Consideriamo il permanganato $\rm MnO_4^-$. Esso in ambiente fortemente acido acquista 5 elettroni e diventa $\rm Mn^{2+}$, quindi $\rm MnO_4^-$ è lo stato ossidato e $\rm Mn^{2+}$ quello ridotto.

Va da ricordare che in ambiente fortemente acido ci sono ioni $\rm H_3O^+$, quindi la forma ossidata non è solo il permanganato ma anche gli ioni $\rm H_3O^+$. Anche in questo caso se usiamo il platino sarà un elettrodo di terza specie perché viene semplicemente immerso e funge da trasportatore di elettroni.

Un altro esempio è il chinidrone, una miscela di due polveri bianche, il chinone e l'idrochinone, che sono la forma ossidata e la forma ridotta. Sono utili perché permettono di misurare in maniera veloce il pH. Se immergiamo un filo di platino in una soluzione di questi composti, agirà come un elettrodo di terza specie.

\vspace{0.2cm}Scriviamo le equazioni di Nernst relative ai vari elettrodi:

\vspace{0.2cm} $\bullet$\textbf{Elettrodo di prima specie}

$$\ce{Cu^{2+} + 2e^- <--> Cu} \quad E= E_0 + \frac{0.059}{2}\log \frac{a_{\text{Cu}^{2+}}}{a_{\text{Cu}}}$$

$$a_{\text{Cu}}=1 \implies E= E_0 + \frac{0.059}{2}\log a_{\text{Cu}^{2+}}$$

Questa sarà la d.d.p. di questo semielemento/elettrodo.

Va da ricordare che poi nei calcoli usiamo la concentrazione, non l'attività.

\vspace{0.2cm} $\bullet$\textbf{Elettrodo di terza specie}

Per convenzione le reazioni si scrivono come riduzione, quindi

$$\rm Pt\,/\,Fe^{2+}, \; Fe^{3+} \implies \ce{Fe^{3+} + e^- <--> Fe^{2+}} $$

$$E = E_0 + \frac{0.059}{1} \log \frac{a_{\text{Fe}^{3+}}}{a_{\text{Fe}^{2+}}}$$

$$\rm Pt\,/\,Mn^{2+}, \; MnO_4^- \implies \ce{MnO_4^- + 8H^+ + 5e^- <--> Mn^{2+} + 4H_2O} $$

Questa è la semi-reazione relativa al permanganato. La d.d.p. relativa ad un filo di platino immerso in una soluzione di $\rm MnO_4^-$ e $\rm Mn^{2+}$ sarà

$$E = E_0 + \frac{0.059}{5} \log \frac{a_{\text{MnO}_4^-} \cdot a^8_{\text{H}^+}}{a_{\text{Mn}^{2+}}}$$

Attenzione! Abbiamo scritto d.d.p. ma è corretto dire il potenziale
\subsection{Elettrodo normale standard ad idrogeno}
Finora non siamo stati in grado di misurare i potenziali. Per fare ciò si sceglie un elettrodo e si usa come riferimento. Ad esso si assegna un potenziale arbitrario (nel caso particolare si è assegnato valore zero) e si usa questo come secondo elettrodo per fare le misure di tutti i potenziali di qualunque altro elettrodo.

Quindi la d.d.p. tra elettrodo e soluzione non è misurabile perché dovremmo introdurre un tester che farebbe da secondo elettrodo.

Come riferimento si è scelto l'elettrodo normale standard a idrogeno:

\begin{minipage}{0.55\textwidth}
    \begin{figure}[H]
        \centering
        \includegraphics[width=8cm]{immagini/Elettrodo_a_idrogeno.png}
    \end{figure}
\end{minipage}
\begin{minipage}{0.45\textwidth}
    \vspace{0.7cm}Immaginiamo di avere un beker contenente una soluzione 1-molare di HCl in cui mettiamo un filo di platino saldato ad una lamina di platino, la quale in superficie è ricoperta di una polvere finissima di platino. Tale oggetto si chiama \textit{platino platinato}. Viene rivestito per aumentare la superficie di disposizione, perché si conta la superficie di ogni granello. Facciamo gorgogliare (quindi sotto la lamina) l'idrogeno che è un gas. Esso gorgogliando aderisce alla superficie della lamina, in quanto i grani finissimi di platino che ricoprono la lamina di platino fungono da supporto per le bollicine di idrogeno che vanno ad aderire sulla lamina.
\end{minipage}

\vspace{0.2cm}Perché si chiama standard?

Dato che abbiamo idrogeno che gorgoglia alla pressione di 1 atm e a 25° C, siamo in condizioni standard.

Perché si chiama normale?

La normalità è definita come il numero di equivalenti su un litro di soluzione. Nel caso dell'acido cloridrico che libera un solo protone, molarità e normalità coincidono perchè coincidono moli ed equivalenti, per cui concentrazione 1-molare significa anche concentrazione 1-normale in questo caso.

La reazione che avviene è

$$\ce{2H_3O^+ + 2e^- <--> 2H_2O + H_2}$$

Può anche essere immaginata come due ioni $\rm H^+$ che acquistano 2 elettroni e formano $\rm H_2$. In entrambi i casi è una riduzione.

Il potenziale di questa reazione è

$$E = E_0 + \frac{0.059}{2} \log \frac{a^2_{\text{H}_3\text{O}^+}}{a_{\text{H}_2}}$$

L'attività di un gas che gorgoglia a pressione atmosferica è pari a 1, per cui

$$E = E_0 + 0.059 \log a_{\text{H}_3\text{O}^+}$$

L'attività degli ioni $\rm H_3O^+$ è pari a 1 (perché 1-molare), quindi il logaritmo si azzera e otteniamo $E=E_0$. In queste condizioni il potenziale dell'elettrodo normale standard ad idrogeno è pari a $0.000 V$.

Dunque tutte le volte che vogliamo misurare il potenziale di un qualunque elettrodo, dato che per fare una misura serve un secondo elettrodo useremo questo come riferimento. Ne segue che tutti i potenziali che misureremo saranno dei \textbf{potenziali standard} $E_0$.

In realtà è difficoltoso usare questo elettrodo, perché si deve avere una bombola che eroghi costantemente idrogeno ad 1 atm esatto. Pertanto questo elettrodo è il riferimento teorico ma non quello che si usa praticamente.
\subsection{Elettrodo a calomelano saturo}
\E un elettrodo a mercurio ed è quello che poi viene usato nella pratica. Si chiama così perché è presente il composto $\rm Hg_2Cl_2$, che si chiama calomelano.

Perché non possiamo semplificare i pedici in $\rm Hg_2Cl_2$?

Perché lo ione in soluzione non è $\rm Hg^+$ ma è $\rm Hg_2^{2+}$, cioè ci sono due ioni che stanno assieme formando un dimero.

\begin{minipage}{0.35\textwidth}
    \begin{figure}[H]
        \centering
        \includegraphics[width=5cm]{immagini/Elettrodo_a_calomelano.png}
    \end{figure}
\end{minipage}
\begin{minipage}{0.63\textwidth}
    \vspace{0.4cm}Tale elettrodo è costituito da un bulbo di vetroin cui abbiamo mercurio Hg liquido. Per realizzare il contatto elettrico è saldato al bulbo un filo di platino che tocca il mercurio dentro ed esce fuori. In questo modo il contatto elettrico è assicurato sia dentro che fuori.

    Il metallo deve essere a contatto con una soluzione di un suo sale poco solubile, che è l'$\rm Hg_2Cl_2$. Questo a sua volta è a contatto con un sale con cui ha in comune l'anione, che è il KCl. Quest'ultimo si trova nella gelatina di agar-agar che costituisce il ponte salino. In questo modo in questo elettrodo abbiamo entrambi i collegamenti: il filo metallico e il ponte salino per collegarlo ad un'altra soluzione.
\end{minipage}

\vspace{0.2cm}Ciò che avviene è che l'$\rm Hg_2Cl_2$ si dissocia in $\rm Hg_2^{2+}$ più due ioni cloruro $\rm Cl^-$:

$$\ce{Hg_2Cl_2(s) <--> Hg_2^{2+} + 2Cl^-}$$

Se poi ad $\rm Hg_2^{2+}$ forniamo due elettroni otteniamo due atomi di mercurio:

$$\ce{ Hg_2^{2+} + 2e^- <--> 2Hg}$$

Globalmente la reazione sarà

$$\ce{Hg_2Cl_2(s) + 2e^- <--> 2Hg + 2Cl^-}$$

Il potenziale di questa reazione sarà

$$E=E_0 + \frac{0.059}{2} \log \frac{a_{\text{Hg}_2\text{Cl}_2(s)}}{a^2_{\text{Hg}} \cdot a^2_{\text{Cl}^-}}$$

Il mercurio è un metallo puro, quindi la sua attività è unitaria. Il calomelano saturo è un solido puro e l'attività dei solidi è sempre unitaria. Quindi resta

$$E=E_0 + \frac{0.059}{2} \log \frac{1}{a^2_{\text{Cl}^-}}\quad
\implies E=E_0 + \frac{0.059}{2} \log a^{-2}_{\text{Cl}^-}$$

$$\implies E=E_0 - \frac{0.059}{2} \cdot 2 \log a_{\text{Cl}^-}
\quad
\implies E=E_0 - 0.059 \log a_{\text{Cl}^-}$$

Il potenziale di Nernst di questo elettrodo dipenderà allora dalla concentrazione dello ione cloruro (infatti se misuriamo una d.d.p. tra questo elettrodo e quello normale standard a idrogeno, dato che per definzione il potenziale del secondo è pari a zero, quello che misuriamo è imputabile tutto a questo elettrodo). Abbiamo così scoperto un modo per misurare la concentrazione di uno ione. Se ad esempio vogliamo sapere se dell'acqua è potabile, misurando la d.d.p. è possibile calcolare la concentrazione degli ioni $\rm Cl^-$.

Con questo metodo siamo in grado di costruire ina reazione che ci permette, tramite una misura di d.d.p.\,, di quantificare la presenza (cioè la concentrazione) di uno ione, e ciò per qualunque specie chimica. Gli oggetti che permettono tali misure si chiamano \textit{elettrodi ion-selettivi}. Essi sono sensori elettro-chimici, che una volta immersi leggono la concentrazione dello ione. Ce n'è uno specifico per ogni ione.

Quindi ciò che si deve fare è andare a cercare la reazione giusta, quella più adatta, poi costruiamo un elettrodo il cui potenziale dipenda, tramite questa reazione, da ciò che vogliamo misurare.

La maggior parte dei sensori usa l'elettrochimica.

Nel nostro esempio tale elettrodo ci permette di misurare la concentrazione del KCl. Se questo dovesse essere 1-molare, $a_{\text{Cl}^-}=1$ e allora diventa $E=E_0=0.268 V$

\vspace{0.2cm}Nota: nelle misure di potenziale si usano 3 cifre dopo la virgola.

\vspace{0.2cm}Quindi l'elettrodo di riferimento è quello normale standard a idrogeno, ma quello che poi si usa è l'elettrodo a calomelano saturo che, se costruito in modo tale che l'attività dello ione cloruro sia unitaria avrà un potenziale pari a quello sopra. Esso è semplice da usare perché per toccare qualunque cosa c'è solo un filo di platino e nella sua struttura è già incluso il ponte salino.

Quello che si fa nella pratica è immergere il ponte salino nel beker contenente l'altro elettrodo e col filo di platino tocchiamo il tester. A questo punto poggiamo un puntale sul filo di platino e l'altro sul secondo elettrodo.

Inizialmente non si era pensato di mettere il filo di platino. Al suo posto il bulbo aveva un piccolo foro da cui il mercurio gocciolava lentamente. La goccia che stava per cadere era a contatto con il mercurio dentro e con la soluzione esterna. Tale sistema si chiamava \textit{elettrodo a goccia di mercurio}. La limitazione era che il mercurio si esauriva, quindi si è messo il filo di platino per mantenere il contatto.
\subsection{Elettrodo d'argento}
Simile a quello a calomelano, in esso abbiamo un filo di argento Ag che è a contatto col cloruro di argento, il quale a sua volta è a contatto con cloruro di potassio. La reazione che avviene è
$$\ce{AgCl + 1e^- <--> Ag + Cl^-}$$
Il potenziale sarà
$$E = E_0 -0.059 \log a_{\text{Cl}^-}$$

Anche questo è un elettrodo il cui potenziale dipende dalla concentrazione degli ioni cloruro.

Se questa concentrazione è pari a 1 il potenziale sarà $E=0.2355 V$
\subsection{Elettrodo al chinidrone}
Il chinidrone è formato da benzochinone e idrochinone, forma ossidata e forma ridotta:

\begin{center}
    \begin{tabular}{p{2.4cm}p{2.5cm}p{2.1cm}p{2cm}}
    \chemfig{*6(-(=O)-=-(=O)-=)} & \vspace{-0.8cm}$\rm + \; 2H^+ \; + \; 2e^-$ & \vspace{-0.9cm} \schemestart \arrow{<=>}      \schemestop &
    \chemfig{*6(=(-OH)-=-(-OH)=-)}\\
    \vspace{0.2cm}\hspace{-0.1cm}Benzochinone & & & \vspace{0.2cm}Idrochinone
    \end{tabular}
\end{center}

Esse sono due polveri bianche che per questo tipo di misure vengono vendute già mescolate insieme in quantità equimolari.

Il potenziale sarà

$$E = E_0 + \frac{0.059}{2} \log \frac{a_{\text{Benzochinone}} \cdot a^2_{\text{H}^+}}{a_{\text{Idrochinone}}}$$

Essendo equimolari, qualunque sia il volume di chinidrone preso avremo le stesse moli di benzochinone e idrochinone, per cui la concentrazione sarà identica, quindi le loro attività si semplificano qualunque sia il volume. Avremo allora

$$E = E_0 + 0.059 \log a_{\text{H}^+}$$

Il logaritmo ricorda molto il pH, che può essere calcolato se scriviamo $E$ come

$$E = E_0 - 0.059 \log \left( \frac{1}{a_{\text{H}^+}} \right)$$

Vediamo adesso una tabella di valori dei potenziali per i vari elettrodi. Il riferimento è l'elettrodo a idrogeno, per il quale la reazione di riduzione ha potenziale di riduzione $E_0$ pari a 0
\begin{center}
    \scriptsize\begin{tabular}{|ll|ll|}
        \hline
        &&&\\
        Reazione & $E_0 \; (V)$ & Reazione & $E_0 \; (V)$\\
        &&&\\
        \hline
        &&&\\
        \ce{Li^+ + e^- <--> Li} & -3.045 & \ce{S_4O_6^{2-} <--> 2S_2O_3^{2-}} & 0.10\\[0.7ex]
        \ce{K^+ + e^- <--> K} & -3.924 & \ce{S + 2H_3O^+ + 2e^- <--> H_2S + 2H_2O} & 0.14\\[0.7ex]
        \ce{Ca^{2+} + 2e <--> Ca} & -2.76 & \ce{Sn^{4+} 2e^- <--> Sn^{2+} (HCl \; 1F)} & 0.15\\[0.7ex]
        \ce{Na^+ + e^- <--> Na} & -2.7109 & \ce{Cu^{2+} + e^- <--> Cu^+} & 0.158\\[0.7ex]
        \ce{Mg^{2+} + 2e^- <--> Mg} & -2.375 & \ce{Hg_2Cl_2 + 2e^- <--> 2Hg + 2Cl^-} & 0.2682\\[0.7ex]
        \ce{H_3O^+ + e^- <--> H_2O + H} & -2.10 & \ce{Cu^{2+} + 2e^- <--> Cu} & 0.337\\[0.7ex]
        \ce{Al^{3+} + 3e^- <--> Al} & -1.71 & \ce{O_2 + 2H_2O + 4e^- <--> 4OH^-} & 0.401\\[0.7ex]
        \ce{Ti^{2+} + 2e^- <--> Ti} & -1.63 & \ce{Cu^+ + e^- <--> Cu} & 0.521 \\[0.7ex]
        \ce{ZnO_2^{2-} + 2H_2O + 2e^- <--> Zn + 4OH^-} & -1.22 & \ce{I_2 + 2e^- <--> 2I^-} & 0.536 \\[0.7ex]
        \ce{Mn^{2+} + 2e^- <--> Mn} & -1.03 & \ce{O_2 + 2H_3O^+ + 2e^- <--> H_2O_2 + 2H_2O} & 0.682 \\[0.7ex]
        \ce{2H_2O + 2e^- <--> H_2 + 2OH^-} & -0.828 & \ce{Fe^{3+} + e^- <--> Fe^{2+}} & 0.771\\[0.7ex]
        \ce{Zn^{2+} + 2e^- <--> Zn} & -0.7628 & \ce{Hg_2^{2+} + 2e^- <--> 2Hg} & 0.7961 \\[0.7ex]
        \ce{Cr^{3+} + 3e^- <--> Cr} & -0.74 & \ce{Ag + e^- <--> Ag} & 0.7996\\[0.7ex]
        \ce{Te + 2H_3O^+ +2e^- <--> H_2Te + 2H_2O} & -0.72 & \ce{2NO_3^- + 4H_3O^+ + 2e^- <--> N_2O_4 + 6H_2O} & 0.80\\[0.7ex]
        \ce{As + 3H_3O^+ + 3e^- <--> AsH_3 + 3H_2O} & -0.60& \ce{NO_3^- + 3H_3O^+ + 2e^- <--> HNO_2 + 4H_2O} & 0.94\\[0.7ex]
        \ce{Cr^{2+} + 2e^- <--> Cr} & -0.557 & \ce{NO_3^- + 4H_3O^+ + 3e^- <--> NO + 6H_2O} & 0.96\\[0.7ex]
        \ce{H_3PO_2 + H_3O^+ + e^- <--> P + 3H_2O} & -0.51 & \ce{} & 1.087\\[0.7ex]
        \ce{Fe^{2+} + 2e^- <--> Fe} & -0.409& \ce{} & 1.2\\[0.7ex]
        \ce{Cr^{3+} + e^- <--> Cr^{2+}} & -0.41 & \ce{} & 1.21\\[0.7ex]
        \ce{Cd^{2+} + 2e^- <--> Cd} & -0.4026 & \ce{} & 1.229\\[0.7ex]
        \ce{Se + 2H_3O^+ + 2e^- <--> H_2Se + 2H_2O} & -0.40 & \ce{} & 1.33 \\[0.7ex]
        \ce{Tl^+ + e^- <--> Tl} & -0.3363 & \ce{} & 1.358 \\[0.7ex]
        \ce{Co^{2+} +2e^- <--> Co} & -0.277 & \ce{} & 1.45\\[0.7ex]
        \ce{Ni^{2+} + 2e^- <--> Ni} & -0.230 & \ce{} & 1.455\\[0.7ex]
        \ce{N_2 + 5H_3O^+ -2e^- <--> N_2H_5^+ + 5H_2O} & -0.23 & \ce{MnO_4^- + 8H_3O^+ +5e^- <--> Mn^{2+} + 12H_2O} & 1.50\\[0.7ex]
        \ce{Sn^{2+} + 2e^- <--> Sn} & -0-1364 & \ce{HClO + H_3O^+ + e^- <--> \frac{1}{2} Cl_2 + 2H_2O} & 1.63\\[0.7ex]
        \ce{Pb + 2e^- <--> Pb} & -0.1263 & \ce{H_2O_2 + 2H_3O^+ + 2e^- <--> 4H_2O} & 1.776\\[0.7ex]
        \ce{2H_3O^+ + 2e^- <--> H_2 + 2H_2O} & 0.000 & \ce{Co^{3+} + e^- <--> Co^{2+} (HNO_3 \; 3F)} & 1.842\\[0.7ex]
        \ce{NO_3^- + H_2O + 2e^- <--> NO_2^- + 2OH^-} & 0.0 & \ce{F_2 + 2e <--> 2F^-} & 2.87\\[0.7ex]
        \hline
    \end{tabular}
\end{center}
\normalsize


Quindi si riducono le speci che hanno il potenziale standard maggiore. Tutti gli elementi a potenziale negativo non vedranno mai sviluppata la loro riduzione in soluzione, al loro posto sarà lo ione $\rm H^+$ a ridursi in idrogeno: \ce{2H^+ <--> H_2}.

\textbf{ES.1}
$$\ce{2Fe^{2+} + Sn^{4+} <--> Sn^{2+} + 2Fe^{3+}}$$

$$E_0 \quad \text{Sn}^{4+}/ \, \text{Sn}^{2+}=0.15 V$$

$$E_0 \quad \text{Fe}^{3+}/ \, \text{Fe}^{2+}=0.77 V$$

$$\ce{Sn^{2+} + 2Fe^{3+} <--> Sn^{4+} + 2Fe^{2+}}$$

\textbf{ES.2}

\textbf{ES.3}

\subsubsection{Excursus: la placcatura}
Placcare qualcosa in un certo metallo significa avere una solzione di questo in cui immergiamo l'oggetto che vogliamo ricoprire e alla quale applichiamo una d.d.p. in modo tale che il metallo si depositi come un film sottile omogeneo.

Quali metalli possiamo far depositare in acqua? Tutti quelli che hanno il potenziali maggiore di zero, perché altrimenti anziché ridursi il metallo gli elettroni saranno presi dall'idrogeno piuttosto che dagli ioni metallici.
