\subsection{Schematismo}
\begin{center}
    \begin{tabular}{p{2cm}|p{2cm}||p{2cm}|p{2cm}}
        \vspace{0.3cm}elettrodo 1 & soluzione 1 & soluzione 2 & \vspace{0.3cm}elettrodo 2\\
        & attività 1 & attività 2 &
    \end{tabular}
\end{center}

\begin{center}
    \begin{tabular}{p{2.3cm}|p{2cm}||p{1.8cm}|p{2cm}}
        Pt($\rm H_2$) & $\rm H_3O^+$ & $\rm Ag^+$ & Ag\\[0.5ex]
        P$\rm _{H_2}$=1 atm & $a_{\text{H}_3\text{O}^+}=1$ & $a_{\text{Ag}^+}=x$&\\[0.5ex]
    \end{tabular}
\end{center}

\begin{center}
    \begin{tabular}{p{0.6cm}|p{1.6cm}||p{1.6cm}|p{2cm}}
        Pt & $\rm Sn^{2+}$ $a_1$ & $\rm Fe^{2+}$ $a_3$ & Pt\\[0.5ex]
         & $\rm Sn^{4+}$ $a_2$ & $\rm Fe^{3+}$ $a_4$&\\[0.5ex]
    \end{tabular}
\end{center}

$$\ce{ox_A + ne^- <--> rid_A}$$

$$\ce{ox_B + ne^- <--> rid_B}$$
\subsection{Reazioni di disproporzione}
La disproporzione è una reazione in cui la stessa specie chimica si ossida e si riduce. Un esempio è il cloro in acqua usato come agente disinfettante: esso diventa ipoclorito (ossidandosi) e cloruro (riducendosi) a basse temperature, mentre ad alte temperature diventa clorato (ossidandosi) e cloruro (riducendosi).

Vediamo degli esempi:

\vspace{0.2cm}$\bullet$\textbf{ES.1}

\begin{center}
    \begin{tabular}{p{5.8cm}}
         \hspace{-0.6cm}$\begin{cases}
         \ce{Fe^{2+} + 2e^- -> Fe}, \; E_0= -0.41V\\
         \ce{Fe^{2+} -> Fe^{3+} + e^-}, \; E_0= 0.77V
         \end{cases}$\\
         \\[-1.5ex]
         \hline
         \\[-1.5ex]
         \hspace{-0.2cm}\ce{3Fe^{2+} -> 2Fe^{3+} + Fe}
    \end{tabular}
    \end{center}

Immaginiamo di avere $\rm Fe^{2+}$. Esso può acquistare 2 elettroni e diventare $\rm Fe^0$ o può perdere un elettrone e diventare $\rm Fe^{3+}$ (i suoi stati di ossidazione sono 0, +2 e +3).

Se quindi abbiamo il $\rm Fe^{2+}$ che è l'intermedio possiamo passare da uno stato all'altro.

Se mettiamo $\rm Fe^{2+}$ in acqua creiamo disproporzione di questo?

Per capirlo andiamo a vedere i potenziali delle due diverse reazioni: quella di riduzione da $\rm Fe^{2+}$ a $\rm Fe^0$ e quella di ossi da $\rm Fe^{2+}$ a $\rm Fe^{3+}$. Siccome l'$E_0$ della reazione di riduzione è minore di quello della reazione di ossidazione, la disproporzione NON avviene.

\vspace{0.2cm}$\bullet$\textbf{ES.2}

\begin{center}
    \begin{tabular}{p{5.7cm}}
        \hspace{-0.6cm}$\begin{cases}
        \ce{Cu^+ + e^- -> Cu}, \; E_0= 0.52V\\
        \ce{Cu^+ -> Cu^{2+} + e^-}, \; E_0= 0.15V
        \end{cases}$\\
        \\[-1.5ex]
        \hline
        \\[-1.5ex]
        \hspace{-0.2cm}\ce{2Cu^+ -> Cu + Cu^{2+}}
    \end{tabular}
\end{center}

\subsection{Determinazione del pH di una soluzione}