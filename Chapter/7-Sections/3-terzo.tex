\subsection{Esempi}
Reazioni in cui mettiamo a reagire un acido e una base.

Esempi:

\vspace{0.2cm}\ce{HCl(aq) + NH_3(aq) <--> NH_4^+(aq) + Cl^-(aq)}

\begin{center}
    \begin{tabular}{llllllll}
        \textbf{Nome} & \textbf{Acido 1} & & \textbf{Base 2} & & \textbf{Base 1} & & \textbf{Acido 2}\\[0.3ex]
        Acido cloridico & HCl & + & $\rm H_2O$ & \ce{<-->} & $\rm Cl^-$ & + & $\rm H_3O^+$\\[0.3ex]
        Acido nitrico & $\rm HNO_3$ & + & $\rm H_2O$ & \ce{<-->} & $\rm NO_3^-$ & + & $\rm H_3O^+$\\[0.3ex]
        Acido carbonico & $\rm H_2CO_3$ & + & $\rm H_2O$ & \ce{<-->} & $\rm HCO_3^-$ & + & $\rm H_3O^+$\\[0.3ex]
        Acido acetico & $\rm CH_3CO_2H$ & + & $\rm H_2O$ & \ce{<-->} & $\rm CH_3CO_2^-$ & + & $\rm H_3O^+$\\[0.3ex]
        Acido cianidrico & $\rm HCN$ & + & $\rm H_2O$ & \ce{<-->} & $\rm CN^-$ & + & $\rm H_3O^+$\\[0.3ex]
        Acido solfidrico & $\rm H_2S$ & + & $\rm H_2O$ & \ce{<-->} & $\rm HS^-$ & + & $\rm H_3O^+$\\[0.3ex]
        Ammoniaca & $\rm H_2O$ & + & $\rm NH_3$ & \ce{<-->} & $\rm OH^-$ & + & $\rm NH_4^+$\\[0.3ex]
        Ione carbonato & $\rm H_2O$ & + & $\rm CO_3^{2-}$ & \ce{<-->} & $\rm OH^-$ & + & $\rm HCO_3^-$\\[0.3ex]
        Acqua & $\rm H_2O$ & + & $\rm H_2O$ & \ce{<-->} & $\rm OH^-$ & + & $\rm H_3O^+$\\[0.3ex]
    \end{tabular}
\end{center}
\subsection{Forza degli acidi e delle basi}
\textbf{ES.}

$$\ce{HCOOH(aq) + H_2O(l) <--> HCOO^-(aq) + H_3O^+(aq)}$$

L'acido formico in acqua dà lo ione formiato, e il protone che si libera si associa all'acqua per formare $\rm H_3O^+$.

$$k= \rm{\frac{[HCOO^-]\cdot[H_3O^+]}{[HCOOH]}} = \frac{\alpha^2}{1- \alpha}\textit{c}$$
\subsection{Autodissociazione dell'acqua}
Abbiamo visto la reazione

$$\ce{H_2O + H_2O <--> H_3O^+ + OH^-}$$

In essa stiamo immaginando di avere soltanto acqua a disposizione. Ciononostante l'acqua si autodissocia producendo, tramite un equilibrio, un po' di ioni $\rm H_3O^+$ e un po' di ioni $\rm OH^-$.

Se così fosse l'acqua pura dovrebbe condurre. Con una misura di conducibilità possiamo risalire alla concentrazione di questi ioni. Vedremo che essa è molto piccola, per cui l'autodissociazione dell'acqua sarà trascurabile.

La costante è data da

$$\ce{2 H_2O <--> H_3O^+ + OH^-}
\implies
k= \frac{[\text{H}_3\text{O}^+] [\text{OH}^-]}{[\text{H}_2\text{O}]^2} \; (=1.8 \cdot 10^{-16})$$

$$\implies k \cdot [\text{H}_2\text{O}]^2 = [\text{H}_3\text{O}^+] \cdot [\text{OH}^-] = k_w$$

Poniamo $k_w$ il prodotto di $k$ per la concentrazione dell'acqua all'equilibrio al quadrato, che si vede essere pari anche al prodotto della concentrazione dello ione $\rm H_3O^+$ per quella dello ione $\rm OH^-$. Inoltre quest'ultime due concentrazioni saranno uguali, perché ogni due molecole di acqua si producono uno ione $\rm H_3O^+$ e uno ione $\rm OH^-$. Si trova infatti che entrambe sono pari a $10^{-7}$ mol/L.

Va da notare che se le quantità degli ioni sono uguali, non ci sarà un eccesso di $\rm H_3O^+$ che porta l'acqua ad essere acida né un eccesso di ioni $\rm OH^-$ che porta l'acqua ad essere basica: l'acqua è neutra, poiché questi ioni hanno stessa concentrazione.

$$k_w = 10^{-7} \, (mol/L) \cdot 10^{-7} \, (mol/L) = 10^{-14}$$

$$\log \left( \frac{1}{k_w} \right) = \log \left( \frac{1}{\rm{[H_3O^+]}} \right) + \log \left( \frac{1}{\rm{[OH^-]}} \right) = \log \frac{1}{10^{-14}}=14$$

$$\text{p}k = \log \frac{1}{k}; \quad \rm pH = \log \frac{1}{[H_3O^+]}; \quad pOH = \log \frac{1}{[OH^-]}; \quad$$

$$\implies \rm pH + pOH = 1$$

\subsubsection{Definizioni: pH e P$\boldsymbol{k_a}$}

\subsubsection{Definizioni: pOH e P$\boldsymbol{k_w}$}

\subsubsection{Definizioni: P$\boldsymbol{k_a}$ vs P$\boldsymbol{k_b}$}

\ce{HA <--> H^+ + A^-}

$$k_a = \left( \rm \frac{[H^+] \cdot [A^-]}{[HA]} \right)$$

\ce{A^- + H_2O <--> HA + OH^-}

$$k_b = \left( \rm \frac{[OH^-] \cdot [HA]}{[A^-]} \right)$$

$$k_a \cdot k_b = \left( \rm \frac{[H^+] \cdot [A^-]}{[HA]} \right) \cdot \left( \rm \frac{[OH^-] \cdot [HA]}{[A^-]} \right)$$

$$\implies k_a \cdot k_b = \rm [H^+] \cdot [OH^-] \equiv \textit{k}_\textit{w}$$