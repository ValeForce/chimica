La titolazione è una tipica operazione dell'analisi chimica che consiste nel determinare la concentrazione o \textbf{titolo} di una specie chimica in soluzione, facendola reagire con una quantità nota di un dato reagente, detto \textbf{titolante}.
\subsection{Titolazione acido forte-base forte}

$$\ce{HCl(aq) + NaOH(aq) -> NaCl(aq) + H_2O}$$

\begin{center}
    \begin{tabular}{ccccccc}
        0.01 &  & 0.001 & & / & &\\
        HCl & + & NaOH & \ce{->} & NaCl & + & $\rm H_2O$\\
        0.009 &  &  / & & 0.001 & &\\
    \end{tabular}
\end{center}

\begin{center}
    \begin{tabular}{ccccccc}
        0.01 &  & 0.002 & & / & &\\
        HCl & + & NaOH & \ce{->} & NaCl & + & $\rm H_2O$\\
        0.008 &  &  / & & 0.002 & &\\
    \end{tabular}
\end{center}

\begin{center}
    \begin{tabular}{ccccccc}
        0.01 &  & 0.004 & & / & &\\
        HCl & + & NaOH & \ce{->} & NaCl & + & $\rm H_2O$\\
        0.006 &  &  / & & 0.004 & &\\
    \end{tabular}
\end{center}

\begin{center}
    \begin{tabular}{ccccccc}
        0.01 &  & 0.005 & & / & &\\
        HCl & + & NaOH & \ce{->} & NaCl & + & $\rm H_2O$\\
        0.005 &  &  / & & 0.005 & &\\
    \end{tabular}
\end{center}

\begin{center}
    \begin{tabular}{ccccccc}
        0.01 &  & 0.008 & & / & &\\
        HCl & + & NaOH & \ce{->} & NaCl & + & $\rm H_2O$\\
        0.002 &  &  / & & 0.008 & &\\
    \end{tabular}
\end{center}

\begin{center}
    \begin{tabular}{ccccccc}
        0.01 &  & 0.009 & & / & &\\
        HCl & + & NaOH & \ce{->} & NaCl & + & $\rm H_2O$\\
        0.001 &  &  / & & 0.009 & &\\
    \end{tabular}
\end{center}

\begin{center}
    \begin{tabular}{ccccccc}
        0.01 &  & $9.5 \cdot 10^{-3}$  & & / & &\\
        HCl & + & NaOH & \ce{->} & NaCl & + & $\rm H_2O$\\
        $5 \cdot 10^{-4}$ &  &  / & & $9.5 \cdot 10^{-3}$ & &\\
    \end{tabular}
\end{center}

\begin{center}
    \begin{tabular}{ccccccc}
        0.01 &  & $9.9 \cdot 10^{-3}$  & & / & &\\
        HCl & + & NaOH & \ce{->} & NaCl & + & $\rm H_2O$\\
        $1 \cdot 10^{-4}$ &  &  / & & $9.9 \cdot 10^{-3}$ & &\\
    \end{tabular}
\end{center}

\begin{center}
    \begin{tabular}{ccccccc}
        0.01 &  & $9.99 \cdot 10^{-3}$  & & / & &\\
        HCl & + & NaOH & \ce{->} & NaCl & + & $\rm H_2O$\\
        $1 \cdot 10^{-5}$ &  &  / & & $9.99 \cdot 10^{-3}$ & &\\
    \end{tabular}
\end{center}
\begin{center}
    \begin{tabular}{ccccccc}
        0.01 &  & $9.95 \cdot 10^{-3}$  & & / & &\\
        HCl & + & NaOH & \ce{->} & NaCl & + & $\rm H_2O$\\
        $5 \cdot 10^{-5}$ &  &  / & & $9.95 \cdot 10^{-3}$ & &\\
    \end{tabular}
\end{center}

\begin{center}
    \begin{tabular}{|p{1.5cm}|p{1.5cm}||p{1.5cm}||p{1.5cm}|}
        \textbf{pH} & \textbf{mL} & \textbf{pH} & \textbf{mL}\\
        1 & 0 & 2.59 & 95\\
        1.09 & 10 & 3.30 & 99\\
        1.18 & 20 & 3.60 & 99.5\\
        1.37 & 40 & 4.30 & 99.9\\
        1.48 & 50 & 4.60 & 99.95\\
        1.95 & 80 & 7 & 100\\
        2.28 & 90 &&\\
    \end{tabular}
\end{center}

\subsection{Acidi deboli}

$$\ce{CH_3COOH(aq) + H_2O <--> CH_3COO^-(aq) + H_3O^+(aq)}$$

$$k_a=\frac{\rm{[CH_3COO^-]} \cdot \rm{[H_3O^+]}}{\rm{[CH_3COOH]}}$$

$$k_a=\frac{\rm{[H_3O^+]^2}}{\rm{[CH_3COOH]}}$$

\begin{center}
    \begin{tabular}{ccccccc}
        1 &  &  & & / & &\\
        $\rm CH_3COOH(aq)$ & + & $\rm H_2O$ & \ce{<-->} & $\rm CH_3COO^-(aq)$ & + & $\rm H_3O^+$\\
        $1-x$ & & & & $x$ & & $x$\\
    \end{tabular}
\end{center}

$$k_a = \frac{x^2 \, c}{(1-x)c}= c \frac{x^2}{(1-x)}$$

$$[\text{H}_3\text{O}^+]^2= k_a \cdot c_a$$

$$\implies [\text{H}_3\text{O}^+] = \sqrt{k_a \cdot c_a}$$

\subsection{Basi deboli}

$$\ce{NH_4OH <--> NH_4^+ + OH^-}$$

$$\ce{NH_3 \cdot H_2O <--> NH_4^+ + OH^-}$$

$$k_b= \frac{\rm{[NH_4^+] \cdot [OH^-]}}{\rm{[NH_3 \cdot H_2O]}}$$

$$k_b= \frac{\rm{[OH^-]^2}}{\rm{[NH_3 \cdot H_2O]}}$$

$$\implies \rm [OH^-] = \sqrt{ \textit{k}_{\textit{b}} \cdot [NH_3 \cdot H_2O]}$$

$$\implies  [\text{OH}^-] = \sqrt{ k_b \cdot c_b}$$
\subsection{Reazioni di idrolisi}

$$\ce{NaCl(aq) -> Na^+(aq) + Cl^-(aq)}$$

$$\ce{CH_3COONa -> CH_3COO^-(aq) + Na^+(aq)}$$

$$\ce{CH_3COO^-(aq) + H_2O <--> CH_3COOH(aq) + OH^-(aq)}$$

$$k_i = \rm{\frac{[CH_3COOH] [OH^-]}{[CH_3COO^-]}}$$

$$k_i = \rm{\frac{[CH_3COOH] [OH^-]}{[CH_3COO^-]} \cdot \frac{[H_3O^+]}{[H_3O^+]}}$$

$$k_i = \frac{k_w}{k_a}$$

$$k_i = \rm{\frac{[OH^-]^2}{gg}}$$

\subsubsection{Calcolare le moli conoscendo il volume ma non la massa}

\subsection{Soluzioni tampone}

$$\rm{[H_3O^+]}=\textit{k}_\textit{a} \frac{c_a + \rm{[H_3O^+]}}{c_s - \rm{[H_3O^+]}}$$

$$\rm{[H_3O^+]}=\textit{k}_a \frac{c_a - \rm{[OH^-]}}{c_s + \rm{[OH^-]}}$$

\subsection{Titolazione acido debole-base forte}

\subsection{Titolazione base debole-acido forte}

\subsection{Indicatori}

$$\ce{HIn(aq) + H_2O <--> H_3O^+ + In^-}$$

$$\implies k_{In}=\frac{\rm{[H_3O^+]} \cdot \rm{[In^-]}}{\rm{[HIn]}}$$

$$\rm{[H_3O^+]}=\textit{k}_{\textit{In}} \frac{\rm{[HIn]}}{\rm{[In^-]}}$$

$$\implies \rm pH = log \left( \frac{1}{[H_3O^+]} \right) = log \left( \frac{1}{\textit{k}_{\textit{In}}} \right) + log \left( \frac{\rm{[In^-]}}{\rm{[HIn]}} \right)$$

$$\implies \rm pH= p\textit{k}_\textit{a} + log \left( \frac{\rm{[In^-]}}{\rm{[HIn]}} \right)$$
\subsection{Solubilità}