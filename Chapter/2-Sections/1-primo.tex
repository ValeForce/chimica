\subsection{Dubbi esistenziali}
Su scala macroscopica la materia è continua, ma per spiegare alcune proprietà dobbiamo assumere sia discontinua, costituita da atomi, i quali possono avere carica e quindi esistono gli ioni. Inoltre esistono le molecole.

Cerchiamo ora di spiegare:
\begin{enumerate}
    \item Come sono fatti gli atomi;
    \item Perché e come si legano insieme producendo molecole;
    \item Perché abbiamo atomi che portano cariche se fondamentalmente dovrebbero essere neutri;
    \item Se quando formiamo molecole, esse abbiano o meno relazioni con gli atomi costituenti (cioè se hanno proprietà che dipendono da essi).
  \end{enumerate}
  
  \subsection{Onde elettromagnetiche}
Le onde elettromagnetiche sono costituite da due vettori: campo elettrico $\vec{E}$ e campo magnetico $\vec{H}$, perpendicolari l'un l'altro che oscillano nel tempo. La velocità dell'onda è la velocità della luce $c$.

\textbf{DEF} Ampiezza: altezza massima di un'onda rispetto alla direzione di propagazione. 

\textbf{DEF} Frequenza ($\nu$): numero di oscillazioni per unità di tempo. È inversamente proporzionale alla lunghezza d'onda ($\lambda$): $\lambda\nu=c$, cioè $\lambda\nu=c/\nu$

\subsection{La luce: onda o corpuscolo?}
Nella fisica classica (?) moto ondulatorio e moto dei corpi sono due teorie a sè stanti, ognuna con le proprie leggi.
In particolare per i fenomeni ondulatori esiste il fenomeno dell'interferenza, che consiste nella sovrapposizione di due onde.
Per la luce ci sono due ipotesi:

$\bullet$ La teoria di Newton: la luce ha natura corpuscolare, cioè è formata da fotoni;

$\bullet$ La teoria di Huygens: la luce è un fenomeno ondulatorio, priva di massa e dotata solo di energia.

\subsection{Fenomeni che supportano la natura ondulatoria della luce (e della materia)}
$\bullet$ Diffrazione: è un fenomeno associato alla deviazione della traiettoria di propagazione delle onde quando queste incontrano un ostacolo sul loro cammino.

$\bullet$ Interferenza: due o più onde elettromagnetiche si sovrappongono in un punto dello spazio in modo costruttivo (si intensificano) o distruttivo (si indeboliscono fin quando si annullano a vicenda).

$\bullet$ Riflessione: un'onda che si propaga lungo l'interfaccia tra differenti mezzi, cambia direzione a causa di un impatto con un materiale riflettente.

$\bullet$ Rifrazione: deviazione subita da un'onda che ha luogo quando questa passa da un mezzo ad un altro otticamente differenti nel quale la sua velocità di propagazione cambia.

\subsection{Esperimento di Young (1801)}
Rafforza la teoria di Huygens. Si pone una sorgente luminosa dietro una superficie con una fenditura. La fenditura diventa a sua volta sorgente di fronti d'onda sferici (diffrazione). Oltre questa superficie ce n'è un'altra con 2 fenditure che a loro volta diventano sorgenti di fronti d'onda sferici. Proiettando il risultato su uno schermo si ottengono zone luminose e zone buie alternate. Se la luce da origine alla diffrazione, allora deve avere natura ondulatoria e non corpuscolare.

\subsection{Conseguenze della luce come onda}
Se la luce è un'onda, allora la sua energia è proporzionale a $E^2$ ed $H^2$, cioè dipende dalla ampiezza e non dalla sua frequenza. 

Ciò fu messo in crisi dalle evidenze sperimentali sulla radiazione emessa da un corpo caldo: quando si riscalda un corpo, esso emetterà radiazione \comment{(per raffreddarsi) mi pare inutile scriverlo fai te pd negridimerda} con frequenza che \textbf{dipende} dalla sua temperatura e dalla sua composizione chimica. Ma ciò non dovrebbe accadere se l'energia non dipende dalla frequenza! Dunque la temperatura del corpo (e quindi la sua energia) è legata alla frequenza della radiazione che emette. 

Un corpo nero è un oggetto ideale capace di assorbire radiazione incidente di qualunque frequenza\footnote{Per conservazione dell'energia emette anche radiazione a tutte le frequenze.}. La migliore approssimazione reale di corpo nero è una sfera di nerofumo con un piccolo foro in cui inviamo radiazione. Essa verrà riflessa varie volte all'interno riscaldando la sfera. Aumentando la temperatura, gli atomi si eccitano ed emettono radiazione diseccitandosi. Classicamente avremmo dovuto osservare uno spettro di emissione su tutte le frequenze fino all'UV, invece ci sono delle frequenze in cui l'intensità è maggiore: maggiore è la temperatura, più il picco di emissione si sposta a frequenze maggiori. Inoltre secondo la teoria classica l'energia della radiazione elettromagnetica doveva dipendere soltanto dalla sua ampiezza e non dalla frequenza. %forse quest'ultima frase è una ripetizione? Dio porco? scemo chi legge

Il massimo dello spettro è legato alla temperatura dalla legge di Wien 
$$T\cdot\lambda_{max}= \text{cost}= 2.8977\cdot 10^{-3} \text{m K}$$

%oh merda a fine pagina 59 pure tu non chiudi una parentesi, tutto il resto del corso di chimica è solo una parentesi

\subsection{Quantizzazione dell'energia}%devo andare mi sto pisciando addosso