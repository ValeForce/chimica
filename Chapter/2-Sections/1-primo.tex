\subsection{Dubbi esistenziali}
Su scala macroscopica la materia è continua, ma per spiegare alcune proprietà dobbiamo assumere sia discontinua, costituita da atomi, i quali possono avere carica e quindi esistono gli ioni. Inoltre esistono le molecole.

Cerchiamo ora di spiegare:
\begin{enumerate}
    \item Come sono fatti gli atomi;
    \item Perché e come si legano insieme producendo molecole;
    \item Perché abbiamo atomi che portano cariche se fondamentalmente dovrebbero essere neutri;
    \item Se quando formiamo molecole esse abbiano o meno relazioni con gli atomi costituenti (cioè se hanno proprietà che dipendono da essi).
  \end{enumerate}
  
  \subsection{Onde elettromagnetiche}
Le onde elettromagnetiche sono costituite da due vettori: campo elettrico $\vec{E}$ e campo magnetico $\vec{H}$, perpendicolari l'un l'altro che oscillano nel tempo. La velocità dell'onda è la velocità della luce $c$.

\textbf{DEF} Ampiezza: altezza di un'onda rispetto all'asse centrale (è l'altezza massima). 

\textbf{DEF} Frequenza ($\nu$): numero di oscillazioni per unità di tempo. È inversamente proporzionale alla lunghezza d'onda ($\lambda$): $\lambda\nu=c$, cioè $\lambda\nu=c/\nu$

\subsection{La luce: onda o corpuscolo?}
Nella fisica classica moto ondulatorio e moto dei corpi sono due teorie a sè stanti, ognuna con le proprie leggi.
In particolare per i fenomeni ondulatori esiste il fenomeno dell'interferenza, che consiste nella sovrapposizione di due onde.
Per la luce c'erano due ipotesi:

$\bullet$ La teoria di Newton: la luce ha natura corpuscolare, cioè è formata da particelle;

$\bullet$La teoria di Huygens: la luce è un fenomeno ondulatorio, privo di massa e dotato solo di energia.
\subsection{L'esperimento di Young}