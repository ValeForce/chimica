Osserviamo l'andamento di alcune quantità che dipendono dal riempimento con elettroni dei livelli di un atomo.

\subsection{Raggio atomico}

Questa grandezza è in teoria infinita, ma abbiamo deciso di assegnare un raggio ad ogni atomo e dire che il nucleo è sostanzialmente piccolissimo e le dimensioni dell'atomo sono dovute agli orbitali, quindi agli elettroni (in particolar modo a quelli più esterni, che poi sono quelli di valenza). Stiamo inoltre imponendo che la probabilità di trovare l'elettrone sia confinata entro un certo valore che riteniamo accettabile. Fare questo ragionamento ha senso, perché nell'istante in cui avessi una molecola, se essa è formata da due atomi a raggi infiniti, quale sarebbe la distanza di legame tra i due atomi? Sarebbe infinita, ma ciò non è vero, perché noi siamo in grado di misuare le distanze nucleari e le distanze interatomiche nelle molecole. Quindi l'assunzione di confinare le dimensioni dei raggi degli atomi e degli ioni ragionando sul concetto probabilistico è indispensabile, ed è ciò che i parametri metrici delle molecole ci suggeriscono, ossia possiamo fare misure di distanze interatomiche sia in solidi che in sistemi molecolari gassosi, e pertanto l'avere pensato che gli ioni abbiano dimensioni ben precise è più che ragionevole.\\

Si osserva che i raggi degli atomi crescono all'aumentare del numero quantico principale n, in quanto l'energia dell'elettrone cresce in valore assoluto e dunque aumenta la distanza elettrone-nucleo. Quindi scendendo lungo un gruppo il raggio atomico aumenta.

Alcuni elementi non rispettano questa regola, come ad esempio il gallio, la cui anomalia è causata dal fatto che esso è un metallo allo stato liquido, per cui entrano in gioco altri fattori.\\

Lungo i periodi invece si nota esattamente il contrario: le dimensioni dell'atomo diminuiscono. Il motivo è che a differenza dei gruppi dove il fattore preponderante è n, qui lo è l'aumento di carica nucleare. Se la carica nucleare aumenta, ma n rimane uguale, gli elettroni si schermano poco reciprocamente e quindi l'aumento di carica nucleare fa si che le dimensioni dell'atomo diminuiscano.

Andando un gruppo succede l'esatto contrario: i livelli interni schermano molto gli elettroni esterni dalla carica nucleare, per cui questi ultimi non la sentono tutta ma solo una parte, col risultato di essere meno legati e quindi più distanti dal nucleo.

Quindi la carica nucleare:
\begin{itemize}
    \item Si sente molto lungo un periodo perché gli elettroni a parità di n si schermano poco reciprocamente.
    \item Si sente meno lungo un gruppo perché i livelli interni sono pieni (quando si arriva al gas nobile abbiamo riempito tutti i livelli interni), gli elettroni al loro interno schermano tantissimo quelli esterni dalla carica nucleare.
\end{itemize}
\subsection{Raggio degli ioni}
Quanto osservato per i raggi atomici in larga misura vale anche per i raggi degli ioni: crescono lungo un gruppo perché cresce il valore di n, mentre per quello che succede lungo un periodo bisogna ragionare in termini di stato d'ossidazione dato che abbiamo a che fare con ioni, e non ha senso confrontare ad esempio uno ione +1 con uno +2 o +3, dobbiamo confrontare ioni con la stessa carica. Infatti non avrebbe senso confrontare ad esempio l'\ce{N^{3-}} con l'N: il primo ha acquistato elettroni, e risulta quindi più grande.

Quindi quando si va lungo un periodo, per i raggi ionici bisogna fare ragionamenti che tengano conto della carica, mentre per un gruppo il ragionamento è più immediato: lungo esso le dimensioni aumentano così come aumentavano per gli atomi, perché all'interno di un gruppo si ha la stessa carica (Es 1° gruppo $\rightarrow$ un solo elettrone esterno. Se diventano ioni hanno tutti perso l'unico elettrone esterno, quindi sono tutti nella stessa situazione, avranno la configurazione elettronica del gas nobile che li precede, perché avevano un elettrone 2s, 3s, 4s ecc. da atomo, lo hanno perso e quindi hanno acquisito la configurazione dell'atomo che li precede, simile al gas nobile. Quindi ciò che osservo nel gas nobile lo riosservo nella sequenza. Analogamente si ragiona per gli elementi del secondo gruppo se perdono 2 elettroni, i quali avranno configurazione elettronica del gas nobile che li precede. Infine per gli elementi del terzo gruppo, se hanno stato di ossidazione +3 devo osservare lo stesso fenomeno. Se invece lo stato di ossidazione cambia la situazione si complica). Quindi quando si parla di raggi ionici si devono confrontare ioni con la stessa carica.\\

I lantanidi sono elettroluminescenti, infatti gli schermi, i display, sono costruiti con essi.

In essi si osserva che la dimensione dell'atomo diminuisce man mano che si va lungo la serie. Si parla dunque di "\textbf{contrazione lantanidica}", perché interviene un effetto relativistico: la massa di un elettrone aumenta aumenta all'aumentare della sua velocità, man mano che si avvicina alla velocità della luce. Gli elettroni s e p degli elementi "pesanti" hanno aumenti di massa fino al 20\%. Il raggio di un atomo è inversamente proporzionale alla massa di un elettrone.
\subsection{Potenziale di ionizzazione}
Quando due elementi o due composti reagiscono tra loro, ciò che avviene è uno scambio di elettroni (scambio non significa cessione totale, ma in qualche modo c'è una messa in comune, una condivisione di elettroni, ossia cambia l'assetto elettronico degli atomi, si ha una struttura elettronica diversa, pertanto l'idea di perdere o di acquistare elettroni sono due situazioni limite valide solo per sistemi puramente ionici. Ciò è da tenere a mente perché se un elemento ha i suoi elettroni esterni fortemente legati sarà poco reattivo, se si verifica il contrario sarà propenso a reagire.).

In generale con potenziale di ionizzazione indichiamo l'idea concettuale di Einstein: l'energia necessaria per far si che l'elettrone vinca l'attrazione elettrone-nucleo.\\

Oggi misurare il potenziale di ionizzazione è semplice: la misura può essere fatta strappando elettroni e verificando l'energia richiesta, così da costruire la configurazione elettronica in funzione delle varie energie richieste.

Ragioniamo sull'aspetto formale di questo fenomeno: immaginiamo di avere un elemento e vogliamo che esso sia libero, cioè non abbia costrizioni di natura geometrica dovute ad un eventuale solido nel quale esso si trovi, in altre parole consideriamo atomi gassosi in sistemi rarefatti, in modo tale che il generico atomo o molecola sia effettivamente immaginabile come isolato e quindi non sottoposto ad alcun'altra interazione. A quest'atomo inviamo un'energia $h\nu$, e se è sufficiente l'atomo espellerà un elettrone e si caricherà positivamente perché avrà un eccesso di carica positiva:

$$\ce{M ->[$h\nu$] M+ + e-} \qquad \text{Potenziale di prima ionizzazione} \; \ce{(PI_1)}$$
È chiaro che \ce{M+} ha un tempo di vita molto piccolo ed è difficile studiarlo perché è difficile fare misure con un tempo di vita basso, pertanto la probabilità che si abbia una seconda ionizzazione, ossia che inviando un'alta energia questa ionizzi \ce{M+} ad \ce{M^{2+}} è scarsa, ma supponendo che avvenga chiameremo l'energia necessaria potenziale di seconda ionizzazione \ce{PI_2}
$$\ce{M+ -> M^{2+} + e-}$$
E così a crescere: terza, quarta ecc.

Va da notare che nei composti è più facile da ottenere una seconda ionizzazione, infatti si conoscono alcuni di questi che cedono anche 3 elettroni (soprattutto nei composti ionici).

\textbf{ES1}\\

\begin{tabular}{ m{4cm} m{4cm} }
\ce{Li -> Li+ + e-} & \ce{PI_1}\text{=5.39 eV} \\ 
\ce{Li+ -> Li^{2+} + e-} & \ce{PI_2}\text{=50.0 eV}  \\  
\ce{Li^{2+} -> Li^{3+} + e-} & \ce{PI_3}\text{=122.4 eV}
\end{tabular}\\

Il litio sta sotto l'idrogeno, quindi ha gli orbitali 1s totalmente pieni, e un terzo elettrone nell'orbitale 2s, quindi se volessi strappargli questo elettrone esterno, avrei:
$$\ce{Li_{gassoso} ->[$+h\nu$] Li+ + e-}$$
L'energia richiesta non è elevatissima. Lo ione \ce{Li+} avrà la configurazione elettronica dell'elio.

Se volessimo strappare un secondo elettrone al litio, stavolta dovremmo prenderlo dal livello 1s, a numero quantico inferiore, mentre il primo stava nel 2s, che è un orbitale di valenza perché è il più esterno, a numero quantico massimo per il litio. L'orbitale 1s è estremamente più vicino al nucleo, e il valore del potenziale di ionizzazione cambia drasticamente. Ciò dipende dal fatto che non sto strappando più elettroni di valenza ma "elettroni di core", interni, ad un livello a numero quantico inferiore.

Se volessi strappare anche il terzo ed ultimo elettrone del litio, l'energia richiesta aumenta ulteriormente.

È quindi ragionevole pensare che, nei suoi composti, la reattività del litio sia confinata all'esistenza dello ione \ce{Li+}, cioè nelle reazioni chimiche lavoriamo con gli elettroni di valenza, non riusciremo mai a coinvolgere così attivamente gli elettroni interni: essi sentiranno l'intorno chimico, ma non avremo stati di ossiazione che dipendono dal mettere in gioco anche elettroni a numero quantico inferiore, ossia elettroni interni.

\textbf{ES2}\\

\begin{tabular}{ m{4cm} m{4cm} }
 \ce{Na -> Na+ + e-} & \ce{PI_1}\text{=5.12 eV} \\ 
 \ce{Na+ -> Na^{2+} + e-} & \ce{PI_2}\text{=47.05 eV}  \\  
 \ce{Na^{2+} -> Na^{3+} + e-} & \ce{PI_3}\text{=70.70 eV}
\end{tabular}\\
\subsection{Affinità elettronica}

\subsection{Scala di elettronegatività di Pauling}
