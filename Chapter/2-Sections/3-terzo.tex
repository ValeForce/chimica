Osserviamo l'andamento di alcune quantità che dipendono dal riempimento con elettroni dei livelli di un atomo.

\subsection{Raggio atomico}

Questa grandezza è in teoria infinita, ma abbiamo deciso di assegnare un raggio ad ogni atomo e dire che il nucleo è sostanzialmente piccolissimo e le dimensioni dell'atomo sono dovute agli orbitali, quindi agli elettroni (in particolar modo a quelli più esterni, che poi sono quelli di valenza). Stiamo inoltre imponendo che la probabilità di trovare l'elettrone sia confinata entro un certo valore che riteniamo accettabile. Fare questo ragionamento ha senso, perché nell'istante in cui avessi una molecola, se essa è formata da due atomi a raggi infiniti, quale sarebbe la distanza di legame tra i due atomi? Sarebbe infinita, ma ciò non è vero, perché noi siamo in grado di misuare le distanze nucleari e le distanze interatomiche nelle molecole. Quindi l'assunzione di confinare le dimensioni dei raggi degli atomi e degli ioni ragionando sul concetto probabilistico è indispensabile, ed è ciò che i parametri metrici delle molecole ci suggeriscono, ossia possiamo fare misure di distanze interatomiche sia in solidi che in sistemi molecolari gassosi, e pertanto l'avere pensato che gli ioni abbiano dimensioni ben precise è più che ragionevole.\\

Si osserva che i raggi degli atomi crescono all'aumentare del numero quantico principale n, in quanto l'energia dell'elettrone cresce in valore assoluto e dunque aumenta la distanza elettrone-nucleo. Quindi scendendo lungo un gruppo il raggio atomico aumenta.

Alcuni elementi non rispettano questa regola, come ad esempio il gallio, la cui anomalia è causata dal fatto che esso è un metallo allo stato liquido, per cui entrano in gioco altri fattori.\\

Lungo i periodi invece si nota esattamente il contrario: le dimensioni dell'atomo diminuiscono. Il motivo è che a differenza dei gruppi dove il fattore preponderante è n, qui lo è l'aumento di carica nucleare. Se la carica nucleare aumenta, ma n rimane uguale, gli elettroni si schermano poco reciprocamente e quindi l'aumento di carica nucleare fa si che le dimensioni dell'atomo diminuiscano.

Andando un gruppo succede l'esatto contrario: i livelli interni schermano molto gli elettroni esterni dalla carica nucleare, per cui questi ultimi non la sentono tutta ma solo una parte, col risultato di essere meno legati e quindi più distanti dal nucleo.

Quindi la carica nucleare:
\begin{itemize}
    \item Si sente molto lungo un periodo perché gli elettroni a parità di n si schermano poco reciprocamente.
    \item Si sente meno lungo un gruppo perché i livelli interni sono pieni (quando si arriva al gas nobile abbiamo riempito tutti i livelli interni), gli elettroni al loro interno schermano tantissimo quelli esterni dalla carica nucleare.
\end{itemize}
\subsection{Raggio degli ioni}
Quanto osservato per i raggi atomici in larga misura vale anche per i raggi degli ioni: crescono lungo un gruppo perché cresce il valore di n, mentre per quello che succede lungo un periodo bisogna ragionare in termini di stato d'ossidazione dato che abbiamo a che fare con ioni, e non ha senso confrontare ad esempio uno ione +1 con uno +2 o +3, dobbiamo confrontare ioni con la stessa carica.
\subsection{Potenziale di ionizzazione}

\subsection{Affinità elettronica}

\subsection{Scala di elettronegatività di Pauling}
