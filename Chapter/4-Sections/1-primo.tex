\subsection{Gas ideali}
Un gas ideale è costituito da particelle puntiformi, che non abbiano un proprio volume, e contenute in un recipiente cui è praticato il vuoto in modo da avere solo poche particelle del gase e dove abbiamo dei misuratori di temperatura e pressione.

Definiamo un gase "ideale" quando esso si trova a bassa pressione, quindi rarefatto e ad alta temperatura.

In queste condizioni è ragionevole assumere che non esistano interazioni fra le varie particelle, per cui esse si muovono in modo del tutto casuale. Qualora dovessero esserci urti, si assumono essere perfettamente elastici.

Un gas occupa l'intero volume che trova a sua disposizione ed esercita una pressione dovuta
\subsection{Significato molecolare della pressione}
\subsection{Significato molecolare della temperatura}
\subsection{Legge di Boyle}
\subsection{Legge di Charles (Gay-Lussac)}
\subsection{Legge di Gay-Lussac}
\subsection{Legge di Dalton e delle pressioni parziali}
\subsection{Legge dei volumi molare e di Avogadro}
\subsection{Equazione di stato}
\subsection{Determinazione dei pesi molecolari}
\subsection{Gas e vapori}