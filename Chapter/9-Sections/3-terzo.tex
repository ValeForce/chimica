\textbf{3.1}$\bigstar$ Si consideri una soluzione al 10\% di nitrato di calcio $\rm Ca(NO_3)_2$ e densità $1.1\;g/mL$. Calcolarne la molalità $m$.

\vspace{0.2cm}\large\textbf{Svolgimento}\normalsize

\vspace{0.2cm}La molarità è definita come il numero di moli in un chilogramo di solvente puro. La densità pari a $1.1\;g/mL$ ci dice che 1 mL ha una massa uguale a 1.1 grammi, ma solo il 10\% di questo è $\rm Ca(NO_3)_2$, cioè di tale massa 0.11 grammi sono di sale, e questo su 1 mL, cioè 1 mL contiene 0.11 grammi di sale ($\rm Ca(NO_3)_2$ è un sale), 1 litro ne contiene 110 grammi. La soluzione di un litro avrà massa pari a 1100 grammi, di cui 110 grammi sono di sale, per cui avremo un numero di grammi di acqua pari a $1100-110=990$.

A questo punto dobbiamo fare la proporzione per sapere quanti grammi ci sono in 1000 grammi di solvente:

$$110:990=g:1000$$

$$g=\frac{110 \cdot 1000}{990}$$

la molalità sarà data da

$$m=\frac{110 \cdot 1000}{990 \cdot 164.09}=0.677$$

dove le moli sono espresse come grammi diviso peso molecolare.

\vspace{0.2cm}\textbf{3.2}$\bigstar$ Una soluzione è stata preparata sciogliendo 684 grammi di saccarosio ($MM=342$) in 900 grammi di acqua. Determinare le frazioni molari dei due componenti.

\vspace{0.2cm}\large\textbf{Svolgimento}\normalsize

\vspace{0.2cm}Si ha che

$$\rchi_{\text{sacc.}}=\frac{\frac{684}{342}}{\frac{684}{342} + \frac{900}{18}}=\frac{2}{52}=0.038
\quad;\quad
\rchi_{\text{H}_2\text{O}}=\frac{\frac{900}{18}}{\frac{684}{342} + \frac{900}{18}}=\frac{50}{52}=0.962$$

\vspace{0.2cm}Infatti $\rchi_{\text{sacc.}} + \rchi_{\text{H}_2\text{O}}\approx 1$.

\vspace{0.2cm}\textbf{3.3}$\bigstar$ 350 grammi di cloruro di zinco $\rm ZnCl_2$ con densità $d=2.91 g/cm^3$ (attenzione! Questa non è la densità di tutta la soluzione, ma solo del solido) vengono sciolti in 650 grammi di acqua. Si ottiene una soluzione che avrà un volume di 740 mL. Calcolare molarità, normalità, molalità, frazioni molari, percentuali in massa e percentuali in volume.

\vspace{0.2cm}Nota: $d_{\text{H}_3\text{O}}=1$, quindi 650 grammi sono anche 650 mL.

\vspace{0.2cm}\large\textbf{Svolgimento}\normalsize

\vspace{0.2cm}Calcoliamo la molarità

$$M=\frac{350 \cdot 1000}{136.28 \cdot 740}=3.4706\;mol/L$$

(Il fattore 1000 compare perché facciamo una proporzione)

Calcoliamo la normalità. In acqua il sale si scioglie così:

$$\ce{ZnCl_2 -> Zn^{2+} + 2Cl^-}$$

La carica è +2, il peso equivalente sarà la metà del peso molecolare e quindi la normalità sarà il doppio della molarità:

$$N=M \cdot 2=6.941\;eq/L$$

Calcoliamo la molalità

$$m=\frac{350 \cdot 1000}{136.28 \cdot 650}=3.9511\;mol/kg$$
\textbf{Incompleto}

\vspace{0.2cm}\textbf{3.2}$\bigstar$ Consideriamo un composto avente la seguente composizione percentuale: 14.3\% carbonio, 1.2\% idrogeno, 84.5\% cloro. 1 grammo di questo composto viene vaporizzato a $120^{\circ}C$ e alla pressione di 752 torr. In queste condizioni occupa un volume di 194 mL. Calcolare la formula del composto.

\vspace{0.2cm}\large\textbf{Svolgimento}\normalsize

\vspace{0.2cm}Le composizioni percentuali date ci dicono che su 100 grammi di composto avremo 14.3 grammi di C, 1.2 grammi di H e 84.5 grammi di Cl. Vediamo quindi, su 100 grammi di composto, quante sono le moli di carbonio, di idrogeno e di cloro

$$n_{\text{C}}=\frac{14.3}{12}=1.19
\quad;\quad
n_{\text{H}}=\frac{1.2}{1.008}=1.19
\quad;\quad
n_{\text{Cl}}=\frac{84.5}{35.453}=2.38$$

Abbiamo trovato che per 100 grammi di composto le moli di carbonio e idrogeno sono le stesse, mentre quelle di cloro sono pari al doppio. Ne segue che la \textit{formula minima}\footnote{La formula minima è una particolare formula molecolare in cui il numero di atomi di ciascun elemento costituente è ridotto al massimo comun divisore relativo.} del composto è $\rm CHCl_2$.

Il peso molecolare di tale formula è 83.916, ma abbiamo tutti i dati necessari per calcolare il peso molecolare del composto dall'equazione di stato dei gas perfetti:

$$MM=\frac{gRT}{PV} \implies MM=\frac{1 \cdot 0.082 \cdot 393.15 \cdot 760}{752 \cdot 0.194}=167.94$$

A questo punto vediamo quante volte questo peso molecolare contiene il valore del peso molecolare trovato per la formula minima. Nel nostro caso lo contiene 2 volte, ciò significa che il peso molecolare vero è il doppio di quello della formula minima, per cui la formula reale del composto è $\rm C_2H_2Cl_4$, in quanto contiene 2 volte la formula minima.

\vspace{0.2cm}\textbf{3.3}$\bigstar$ Una miscela di 0.278 grammi di etere e 0.568 grammi di cloroformio si trova alla temperatura di $100^{\circ}C$ in un contenitore di volume 1 L in cui era stato preventivamente fatto il vuoto. Calcolare le pressioni parziali dell'etere e del cloroformio.

\vspace{0.2cm}\large\textbf{Svolgimento}\normalsize

\vspace{0.2cm}

\vspace{0.2cm}\textbf{3.4}$\bigstar$ Un gas a 20°C e alla pressione di 730 torr occupa un volume di 20 L. Calcolare

\begin{enumerate}
    \item Il volume occupato quando la pressione diventa pari a 2 atm, mantenendo la temperatura costante;
    \item Il volume occupato in condizioni normali;
    \item La temperatura alla quale si deve portare il gas affinché abbia una pressione di 700 mmHg, mantenendo costante il volume.
\end{enumerate}

\vspace{0.2cm}\large\textbf{Svolgimento}\normalsize

\vspace{0.2cm} 1) Nel primo caso i dati sono

$$T_1=20^{\circ}\;C
\quad;\quad
P_1=730\;torr
\quad;\quad
V_1=20\;L$$
$$T_2=T_1
\quad;\quad
P_2=2\;atm
\quad;\quad
V_2=?$$

Visto che la temperatura è costante, l'equazione di stato diventa

$$P_1V_1=P_2V_2$$

Vogliamo trovare $V_2$, per cui

$$V_2=\frac{P_1V_1}{P_2}
=\frac{730 \cdot 20}{760 \cdot 2}=9.6052\;L$$

2) Nel secondo caso invece i dati sono

$$T_1=20^{\circ}\;C
\quad;\quad
P_1=730\;torr
\quad;\quad
V_1=20\;L$$
$$T_2=0^{\circ}\;C
\quad;\quad
P_2=1\;atm
\quad;\quad
V_2=?$$

L'equazione di stato si scriverà come

$$\frac{730 \cdot 20}{760 \cdot 293.15}=\frac{1 \cdot V_2}{273.15}$$

$$V_2=\frac{730 \cdot 20 \cdot 273.15}{760 \cdot 293.15}=17.9\;L$$

3) I dati sono

$$T_1=20^{\circ}\;C
\quad;\quad
P_1=730\;torr
\quad;\quad
V_1=20\;L$$
$$T_2=?
\quad;\quad
P_2=700\;torr
\quad;\quad
V_2=V_1$$

\vspace{0.2cm}\textbf{3.1} Calcolare la pressione osmotica, in mmHg, di una soluzione acquosa 0,001 M di acido acetico dissociato al 10\% a 25 °C.

\vspace{0.2cm}\large\textbf{Svolgimento}\normalsize

\vspace{0.2cm}

\vspace{0.2cm}\textbf{3.2} Calcolare la pressione osmotica di una soluzione di 923 mL, contenente 4.1234 grammi di idrossido di magnesio alla temperatura di 27°C. 

\vspace{0.2cm}\large\textbf{Svolgimento}\normalsize

\vspace{0.2cm}

\vspace{0.2cm}\textbf{3.3} Calcolare la pressione osmotica di una soluzione di 875 mL, contenente 3.05 grammi di idrossido
di potassio alla temperatura di 27°C.

\vspace{0.2cm}\large\textbf{Svolgimento}\normalsize

\vspace{0.2cm}

\vspace{0.2cm}\textbf{3.4} Calcolare a 25°C la pressione osmotica di una soluzione di un acido debole 0.15 M la cui $k_a=0.17$.

\vspace{0.2cm}\large\textbf{Svolgimento}\normalsize

\vspace{0.2cm}

\vspace{0.2cm}\textbf{3.5} 50.00 mL di una soluzione acquosa di nitrato di argento vengono trattati con un eccesso di acido
cloridrico. Si formano 1.7658 g di cloruro di argento. Calcolare la molarità della soluzione di nitrato
di argento.

\vspace{0.2cm}\large\textbf{Svolgimento}\normalsize

\vspace{0.2cm}

\vspace{0.2cm}\textbf{3.6} In un recipiente di 5 litri sono introdotti 300 mL di cloro misurati a 20°C e 750 torr, 800 mL di azoto misurati a 15°C e 700 torr, 500 mL di ossigeno misurati a 18°C e 850 torr e 450 mL di anidride carbonica misurati a 20°C e 760 torr. Calcolare la pressione totale della miscela a 25°C e la frazione molare dell’ossigeno.

\vspace{0.2cm}\large\textbf{Svolgimento}\normalsize

\vspace{0.2cm}

\vspace{0.2cm}\textbf{3.7} Calcolare a 25°C la pressione osmotica di una soluzione 0.15 M di acido iodico ($k_a= 1.7 \cdot 10-1 $)

\vspace{0.2cm}\large\textbf{Svolgimento}\normalsize

\vspace{0.2cm}

\vspace{0.2cm}\textbf{3.8} In un recipiente di 1 L sono state introdotte 2.5 moli di NOCl. Dopo opportuno riscaldamento ad una fissata temperatura si realizza il seguente equilibrio:

\begin{center}
    \ce{2 NOCl <--> 2NO + Cl_2}
\end{center}

All’equilibrio sono presenti 0.86 moli di NO. Calcolare la relativa costante di equilibrio in funzione delle concentrazioni.

\vspace{0.2cm}\large\textbf{Svolgimento}\normalsize

\vspace{0.2cm}

\vspace{0.2cm}\textbf{3.9} Sia data una soluzione composta da due liquidi A e B. Ad una data temperatura la $P^0_A = 152$ torr
e la $P^0_B = 324$ torr. Calcolare la pressione totale dei vapori di A e B in equilibrio con la soluzione sapendo che la frazione molare di A in soluzione è 0.75.

\vspace{0.2cm}\large\textbf{Svolgimento}\normalsize

\vspace{0.2cm}

\vspace{0.2cm}\textbf{3.10} Calcolare la pressione osmotica di una soluzione di 900 mL, contenente 5.56 grammi di KOH alla temperatura di 26°C.

\vspace{0.2cm}\large\textbf{Svolgimento}\normalsize

\vspace{0.2cm}

\vspace{0.2cm}\textbf{3.11} A 400 mL di una soluzione acquosa di $\rm FeCl_3$ 0.0132-M sono stati aggiunti 350 mg di un composto organico senza variazione apprezzabile di volume. A 25 °C si misura una pressione osmotica di 1072 torr. Calcolare la massa molare del composto organico.

\vspace{0.2cm}\large\textbf{Svolgimento}\normalsize

\vspace{0.2cm}

\vspace{0.2cm}\textbf{3.12} Un volume di 10 L di una miscela gassosa di 80 grammi alla temperatura di 90°C contiene il
30\% in peso di $\rm SO_3$ ed il 70\% di $\rm O_2$. Calcolare le pressioni parziali dei singoli gas.

\large\textbf{Svolgimento}\normalsize

\vspace{0.2cm}

\vspace{0.2cm}\textbf{3.13} In un recipiente di 5.6 litri, alla temperatura di 80°C, vi sono 4,5432 g di metano, 1,2345 g di idrogeno e 14,5678 g di azoto. Calcolare i valori delle pressioni parziali dei tre gas. 

\vspace{0.2cm}\large\textbf{Svolgimento}\normalsize

\vspace{0.2cm}

\vspace{0.2cm}\textbf{3.14} Una soluzione al 5\% in peso di $\rm CaX_2$ (ove X è un alogeno), inizia a congelare a - 1,396°C. $k_cr_{\rm H_2O} = 1,860$. Qual è l'alogeno? 

\vspace{0.2cm}\large\textbf{Svolgimento}\normalsize

\vspace{0.2cm}

\vspace{0.2cm}\textbf{3.15} Calcolare la pressione osmotica di 789 ml di soluzione contenente 5.8440 grammi di NaCl e  21.2996 grammi di $\rm Al(NO_3)_3$ alla temperatura di 27°C.

\vspace{0.2cm}\large\textbf{Svolgimento}\normalsize

\vspace{0.2cm}

\vspace{0.2cm}\textbf{3.16} Calcolare la pressione osmotica di una soluzione di 850 mL contenente 89.1234 grammi di solfato di alluminio alla temperatura di 80 °C.

\vspace{0.2cm}\large\textbf{Svolgimento}\normalsize

\vspace{0.2cm}