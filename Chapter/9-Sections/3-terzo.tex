\textbf{3.1} Calcolare la pressione osmotica, in mm Hg, di una soluzione acquosa 0,001 M di acido acetico dissociato al 10\% a 25 °C.

\vspace{0.2cm}\textbf{3.2} Calcolare la pressione osmotica di una soluzione di 923 mL, contenente 4.1234 grammi di idrossido di magnesio alla temperatura di 27°C. 

\vspace{0.2cm}\textbf{3.3} Calcolare la pressione osmotica di una soluzione di 875 mL, contenente 3.05 grammi di idrossido
di potassio alla temperatura di 27°C.

\vspace{0.2cm}\textbf{3.4} Calcolare a 25°C la pressione osmotica di una soluzione di un acido debole 0.15 M la cui $k_a=0.17$.

\vspace{0.2cm}\textbf{3.5} 50.00 mL di una soluzione acquosa di nitrato di argento vengono trattati con un eccesso di acido
cloridrico. Si formano 1.7658 g di cloruro di argento. Calcolare la molarità della soluzione di nitrato
di argento.

\vspace{0.2cm}\textbf{3.6} In un recipiente di 5 litri sono introdotti 300 mL di cloro misurati a 20°C e 750 torr, 800 mL di azoto misurati a 15°C e 700 torr, 500 mL di ossigeno misurati a 18°C e 850 torr e 450 mL di anidride carbonica misurati a 20°C e 760 torr. Calcolare la pressione totale della miscela a 25°C e la frazione molare dell’ossigeno.

\vspace{0.2cm}\textbf{3.7} Calcolare a 25°C la pressione osmotica di una soluzione 0.15 M di acido iodico ($k_a= 1.7 \cdot 10-1 $)

\vspace{0.2cm}\textbf{3.8} In un recipiente di 1 L sono state introdotte 2.5 moli di NOCl. Dopo opportuno riscaldamento ad una fissata temperatura si realizza il seguente equilibrio:

\begin{center}
    \ce{2 NOCl <--> 2NO + Cl_2}
\end{center}

All’equilibrio sono presenti 0.86 moli di NO. Calcolare la relativa costante di equilibrio in funzione delle concentrazioni.

\vspace{0.2cm}\textbf{3.9} Sia data una soluzione composta da due liquidi A e B. Ad una data temperatura la $P^0_A = 152$ torr
e la $P^0_B = 324$ torr. Calcolare la pressione totale dei vapori di A e B in equilibrio con la soluzione sapendo che la frazione molare di A in soluzione è 0.75.

\vspace{0.2cm}\textbf{3.10} Calcolare la pressione osmotica di una soluzione di 900 mL, contenente 5.56 grammi di KOH alla temperatura di 26°C.

\vspace{0.2cm}\textbf{3.11} A 400 mL di una soluzione acquosa di $\rm FeCl_3$ 0.0132-M sono stati aggiunti 350 mg di un composto organico senza variazione apprezzabile di volume. A 25 °C si misura una pressione osmotica di 1072 torr. Calcolare la massa molare del composto organico.

\vspace{0.2cm}\textbf{3.12} Un volume di 10 L di una miscela gassosa di 80 grammi alla temperatura di 90°C contiene il
30\% in peso di $\rm SO_3$ ed il 70\% di $\rm O_2$. Calcolare le pressioni parziali dei singoli gas.

\vspace{0.2cm}\textbf{3.13}In un recipiente di 5.6 litri, alla temperatura di 80°C, vi sono 4,5432 g di metano, 1,2345 g di idrogeno e 14,5678 g di azoto. Calcolare i valori delle pressioni parziali dei tre gas. 

\vspace{0.2cm}\textbf{3.14} Una soluzione al 5\% in peso di $\rm CaX_2$ (ove X è un alogeno), inizia a congelare a - 1,396°C. $k_cr_{\rm H_2O} = 1,860$. Qual è l'alogeno? 

\vspace{0.2cm}\textbf{3.15} Calcolare la pressione osmotica di 789 ml di soluzione contenente 5.8440 grammi di NaCl e  21.2996 grammi di $\rm Al(NO_3)_3$ alla temperatura di 27°C.

\vspace{0.2cm}\textbf{3.16} Calcolare la pressione osmotica di una soluzione di 850 mL contenente 89.1234 grammi di solfato di alluminio alla temperatura di 80 °C.