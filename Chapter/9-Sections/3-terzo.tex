\begin{esercizio}[$\bigstar$]
    Si consideri una soluzione al 10\% di nitrato di calcio $\rm Ca(NO_3)_2$ e densità 1.1 g/mL. Calcolarne la molalità $m$.
\end{esercizio}
\begin{soluzione}
    La molalità è definita come il numero di moli in un chilogrammo di solvente puro. La densità pari a $1.1 \; \rm g/mL$ ci dice che 1 mL ha una massa uguale a 1.1 grammi, ma solo il 10\% di questo è $\rm Ca(NO_3)_2$, cioè di tale massa 0.11 grammi sono di sale, e questo su 1 mL, cioè 1 mL contiene 0.11 grammi di sale ($\rm Ca(NO_3)_2$ è un sale), per cui 1 litro ne conterrà 110 grammi. La soluzione di 1 litro avrà massa pari a 1100 grammi, di cui 110 grammi sono di sale, per cui avremo un numero di grammi di acqua pari a $1100-110=990$.

A questo punto dobbiamo fare la proporzione per sapere quanti grammi ci sono in 1000 grammi di solvente:

$$110:990=g:1000$$

$$g=\frac{110 \cdot 1000}{990} \; \rm grammi$$

la molalità sarà data da

$$m=\frac{110 \cdot 1000}{990 \cdot 164.09}=0.677 \; \rm mol/kg$$

dove le moli sono espresse come grammi diviso peso molecolare.
\end{soluzione}

\newpage

\begin{esercizio}[$\bigstar$]
    Una soluzione è stata preparata sciogliendo 684 grammi di saccarosio ($MM=342$) in 900 grammi di acqua. Determinare le frazioni molari dei due componenti.
\end{esercizio}
\begin{soluzione}
    Si ha che
    \begin{equation*}
        \rchi_{\rm sacc.}
        =\frac{\displaystyle \frac{684}{342}}{\displaystyle \frac{684}{342} + \frac{900}{18}}
        =\frac{2}{52}=0.038
        \quad;\quad
        \rchi_{\rm H_2O}
        =\frac{\displaystyle \frac{900}{18}}{\displaystyle \frac{684}{342} + \frac{900}{18}}
        =\frac{50}{52}
        =0.962
    \end{equation*}
    Infatti $\rchi_{\rm sacc.} + \rchi_{\rm H_2O}\approx 1$.
\end{soluzione}

\newpage

\begin{esercizio}[$\bigstar$]
    350 grammi di cloruro di zinco $\rm ZnCl_2$ con densità\footnotemark\;$d=2.91 \; \rm g/cm^3$ vengono sciolti in 650 grammi di acqua. Si ottiene una soluzione che avrà un volume di 740 mL. Calcolare molarità, normalità, molalità, frazioni molari, percentuali in massa e percentuali in volume.\\[0.2cm]
    Nota: $d_{\text{H}_2\text{O}}=1$, quindi 650 grammi sono anche 650 mL.
\end{esercizio}
\begin{soluzione}
    \footnotetext{Attenzione! Questa non è la densità di tutta la soluzione, ma solo del solido.}
    Calcoliamo la molarità:

$$M=\frac{350 \cdot 1000}{136.28 \cdot 740}=3.4706 \; \rm mol/L$$

(Il fattore 1000 compare per passare da mL a L)

Calcoliamo la normalità. In acqua il sale si scioglie così:

$$\ce{ZnCl_2 -> Zn^{2+} + 2Cl^-}$$

La carica è +2, il peso equivalente sarà la metà del peso molecolare e quindi la normalità sarà il doppio della molarità:

$$N=M \cdot 2=6.941 \; \rm eq/L$$

Calcoliamo la molalità

$$m=\frac{350 \cdot 1000}{136.28 \cdot 650}=3.9511 \; \rm mol/kg$$

Calcoliamo le frazioni molari

$$\rchi_{\rm ZnCl_2}=\frac{\frac{350}{136.28}}{\frac{350}{136.28} + \frac{650}{18}}
\quad;\quad
\rchi_{\rm H_2O}=\frac{\frac{650}{18}}{\frac{350}{136.28} + \frac{650}{18}}$$

$$\implies
\rchi_{\rm ZnCl_2}=0.0665
\quad;\quad
\rchi_{\rm H_2O}=0.933
$$

Infatti la loro somma è circa 1.

Calcoliamo la percentuale in massa

$$\text{\% in massa}
=\frac{\text{massa sale}}{\text{massa totale}} \cdot 100
=\frac{350 \cdot 100}{350 + 650}=35\%$$

Calcoliamo la percentuale in volume

$$\text{\% in volume}
=\frac{\text{volume sale}}{\text{volume totale}} \cdot 100$$

Il volume del sale va calcolato tramite la densità: $d=2.91 \; \rm g/cm^3$. Il volume sarà dato da\footnote{Ricorda che $\rm 1 \; cm^3 = 1 \; mL$.}

$$V=\frac{m}{d}=\frac{350}{2.91}=120.27 \; \rm mL$$

da cui

$$\text{\% in volume}
=\frac{120.27 \cdot 100}{740}=16.25\%$$
\end{soluzione}

\newpage

\begin{esercizio}[$\bigstar$]
    Consideriamo un composto avente la seguente composizione percentuale: 14.3\% carbonio, 1.2\% idrogeno, 84.5\% cloro. 1 grammo di questo composto viene vaporizzato a $120 \; \rm ^{\circ}C$ e alla pressione di 752 Torr. In queste condizioni occupa un volume di 194 mL. Calcolare la formula del composto.
\end{esercizio}
\begin{soluzione}
    Le composizioni percentuali date ci dicono che su 100 grammi di composto avremo 14.3 grammi di C, 1.2 grammi di H e 84.5 grammi di Cl. Vediamo quindi, su 100 grammi di composto, quante sono le moli di carbonio, di idrogeno e di cloro:

$$n_{\text{C}}=\frac{14.3}{12}=1.19 \; \text{mol}
\quad;\quad
n_{\text{H}}=\frac{1.2}{1.008}=1.19 \; \text{mol}
\quad;\quad
n_{\text{Cl}}=\frac{84.5}{35.453}=2.38 \; \text{mol}$$

Abbiamo trovato che per 100 grammi di composto le moli di carbonio e idrogeno sono le stesse, mentre quelle di cloro sono pari al doppio. Ne segue che la \textit{formula minima}\footnote{La formula minima è una particolare formula molecolare in cui il numero di atomi di ciascun elemento costituente è ridotto al massimo comun divisore relativo.} del composto è $\rm CHCl_2$.

Il peso molecolare di tale formula è 83.916, ma abbiamo tutti i dati necessari per calcolare il peso molecolare del composto dall'equazione di stato dei gas perfetti:

$$MM=\frac{gRT}{PV} \implies MM=\frac{1 \cdot 0.082 \cdot 393.15 \cdot 760}{752 \cdot 0.194}=167.94$$

A questo punto vediamo quante volte questo peso molecolare contiene il valore del peso molecolare trovato per la formula minima. Nel nostro caso lo contiene 2 volte, ciò significa che il peso molecolare vero è il doppio di quello della formula minima, per cui la formula reale del composto è $\rm C_2H_2Cl_4$, in quanto contiene 2 volte la formula minima.

\end{soluzione}

\newpage

\begin{esercizio}[$\bigstar$]
    Una miscela di 0.278 grammi di etere e 0.568 grammi di cloroformio si trova alla temperatura di $100 \; \rm ^{\circ}C$ in un contenitore di volume 1 L in cui era stato preventivamente fatto il vuoto. Calcolare le pressioni parziali dell'etere e del cloroformio.
\end{esercizio}
\begin{soluzione}
    Se siamo in condizioni di idealità ogni composto si comporta come se fosse l'unico presente, quindi possiamo calcolare le pressioni dei singoli composti con l'equazione di stato:

$$PV=nRT$$

$$P_{\rm etere}
=\frac{g}{MM}\frac{RT}{V}
=\frac{0.278}{74}\cdot\frac{0.082 \cdot 373.15}{1}
=0.1187 \; \rm atm$$

$$P_{\rm cloroformio}
=\frac{0.568}{119.5}\cdot\frac{0.082 \cdot 373.15}{1}
=0.1454 \; \rm atm$$

\end{soluzione}

\newpage

\begin{esercizio}[$\bigstar$]
    Quanti litri di solfuro di idrogeno $\rm H_2S$ a condizioni normali ($1 \; \rm atm$ e $20 \; ^{\circ}\rm C$) sono stati necessari per precipitare l'antimonio come solfuro da una soluzione contenente 10 grammi di tricloruro di antimonio $\rm SbCl_3$?
\end{esercizio}
\begin{soluzione}
    La reazione che avviene è
    \begin{equation*}
        \ce{2SbCl_3 + 3H_2S -> Sb_2S_3 + 6HCl}
    \end{equation*}

(si ottiene solfuro di antimonio e acido cloridrico)

Visto che dobbiamo calcolare l'$\rm H_2S$ necessario, bisogna ricordare che i rapporti stechiometrici sono: due moli di $\rm SbCl_3$ reagiscono con tre moli di $\rm H_2S$.

Le moli iniziali di antimonio saranno

$$n_{\rm SbCl_3}=\frac{10}{228.109} \; \rm mol$$

Le moli di $\rm H_2S$ necessarie saranno pari a $3/2$ di quelle di antimonio:

$$n_{\rm H_2S}=\frac{10 \cdot 3}{2 \cdot 228.109} \; \rm mol$$

A condizioni normali una mole di qualunque gas occupa sempre un volume di $22.414 \; \rm L$, per cui per trovare i litri di $\rm H_2S$ basterà moltiplicare tale valore per il numero di moli

$$V_{\rm H_2S}(\text{L})
=\frac{10 \cdot 3}{2 \cdot 228.109} \cdot 22.414
=1.4739 \; \rm L$$

\end{soluzione}

\newpage

\begin{esercizio}[$\bigstar$]
    Un gas a $20 \; ^{\circ} \rm C$ e alla pressione di 730 Torr occupa un volume di 20 L. Calcolare
    \begin{enumerate}
        \item Il volume occupato quando la pressione diventa pari a 2 atm, mantenendo la temperatura costante;
        \item Il volume occupato in condizioni normali;
        \item La temperatura alla quale si deve portare il gas affinché abbia una pressione di 700 mmHg, mantenendo costante il volume.
        \item La pressione esercitata dal gas, mantenendo costante il volume, quando la temperatura diventa $-50 \; ^{\circ} \rm C$.
        \item La temperatura a cui si deve riscaldare il gas perché alla pressione di 650 mmHg il volume sia 26 L.
    \end{enumerate}
\end{esercizio}
\begin{soluzione}
    \begin{enumerate}
        \item I dati sono

    $$T_1=20 \; ^{\circ}\text{C}
    \quad;\quad
    P_1=730 \; \text{Torr}
    \quad;\quad
    V_1=20 \; \text{L}$$
    $$T_2=T_1
    \quad;\quad
    P_2=2 \; \text{atm}
    \quad;\quad
    V_2=?$$
    
    Visto che la temperatura è costante, l'equazione di stato diventa
    
    $$P_1V_1=P_2V_2$$
    
    Vogliamo trovare $V_2$, per cui
    
    $$V_2=\frac{P_1V_1}{P_2}
    =\frac{730 \cdot 20}{760 \cdot 2}=9.6052 \; \rm L$$
    \item I dati sono

    $$T_1=20 \; ^{\circ}\text{C}
    \quad;\quad
    P_1=730 \; \text{Torr}
    \quad;\quad
    V_1=20 \; \text{L}$$
    $$T_2=0 \; ^{\circ}\text{C}
    \quad;\quad
    P_2=1 \; \text{atm}
    \quad;\quad
    V_2=?$$
    
    L'equazione di stato si scriverà come
    
    $$\frac{730 \cdot 20}{760 \cdot 293.15}=\frac{1 \cdot V_2}{273.15}$$
    
    $$V_2=\frac{730 \cdot 20 \cdot 273.15}{760 \cdot 293.15}=17.9 \; \rm L$$
    \item I dati sono

    $$T_1=20 \; ^{\circ}\text{C}
    \quad;\quad
    P_1=730 \; \text{Torr}
    \quad;\quad
    V_1=20 \; \text{L}$$
    $$T_2=?
    \quad;\quad
    P_2=700 \; \text{Torr}
    \quad;\quad
    V_2=V_1$$
    
    L'equazione di stato diventa
    
    $$\frac{P_1}{T_1}=\frac{P_2}{T_2}
    \implies
    T_2=\frac{P_2}{P_1}T_1$$
    
    $$\implies
    T_2=\frac{700 \cdot 760}{760 \cdot 730}\cdot 293.15
    =281.10 \; \rm K
    \quad(\approx 8 \; ^{\circ}C)$$
    \item I dati sono 

    $$T_1=20 \; ^{\circ}\text{C}
    \quad;\quad
    P_1=730 \; \text{Torr}
    \quad;\quad
    V_1=20 \; \text{L}$$
    $$T_2=-50 \; ^{\circ}\text{C}
    \quad;\quad
    P_2=?
    \quad;\quad
    V_2=V_1$$
    
    L'equazione di stato diventa
    
    $$\frac{P_1}{T_1}=\frac{P_2}{T_2}
    \implies
    P_2=\frac{T_2}{T_1}P_1$$
    
    $$\implies
    P_2=\frac{223.15}{293.15}\cdot \frac{730}{760}
    =0.73 \; \rm atm
    \quad(\approx 556 \; Torr)$$
    \item I dati sono

    $$T_1=20 \; ^{\circ}\text{C}
    \quad;\quad
    P_1=730 \; \text{Torr}
    \quad;\quad
    V_1=20 \; \text{L}$$
    $$T_2=?
    \quad;\quad
    P_2=650 \; \text{Torr}
    \quad;\quad
    V_2=26 \; \text{L}$$
    
    L'equazione di stato diventa 
    
    $$\frac{P_1V_1}{T_1}=\frac{P_2V_2}{T_2}
    \implies
    T_2=\frac{P_2V_2}{P_1V_1}T_1$$
    
    $$\implies
    T_2=\frac{650 \cdot 26 \cdot 760}{760 \cdot 20 \cdot 730}\cdot 293.15
    =339.3 \; \rm K
    \quad(\approx 66 \; \rm ^{\circ} C)$$
\end{enumerate}
\end{soluzione}

\newpage

\begin{esercizio}[$\bigstar$]
    Una soluzione 0.1 m di acido dicloro acetico congela a $0.278 \; \rm ^{\circ} C$. Calcolare la percentuale di soluto presente in soluzione come ione idrogeno $\rm H^+$, ione dicloro acetato $\rm CHCl_2COO^-$ e molecole indissociate dell'acido.
\end{esercizio}
\begin{soluzione}
    \begin{equation*}
        \schemestart
        \chemname{\chemfig{H-C(-[6]H)(-[2]H)-C(=[1]O)-OH}}{acido acetico}
        \hspace{1.5cm}
        \chemname{
        \chemfig{Cl-C(-[6]H)(-[2]Cl)-C(=[1]O)-OH}}{acido dicloroacetico}
        \schemestop
    \end{equation*}
    \vspace{0.7cm}Due idrogeni nel gruppo $\rm CH_3$ vengono sostituiti da due atomo di cloro, diventando $\rm CHCl_2COOH$. Esso è un acido organico che si dissocia liberando il protone del gruppo OH, mai quello del CH.\\[0.2cm]
    Chiameremo per semplicità lo ione dicloro acetato con $\rm A$. Il testo ci fa capire che questa specie è un elettrolita debole, ossia si dissocia in parte, non totalmente. Quando si dissocia dà luogo allo ione $\rm H^+$ più lo ione $\rm A^-$:
    \begin{equation*}
        \ce{HA <--> H^+ + A^-}
    \end{equation*}
    Siccome non si dissocia del tutto, dobbiamo calcolare il grado di dissociazione.\\
    Sappiamo che la concentrazione è 0.1 molale, però dobbiamo valutare quanto se ne dissoci. Immaginiamo allora di partire da una soluzione che contiene una mole della specie HA e scriviamo sopra la reazione le quantità prima della dissociazione, sotto quelle dopo la dissociazione:
    \begin{center}
        \begin{tabular}{ccccc}
            $1$ & & / & & /\\
            HA & \ce{<-->} & $\rm H^+$ & + & $\rm A^-$\\
            $1 - \alpha$ &  &  $\alpha$ & & $\alpha$\\
        \end{tabular}
    \end{center}
    Dopo che avviene la dissociazione e il sistema raggiunge l'equilibrio, la quantità che si è dissociata è pari ad $\alpha$. Se $\alpha$ sono le moli che si dissociano, alla fine avremo $1-\alpha$ moli di HA non dissociato. Delle $\alpha$ moli che si dissociano, otteniamo da ciascuna uno ione $\rm H^+$ e ione $\rm A^-$, quindi avremo $\alpha$ moli di $\rm H^+$ e $\alpha$ moli di $\rm A^-$.\\
    Alla fine il numero totale di moli sarà la somma di tutte e tre le specie, cioè
    \begin{equation*}
        n_{\rm tot}
        =1 - \alpha + \alpha + \alpha
        =1+\alpha
    \end{equation*}
    Tale valore andrà a moltiplicare la molalità. L'espressione allora si riduce a
    \begin{gather*}
        \Delta t
        =k_{cr} \cdot m \cdot (1 + \alpha)
        \\
        \implies
        0.278 = 1.86 \cdot 0.1 \cdot (1+\alpha)
        \implies
        \frac{0.278}{0.186}= 1 + \alpha
        \implies
        \alpha=\frac{0.278}{0.186} - 1
        \\
        \implies \alpha=0.4946
    \end{gather*}
    Quindi si è dissociato il 49.46\%.
\end{soluzione}

\newpage

\begin{esercizio}
    Qual è la percentuale in massa di acetato di etile in una miscela di questo e acetato di propile, sapendo che $P_{et.}^0=415 \; \rm Torr$ e $P_{pr.}^0=172 \; \rm Torr$ e che a $60 \; ^{\circ} \rm C$ questa miscela ha una tensione di vapore di $250 \; \rm Torr$?
\end{esercizio}
\begin{soluzione}
    Tutti e due i composti sono volatili, quindi la tensione di vapore sarà
    \begin{equation*}
        P=P_{\rm et.}^0 \rchi_{\rm et.} + P_{\rm pr.}^0 \rchi_{\rm pr.}
    \end{equation*}
    Poiché $\rchi_{et.} + \rchi_{pr.}=1$ allora $\rchi_{pr.}=1 - \rchi_{et.}$. L'espressione diventa
    \begin{gather*}
        250
        =415 \rchi_{\rm et.} + 172(1 - \rchi_{\rm et.})
        =415 \rchi_{\rm et.} - 172 \rchi_{\rm et.} + 172
        \\
        243\rchi_{\rm et.}=78
        \implies
        \rchi_{\rm et.}=\frac{78}{243}
        =0.32
        \quad
        (\implies \rchi_{\rm pr.}=0.68)
    \end{gather*}
    Le percentuali allora sono: 32\% di acetato di etile e 68\% di acetato di propile.
\end{soluzione}

\newpage

\begin{esercizio}
    Calcolare la pressione osmotica, in mmHg, di una soluzione acquosa 0.001 M di acido acetico dissociato al 10\% a $25 \; ^{\circ} \rm C$.
\end{esercizio}
\begin{soluzione}
    Innanzitutto notiamo che siamo davanti ad una specie debole, per cui il coefficiente di Van't Hoff si calcolerà come

$$i=\big[1 + (\gamma -1)\alpha\big]$$

con $\alpha$ grado di dissociazione e $\gamma$ numero di particelle in cui il composto si dissocia.

L'acido acetico in acqua si dissocia in ione $\rm H^+$ e ione acetato, per cui $\gamma=2$; inoltre il testo ci dice che $\alpha=0.1$, quindi

$$i=\big[1 + (2-1) \cdot 0.1\big]=1.1$$

La pressione osmotica sarà quindi data da

$$\Pi=1.1 \cdot CRT \cdot 760
=1.1 \cdot 0.001 \cdot 0.082 \cdot 298.15 \cdot 760=20.4388 \; \rm mmHg$$

\end{soluzione}

\newpage

\begin{esercizio}
    Calcolare la pressione osmotica di una soluzione di 923 mL, contenente 4.1234 grammi di idrossido di magnesio alla temperatura di $27 \; \rm ^{\circ} C$.
\end{esercizio}
\begin{soluzione}
    Per prima cosa calcoliamo le moli di $\rm Mg(OH)_2$:
    \begin{equation*}
        n=\frac{g}{MM}=\frac{4.1234}{58.3197}=7.0703 \cdot 10^{-2} \; \rm mol
    \end{equation*}
    Calcoliamo la concentrazione
    \begin{equation*}
        C=\frac{n}{V(\text{L})}
        =\frac{7.0703 \cdot 10^{-2} \cdot 1000}{923}
        =7.6601 \cdot 10^{-2} \; \rm mol/L
    \end{equation*}
    A questo punto possiamo calcolare la pressione osmotica. Siccome l'idrossido di magnesio in acqua si dissocia in uno ione $\rm Mg^{2+}$ e due ioni $\rm OH^-$ il coefficiente di Van't Hoff sarà pari a 3, per cui
    \begin{equation*}
        \Pi=3 \cdot CRT
        =3 \cdot 7.6601 \cdot 10^{-2} \cdot 0.082 \cdot 300.15
        =5.6560 \; \rm atm
    \end{equation*}
\end{soluzione}

\newpage

\begin{esercizio}
    Calcolare la pressione osmotica di una soluzione di 875 mL, contenente 3.05 grammi di idrossido di potassio alla temperatura di $27 \; \rm ^{\circ} C$.
\end{esercizio}
\begin{soluzione}
    Per prima cosa calcoliamo le moli di $\rm KOH$:

$$n=\frac{g}{MM}
=\frac{3.05}{56.1056}
=5.4362 \cdot 10^{-2} \; \rm mol$$

Calcoliamo la concentrazione

$$C=\frac{n}{V(L)}
=\frac{5.4362 \cdot 10^{-2} \cdot 1000}{875}
=6.2128 \cdot 10^{-2} \; \rm mol/L$$

A questo punto possiamo calcolare la pressione osmotica. Siccome l'idrossido di potassio in acqua si dissocia in uno ione $\rm K^{+}$ e uno ione $\rm OH^-$ il coefficiente di Van't Hoff sarà pari a 2, per cui

$$\Pi=2 \cdot CRT
=2 \cdot 6.2128 \cdot 10^{-2} \cdot 0.082 \cdot 300.15
=3.0582 \; \rm atm$$
\end{soluzione}
\newpage

\begin{esercizio}
    Calcolare a $25 \; ^{\circ} \rm C$ la pressione osmotica di una soluzione 0.15 M di acido iodico ($K_a=1.7 \cdot 10^{-1}$).
\end{esercizio}
\begin{soluzione}
    Il testo non fornisce sufficienti dati, quindi ai fini della risoluzione del problema supporremo che l'acido debole sia monoprotico. Sotto tale ipotesi, chiamiamo HA il generico acido debole, il quale in acqua darà luogo all'equilibrio

$$\ce{HA <--> H^+ + A^-}$$

Se partiamo da una mole di acido, a fine reazione resteranno $1 - \alpha$ moli indissociate di acido e si produrranno $\alpha$ moli sia di ione $\rm H^+$ che di ione $\rm A^-$. 

La costante di dissociazione sarà allora pari, avendo una concentrazione iniziale $C$, a

$$K_a=C\,\frac{\alpha^2}{1-\alpha}$$

Poiché $K_a=1.7 \cdot 10^{-1}$ non è possibile trascurare l'$\alpha$ al denominatore. Infatti è lecito adoperare tale approssimazione solo nel caso in cui $C_a \geq 10^{-2}$ e $K_a \leq 10^{-5}$. Non essendo soddisfatta la condizione sull'acido siamo costretti a dover risolvere la seguente equazione di secondo grado, che si ottiene con semplici passaggi algebrici a partire dalla definizione di $K_a$ di sopra:

$$\frac{C}{K_a}\alpha^2 + \alpha -1=0$$

Tale equazione di secondo grado in $\alpha$ ammette una soluzione positiva (e una negativa che scartiamo) pari a:

$$\alpha=\frac{-1 + \sqrt{1 + 4 \cdot \displaystyle \frac{C}{K_a}}}{2 \cdot \displaystyle \frac{C}{K_a}}
=\frac{-1 + \sqrt{1 + 3.5294}}{1.7647}=0.6393\approx0.64$$

Il coefficiente di Van't Hoff sarà allora dato da

$$i=\big[1 + (2-1) \cdot 0.64\big]=1.64$$

da cui la pressione osmotica sarà pari a

$$\Pi=1.64 \cdot CRT
=1.64 \cdot 0.15 \cdot 0.082 \cdot 298.15
=6.0143 \; \rm atm$$
\end{soluzione}

\newpage

\begin{esercizio}
    50.00 mL di una soluzione acquosa di nitrato di argento vengono trattati con un eccesso di acido cloridrico. Si formano 1.7658 g di cloruro di argento. Calcolare la molarità della soluzione di nitrato
di argento.
\end{esercizio}
\begin{soluzione}
    La reazione che avviene è

$$\ce{AgNO_3 + HCl -> AgCl + HNO_3}$$

il testo ci dice che siamo in eccesso di acido cloridrico, quindi il reagente limitante sarà il nitrato d'argento, che ha quindi reagito completamente. Inoltre, poiché il rapporto stechiometrico tra questo e il cloruro d'argento è 1:1, il numero di moli di AgCl prodotte corrisponderà al numero iniziale di moli di $\rm AgNO_3$. Quindi:

$$n_{\rm AgNO_3}=n_{\rm AgCl}=\frac{g}{MM_{\rm AgCl}}
=\frac{1.7658}{143.32}=1.2321 \cdot 10^{-2} \; \rm mol$$

La concentrazione sarà data da

$$C=\frac{n_{\rm AgCl}}{V(\text{L})}
=\frac{1.2321 \cdot 10^{-2} \cdot 1000}{50}
=0.2464 \; \rm mol/L$$
\end{soluzione}

\newpage

\begin{esercizio}
    In un recipiente di 5 litri sono introdotti 300 mL di cloro misurati a $20 \; ^{\circ} \rm C$ e 750 Torr, 800 mL di azoto misurati a $15 \; ^{\circ} \rm C$ e 700 Torr, 500 mL di ossigeno misurati a $18 \; ^{\circ} \rm C$ e 850 Torr e 450 mL di anidride carbonica misurati a $20 \; ^{\circ} \rm C$ e 760 Torr. Calcolare la pressione totale della miscela a $25 \; ^{\circ} \rm C$ e la frazione molare dell'ossigeno.
\end{esercizio}
\begin{soluzione}
    Attraverso l'equazione di stato dei gas, il numero di moli è dato da
    \begin{equation*}
        PV=nRT
        \implies
        n=\frac{PV}{RT}
    \end{equation*}
    Per prima cosa calcoliamo le moli di ciascun componente:
    \begin{gather*}
        n_{\rm Cl_2}
        =\frac{750 \cdot 300}{0.082 \cdot 293.15 \cdot 1000 \cdot 760}
        =1.2316 \cdot 10^{-2} \; \rm mol
        \\[0.2cm]
        n_{\rm N_2}
        =\frac{700 \cdot 800}{0.082 \cdot 288.15 \cdot 1000 \cdot 760}
        =3.1185 \cdot 10^{-2} \; \rm mol
        \\[0.2cm]
        n_{\rm O_2}
        =\frac{850 \cdot 500}{0.082 \cdot 291.15 \cdot 1000 \cdot 760}
        =2.3423 \cdot 10^{-2} \; \rm mol
        \\[0.2cm]
        n_{\rm CO_2}
        =\frac{760 \cdot 450}{0.082 \cdot 293.15 \cdot 1000 \cdot 760}
        =1.8720 \cdot 10^{-2} \; \rm mol
    \end{gather*}
    Nota: abbiamo diviso per 1000 per passare da $\rm mL$ a $\rm L$ e per 760 per passare da $\rm Torr$ a $\rm atm$.\\
    A questo punto calcoliamo le pressioni parziali. Sempre dall'equazione di stato dei gas, si ha
    \begin{equation*}
        PV=nRT
        \implies
        P=\frac{nRT}{V}
    \end{equation*}
    dunque
    \begin{gather*}
        P_{\rm Cl_2}
        =\frac{1.2316 \cdot 10^{-2} \cdot 0.082 \cdot 298.15}{5}
        =0.6022 \cdot 10^{-1} \; \rm atm
        \\[0.2cm]
        P_{\rm N_2}
        =\frac{3.1185 \cdot 10^{-2} \cdot 0.082 \cdot 298.15}{5}
        =0.1525 \cdot 10^{-1} \; \rm atm
        \\[0.2cm]
        P_{\rm O_2}
        =\frac{2.3423 \cdot 10^{-2} \cdot 0.082 \cdot 298.15}{5}
        =1.1453 \cdot 10^{-1} \; \rm atm
        \\[0.2cm]
        P_{\rm CO_2}
        =\frac{1.8720 \cdot 10^{-2} \cdot 0.082 \cdot 298.15}{5}
        =0.9642 \cdot 10^{-1} \; \rm atm
    \end{gather*}
    La pressione totale sarà pari a
    \begin{equation*}
        P_{\rm tot}
        =(0.6022 + 0.1525 + 1.1453 + 0.9642) \cdot 10^{-1}
        =2.8642 \cdot 10^{-1} \; \rm atm
    \end{equation*}
    Inoltre la frazione molare dell'ossigeno sarà pari a
    \begin{equation*}
        \rchi_{\rm O_2}
        =\frac{n_{\rm O_2}}{n_{\rm tot}}
        =\frac{2.3423 \cdot 10^{-2}}{(1.2316 + 3.1185 + 2.3423 + 1.9720) \cdot 10^{-2}}
        =0.4223
    \end{equation*}
\end{soluzione}

\newpage

\begin{esercizio}
    In un recipiente di 1 L sono state introdotte 2.5 moli di NOCl. Dopo opportuno riscaldamento ad una fissata temperatura si realizza il seguente equilibrio:
    \begin{equation*}
        \ce{2 NOCl <--> 2NO + Cl_2}
    \end{equation*}
    All'equilibrio sono presenti 0.86 moli di NO. Calcolare la relativa costante di equilibrio in funzione delle concentrazioni.
\end{esercizio}
\begin{soluzione}
    Scriviamo innanzitutto la reazione che avviene con le quantità iniziali sopra e quelle finali sotto: 

\begin{center}
    \begin{tabular}{ccccccc}
        $2.5$ & & / & & /\\
        $\rm 2NOCl$ & \ce{<-->} & $\rm 2NO$ & + & $\rm Cl_2$\\
        $1.64$ & & $0.86$ & & $0.43$\\
    \end{tabular}
\end{center}

Vediamo perché abbiamo scritto tali quantità finali:

Il testo ci dice che a fine reazione abbiamo 0.86 moli di NO, e siccome esso è stato prodotto a partire dall'NOCl di quest'ultimo ci resteranno un numero di moli pari proprio alla differenza, cioè pari a 1.64; per quanto riguarda il $\rm Cl_2$ invece abbiamo scritto 0.43 perché il suo rapporto stechiometrico con l'NO è 2:1, cioè il numero di moli di tale specie sarà pari alla metà di quelle dell'NO.

Scriviamo la concentrazione di tutte le specie. Attenzione! Dovremo usare le moli finali. Inoltre, siccome abbiamo un litro di soluzione, le concentrazioni coincideranno in numero col numero di moli, per cui

$$C_{\rm NOCl}=1.64 \; \text{mol/L}
\quad;\quad
C_{\rm NO}=0.86 \; \text{mol/L}
\quad;\quad
C_{\rm Cl_2}=0.43 \; \text{mol/L}$$

La costante di equilibrio sarà data da

$$K=\rm \frac{[NO]^2 \cdot [Cl_2]}{[NOCl]^2}
=\frac{0.86^2 \cdot 0.44}{1.64^2}=1.1824 \cdot 10^{-1}$$
\end{soluzione}

\newpage

\begin{esercizio}
    Sia data una soluzione composta da due liquidi $A$ e $B$. Ad una data temperatura la $P^0_A = 152$ Torr e la $P^0_B = 324$ Torr. Calcolare la pressione totale dei vapori di $A$ e $B$ in equilibrio con la soluzione sapendo che la frazione molare di $A$ in soluzione è $0.75$.
\end{esercizio}
\begin{soluzione}
    Innanzitutto, poiché $\rchi_A=0.75$, segue che $\rchi_B=0.25$ (deve sempre valere $\rchi_A + \rchi_B =1$).\\
    Dalla legge di Dalton unita alla legge di Raoult segue che la pressione totale dei vapori è data da
    \begin{equation*}
        P=\rchi_A P_A^0 + \rchi_B P_B^0
    \end{equation*}
    da cui
    \begin{equation*}
        P=0.75 \cdot 152 + 0.25 \cdot 324
        =195 \; \rm Torr
    \end{equation*}
\end{soluzione}

\newpage

\begin{esercizio}
    Calcolare la pressione osmotica delle seguenti soluzioni a $T=25 \; \rm ^{\circ}C$:
    \begin{enumerate}
        \item Glucosio 0.3 M;
        \item Cloruro di bario 0.1 M;
        \item Cloruro di alluminio 0.15 M.
    \end{enumerate}
\end{esercizio}
\begin{soluzione}
    La pressione osmotica è data da
    \begin{equation*}
        \Pi=CRTi
    \end{equation*}
    Calcoliamola per i vari casi:
    \begin{enumerate}
        \item Il glucosio è una specie molecolare, per cui in acqua non si dissocia e quindi $i=1$:
        \begin{equation*}
            \Pi_{\rm glucosio}
            =0.3 \cdot 0.082 \cdot 298.15
            =7.3345 \; \rm atm
        \end{equation*}
        \item Il cloruro di bario $\rm BaCl_2$ in acqua si dissocia in uno ione $\rm Ba^{2+}$ e due ioni $\rm Cl^-$, per cui $i=3$:
        \begin{equation*}
            \Pi_{\rm BaCl_2}
            =0.1 \cdot 0.082 \cdot 298.15 \cdot 3
            =7.3345 \; \rm atm
        \end{equation*}
        \item Il cloruro di alluminio $\rm AlCl_3$ in acqua si dissocia in uno ione $\rm Al^{3+}$ e tre ioni $\rm Cl^-$, per cui $i=4$:
        \begin{equation*}
            \Pi_{\rm BaCl_2}
            =0.15 \cdot 0.082 \cdot 298.15 \cdot 4
            =14.6690 \; \rm atm
        \end{equation*}
    \end{enumerate}
\end{soluzione}

\newpage

\begin{esercizio}
    Calcolare la pressione osmotica di una soluzione di 900 mL, contenente 5.56 grammi di KOH alla temperatura di $26 \; ^{\circ} \rm C$.
\end{esercizio}
\begin{soluzione}
    Calcoliamo innanzitutto le moli di KOH

$$n_{\rm KOH}=\frac{5.56}{56.1056}
=9.9099 \cdot 10^{-2} \; \rm mol$$

da cui si ha che la concentrazione è pari a

$$C=\frac{n}{V(L)}
=\frac{9.9099 \cdot 10^{-1}}{0.9}
=1.1011 \cdot 10^{-2} \; \rm mol/L$$

Infine notiamo che l'idrossido di potassio in acqua si dissocia in uno ione $\rm K^+$ e uno ione $\rm OH^-$, per cui il coefficiente di Van't Hoff è pari a 2. Pertanto la pressione osmotica sarà data da

$$\Pi=2CRT
=2 \cdot 1.1011 \cdot 10^{-1} \cdot 0.082 \cdot 299.15
=5.4026 \; \rm atm$$
\end{soluzione}

\newpage

\begin{esercizio}
    A 400 mL di una soluzione acquosa di $\rm FeCl_3$ 0.0132 M sono stati aggiunti 350 mg di un composto organico senza variazione apprezzabile di volume. A $25 \; ^{\circ} \rm C$ si misura una pressione osmotica di 1072 Torr. Calcolare la massa molare del composto organico.
\end{esercizio}
\begin{soluzione}
    Quando in soluzione abbiamo più di un composto, la pressione osmotica finale sarà data dalla somma delle singole pressioni osmotiche. Per trovare allora la pressione osmotica del composto organico X, calcoliamo quella dell'$\rm FeCl_3$ e sottraiamola a quella totale.

Il cloruro ferrico in acqua si dissocia in uno ione ferrico $\rm Fe^{3+}$ e 3 ioni $\rm Cl^-$, per cui il coefficiente di Van't Hoff sarà pari a 4 e dunque

$$\Pi_{\rm FeCl_3}=4CRT
=4 \cdot 0.0132 \cdot 0.082 \cdot 298.15
=1.2909 \; \rm atm$$

La pressione totale in atm è pari a

$$\Pi=\frac{1072}{760}=1.4105 \; \rm atm$$

quindi la pressione osmotica del composto X sarà pari a

$$\Pi_{\rm X}=1.4105 - 1.2909 = 0.1197 \; \rm atm$$

A questo punto ricaviamo la concentrazione del composto X come

$$C=\frac{\Pi_{\rm X}}{RT}=\frac{0.1197}{0.082 \cdot 298.15}
=4.8960 \cdot 10^{-3} \; \rm mol/L$$

Nota: i composti organici non si sciolgono in acqua, quindi $i=1$!

Calcoliamo poi le moli attraverso la proporzione

$$n_{\rm X}:400=4.8960 \cdot 10^{-3}:1000
\implies
n_{\rm X}=\frac{400 \cdot 4.8960 \cdot 10^{-3}}{1000}
=1.9584 \cdot 10^{-3} \; \rm mol$$

Infine, la massa molare si calcolerà come

$$n_{\rm X}=\frac{g}{MM_{\rm X}}
\implies
MM_{\rm X}=\frac{g}{n_{\rm X}}=\frac{0.350}{1.9584 \cdot 10^{-3}}=178.7173$$
\end{soluzione}

\newpage

\begin{esercizio}
    Un volume di 10 L di una miscela gassosa di 80 grammi alla temperatura di $90 \; ^{\circ} \rm C$ contiene il 30\% in peso di $\rm SO_3$ ed il 70\% di $\rm O_2$. Calcolare le pressioni parziali dei singoli gas.
\end{esercizio}
\begin{soluzione}
    Calcoliamo le moli dei singoli composti. Per tale scopo dobbiamo conoscere i grammi di ogni composto:

$$g_{\rm SO_3}=30\% \cdot 80=24 \; \text{grammi}
\quad;\quad
g_{\rm O_2}=70\% \cdot 80=56 \; \text{grammi}$$

da cui

\vspace{-0.2cm}$$n_{\rm SO_3}=\frac{g}{MM_{\rm SO_3}}=\frac{24}{80.06}=0.2998 \; \text{mol}
\quad;\quad
n_{\rm O_2}=\frac{g}{MM_{\rm O_2}}=\frac{56}{32}=1.7500 \; \text{mol}$$

Essendo nel caso di gas reali, ogni specie occuperà l'intero volume come se fosse l'unica presente all'interno del contenitore. Le pressioni delle singole specie si otterranno, partendo dall'equazione di stato dei gas, come

$$PV=nRT \implies
P=n\frac{RT}{V}$$

$$P_{\rm SO_3}=n_{\rm SO_3} \displaystyle \frac{RT}{V}
=0.2998 \cdot \frac{0.082 \cdot 363.15}{10}
=0.8927 \; \rm atm$$

$$P_{\rm O_2}=n_{\rm O_2} \displaystyle \frac{RT}{V}
=1.7500 \cdot \frac{0.082 \cdot 363.15}{10}
=5.2112 \; \rm atm$$
\end{soluzione}

\newpage

\begin{esercizio}
    In un recipiente di 5.6 litri, alla temperatura di $80 \; ^{\circ} \rm C$, vi sono 4.5432 g di metano, 1.2345 g di idrogeno e 14.5678 g di azoto. Calcolare i valori delle pressioni parziali dei tre gas.
\end{esercizio}
\begin{soluzione}
    Essendo nel caso di gas ideali, ognuno di questi si comporterà come se fosse l'unico presente all'interno del recipiente, occupando l'intero volume. Le pressioni parziali si calcoleranno a partire dell'equazione di stato dei gas:
    \begin{equation*}
        PV=nRT
        \implies
        P=\frac{nRT}{V}
    \end{equation*}
    Calcoliamo le moli e conseguentemente le pressioni:
    \begin{gather*}
        n_{\rm CH_4}
        =\frac{4.5432}{16.04}
        =0.2832
        \implies
        P_{\rm CH_4}
        =\frac{0.2832 \cdot 0.082 \cdot 353.15}{5.6}
        =1.4645 \; \rm atm
        \\[0.2cm]
        n_{\rm H_2}
        =\frac{1.2345}{2.0158}
        =0.6124
        \implies
        P_{\rm H_2}
        =\frac{0.6124 \cdot 0.082 \cdot 353.15}{5.6}
        =3.1668 \; \rm atm
        \\[0.2cm]
        n_{\rm N_2}
        =\frac{14.5678}{28.0134}
        =0.5200
        \implies
        P_{\rm N_2}
        =\frac{0.5200 \cdot 0.082 \cdot 353.15}{5.6}
        =2.6891 \; \rm atm
    \end{gather*}
\end{soluzione}

\newpage

\begin{esercizio}
    Una soluzione al 5\% in peso di $\rm CaX_2$ (ove X è un alogeno), inizia a congelare a $-1.396 \; \rm ^{\circ}C$. $k_{cr_{\rm H_2O}} = 1.860$. Qual è l'alogeno? 
\end{esercizio}
\begin{soluzione}
    \textbf{Nota}: questo esercizio è incorretto. Infatti per risolverlo correttamente dovremmo considerare 1000 grammi di soluzione, ma in tal caso non si perverrebbe ad un alogeno.

\vspace{0.2cm}Supponiamo innanzitutto di avere 1000 grammi di solvente. La $\Delta t_{cr}$ è data da

$$\Delta t_{cr}=k_{cr} m i
\quad\text{dove}\quad
m=\frac{n_{\rm soluto}}{1000\;g_{\rm solvente}}$$

La specie è $\rm CaX_2$, la quale si dissocia in uno ione $\rm Ca^{2+}$ e due ioni $\rm X^-$, per cui $i=3$. Dobbiamo inoltre considerare il fatto che la massa è solo il 5\%, per cui

$$1.396=1.860 \cdot 10^3 \cdot \frac{\displaystyle \frac{5}{100}\;g_{\rm solvente}}{1000 \cdot MM_{\rm soluto}} \cdot 3$$

Nota: abbiamo moltiplicato per un fattore $10^3$ per l'unità di misura della $k_{cr}$.

$$\implies
1.396=1.860 \cdot \frac{50}{MM} \cdot 3
\implies
MM=199.8567$$

A tale peso molecolare dobbiamo togliere il peso atomico del calcio:

$$MM_{\rm X_2}=199.8567-40.08=1597767
\implies
MM_{\rm X} \approx 79.8884$$

Dunque l'alogeno incognito è il bromo.

\end{soluzione}

\newpage

\begin{esercizio}
    Calcolare la pressione osmotica di 789 ml di soluzione contenente 5.8440 grammi di NaCl e 21.2996 grammi di $\rm Al(NO_3)_3$ alla temperatura di $27 \; \rm ^{\circ}C$.
\end{esercizio}
\begin{soluzione}
    Calcoliamo innanzitutto le moli:

$$n_{\rm NaCl}=\frac{g}{MM_{\rm NaCl}}
=\frac{5.8440}{58.44}=0.1 \; \rm mol$$

$$n_{\rm Al(NO_3)_3}=\frac{g}{MM_{\rm Al(NO_3)_3}}
=\frac{21.2996}{212.996}=0.1 \; \rm mol$$

da cui le concentrazioni sono uguali e pari a

$$C_{\rm NaCl}=C_{\rm Al(NO_3)_3}=\frac{0.1 \cdot 1000}{789}
=0.1267 \; \rm mol/L$$

Calcoliamo le singole pressioni osmotiche.

L'NaCl in acqua si dissocia in ione $\rm Na^+$ e ione $\rm Cl^-$, quindi il coefficiente di Van't Hoff è pari a 2, da cui

$$\Pi_{\rm NaCl}=2CRT=2 \cdot 0.1267 \cdot 0.082 \cdot 300.15=6.2368 \; \rm atm$$

L'$\rm Al(NO_3)_3$ in acqua si dissocia in ione $\rm Al^{3+}$ e 3 ioni $\rm NO_3^-$, quindi il coefficiente di Van't Hoff è pari a 4, da cui

$$\Pi_{\rm Al(NO_3)_3}=4CRT=4 \cdot 0.1267 \cdot 0.082 \cdot 300.15=12.4735 \; \rm atm$$

La pressione totale sarà pari alla somma

$$\Pi_{\rm tot}=6.2368 + 12.4735=18.7103 \; \rm atm$$
\end{soluzione}

\newpage

\begin{esercizio}
    Calcolare la pressione osmotica di una soluzione di 850 mL contenente 89.1234 grammi di solfato di alluminio alla temperatura di $80 \; ^{\circ} \rm C$.
\end{esercizio}
\begin{soluzione}
    Per prima cosa calcoliamo le moli, per poi calcolare la concentrazione:

$$n_{\rm Al_2(SO_4)_3}=\frac{g}{MM_{\rm Al_2(SO_4)_3}}=\frac{89.1234}{342.15}=2.6048 \cdot 10^{-1} \; \rm mol$$

$$\implies C
=\frac{n}{V(\text{L})}
=\frac{2.6048 \cdot 10^{-1} \cdot 1000}{850}
=3.0645 \cdot 10^{-1} \; \rm mol/L$$

A questo punto possiamo calcolare la pressione osmotica.

L'$\rm Al_2(SO_4)_3$ in acqua si dissocia in due ioni $\rm Al^{3+}$ e tre ioni $\rm SO_4^-$, quindi il coefficiente di Van't Hoff è pari a 5, da cui

$$\Pi=5CRT
=5 \cdot 3.0645 \cdot 10^{-1} \cdot 0.082 \cdot 353.15
=44.3713 \; \rm atm$$

\end{soluzione}

\newpage

\begin{esercizio}
    Una soluzione di acido nitrico al 27\% in massa ha una densità di $1.16 \; \rm g/mL$. Calcolarne la molarità, molalità e temperatura di ebollizione sapendo che $k_{eb_{\rm H_2O}}=1.86 \; \rm ^{\circ}C \cdot kg/mol$.
\end{esercizio}
\begin{soluzione}
    Supponiamo di avere un litro di soluzione. Essa in grammi avrà una massa pari a

$$g=V \cdot d \implies g=1000 \cdot 1.16 = 1160 \; \rm grammi$$

Di questi, solo il 27\% costituisce l'acido nitrico $\rm HNO_3$, per cui avremo che

$$g_{\rm HNO_3}=\frac{1160 \cdot 27}{100}=313.2 \; \rm grammi$$

Calcoliamo quindi il corrispondente numero di moli:

$$n_{\rm HNO_3}=\frac{g}{MM_{\rm HNO_3}}
=\frac{313.2}{63.018}=4.97 \; \rm mol$$

Avendo supposto di avere 1 litro di soluzione, esso sarà anche il valore della molarità: $M=4.97 \; \rm mol/L$.

\vspace{0.2cm}Per quanto riguarda la molalità dobbiamo stare attenti, poiché essa è definita come il rapporto tra le moli di soluto su 1000 grammi di solvente puro. Noi però abbiamo la massa della soluzione totale, ma avendo calcolato i grammi di soluto possiamo ricavare i grammi di solvente puro come

$$g_{\rm solvente}=g_{\rm soluzione} - g_{\rm soluto}
\implies
1160 -313.2=846.8 \; \rm grammi$$

A questo punto possiamo ricavare la molalità attraverso la proporzione

$$m:1000=4.97:846.8
\implies
m=\frac{1000 \cdot 4.97}{846.8}=5.8691 \; \rm mol/kg$$

Avendo ricavato la molalità, possiamo calcolare la temperatura di ebollizione.

Essendo l'acido nitrico una specie forte, essa in acqua si dissocia totalmente in ione $\rm H_3O^+$ e ione $\rm NO_3^-$, per cui il coefficiente $i$ di Van't Hoff sarà pari a 2. Dunque

$$\Delta t_{eb}=t_f - t_i=t_f - 0 \; ^{\circ}\text{C}=t_f$$

$$\implies t_f
=k_{eb} \cdot m \cdot 2
=1.86 \cdot 5.8691 \cdot 2
=21.83 \; \rm ^{\circ}C$$
\end{soluzione}

\newpage

\begin{esercizio}
    Una miscela di 0.2870 grammi di etere ($MM=74.000$) ed una certa quantità di cloroformio ($MM=119.5$) viene vaporizzata a $100 \; \rm ^{\circ} C$ in un recipiente chiuso del volume di 1 L in cui era stato fatto il vuoto. La pressione totale è 0.26411 atm. Calcolare le pressioni parziali dell'etere e del cloroformio e i grammi di cloroformio.
\end{esercizio}
\begin{soluzione}
    Ponendoci nel caso di gas ideali, ciascun gas occuperà interamente il volume del recipiente come se fosse l'unico presente. Inoltre dalla legge di Dalton sappiamo che

$$P_{\rm tot}=P_{\rm etere} + P_{\rm cloroformio}$$

La pressione dell'etere può essere ricavata dall'equazione di stato dei gas. Per prima cosa calcoliamone le moli:

$$n_{\rm etere}=\frac{0.2870}{74.000}
=3.8784 \cdot 10^{-3} \; \rm mol$$

da cui

$$PV=nRT 
\implies
P=\frac{nRT}{V}$$

$$\implies P_{\rm etere}
=\frac{3.8784 \cdot 10^{-3} \cdot 0.082 \cdot 373.15}{1}
=0.11867 \; \rm atm$$

La pressione del cloroformio sarà allora data da

$$P_{\rm cloroformio}
=P_{\rm tot} - P_{\rm etere}
=0.26411-0.11867
=0.14544 \; \rm atm$$

Le moli di cloroformio le possiamo ricavare dall'equazione di stato dei gas

$$PV=nRT
\implies
n=\frac{PV}{RT}$$

$$\implies
n_{\rm cloroformio}=\frac{0.14544 \cdot 1}{0.082 \cdot 373.15}
=4.7532 \cdot 10^{-3} \; \rm mol$$

e i grammi saranno pari a

$$g=n \cdot MM=4.7532 \cdot 10^{-3} \cdot 119.5=0.5680 \; \rm grammi$$
\end{soluzione}

\newpage

\begin{esercizio}
    Determinare la molarità in ioni cloruro di una soluzione ottenuta solubilizzando 1.40 g di NaCl ($MM=58.44$) e 2.30 g di $\rm MgCl_2$ ($MM=95.21$) in 100.0 mL di acqua. Calcolare poi quanti mL di una soluzione 0.11 M di $\rm AgNO_3$ si dovrebbero usare per precipitare completamente lo ione cloruro sotto forma di cloruro di argento (insolubile).
\end{esercizio}
\begin{soluzione}
    Vediamo innanzitutto come si dissociano le specie in acqua:

$$\begin{cases}
    \ce{NaCl -> Na^+ + Cl^-}\\
    \ce{MgCl_2 -> Mg^{2+} + 2Cl^-}
\end{cases}$$

Calcoliamo le moli:

$$n_{\rm NaCl}=\frac{1.4}{58.44}=2.3956 \cdot 10^{-2} \; \text{mol}
\quad,\quad
n_{\rm MgCl_2}=\frac{2.3}{95.21}=2.4157 \cdot 10^{-2} \; \text{mol}$$

Lo ione cloruro proviene da entrambe le dissociazioni. In particolare vediamo come nella prima reazione il rapporto stechiometrico tra NaCl e ione cloruro è 1:1, mentre nella seconda il rapporto tra $\rm MgCl_2$ e ione cloruro è 1:2. Pertanto si ha:

$$n_{\rm Cl^-}
=n_{\rm NaCl} + 2 \cdot n_{\rm MgCl_2}
=7.2270 \cdot 10^{-2} \; \rm mol$$

La molarità in ioni cloruro sarà allora pari a

$$M=\frac{7.2270 \cdot 10^{-2}}{0.100}
=7.2270 \cdot 10^{-1} \; \rm mol/L$$

A questo punto calcoliamo i mL di soluzione di $\rm AgNO_3$ necessari per precipitare completamente lo ione cloruro sotto forma di cloruro di argento.

La reazione che avviene è

$$\ce{AgNO_3 + Cl^- -> Ag Cl v + NO_3^-}$$

Il rapporto stechiometrico tra $\rm AgNO_3$ e $\rm Cl^-$ è 1:1, per cui il numero di moli necessarie sarà lo stesso. Il volume allora sarà

$$n_{\rm AgNO_3}=n_{\rm Cl^-}
\implies
V_{\rm AgNO_3}=\frac{n_{\rm Cl^-}}{M}=\frac{7.2270 \cdot 10^{-1}}{0.11}=657 \; \rm mL$$
\end{soluzione}

\newpage

\begin{esercizio}
    11.0000 g di un non elettrolita, non volatile, costituito al 40.00\% da carbonio, al 6.72\% da idrogeno e al 53.28\% da ossigeno, sciolti in 599 mL di acqua determinano un abbassamento crioscopico di $0.189 \; ^{\circ} \rm C$. $k_{cr}(\text{H}_2\text{O})=1.86 \; ^{\circ} \rm C/mol$. Determinare la formula molecolare del composto.
\end{esercizio}
\begin{soluzione}
    Per determinare la formula del composto dobbiamo innanzitutto determinare la sua formula minima. Per fare ciò calcoliamo i grammi e quindi le moli degli atomi che lo compongono:

$$g_{\rm C}=11 \cdot \frac{40}{100}=4.4000
\implies
n_{\rm C}=3.6636 \cdot 10^{-1} \; \text{mol}$$
$$g_{\rm H}=11 \cdot \frac{6.72}{100}=7.3920
\implies
n_{\rm H}=7.3333 \; \text{mol}$$
$$g_{\rm O}=11 \cdot \frac{53.28}{100}=5.8608
\implies
n_{\rm O}=3.6630 \cdot 10^{-1} \; \text{mol}$$

Confrontando i valori ottenuti, vediamo che la relazione tra le moli è

$$n_{\rm C} \sim n_{\rm O} \sim \frac{n_{\rm H}}{2} \; \rm mol$$

da ciò deduciamo che la formula minima del composto sarà $\rm CH_2O$, ma non sappiamo il numero per cui dobbiamo moltiplicare ogni elemento. Per trovarlo notiamo che la massa molecolare di tale formula è 30.03, per cui se otteniamo la massa molecolare del composto, facendo il rapporto tra le due è possibile ottenerlo.

Dalla formula dell'abbassamento crioscopico otteniamo che

$$m=\frac{\Delta t}{k_{cr} i}=\frac{0.189}{1.86}
=1.0161 \cdot 10^{-1} \; \rm mol/kg$$

dove abbiamo posto $i=1$ perché la specie è un non elettrolita. Dalla molalità possiamo ricavare il numero di moli del composto (ricordiamo che $d_{\rm H_2O}=1$):

$$n=m \cdot 0.599=6.0866 \cdot 10^{-2} \; \rm mol$$

e il peso molecolare sarà

$$MM=\frac{g}{n}=180.72$$

il rapporto tra le due masse molecolari è circa 6, per cui il composto che stiamo cercando è il glucosio $\rm C_6H_{12}O_6$.
\end{soluzione}

\newpage

\begin{esercizio}
    Calcolare la temperatura di congelamento di una soluzione preparata sciogliendo in 1.2 kg di acqua 140.26 grammi di cloruro di sodio, sapendo che la costante crioscopica dell'acqua è $1.86 \; ^{\circ} \rm C/mol$.
\end{esercizio}
\begin{soluzione}
    Per prima cosa calcoliamo le moli di NaCl:

$$n=\frac{140.26}{58.44}=2.4000 \; \rm mol$$

La molalità si otterrà come

$$m=\frac{n_{\rm soluto} \cdot 1000}{g_{\rm solvente}}
=\frac{2.4000 \cdot 1000}{1200}
=2 \; \rm mol/kg$$

Il cloruro di sodio è un elettrolita forte che si dissocia in $\rm Na^+$ e $\rm Cl^-$, per cui $i=2$. Dunque:

$$\Delta t_{cr}=0^{\circ} - t_f
\implies
tf=-k_{cr} m i= - 1.86 \cdot 2 \cdot 2=-7.44 \; \rm ^{\circ}C$$
\end{soluzione}

\newpage

\begin{esercizio}
    0.8664 grammi di ossido di mercurio (II), posti in un recipiente di $0.5 \; \rm L$ nel quale era stato preventivamente fatto il vuoto, ad $800 \; \rm K$ si sono decomposti nei loro elementi esercitando una pressione di $0.788 \; \rm atm$. Dimostrare se i risultanti vapori di mercurio sono monoatomici o biatomici.
\end{esercizio}
\begin{soluzione}
    Scriviamo le due possibili reazioni:
    \begin{gather*}
        \ce{2HgO -> 2Hg + O_2}
        \quad(\text{monoatomica})
        \\
        \ce{2HgO -> Hg_2 + O_2}
        \quad(\text{biatomica})
    \end{gather*}
    Calcoliamo le moli di mercurio risultanti per entrambe le reazioni.\\
    Nel caso monoatomico il rapporto stechiometrico è 2:2, dunque 1:1, per cui avremo
    \begin{equation*}
        n_{\rm Hg}=n_{\rm HgO}
        =\frac{0.8664}{216.5894}
        =4.0001 \cdot 10^{-3} \; \rm mol
    \end{equation*}
    Nel caso biatomico il rapporto stechiometrico è 2:1, per cui le moli di $\rm Hg_2$ saranno la metà di quelle di $\rm HgO$:
    \begin{equation*}
        n_{\rm Hg_2}=\frac{n_{\rm HgO}}{2}
        =\frac{1}{2} \cdot \frac{0.8664}{216.5894}
        =2.0000 \cdot 10^{-3} \; \rm mol
    \end{equation*}
    Confrontiamo i due risultati col numero di moli ricavato a partire dall'equazione di stato dei gas. Notiamo che questa ci darà le moli totali, per cui, visto che in entrambe le reazioni appare la molecola $\rm O_2$ tra i prodotti, dovremo sottrarre le moli di questa:
    \begin{gather*}
        PV=n_{\rm tot}RT
        \implies
        n_{\rm tot}
        =\frac{PV}{RT}
        =\frac{0.788 \cdot 0.5}{0.082 \cdot 800}
        =6.0061 \cdot 10^{-3} \; \rm mol
        \\[0.2cm]
        n_{\rm O_2}
        =\frac{1}{2} \cdot \frac{0.8664}{216.5894}
        =2.0000 \cdot 10^{-3} \; \rm mol
    \end{gather*}
    in definitiva
    \begin{equation*}
        n_{\rm tot} - n_{\rm O_2}
        =(6.0061 - 2.0000) \cdot 10^{-3}
        =4.0060 \cdot 10^{-3} \; \rm mol
    \end{equation*}
    dunque il vapore di mercurio è monoatomico.
\end{soluzione}

\newpage

\begin{esercizio}
    A $80 \; \rm ^{\circ} C$ la tensione di vapore del butirrato di metile (composto A, $\rm C_5H_{10}O_2$) è di $361 \; \rm mmHg$ e quella dell'acetato di etile (composto B, $\rm C_4H_8O_2$) è di $833 \; \rm mmHg$. Calcolare la percentuale in massa dei due componenti sapendo che tale soluzione bolle proprio a $80 \; \rm ^{\circ} C$.
\end{esercizio}
\begin{soluzione}
    Ricordiamo che, per definizione, la temperatura di ebollizione è la temperatura in cui la tensione di vapore della soluzione eguaglia la pressione esterna. Poiché il testo non fornisce particolari dati, possiamo supporre per semplicità che la pressione esterna sia quella atmosferica, dunque pari a $1 \; \rm atm$. Allora, dalla legge di Raoult seguirà che
    \begin{equation*}
        \rchi_A P_A^0 + \rchi_B P_B^0=1 \; \rm atm
    \end{equation*}
    dove $P_A^0$ e $P_B^0$ sono le tensioni di vapore delle singole specie e $\rchi_A$ e $\rchi_B$ le rispettive frazioni molari.\\
    D'altra parte, poiché la soluzione è composta solo da A e B, deve valere
    \begin{equation*}
        \rchi_A + \rchi_B=1
    \end{equation*}
    e dunque mettendo insieme le condizioni avremo:
    \begin{gather*}
        \begin{dcases}
        \frac{833}{760}\rchi_A + \frac{361}{760}\rchi_B=1\\
        \rchi_A + \rchi_B=1
    \end{dcases}
    \implies
    \begin{cases}
        1.0960\rchi_A + 0.4750\rchi_B=1\\
        \rchi_A + \rchi_B=1
    \end{cases}
    \\[0.3cm]
    \implies
    \begin{cases}
        \rchi_A=1 - \rchi_B\\
        1.0960(1 - \rchi_B) + 0.4750\rchi_B=1
    \end{cases}
    \implies
    \begin{cases}
        \rchi_A=1 - \rchi_B\\
        0.6210 \rchi_B=0.0960
    \end{cases}
    \\[0.3cm]
    \implies
    \begin{cases}
        \rchi_B=0.1546\\
        \rchi_A=0.8454
    \end{cases}
    \end{gather*}
\end{soluzione}