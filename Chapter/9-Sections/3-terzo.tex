\textbf{3.1}$\bigstar$ Si consideri una soluzione al 10\% di nitrato di calcio $\rm Ca(NO_3)_2$ e densità $1.1\;g/mL$. Calcolarne la molalità $m$.

\vspace{0.2cm}\large\textbf{Svolgimento}\normalsize

\vspace{0.2cm}La molarità è definita come il numero di moli in un chilogramo di solvente puro. La densità pari a $1.1\;g/mL$ ci dice che 1 mL ha una massa uguale a 1.1 grammi, ma solo il 10\% di questo è $\rm Ca(NO_3)_2$, cioè di tale massa 0.11 grammi sono di sale, e questo su 1 mL, cioè 1 mL contiene 0.11 grammi di sale ($\rm Ca(NO_3)_2$ è un sale), 1 litro ne contiene 110 grammi. La soluzione di un litro avrà massa pari a 1100 grammi, di cui 110 grammi sono di sale, per cui avremo un numero di grammi di acqua pari a $1100-110=990$.

A questo punto dobbiamo fare la proporzione per sapere quanti grammi ci sono in 1000 grammi di solvente:

$$110:990=g:1000$$

$$g=\frac{110 \cdot 1000}{990}$$

la molalità sarà data da

$$m=\frac{110 \cdot 1000}{990 \cdot 164.09}=0.677$$

dove le moli sono espresse come grammi diviso peso molecolare.

\vspace{0.2cm}\textbf{3.2}$\bigstar$ Una soluzione è stata preparata sciogliendo 684 grammi di saccarosio ($MM=342$) in 900 grammi di acqua. Determinare le frazioni molari dei due componenti.

\vspace{0.2cm}\large\textbf{Svolgimento}\normalsize

\vspace{0.2cm}Si ha che

$$\rchi_{\text{sacc.}}=\frac{\frac{684}{342}}{\frac{684}{342} + \frac{900}{18}}=\frac{2}{52}=0.038
\quad;\quad
\rchi_{\text{H}_2\text{O}}=\frac{\frac{900}{18}}{\frac{684}{342} + \frac{900}{18}}=\frac{50}{52}=0.962$$

\vspace{0.2cm}Infatti $\rchi_{\text{sacc.}} + \rchi_{\text{H}_2\text{O}}\approx 1$.

\vspace{0.2cm}\textbf{3.3}$\bigstar$ 350 grammi di cloruro di zinco $\rm ZnCl_2$ con densità $d=2.91 g/cm^3$ (attenzione! Questa non è la densità di tutta la soluzione, ma solo del solido) vengono sciolti in 650 grammi di acqua. Si ottiene una soluzione che avrà un volume di 740 mL. Calcolare molarità, normalità, molalità, frazioni molari, percentuali in massa e percentuali in volume.

\vspace{0.2cm}Nota: $d_{\text{H}_3\text{O}}=1$, quindi 650 grammi sono anche 650 mL.

\vspace{0.2cm}\large\textbf{Svolgimento}\normalsize

\vspace{0.2cm}Calcoliamo la molarità

$$M=\frac{350 \cdot 1000}{136.28 \cdot 740}=3.4706\;mol/L$$

(Il fattore 1000 compare perché facciamo una proporzione)

Calcoliamo la normalità. In acqua il sale si scioglie così:

$$\ce{ZnCl_2 -> Zn^{2+} + 2Cl^-}$$

La carica è +2, il peso equivalente sarà la metà del peso molecolare e quindi la normalità sarà il doppio della molarità:

$$N=M \cdot 2=6.941\;eq/L$$

Calcoliamo la molalità

$$m=\frac{350 \cdot 1000}{136.28 \cdot 650}=3.9511\;mol/kg$$
\textbf{Incompleto}

\vspace{0.2cm}\textbf{3.2}$\bigstar$ Consideriamo un composto avente la seguente composizione percentuale: 14.3\% carbonio, 1.2\% idrogeno, 84.5\% cloro. 1 grammo di questo composto viene vaporizzato a $120^{\circ}C$ e alla pressione di 752 torr. In queste condizioni occupa un volume di 194 mL. Calcolare la formula del composto.

\vspace{0.2cm}\large\textbf{Svolgimento}\normalsize

\vspace{0.2cm}Le composizioni percentuali date ci dicono che su 100 grammi di composto avremo 14.3 grammi di C, 1.2 grammi di H e 84.5 grammi di Cl. Vediamo quindi, su 100 grammi di composto, quante sono le moli di carbonio, di idrogeno e di cloro

$$n_{\text{C}}=\frac{14.3}{12}=1.19
\quad;\quad
n_{\text{H}}=\frac{1.2}{1.008}=1.19
\quad;\quad
n_{\text{Cl}}=\frac{84.5}{35.453}=2.38$$

Abbiamo trovato che per 100 grammi di composto le moli di carbonio e idrogeno sono le stesse, mentre quelle di cloro sono pari al doppio. Ne segue che la \textit{formula minima}\footnote{La formula minima è una particolare formula molecolare in cui il numero di atomi di ciascun elemento costituente è ridotto al massimo comun divisore relativo.} del composto è $\rm CHCl_2$.

Il peso molecolare di tale formula è 83.916, ma abbiamo tutti i dati necessari per calcolare il peso molecolare del composto dall'equazione di stato dei gas perfetti:

$$MM=\frac{gRT}{PV} \implies MM=\frac{1 \cdot 0.082 \cdot 393.15 \cdot 760}{752 \cdot 0.194}=167.94$$

A questo punto vediamo quante volte questo peso molecolare contiene il valore del peso molecolare trovato per la formula minima. Nel nostro caso lo contiene 2 volte, ciò significa che il peso molecolare vero è il doppio di quello della formula minima, per cui la formula reale del composto è $\rm C_2H_2Cl_4$, in quanto contiene 2 volte la formula minima.

\vspace{0.2cm}\textbf{3.3}$\bigstar$ Una miscela di 0.278 grammi di etere e 0.568 grammi di cloroformio si trova alla temperatura di $100^{\circ}C$ in un contenitore di volume 1 L in cui era stato preventivamente fatto il vuoto. Calcolare le pressioni parziali dell'etere e del cloroformio.

\vspace{0.2cm}\large\textbf{Svolgimento}\normalsize

\vspace{0.2cm}

\vspace{0.2cm}\textbf{3.4}$\bigstar$ Un gas a 20°C e alla pressione di 730 torr occupa un volume di 20 L. Calcolare

\begin{enumerate}
    \item Il volume occupato quando la pressione diventa pari a 2 atm, mantenendo la temperatura costante;
    \item Il volume occupato in condizioni normali;
    \item La temperatura alla quale si deve portare il gas affinché abbia una pressione di 700 mmHg, mantenendo costante il volume.
\end{enumerate}

\vspace{0.2cm}\large\textbf{Svolgimento}\normalsize

\vspace{0.2cm} 1) Nel primo caso i dati sono

$$T_1=20^{\circ}\;C
\quad;\quad
P_1=730\;torr
\quad;\quad
V_1=20\;L$$
$$T_2=T_1
\quad;\quad
P_2=2\;atm
\quad;\quad
V_2=?$$

Visto che la temperatura è costante, l'equazione di stato diventa

$$P_1V_1=P_2V_2$$

Vogliamo trovare $V_2$, per cui

$$V_2=\frac{P_1V_1}{P_2}
=\frac{730 \cdot 20}{760 \cdot 2}=9.6052\;L$$

2) Nel secondo caso invece i dati sono

$$T_1=20^{\circ}\;C
\quad;\quad
P_1=730\;torr
\quad;\quad
V_1=20\;L$$
$$T_2=0^{\circ}\;C
\quad;\quad
P_2=1\;atm
\quad;\quad
V_2=?$$

L'equazione di stato si scriverà come

$$\frac{730 \cdot 20}{760 \cdot 293.15}=\frac{1 \cdot V_2}{273.15}$$

$$V_2=\frac{730 \cdot 20 \cdot 273.15}{760 \cdot 293.15}=17.9\;L$$

3) I dati sono

$$T_1=20^{\circ}\;C
\quad;\quad
P_1=730\;torr
\quad;\quad
V_1=20\;L$$
$$T_2=?
\quad;\quad
P_2=700\;torr
\quad;\quad
V_2=V_1$$

\vspace{0.2cm}\textbf{3.17}$\bigstar$ Una soluzione 0.1 m di acido dicloro acetico congela a 0.278° C. Calcolare la percentuale di soluto presente in soluzione come ione idrogeno $\rm H^+$, ione dicloro acetato $\rm CHCl_2COO^-$ e molecole indissociate dell'acido.

\vspace{0.2cm}\large\textbf{Svolgimento}\normalsize

$$\schemestart
\chemname{
\chemfig{H-C(-[6]H)(-[2]H)-C(=[1]O)-OH}}{acido acetico}
\hspace{1.5cm} 
\chemname{
\chemfig{Cl-C(-[6]H)(-[2]Cl)-C(=[1]O)-OH}}{acido dicloroacetico}
\schemestop$$

\vspace{0.2cm}Due idrogeni nel gruppo $\rm CH_3$ vengono sostituiti da due atomo di cloro, diventando $\rm CHCl_2COOH$. Esso è un acido organico che si dissocia liberando il protone del gruppo OH, mai quello del CH.

\vspace{0.2cm}Chiameremo per semplicità lo ione dicloro acetato con $\rm A$. Il testo ci fa capire che questa specie è un elettrolita debole, ossia si dissocia in parte, non totalmente. Quando si dissocia dà luogo allo ione $\rm H^+$ più lo ione $\rm A^-$:

$$\ce{HA <--> H^+ + A^-}$$

Siccome non si dissocia del tutto, dobbiamo calcolare il grado di dissociazione.

Sappiamo che la concentrazione è 0.1 molale, però dobbiamo valutare quanto se ne dissoci. Immaginiamo allora di partire da una soluzione che contiene una mole della specie HA e scriviamo sopra la reazione le quantità prima della dissociazione, sotto quelle dopo la dissociazione:

\begin{center}
    \begin{tabular}{ccccc}
        $1$ & & / & & /\\
        HA & \ce{<-->} & $\rm H^+$ & + & $\rm A^-$\\
        $1 - \alpha$ &  &  $\alpha$ & & $\alpha$\\
    \end{tabular}
\end{center}

Dopo che avviene la dissociazione e il sistema raggiunge l'equilibrio, la quantità che si è dissociata è pari ad $\alpha$. Se $\alpha$ sono le moli che si dissociano, alla fine avremo $1-\alpha$ moli di HA non dissociato. Delle $\alpha$ moli che si dissociano, otteniamo da ciascuna uno ione $\rm H^+$ e ione $\rm A^-$, quindi avremo $\alpha$ moli di $\rm H^+$ e $\alpha$ moli di $\rm A^-$.

Alla fine il numero totale di moli sarà la somma di tutte e tre le speci, cioè

$$n_{tot}=1 - \alpha + \alpha + \alpha=1+\alpha$$

Tale valore andrà a moltiplicare la molalità. L'espressione allora si riduce a

$$\Delta t=k_{cr} \cdot m \cdot (1 + \alpha)$$

$$\implies
0.278 = 1.86 \cdot 0.1 \cdot (1+\alpha)
\implies
\frac{0.278}{0.186}= 1 + \alpha
\implies
\alpha=\frac{0.278}{0.186} - 1$$

$$\implies \alpha=0.4946$$

Quindi si è dissociato il 49.46\%.

\vspace{0.2cm}\textbf{3.1} Calcolare la pressione osmotica, in mmHg, di una soluzione acquosa 0,001 M di acido acetico dissociato al 10\% a 25 °C.

\vspace{0.2cm}\large\textbf{Svolgimento}\normalsize

\vspace{0.2cm}innanzitutto notiamo che siamo davanti ad una specie debole, per cui il coefficiente di Van't Hoff si calcolerà come

$$i=[1 + (\gamma -1)\alpha]$$

con $\alpha$ grado di dissociazione e $\gamma$ numero di particelle in cui il composto si dissocia.

L'acido acetico in acqua si dissocia in ione $\rm H^+$ e ione acetato, per cui $\gamma=2$; inoltre il testo ci dice che $\alpha=0.1$, quindi

$$i=[1 + (2-1) \cdot 0.1]=1.1$$

La pressione osmotica sarà quindi data da

$$\Pi=1.1 \cdot cRT \cdot 760
=1.1 \cdot 0.001 \cdot 0.082 \cdot 298.15 \cdot 760=20.4388\;mmHg$$

\vspace{0.2cm}\textbf{3.2} Calcolare la pressione osmotica di una soluzione di 923 mL, contenente 4.1234 grammi di idrossido di magnesio alla temperatura di 27°C. 

\vspace{0.2cm}\large\textbf{Svolgimento}\normalsize

\vspace{0.2cm}

\vspace{0.2cm}\textbf{3.3} Calcolare la pressione osmotica di una soluzione di 875 mL, contenente 3.05 grammi di idrossido
di potassio alla temperatura di 27°C.

\vspace{0.2cm}\large\textbf{Svolgimento}\normalsize

\vspace{0.2cm}

\vspace{0.2cm}\textbf{3.4} Calcolare a 25°C la pressione osmotica di una soluzione di un acido debole 0.15 M la cui $k_a=0.17$.

\vspace{0.2cm}\large\textbf{Svolgimento}\normalsize

\vspace{0.2cm}Il testo non fornisce sufficienti dati, quindi ai fini della risoluzione del problema supporremo che l'acido debole sia monoprotico. Sotto tale ipotesi, chiamiamo HA il generico acido debole, il quale in acqua darà luogo all'equilibrio

$$\ce{HA <--> H^+ + A^-}$$

Se partiamo da una mole di acido, a fine reazione resteranno $1 - \alpha$ moli indissociate di acido e si produrranno $\alpha$ moli sia di ione $\rm H^+$ che di ione $\rm A^-$. 

La costante di dissociazione sarà allora pari, avendo una concentrazione iniziale $c$, a

$$k_a=c\,\frac{\alpha^2}{1-\alpha}
\implies
\alpha=$$

\textbf{lascia stare mi sono buggato}

\vspace{0.2cm}\textbf{3.5} 50.00 mL di una soluzione acquosa di nitrato di argento vengono trattati con un eccesso di acido
cloridrico. Si formano 1.7658 g di cloruro di argento. Calcolare la molarità della soluzione di nitrato
di argento.

\vspace{0.2cm}\large\textbf{Svolgimento}\normalsize

\vspace{0.2cm}La reazione che avviene è

$$\ce{AgNO_3 + HCl -> AgCl + HNO_3}$$

il testo ci dice che siamo in eccesso di acido cloridrico, quindi il reagente limitante sarà il nitrato d'argento, che ha quindi reagito completamente. Inoltre, poiché il rapporto stechiometrico tra questo e il cloruro d'argento è 1:1, il numero di moli di AgCl prodotte corrisponderà al numero iniziale di moli di $\rm AgNO_3$. Quindi:

$$n_{\rm AgNO_3}=n_{\rm AgCl}=\frac{g}{MM_{\rm AgCl}}
=\frac{1.7658}{143.32}=1.2321 \cdot 10^{-2}\;mol$$

La concentrazione sarà data da

$$c=\frac{n_{\rm AgCl}}{V(L)}
=\frac{1.2321 \cdot 10^{-2} \cdot 1000}{50}
=0.2464\;mol/L$$

\vspace{0.2cm}\textbf{3.6} In un recipiente di 5 litri sono introdotti 300 mL di cloro misurati a 20°C e 750 torr, 800 mL di azoto misurati a 15°C e 700 torr, 500 mL di ossigeno misurati a 18°C e 850 torr e 450 mL di anidride carbonica misurati a 20°C e 760 torr. Calcolare la pressione totale della miscela a 25°C e la frazione molare dell’ossigeno.

\vspace{0.2cm}\large\textbf{Svolgimento}\normalsize

\vspace{0.2cm}

\vspace{0.2cm}\textbf{3.7} Calcolare a 25°C la pressione osmotica di una soluzione 0.15 M di acido iodico ($k_a= 1.7 \cdot 10-1 $)

\vspace{0.2cm}\large\textbf{Svolgimento}\normalsize

\vspace{0.2cm}

\vspace{0.2cm}\textbf{3.8} In un recipiente di 1 L sono state introdotte 2.5 moli di NOCl. Dopo opportuno riscaldamento ad una fissata temperatura si realizza il seguente equilibrio:

\begin{center}
    \ce{2 NOCl <--> 2NO + Cl_2}
\end{center}

All'equilibrio sono presenti 0.86 moli di NO. Calcolare la relativa costante di equilibrio in funzione delle concentrazioni.

\vspace{0.2cm}\large\textbf{Svolgimento}\normalsize

\vspace{0.2cm}Scriviamo innanzitutto la reazione che avviene con le quantità iniziali sopra e quelle finali sotto: 

\begin{center}
    \begin{tabular}{ccccccc}
        $2.5$ & & / & & /\\
        $\rm 2NOCl$ & \ce{<-->} & $\rm 2NO$ & + & $\rm Cl_2$\\
        $1.64$ & & $0.86$ & & $0.43$\\
    \end{tabular}
\end{center}

Vediamo perché abbiamo scritto tali quantità finali:

Il testo ci dice che a fine reazione abbiamo 0.86 moli di NO, e siccome esso è stato prodotto a partire dall'NOCl di quest'ultimo ci resteranno un numero di moli pari proprio alla differenza, cioè pari a 1.64; per quanto riguarda il $\rm Cl_2$ invece abbiamo scritto 0.43 perché il suo rapporto stechiometrico con l'NO è 2:1, cioè avremo il numero di moli di tale specie sarà pari alla metà di quelle dell'NO.

Scriviamo la concentrazione di tutte le speci. Attenzione! dovremo usare le moli finali. Inoltre, siccome abbiamo un litro di soluzione, le concentrazioni coincideranno in numero col numero di moli, per cui

$$c_{\rm NOCl}=1.64\;mol/L
\quad;\quad
c_{\rm NO}=0.86\;mol/L
\quad;\quad
c_{\rm Cl_2}=0.43\;mol/L$$

La costante di equilibrio sarà data da

$$k=\rm \frac{[NO]^2 \cdot [Cl_2]}{[NOCl]^2}
=\frac{0.86^2 \cdot 0.44}{1.64^2}=1.1824 \cdot 10^{-1}$$

\vspace{0.2cm}\textbf{3.9} Sia data una soluzione composta da due liquidi A e B. Ad una data temperatura la $P^0_A = 152$ torr
e la $P^0_B = 324$ torr. Calcolare la pressione totale dei vapori di A e B in equilibrio con la soluzione sapendo che la frazione molare di A in soluzione è 0.75.

\vspace{0.2cm}\large\textbf{Svolgimento}\normalsize

\vspace{0.2cm}innanzitutto, poiché $\rchi_A=0.75$, segue che $\rchi_B=0.25$ (deve sempre valere $\rchi_A + \rchi_B =1$).

Dalla legge di Dalton unita alla legge di Raoult segue che la pressione totale dei vapori è data da

$$P=\rchi_A P_A^0 + \rchi_B P_B^0$$

da cui 

$$P=0.75 \cdot 152 + 0.25 \cdot 324=195\;torr$$

\vspace{0.2cm}\textbf{3.10} Calcolare la pressione osmotica di una soluzione di 900 mL, contenente 5.56 grammi di KOH alla temperatura di 26°C.

\vspace{0.2cm}\large\textbf{Svolgimento}\normalsize

\vspace{0.2cm}Calcoliamo innanzitutto le moli di KOH

$$n_{\rm KOH}=\frac{5.56}{56.1056}=9.9099 \cdot 10^{-2}\;mol$$

da cui si ha che la concentrazione è pari a

$$c=\frac{n}{V(L)}=\frac{9.9099 \cdot 10^{-2}}{0.9}=1.1011 \cdot 10^{-2}\;mol/L$$

Infine notiamo che l'idrossido di potassio in acqua si dissocia in uno ione $\rm K^+$ e uno ione $OH^-$, per cui il coefficiente di Van't Hoff è pari a 2. Pertanto la pressione osmotica sarà data da

$$\Pi=2cRT
=2 \cdot 1.1011 \cdot 10^{-2} \cdot 0.082 \cdot 299.15
=0.5402\;atm$$

\vspace{0.2cm}\textbf{3.11} A 400 mL di una soluzione acquosa di $\rm FeCl_3$ 0.0132 M sono stati aggiunti 350 mg di un composto organico senza variazione apprezzabile di volume. A 25 °C si misura una pressione osmotica di 1072 torr. Calcolare la massa molare del composto organico.

\vspace{0.2cm}\large\textbf{Svolgimento}\normalsize

\vspace{0.2cm}Quando in soluzione abbiamo più di un composto, la pressione osmotica finale sarà data dalla somma delle singole pressioni osmotiche. Per trovare allora la pressione osmotica del composto organico X, calcoliamo quella del $\rm FeCl_3$ e sottraiamola a quella totalmente.

Il cloruro ferrico in acqua si dissocia in uno ione ferrico $\rm Fe^{3+}$ e 3 ioni $\rm Cl^-$, per cui il coefficiente di Van't Hoff sarà pari a 4 e dunque

$$\Pi_{\rm FeCl_3}=4cRT
=4 \cdot 0.0132 \cdot 0.082 \cdot 298.15
=1.2909\;atm$$

La pressione totale in atm è pari a

$$\Pi=\frac{1072}{670}=1.6000\;atm$$

quindi la pressione osmotica del composto X sarà pari a

$$\Pi_{\rm X}=1.6000 - 1.2909 = 0.3091\;atm$$

A questo punto ricaviamo la concentrazione del composto X come

$$c=\frac{\Pi_{\rm X}}{RT}=\frac{0.3091}{0.082 \cdot 298.15}
=0.0126\;mol/L$$

Nota: i composti organici non si sciolgono in acqua, quindi $i=1$!

Calcoliamo poi le moli attraverso la proporzione

$$n_{\rm X}:400=0.0126:1000
\implies
n_{\rm X}=\frac{400 \cdot 0.0126}{1000}
=5.0400 \cdot 10^{-3}$$

Infine, la massa molare si calcolerà come

$$n_{\rm X}=\frac{g}{MM_{\rm X}}
\implies
MM_{\rm X}=\frac{g}{n_{\rm X}}=\frac{0.350}{5.0400 \cdot 10^{-3}}=69.4444$$

\vspace{0.2cm}\textbf{3.12} Un volume di 10 L di una miscela gassosa di 80 grammi alla temperatura di 90°C contiene il
30\% in peso di $\rm SO_3$ ed il 70\% di $\rm O_2$. Calcolare le pressioni parziali dei singoli gas.

\vspace{0.2cm}\large\textbf{Svolgimento}\normalsize

\vspace{0.2cm}

\vspace{0.2cm}\textbf{3.13} In un recipiente di 5.6 litri, alla temperatura di 80°C, vi sono 4,5432 g di metano, 1,2345 g di idrogeno e 14,5678 g di azoto. Calcolare i valori delle pressioni parziali dei tre gas. 

\vspace{0.2cm}\large\textbf{Svolgimento}\normalsize

\vspace{0.2cm}

\vspace{0.2cm}\textbf{3.14} Una soluzione al 5\% in peso di $\rm CaX_2$ (ove X è un alogeno), inizia a congelare a - 1,396°C. $k_cr_{\rm H_2O} = 1,860$. Qual è l'alogeno? 

\vspace{0.2cm}\large\textbf{Svolgimento}\normalsize

\vspace{0.2cm}

\vspace{0.2cm}\textbf{3.15} Calcolare la pressione osmotica di 789 ml di soluzione contenente 5.8440 grammi di NaCl e  21.2996 grammi di $\rm Al(NO_3)_3$ alla temperatura di 27°C.

\vspace{0.2cm}\large\textbf{Svolgimento}\normalsize

\vspace{0.2cm}

\vspace{0.2cm}\textbf{3.16} Calcolare la pressione osmotica di una soluzione di 850 mL contenente 89.1234 grammi di solfato di alluminio alla temperatura di 80 °C.

\vspace{0.2cm}\large\textbf{Svolgimento}\normalsize

\vspace{0.2cm}

\vspace{0.2cm}\textbf{3.17} Una soluzione di acido nitrico al 27\% in massa ha una densità di $1.16\;g/mL$. Calcolarne la molarità, molalità e temperatura di ebollizione sapendo che $k_{eb_{\rm H_2O}}=1.86\,\text{°}C \cdot kg/mol$

\vspace{0.2cm}\large\textbf{Svolgimento}\normalsize

\vspace{0.2cm}Supponiamo di avere un litro di soluzione. Essa in grammi avrà una massa pari a

$$g=V \cdot d \implies g=1000 \cdot 1.16 = 1160\;grammi$$

Di questi, solo il 27\% costituisce l'acido nitrico $\rm HNO_3$, per cui avremo che

$$g_{\rm HNO_3}=\frac{1160 \cdot 27}{100}=313.2\;grammi$$

Calcoliamo quindi il corrispondente numero di moli:

$$n_{\rm HNO_3}=\frac{g}{MM_{\rm HNO_3}}
=\frac{313.2}{63.018}=4.97\;mol$$

Avendo supposto di avere 1 litro di soluzione, esso sarà anche il valore della molarità: $M=4.97\;mol/L$.

\vspace{0.2cm}Per quanto riguarda la molalità dobbiamo stare attenti, poiché essa è definita come il rapporto tra le moli di soluto su 1000 grammi di solvente puro. Noi però abbiamo la massa della soluzione totale, ma avendo calcolato i grammi di soluto possiamo ricavare i grammi di solvente puro come

$$g_{solvente}=g_{soluzione} - g_{soluto}
\implies
1160 -313.2=846.8\;grammi$$

A questo punto possiamo ricavare la molalità attraverso la proporzione

$$m:1000=4.97:846.8
\implies
m=\frac{1000 \cdot 4.97}{846.8}=5.8691\;mol/kg$$

Avendo ricavato la molalità, possiamo calcolare la temperatura di ebollizione.

Essendo l'acido nitrico una specie forte, essa in acqua si dissocia totalmente in ione $\rm H_3O^+$ e ione $NO_3^-$, per cui il coefficiente $i$ di Van't Hoff sarà pari a 2. Dunque

$$\Delta t_{eb}=t_f - t_i=t_f - 0\,\text{°}C=t_f$$

$$\implies t_f=k_{eb} \cdot m \cdot 2
=1.86 \cdot 5.8691 \cdot 2=21.83\,\text{°}C$$