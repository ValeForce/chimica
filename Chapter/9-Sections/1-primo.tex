\subsection{Reazioni di combustione}
\textbf{1.1.1} Calcolare il volume di anidride carbonica, misurato a 750°C ed 1.5 atm di pressione, liberata dalla combustione di 7.1234 grammi di $\rm C_7H_16$ in aria.

\vspace{0.2cm}\textbf{1.1.2} Quanti grammi di anidride carbonica si sviluppano per combustione in eccesso di ossigeno di 3
litri di pentano, prelevati a T = 25 °C e 740 Torr?

\vspace{0.2cm}\textbf{1.1.3} La combustione in eccesso di ossigeno di 12.5 g di un minerale che contiene il 70\% di FeS
produce anidride solforica la quale viene fatta gorgogliare in acqua. La soluzione risultante avrà un
volume totale di 855 mL. Calcolare la molarità della soluzione di acido solforico ottenuta.

\vspace{0.2cm}\textbf{1.1.4} 

\vspace{0.2cm}\textbf{1.1.5} 
\subsection{Reazioni varie}
\textbf{1.2.1} Quanti grammi di idrossido di potassio sono necessari per neutralizzare totalmente 870 ml di una soluzione 0.015 M di acido solforico ? Quali sono i prodotti di reazione ?

\vspace{0.2cm}\textbf{1.2.} Ad un campione di g 0.744 di solfato di magnesio idrato, disciolto in acqua, fu aggiunto cloruro di bario in eccesso ottenendo un precipitato di solfato di bario che, dopo filtrazione ed essiccazione, pesava 0.70 g. Determinare il numero di molecole di acqua contenute nella formula del solfato di magnesio idrato.

\vspace{0.2cm}\textbf{1.2.} Lo zinco in acido cloridrico produce cloruro di zinco ed idrogeno. Calcolare il volume di
idrogeno misurato a c.n. ed i grammi di cloruro di zinco che si ottengono per reazione di 2.42 g di zinco con 100 mL di una soluzione acquosa di HCl 0.5 M.

\vspace{0.2cm}\textbf{1.2.} Quanti grammi di ossigeno sono necessari per trasformare 1.2461 g di potassio in ossido?

\vspace{0.2cm}\textbf{1.2.} Il solfuro ferroso reagisce con l'acido cloridrico liberando solfuro di idrogeno allo stato gassoso. Quale volume di una soluzione di HCl 0.1 N è necessario per la reazione completa con 5.4637 gr di solfuro ferroso?

\vspace{0.2cm}\textbf{1.2.} Una lamina di magnesio di 2.1234 g si scioglie completamente in una soluzione di acido cloridrico. Quanti litri di idrogeno si sviluppano se misurati a 25°C e 1140 torr di pressione?

\vspace{0.2cm}\textbf{1.2.} 
