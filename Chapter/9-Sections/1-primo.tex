\subsection{Reazioni di combustione}
Le reazioni di combustione sono reazioni in cui un composti reagisce con ossigeno per produrre acqua e anidride carbonica.

Per bilanciare una reazione di combustione, innanzitutto bilanciamo gli atomi di carbonio (C), dopodiché bilanciamo gli atomi di idrogeno (H) e infine bilanciamo gli atomi di ossigeno (O). Facciamo un esempio: consideriamo la reazione

$$\ce{C_3H_8 + O_2 -> CO_2 + H_2O}$$

A sinistra abbiamo 3 atomi di carbonio, quindi moltiplichiamo $\rm CO_2$ per 3:

$$\ce{C_3H_8 + O_2 -> 3CO_2 + H_2O}$$

Abbiamo poi 8 atomi di idrogeno a primo membro, quindi ne dobbiamo avere altrettanti a destra. poiché la specie è $\rm H_2O$, moltiplichiamo semplicemente per 4

$$\ce{C_3H_8 + O_2 -> 3CO_2 + 4H_2O}$$

Infine notiamo che a destra abbiamo $3 \times 2 + 4 \times 1=10$ atomi di ossigeno, che dobbiamo avere pure a sinistra. Visto che la specie è $\rm O_2$, moltiplichiamo solo per 5:

$$\ce{C_3H_8 + 5O_2 -> 3CO_2 + 4H_2O}$$

\vspace{0.2cm}\textbf{1.1.1}$\bigstar$ Supponiamo di avere 0.2094 grammi di un composto contenente carbonio. Sappiamo che da esso si ottengono 0.5328 grammi di biossido di carbonio:

$$\ce{C -> CO_2}$$

(abbiamo cioè un composto che contiene carbonio ma non sappiamo quanto, sarà una parte della massa. Alla fine dopo varie reazioni si ottiene quella massa di anidride carbonica)

Qual è la percentuale di carbonio nel campione originale?

\vspace{0.2cm}\large\textbf{Svolgimento}\normalsize

\vspace{0.2cm}L'unica cosa che al momento possiamo scrivere è che, qualunque sia il composto, abbiamo C e avremo $\rm CO_2$, per cui il rapporto stechiometrico tra C e $\rm CO_2$ è 1:1, e dunque una mole di carbonio darà una mole di $\rm CO_2$ (in generale avremo tante moli di $\rm CO_2$ quante sono le molecole di C).

Iniziamo a calcolare il numero di moli di $\rm CO_2$ ottenute, in quanto conosciamo sia la massa che la massa molare di carbonio e ossigeno. avremo

$$MM_{\text{CO}_2}=2 \cdot 15.9994 + 12 = 44.0098 \approx 44$$

Le moli di $\rm CO_2$ saranno date dal rapporto tra la massa e il peso molecolare:

$$n=\frac{0.5328}{44.0098}=0.012106=1.2106 \cdot 10^{-2} \; mol$$

Questo sarà anche il numero di moli di carbonio in quanto c'è lo stesso coefficiente stechiometrico.

Per ottenere i grammi di carbonio che hanno reagito, basterà moltiplicare il numero di moli per la massa dell'atomo di carbonio

$$m=1.2106 \cdot 10^{-2} \cdot 12.011=0.14541 \; gr$$

Notiamo che questi grammi sono inferiori ai grammi del composto di partenza. Se volessimo calcolare la percentuale di carbonio presente nel composto di partenza avremo

$$\%\,C=\frac{0.14541}{0.2094}\cdot 100 = 69.44 \%$$

In questo modo calcoliamo la percentuale di carbonio contenuta in una data massa di un composto contenente sia carbonio che altro.

\textbf{1.1.1} Calcolare il volume di anidride carbonica, misurato a 750°C ed 1.5 atm di pressione, liberata dalla combustione di 7.1234 grammi di $\rm C_7H_{16}$ in aria.

\vspace{0.2cm}\textbf{1.1.2} Quanti grammi di anidride carbonica si sviluppano per combustione in eccesso di ossigeno di 3
litri di pentano, prelevati a T = 25 °C e 740 Torr?

\vspace{0.2cm}\textbf{1.1.3} La combustione in eccesso di ossigeno di 12.5 g di un minerale che contiene il 70\% di FeS
produce anidride solforica la quale viene fatta gorgogliare in acqua. La soluzione risultante avrà un
volume totale di 855 mL. Calcolare la molarità della soluzione di acido solforico ottenuta.

\vspace{0.2cm}\textbf{1.1.4} 

\vspace{0.2cm}\textbf{1.1.5} 
\subsection{Reazioni varie}

\textbf{1.2.1}$\bigstar$ La soda caustica (idrossido di sodio NaOH) viene commercialmente preparata dalla reazione fra carbonato di sodio ($\rm Na_2CO_3$) e calce spenta (idrossido di calcio $\rm Ca(OH)_2$):

$$\ce{Na_2CO_3 + Ca(OH)_2 -> NaOH + CaCO_3}$$
$$\text{Carbonato di sodio} + \text{Idrossido di calcio} \ce{->} \text{Idrossido di sodio} + \text{Carbonato di calcio}$$

Bilanciata, la reazione sarà

$$\ce{Na_2CO_3 + Ca(OH)_2 -> 2NaOH + CaCO_3}$$

Si ha un chilo di carbonato di sodio e lo si fa reagire con tutta la quantità necessaria di idrossido di calcio. Quanto idrossido di sodio si ottiene?

\vspace{0.2cm}\large\textbf{Svolgimento}\normalsize

\vspace{0.2cm}Questa reazione ci dice che 1 mole di carbonato di sodio ci darà 2 moli di idrossido di sodio, quindi l'unica cosa da fare è calcolare le moli di carbonato di sodio, moltiplicarle per 2 in modo da ottenere le moli di idrossido di sodio e quindi moltiplicare quest'ultime per il peso molecolare dell'idrossido di sodio in modo da ottenere i grammi.

Le moli del carbonato di sodio si ottengono dividendone la massa per il peso molecolare che assumiamo essere pari a 106, quindi

$$n_{\text{NaCO}_3}=\frac{1000}{106}=9.434 \; mol$$

$$\implies n_{\text{NaOH}}=2 \cdot 9.434=18.868 \approx 18.87 \; mol$$

I grammi di NaOH saranno dati dal prodotto del numero di moli per il peso molecolare dell'NaOH che è circa 40:

$$g_{\text{NaOH}}=18.87 \cdot 40 \approx 755 g$$


\textbf{1.2.2}$\bigstar$ Quanto cloruro di argento precipita \footnote{Significa che avevamo delle soluzioni limpide e otteniamo una polvere che va subito a fondo del becker.} quando si mescola una soluzione contenente 5.76 grammi di nitrato di argento ($\rm AgNO_3$) con un'altra contenente 1.76 grammi di cloruro di sodio?

\vspace{0.2cm}\large\textbf{Svolgimento}\normalsize

\vspace{0.2cm}La reazione che avviene è

$$\ce{AgNO_3 + NaCl -> AgCl v + NaNO_3}$$
$$\text{Nitrato d'argento} + \text{Cloruro di sodio} \ce{->} \text{Cloruro d'argento} + \text{Nitrato di sodio}$$

La freccia rivolta verso il basso accanto al cloruro d'argento indica che tale composto è precipitato.

L'AgCl quindi si separa dalla soluzione e diventa una gelatina bianca che va al fondo. Inoltre nitrato d'argento, cloruro di sodio e nitrato di sodio sono dissociati in acqua, per cui accanto dovremmo scrivere (aq). \textbf{DA FINIRE}

La reazione ci dà un'informazione: la sua stechiometria è 1:1:1:1, cioè una mole di nitrato d'argento reagirà con una mole di cloruro di sodio per produrre una mole di cloruro d'argento ed una mole di nitrato di sodio. Ma abbiamo le stesse moli per i due reagenti?

I grammi cambiano perché il peso molecolare è diverso, ma dobbiamo calcolarlo.

La prima cosa che facciamo quindi è calcolare le moli di nitrato d'argento:

$$n_{\text{AgNO}_3}=\frac{5.76}{169}=3.41 \cdot 10^{-2} \; mol $$

Adesso calcoliamo le moli di cloruro di sodio:

$$n_{\text{NaCl}}=\frac{1.76}{58.5}=3.0 \cdot 10^{-2} \; mol $$

Le moli sono diverse: non sono in rapporto 1:1, pertanto il cloruro di sodio sarà in difetto e verrà consumato totalmente; al contrario, il nitrato d'argento è in eccesso e pertanto parte di questo non reagirà in quanto non avrà cloruro di sodio a sufficientza. Possiamo quindi dire che il cloruro di sodio in difetto viene consumato tutto, producendo lo stesso numero di moli (per via della stechiometria) di cloruro di argento. Ne segue che avremo $3 \cdot 10^{-2}$ moli di cloruro d'argento prodotte. Per il nitrato d'argento abbiamo un eccesso di $4.1 \cdot 10^{-3}$ moli.

Nota: in alcuni testi il reagente in difetto viene chiamato agente limitante, perché non fa reagire parte del nitrato d'argento, cioè ne limita la reazione completa.

Dobbiamo calcolare i grammi di cloruro di argento: essi valgono

$$g_{\text{AgCl}}=3.0 \cdot 10^{-2} \cdot 142=4.26$$

\textbf{1.2.} Una miscela di ossido di magnesio (MgO) e carbonato di magnesio ($\rm MgCO_3$) riscaldato a 1000° C ha una perdita in peso del 32\%. Calcolarne la composizione.

\vspace{0.2cm}\large\textbf{Svolgimento}\normalsize

\vspace{0.2cm}Quello che succede a 1000° C è che il carbonato di magnesio si decomporrà in ossido, che si aggiungerà a quello che c'era già nella miscela, e in anidride carbonica che invece è un gas e quindi andrà via

Quando si parla di calore in genere si mette il $\Delta$ sopra la freccia, per cui la reazione che avviene è

$$\ce{MgO + MgCO_3 ->[{\Delta}][{1000^{\circ}C}] MgO + CO_2 ^}$$

Il testo non ci dice quanta sostanza abbiamo all'inizio, ma solo la parte che ne abbiamo perso. Se ad esempio ne avessimo 100 grammi, significherebbe perdere 32 grammi e restare con 68 grammi di miscela. Siccome l'unico composto che si può perdere è la $\rm CO_2$, significa che la miscela conteneva 32 grammi di $\rm CO_2$ per ogni 100 grammi di miscela iniziale. La $\rm CO_2$ proviene solo dal carbonato, per cui vogliamo calcolare le moli di $\rm CO_2$ perse portado a 1000°C 100 grammi di questa miscela:

$$n_{\text{CO}_2}=\frac{32}{44}=n_{\text{MgCO}_3}$$

Per la stechiometria, il numero di moli di $\rm CO_2$ sarà uguale al numero di moli di carbonato di magnesio, perché se abbiamo le moli possiamo scrivere

$$g_{\text{MgCO}_3}=\frac{32}{44} \cdot MM_{\text{MgCO}_3}
=\frac{32}{44} \cdot 84.32=61.3$$

Questo significa che la miscela, di cui abbiamo immaginato averne 100 grammi, sarà composta da 61.3 grammi di carbonato di magnesio e da 38.7 grammi di ossido di magnesio.

\vspace{0.2cm}\textbf{1.2.} Data la reazione

$$\ce{NaOH + H_2SO_4 -> Na_2SO_4 + H_2O}$$

Calcolare i grammi dei prodotti ottenuti, partendo da 0.7521 grammi di idrossido di sodio e 0.4613 grammi di acido solforico.

\vspace{0.2cm}\large\textbf{Svolgimento}\normalsize

\vspace{0.2cm}Bilanciamo innanzitutto la reazione:

$$\ce{2NaOH + H_2SO_4 -> Na_2SO_4 + H_2O}$$

La reazione ci dice che due moli di idrossido di sodio reagiranno con una di acido solforico 

\textbf{1.2.} Quanti grammi di idrossido di potassio sono necessari per neutralizzare totalmente 870 ml di una soluzione 0.015 M di acido solforico ? Quali sono i prodotti di reazione ?

\vspace{0.2cm}\textbf{1.2.} Ad un campione di g 0.744 di solfato di magnesio idrato, disciolto in acqua, fu aggiunto cloruro di bario in eccesso ottenendo un precipitato di solfato di bario che, dopo filtrazione ed essiccazione, pesava 0.70 g. Determinare il numero di molecole di acqua contenute nella formula del solfato di magnesio idrato.

\vspace{0.2cm}\textbf{1.2.} Lo zinco in acido cloridrico produce cloruro di zinco ed idrogeno. Calcolare il volume di
idrogeno misurato a c.n. ed i grammi di cloruro di zinco che si ottengono per reazione di 2.42 g di zinco con 100 mL di una soluzione acquosa di HCl 0.5 M.

\vspace{0.2cm}\textbf{1.2.} Quanti grammi di ossigeno sono necessari per trasformare 1.2461 g di potassio in ossido?

\vspace{0.2cm}\textbf{1.2.} Il solfuro ferroso reagisce con l'acido cloridrico liberando solfuro di idrogeno allo stato gassoso. Quale volume di una soluzione di HCl 0.1 N è necessario per la reazione completa con 5.4637 gr di solfuro ferroso?

\vspace{0.2cm}\textbf{1.2.} Una lamina di magnesio di 2.1234 g si scioglie completamente in una soluzione di acido cloridrico. Quanti litri di idrogeno si sviluppano se misurati a 25°C e 1140 torr di pressione?

\vspace{0.2cm}\textbf{1.2.}Il solfuro ferroso reagisce con l'acido cloridrico liberando solfuro di idrogeno allo stato gassoso. Quale volume di una soluzione di HCl 0.1 N è necessario per la reazione completa con 5.4637 gr di solfuro ferroso?

\vspace{0.2cm}\large\textbf{Svolgimento}\normalsize

\vspace{0.2cm}La reazione che avvviene è

$$\ce{FeS + 2HCl -> H_2S ^ FeCl_2}$$
