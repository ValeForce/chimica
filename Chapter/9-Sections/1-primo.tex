\subsection{Reazioni di combustione}
La combustione è una reazione chimica che comporta l'ossidazione di un combustibile da parte di un comburente (che in genere è rappresentato dall'ossigeno presente nell'aria). Nella maggior parte dei casi considereremo combustioni in cui un composto reagisce con ossigeno per produrre acqua e anidride carbonica.

Per bilanciare una reazione di combustione, innanzitutto bilanciamo gli atomi di carbonio (C), dopodiché bilanciamo gli atomi di idrogeno (H) e infine bilanciamo gli atomi di ossigeno (O). Facciamo un esempio: consideriamo la reazione

$$\ce{C_3H_8 + O_2 -> CO_2 + H_2O}$$

A sinistra abbiamo 3 atomi di carbonio, quindi moltiplichiamo $\rm CO_2$ per 3:

$$\ce{C_3H_8 + O_2 -> 3CO_2 + H_2O}$$

Abbiamo poi 8 atomi di idrogeno a primo membro, quindi ne dobbiamo avere altrettanti a destra. poiché la specie è $\rm H_2O$, moltiplichiamo semplicemente per 4

$$\ce{C_3H_8 + O_2 -> 3CO_2 + 4H_2O}$$

Infine notiamo che a destra abbiamo $3 \times 2 + 4 \times 1=10$ atomi di ossigeno, che dobbiamo avere pure a sinistra. Visto che la specie è $\rm O_2$, moltiplichiamo solo per 5:

$$\ce{C_3H_8 + 5O_2 -> 3CO_2 + 4H_2O}$$

\vspace{0.2cm}\textbf{1.1.1}$\bigstar$ Supponiamo di avere 0.2094 grammi di un composto contenente carbonio. Sappiamo che da esso si ottengono 0.5328 grammi di biossido di carbonio:

$$\ce{C -> CO_2}$$

(abbiamo cioè un composto che contiene carbonio ma non sappiamo quanto, sarà una parte della massa. Alla fine dopo varie reazioni si ottiene quella massa di anidride carbonica)

Qual è la percentuale di carbonio nel campione originale?

\vspace{0.2cm}\large\textbf{Svolgimento}\normalsize

\vspace{0.2cm}L'unica cosa che al momento possiamo scrivere è che, qualunque sia il composto, abbiamo C e avremo $\rm CO_2$, per cui il rapporto stechiometrico tra C e $\rm CO_2$ è 1:1, e dunque una mole di carbonio darà una mole di $\rm CO_2$ (in generale avremo tante moli di $\rm CO_2$ quante sono le molecole di C).

Iniziamo a calcolare il numero di moli di $\rm CO_2$ ottenute, in quanto conosciamo sia la massa che la massa molare di carbonio e ossigeno. avremo

$$MM_{\text{CO}_2}=2 \cdot 15.9994 + 12 = 44.0098 \approx 44$$

Le moli di $\rm CO_2$ saranno date dal rapporto tra la massa e il peso molecolare:

$$n=\frac{0.5328}{44.0098}=0.012106=1.2106 \cdot 10^{-2} \; mol$$

Questo sarà anche il numero di moli di carbonio in quanto c'è lo stesso coefficiente stechiometrico.

Per ottenere i grammi di carbonio che hanno reagito, basterà moltiplicare il numero di moli per la massa dell'atomo di carbonio

$$m=1.2106 \cdot 10^{-2} \cdot 12.011=0.14541 \; gr$$

Notiamo che questi grammi sono inferiori ai grammi del composto di partenza. Se volessimo calcolare la percentuale di carbonio presente nel composto di partenza avremo

$$\%\,\text{C}=\frac{0.14541}{0.2094}\cdot 100 = 69.44 \%$$

In questo modo calcoliamo la percentuale di carbonio contenuta in una data massa di un composto contenente sia carbonio che altro.

\vspace{0.2cm}\textbf{1.2.3}$\bigstar$ Una miscela gassosa è costituita da ossido di carbonio CO, metano $\rm CH_4$ e acetilene $\rm C_2H_2$. Il suo volume totale è 50 mL. A questa miscela vengono aggiunti 150 mL di ossigeno e il tutto viene fatto esplodere\footnote{Nota: ciò che finora ha detto il testo è che "aggiungiamo" 150 mL di ossigeno e non che "sono necessari". Bisogna allora capire nel seguito del testo se questo ossigeno è insufficiente, è esattamente quello necessario o è in eccesso.}. Le tre reazioni saranno

$$\ce{2CO + O_2 -> 2CO_2}$$
$$\ce{CH_4 + O_2 -> CO_2 + 2H_2O}$$
$$\ce{2C_2H_2 + 5O_2 -> 4CO_2 + 2H_2O}$$

Poniamo $x=V_{\text{CO}}$, $y=V_{\text{CH}_4}$, $z=V_{\text{C}_2\text{H}_2}$. Una prima equazione relativa al volume sarà

$$x+y+z=50 \; mL$$

I gas post-combustione vengono raffreddati.

Attenzione: mentre a seguito del raffreddamento l'anidride carbonica resta gassosa, l'acqua condensa. Ne segue che nel misurare il volume dei gas post-raffreddamento l'acqua non sarà considerata.

Il volume post-raffreddamento è di 122.5 mL.

Questo gas (che poi nei fatti è $\rm CO_2$ ed eventualmente ossigeno in eccesso) viene fatto assorbire su potassa, ossia l'anidride carbonica, che è un ossido acido, viene fatto reagire con idrossido di potassio KOH per formare il sale carbonato di potassio che è detto potassa:

$$\ce{CO_2 + 2KOH -> K_2CO_3 + H_2O}$$

(In questo modo sottraiamo tutta la $\rm CO_2$ perché il testo dice che viene assorbita)

Il volume residuo sarà di 57.5 mL. Calcolare i volumi finali dei tre composti.

\vspace{0.2cm}\large\textbf{Svolgimento}\normalsize

\vspace{0.2cm} Dal testo educiamo che avevamo ossigeno in eccesso perché i prodotti di reazione sono solo anidride carbonica e acqua, ma avendo fatto raffreddare i gas l'acqua è condensata e avremmo dovuto avere solo $\rm CO_2$ e quindi del volume di $122.5$ mL dopo aver fatto assorbire questo gas in una soluzione di KOH non avremmo dovuto trovare nulla di più perché la $\rm CO_2$ viene fatta reagire tutta, per cui il volume che troviamo è dell'ossigeno in eccesso. Ne segue che possiamo calcolare il volume della $\rm CO_2$ come differenza tra il volume finale ed il volume dell'ossigeno:

$$V_{\rm CO_2}=V_f - V_{\rm O_2}=122.5 -57.5=65\;mL$$

Questa sarà la $\rm CO_2$ prodotta.

A questo punto, se abbiamo messo $150\;mL$ di ossigeno e ne abbiamo un eccesso di 57.5, l'ossigeno utilizzato sarà

$$\rm O{_2}(consumato)=150-57.5=92.5\;mL$$

cioè quello che reagisce.

Possiamo scrivere un'equazione per il volume dell'ossigeno consumato

$$\frac{1}{2}x + 2y + \frac{5}{2}z=92.5$$

e una per l'anidride carbonica prodotta:

$$x+y+2z=65$$

Mettendo insieme le equazioni avremo:

$$\begin{cases}
    x+y+z=50\\
    \displaystyle \frac{1}{2}x + 2y + \displaystyle \frac{5}{2}z=92.5\\
    x+y+2z=65
\end{cases}$$

$$\begin{cases}
    x+y+z=50\\
    x + 4y + 5z=185\\
    x+y+2z=65
\end{cases}
\implies
\begin{cases}
    -3y-4z=-135\\
    3y+3z=120
\end{cases}$$

$$\begin{cases}
    y=40-z\\
    3(40-z) + 4z=135
\end{cases}
\implies
\begin{cases}
    y=40-z\\
    120 - 3z + 4z=135
\end{cases}$$

$$\implies
\begin{cases}
    x=10\;mL\\
    y=25\;mL\\
    z=15\;mL
\end{cases}$$

\vspace{0.2cm}\textbf{1.1.4}$\bigstar$ Una miscela costituita da volumi uguali di idrogeno e metano viene bruciata all'aria\footnote{Significa bruciare in presenza di ossigeno}. Che volume di aria è necessario per la combustione completa di 1 L della miscela?

\vspace{0.2cm}\large\textbf{Svolgimento}\normalsize

\vspace{0.2cm}La reazione che avviene è

$$\ce{H_2 + CH_4 + O_2 -> CO_2 + H_2O}$$

Abbiamo un atomo di carbonio sia a destra che a sinistra, quindi passiamo direttamente a bilanciare gli idrogeni: ne abbiamo 2 dalla molecola di idrogeno e 4 dal metano, per un totale di 6; siccome a destra la specie è $\rm H_2O$ moltiplichiamo per 3:

$$\ce{H_2 + CH_4 + O_2 -> CO_2 + 3H_2O}$$

Bilanciamo gli ossigeni: a destra ne abbiamo 2 dall'anidride carbonica e 3 dall'acqua, per un totale di 5. Siccome la specie è $\rm O_2$ ma il numero per cui dobbiamo moltiplicare è dispari, moltiplichiamo per $\frac{5}{2}$:

$$\ce{H_2 + CH_4 + \frac{5}{2}O_2 -> CO_2 + 3H_2O}$$

oppure, in alternativa, moltiplichiamo tutte le speci per 2, in modo da avere solo coefficienti interi

$$\ce{2H_2 + 2CH_4 + 5O_2 -> 2CO_2 + 6H_2O}$$

Avere un litro di miscela significa avere mezzo litro di ossigeno e mezzo litro di metano: $V_{\text{CH}_4}=V_{\text{H}_2}=0.5\;L$. Dobbiamo calcolare il volume di ossigeno, ricordandoci che solo il 20\% circa dell'aria è composto da ossigeno mentre l'80\% è azoto, per cui poi dovremo prendere un volume 5 volte superiore.

Avendo 0.5 L di reagente e avendo un rapporto stechiometrico di 2:5, sarà

$$V_{\text{O}_2}=V_{\text{H}_2}\cdot\frac{5}{2}=1.25\;L$$

Il volume di aria necessario sarà

$$V=1.25 \cdot 5 = 6.25 \; L$$

\vspace{0.2cm}\textbf{1.1.1} Calcolare il volume di anidride carbonica, misurato a 750°C ed 1.5 atm di pressione, liberata dalla combustione di 7.1234 grammi di $\rm C_7H_{16}$ in aria.

\vspace{0.2cm}\large\textbf{Svolgimento}\normalsize

\vspace{0.2cm}La reazione che avviene è

$$\ce{C_7H_{16} + O_2 -> CO_2 + H_2O}$$

Bilanciamo gli atomi di carbonio. Ne abbiamo 7 nei reagenti, per cui moltiplichiamo per 7 la $\rm CO_2$

$$\ce{C_7H_{16} + O_2 -> 7CO_2 + H_2O}$$

Bilanciamo gli atomi di idrogeno. Tra i reagenti ne abbiamo 16 atomi, essendo la specie $\rm H_2O$ moltiplichiamo per 8

$$\ce{C_7H_{16} + O_2 -> 7CO_2 + 8H_2O}$$

Infine bilanciamo gli ossigeni. A destra abbiamo $7 \times 2 + 8=22$ atomi di ossigeni, essendo la specie $\rm O_2$ moltiplichiamo per 11:

$$\ce{C_7H_{16} + 11O_2 -> 7CO_2 + 8H_2O}$$

A questo punto notiamo che il rapporto stechiometrico tra $\rm C_7H_{16}$ e $\rm CO_2$ è 1:7, per cui ad una mole di $\rm C_7H_{16}$ corrisponderanno 7 moli di $\rm CO_2$. Calcoliamo le moli con la definizione:

$$n_{\text{C}_7\text{H}_{16}}=\frac{gr}{MM_{\text{C}_7\text{H}_{16}}}
=\frac{7.1234}{100.201}=7.1091 \cdot 10^{-2}$$

$$\implies
n_{\rm CO_2}
=7 \cdot 7.1091 \cdot 10^{-2}=0.4976\;mol$$

Il volume di $\rm CO_2$ sarà ottenuto tramite l'equazione di stato dei gas:

$$V_{\rm CO_2}=\frac{n_{\rm CO_2}RT}{P}
=\frac{0.4976 \cdot 0.082 \cdot 1023.15}{1.5}=27.8340\;L$$

\vspace{0.2cm}\textbf{1.1.2} Quanti grammi di anidride carbonica si sviluppano per combustione in eccesso di ossigeno di 3 litri di pentano, prelevati a T = 25 °C e 740 Torr?

\vspace{0.2cm}\large\textbf{Svolgimento}\normalsize

\vspace{0.2cm}La reazione che avviene è

$$\ce{ C_5H_{12} + O_2 -> CO_2 + H_2O}$$

A sinistra abbiamo 5 atomi di carbonio, quindi moltiplichiamo la $\rm CO_2$ per 5:

$$\ce{ C_5H_{12} + O_2 -> 5CO_2 + H_2O}$$

Abbiamo poi 12 atomi di idrogeno tra i reagenti. Siccome la specie è $\rm H_2O$, moltiplichiamo solo per 6:

$$\ce{ C_5H_{12} + O_2 -> 5CO_2 + 6H_2O}$$

Infine abbiamo 10 ossigeni proveniente dalla $\rm CO_2$ e 6 proveniente dall' $\rm H_2O$, per un totale di 16. Siccome la specie è $\rm O_2$, moltiplichiamo per 8:

$$\ce{ C_5H_{12} + 8O_2 -> 5CO_2 + 6H_2O}$$

Nota: quando il testo dice che abbiamo eccesso di ossigeno significa che non abbiamo problemi di quantità.

\vspace{0.2cm}Il rapporto stechiometrico tra pentano e anidride carbonica è 1:5, per cui 3 litri di $\rm C_5H_{12}$ produrranno 15 litri di $\rm CO_2$. Calcoliamo allora le moli di $\rm CO_2$ attraverso l'equazione di stato dei gas:

$$PV=nRT
\implies
n=\frac{PV}{RT}$$

$$\implies n_{\rm CO_2}
=\frac{740 \cdot 15}{760 \cdot 0.082 \cdot 298.15}
=0.5974\;mol$$

i grammi di $\rm CO_2$ saranno dati da

$$n=\frac{g}{MM}
\implies
g=n \cdot MM$$

$$\implies g_{\rm CO_2}=0.5974 \cdot 44.01=26.2916$$

\vspace{0.2cm}\textbf{1.1.3} La combustione in eccesso di ossigeno di 12.5 g di un minerale che contiene il 70\% di FeS
produce anidride solforica la quale viene fatta gorgogliare in acqua. La soluzione risultante avrà un
volume totale di 855 mL. Calcolare la molarità della soluzione di acido solforico ottenuta.

\vspace{0.2cm}\large\textbf{Svolgimento}\normalsize

\vspace{0.2cm}Nelle combustioni ciò che fa l'ossigeno è ossidare il composto con cui reagisce. Sappiamo che la reazione produce anidride solforica $\rm SO_3$, il che significa che lo zolfo, ossidandosi, passa da n.o. -2 a n.o. +6; per quanto riguarda il ferro invece, esso può avere n.o. +2 (che ha nel solfato) oppure ossindadosi n.o. +3. Ne segue che il ferro si lega all'ossigeno per dare luogo all'ossido ferrico $\rm Fe_2O_3$. La reazione che avviene allora sarà

$$\ce{FeS + O_2 -> Fe_2O_3 + SO_3}$$

Per bilanciare, notiamo innanzitutto che a destra abbiamo due atomi di ferro, per cui dobbiamo moltiplicare FeS per 2. In questa maniera otterremo anche due atomi di zolfo, quindi moltiplichiamo la specie $\rm SO_3$ per 2:

$$\ce{2FeS + O_2 -> Fe_2O_3 + 2SO_3}$$

Rimangono da bilanciare gli atomi di ossigeno. Contiamo allora quanti ne abbiamo tra i prodotti: tre provengono dal solfato ferrico, 6 dall'anidride solforica, per un totale di 9; siccome la specie a sinistra è $\rm O_2$, moltiplichiamo per $\frac{9}{2}$

$$\ce{2FeS + \frac{9}{2}O_2 -> Fe_2O_3 + 2SO_3}$$

Oppure, per non lavorare con coefficienti frazionari, moltiplichiamo tutti i coefficienti per 2, ottenendo

$$\ce{4FeS + 9O_2 -> 2Fe_2O_3 + 4SO_3}$$

Il rapporto stechiometrico tra solfuro ferroso e anidride solforica è 4:4, cioè 1:1. Calcoliamo allora le moli, facendo attenzione al fatto che solo il 70\% della massa totale è costituito da FeS:

$$g_{\rm FeS}=\frac{70 \cdot 12.5}{100}=8.75\;grammi$$

$$\implies
n_{\rm FeS}=\frac{8.75}{87.91}=9.9534 \cdot 10^{-2}\;mol$$

Esse saranno anche il numero di moli di $\rm SO_3$. Quest'ultima, immersa in acqua, dà luogo ad acido solforico:

$$\ce{SO_3 + H_2O -> H_2SO_4}$$

la molarità di tale acido sarà data dalla proporzione

$$9.9534 \cdot 10^{-2}:855=M:1000
\implies
M=\frac{9.9534 \cdot 1000}{855}=1.1641\;mol/L$$

\subsection{Reazioni varie}

\large\textbf{Svolgimento}\normalsize

\vspace{0.2cm}

\textbf{1.2.1}$\bigstar$ La soda caustica (idrossido di sodio NaOH) viene commercialmente preparata dalla reazione fra carbonato di sodio ($\rm Na_2CO_3$) e calce spenta (idrossido di calcio $\rm Ca(OH)_2$):

$$\ce{Na_2CO_3 + Ca(OH)_2 -> NaOH + CaCO_3}$$
$$\text{Carbonato di sodio} + \text{Idrossido di calcio} \ce{->} \text{Idrossido di sodio} + \text{Carbonato di calcio}$$

Bilanciata, la reazione sarà

$$\ce{Na_2CO_3 + Ca(OH)_2 -> 2NaOH + CaCO_3}$$

Si ha un chilo di carbonato di sodio e lo si fa reagire con tutta la quantità necessaria di idrossido di calcio. Quanto idrossido di sodio si ottiene?

\vspace{0.2cm}\large\textbf{Svolgimento}\normalsize

\vspace{0.2cm}Questa reazione ci dice che 1 mole di carbonato di sodio ci darà 2 moli di idrossido di sodio, quindi l'unica cosa da fare è calcolare le moli di carbonato di sodio, moltiplicarle per 2 in modo da ottenere le moli di idrossido di sodio e quindi moltiplicare quest'ultime per il peso molecolare dell'idrossido di sodio in modo da ottenere i grammi.

Le moli del carbonato di sodio si ottengono dividendone la massa per il peso molecolare che assumiamo essere pari a 106, quindi

$$n_{\rm Na_2CO_3}=\frac{1000}{106}=9.434 \; mol$$

$$\implies n_{\text{NaOH}}=2 \cdot 9.434=18.868 \approx 18.87 \; mol$$

I grammi di NaOH saranno dati dal prodotto del numero di moli per il peso molecolare dell'NaOH che è circa 40:

$$g_{\text{NaOH}}=18.87 \cdot 40 \approx 755 g$$


\textbf{1.2.2}$\bigstar$ Quanto cloruro di argento precipita\footnote{Significa che avevamo delle soluzioni limpide e otteniamo una polvere che va subito a fondo del becker.} quando si mescola una soluzione contenente 5.76 grammi di nitrato di argento ($\rm AgNO_3$) con un'altra contenente 1.76 grammi di cloruro di sodio?

\vspace{0.2cm}\large\textbf{Svolgimento}\normalsize

\vspace{0.2cm}La reazione che avviene è

$$\ce{AgNO_3 + NaCl -> AgCl v + NaNO_3}$$
$$\text{Nitrato d'argento} + \text{Cloruro di sodio} \ce{->} \text{Cloruro d'argento} + \text{Nitrato di sodio}$$

La freccia rivolta verso il basso accanto al cloruro d'argento indica che tale composto è precipitato.

L'AgCl quindi si separa dalla soluzione e diventa una gelatina bianca che va al fondo. Inoltre nitrato d'argento, cloruro di sodio e nitrato di sodio sono dissociati in acqua, per cui accanto dovremmo scrivere (aq). \textbf{DA FINIRE}

La reazione ci dà un'informazione: la sua stechiometria è 1:1:1:1, cioè una mole di nitrato d'argento reagirà con una mole di cloruro di sodio per produrre una mole di cloruro d'argento ed una mole di nitrato di sodio. Ma abbiamo le stesse moli per i due reagenti?

I grammi cambiano perché il peso molecolare è diverso, ma dobbiamo calcolarlo.

La prima cosa che facciamo quindi è calcolare le moli di nitrato d'argento:

$$n_{\text{AgNO}_3}=\frac{5.76}{169}=3.41 \cdot 10^{-2} \; mol $$

Adesso calcoliamo le moli di cloruro di sodio:

$$n_{\text{NaCl}}=\frac{1.76}{58.5}=3.0 \cdot 10^{-2} \; mol $$

Le moli sono diverse: non sono in rapporto 1:1, pertanto il cloruro di sodio sarà in difetto e verrà consumato totalmente; al contrario, il nitrato d'argento è in eccesso e pertanto parte di questo non reagirà in quanto non avrà cloruro di sodio a sufficienza. Possiamo quindi dire che il cloruro di sodio in difetto viene consumato tutto, producendo lo stesso numero di moli (per via della stechiometria) di cloruro di argento. Ne segue che avremo $3 \cdot 10^{-2}$ moli di cloruro d'argento prodotte. Per il nitrato d'argento abbiamo un eccesso di $4.1 \cdot 10^{-3}$ moli.

Nota: in alcuni testi il reagente in difetto viene chiamato agente limitante, perché non fa reagire parte del nitrato d'argento, cioè ne limita la reazione completa.

Dobbiamo calcolare i grammi di cloruro di argento: essi valgono

$$g_{\text{AgCl}}=3.0 \cdot 10^{-2} \cdot 142=4.26$$

\vspace{0.2cm}\textbf{1.2}$\bigstar$ Alla pressione di $742\;torr$ ed alla temperatura di 27°$\;C$, del biossido di carbonio secco (cioè non contiene acqua) $\rm CO_2$ viene fatto assorbire su potassa KOH 

$$\ce{CO_2 + 2KOH -> K_2CO_3 + H_2O}$$

Viene osservato un aumento in peso della soluzione di $2.3\;g$. Calcolare il volume della $\rm CO_2$ assorbita in tali condizioni.

\vspace{0.2cm}\large\textbf{Svolgimento}\normalsize

\vspace{0.2cm}Ciò che succede è che abbiamo una soluzione di KOH in cui facciamo gorgogliare dentro della $\rm CO_2$. Ci rendiamo conto che si ha un aumento di peso, ovvero sono stati assorbiti di 2.3 grammi di $\rm CO_2$.

Dall'equazione di stato si ha

$$PV=nRT
\implies
V=\frac{NRT}{P}=\frac{2.3 \cdot 0.082 \cdot 300.15 \cdot 760}{44 \cdot 742}$$

$$\implies V=1.3178\;L$$

\textbf{1.2.} Una miscela di ossido di magnesio (MgO) e carbonato di magnesio ($\rm MgCO_3$) riscaldato a 1000° C ha una perdita in peso del 32\%. Calcolarne la composizione.

\vspace{0.2cm}\large\textbf{Svolgimento}\normalsize

\vspace{0.2cm}Quello che succede a 1000° C è che il carbonato di magnesio si decomporrà in ossido, che si aggiungerà a quello che c'era già nella miscela, e in anidride carbonica che invece è un gas e quindi andrà via

Quando si parla di calore in genere si mette il $\Delta$ sopra la freccia, per cui la reazione che avviene è

$$\ce{MgO + MgCO_3 ->[{\Delta}][{1000^{\circ}C}] MgO + CO_2 ^}$$

Il testo non ci dice quanta sostanza abbiamo all'inizio, ma solo la parte che ne abbiamo perso. Se ad esempio ne avessimo 100 grammi, significherebbe perdere 32 grammi e restare con 68 grammi di miscela. Siccome l'unico composto che si può perdere è la $\rm CO_2$, significa che la miscela conteneva 32 grammi di $\rm CO_2$ per ogni 100 grammi di miscela iniziale. La $\rm CO_2$ proviene solo dal carbonato, per cui vogliamo calcolare le moli di $\rm CO_2$ perse portando a 1000°C 100 grammi di questa miscela:

$$n_{\text{CO}_2}=\frac{32}{44}=n_{\text{MgCO}_3}$$

Per la stechiometria, il numero di moli di $\rm CO_2$ sarà uguale al numero di moli di carbonato di magnesio, per cui se abbiamo le moli possiamo scrivere

$$g_{\text{MgCO}_3}=\frac{32}{44} \cdot MM_{\text{MgCO}_3}
=\frac{32}{44} \cdot 84.32=61.3$$

Questo significa che la miscela, di cui abbiamo immaginato averne 100 grammi, sarà composta da 61.3 grammi di carbonato di magnesio e da 38.7 grammi di ossido di magnesio.

\vspace{0.2cm}\textbf{1.2.} Data la reazione

$$\ce{NaOH + H_2SO_4 -> Na_2SO_4 + H_2O}$$

Calcolare i grammi dei prodotti ottenuti, partendo da 0.7521 grammi di idrossido di sodio e 0.4613 grammi di acido solforico.

\vspace{0.2cm}\large\textbf{Svolgimento}\normalsize

\vspace{0.2cm}Bilanciamo innanzitutto la reazione:

$$\ce{2NaOH + H_2SO_4 -> Na_2SO_4 + 2H_2O}$$

La reazione ci dice che due moli di idrossido di sodio reagiranno con una di acido solforico, per cui per avere una reazione completa dovremmo avere un numero di moli di NaOH pari esattamente al doppio di quelle di $\rm H_2SO_4$. Nei fatti però non è detto che sia così, bisogna andare a vedere le moli corrispondenti alle masse di partenza dei composti.

Calcoliamo le moli di NaOH e quelle di $\rm H_2SO_4$

$$n_{\text{NaOH}}=\frac{g}{MM_{\text{NaOH}}}=\frac{0.7521}{40}=1.8803 \cdot 10^{-2}\;mol$$

$$n_{\text{H}_2\text{SO}_4}=\frac{g}{MM_{\text{H}_2\text{SO}_4}}=\frac{0.4613}{98.074}=4.7036 \cdot 10^{-3}\;mol$$

Il doppio delle moli di $\rm H_2SO_4$ è $9.4072 \cdot 10^{-3}\;mol$. Queste sono le moli di idrossido di sodio che reagiranno:

$$n_{\text{NaOH}}\text{(consumate)}=9.4072 \cdot 10^{-3}\;mol$$

Quello che abbiamo visto facendo i conti è che l'acido solforico è in difetto, per cui verrà consumato totalmente e produrrà un ugual numero di moli (perché il rapporto tra i coefficienti stechiometrici è 1:1) di solfato di sodio $\rm Na_2SO_4$, cioè avremo

$$n_{\text{Na}_2\text{SO}_4}=9.4072 \cdot 10^{-3}\;mol$$

L'idrossido di sodio invece sarà in eccesso: ne consumiamo $9.4072 \cdot 10^{-3}$ moli e ne restano $1.8803 \cdot 10^{-2} - 9.4072 \cdot 10^{-3}=9.3958 \cdot 10^{-3}$ moli.

La massa in eccesso di NaOH sarà

$$g_{\text{NaOH}}\text{(eccesso)}=9.3958 \cdot 10^{-3} \cdot 40=0.37583$$

Questa è la massa che NON reagirà.

A questo punto vediamo quanti grammi di $\rm Na_2SO_4$ abbiamo. Le moli le conosciamo, sono tante quante quelle dell'$\rm H_2SO_4$, il peso molecolare è

$$MM_{\text{Na}_2\text{SO}_4}=2\cdot22.98977 + 1 \cdot 32.06 + 4 \cdot 15.9994=142.03714$$

pertanto

$$g_{\text{Na}_2\text{SO}_4}
=4.7036 \cdot 10^{-3} \cdot 142.03714
=0.6681$$

\vspace{0.2cm}\textbf{1.1.4}$\bigstar$ Che volume di ossigeno misurato a condizioni normali si può ottenere dal riscaldamento di 10 grammi di clorato di potassio $\rm KClO_3$ al 90\%?

\vspace{0.2cm}\large\textbf{Svolgimento}\normalsize

\vspace{0.2cm}Se riscaldiamo il $\rm KClO_3$ si produce cloruro di potassio e ossigeno:

$$\ce{KClO_3 ->[{\Delta}] KCl + O_2}$$

La reazione bilanciata è

$$\ce{2KClO_3 ->[{\Delta}] 2KCl + 3O_2}$$

Calcoliamo le moli di $\rm KClO_3$:

$$n_{\text{KClO}_3}=\frac{10}{122.5495}\;mol$$

Le moli di ossigeno saranno pari al numero di moli di $\rm KClO_3$ moltiplicato per $\frac{3}{2}$

$$n_{\text{O}_2}=\frac{3}{2}\cdot\frac{10}{122.5495}\;mol$$

Se vogliamo misurare il volume dell'ossigeno a condizioni normali ($0^{\circ}C$ e $1\;atm$), dobbiamo ricordarci che in tali condizioni una mole occupa 22.414 L, per cui avremo

$$V_{\text{O}_2}=22.414 \cdot n_{\text{O}_2}$$

$$\implies V_{\text{O}_2}=\frac{3 \cdot 10 \cdot 22.414}{2 \cdot 122.5495}$$

Questi sarebbero i litri di $\rm O_2$ se il composto fosse puro, cosa che è solo per il 90\%. Bisogna allora moltiplicare per 0.9:

$$\implies V_{\text{O}_2}(\text{reale})
=\frac{3 \cdot 10 \cdot 22.414}{2 \cdot 122.5495}\cdot 0.9=2.469\;L$$

\textbf{1.2.} 15 grammi di alluminio reagiscono con acido cloridrico, rilasciando ossigeno. Calcolare il volume di idrogeno misurato a $20^{\circ}C$ e 780 torr.

\vspace{0.2cm}\large\textbf{Svolgimento}\normalsize

\vspace{0.2cm}La reazione che avviene è

$$\ce{Al + HCl -> AlCl_3 + H_2}$$

Bilanciata, la reazione è

$$\ce{2Al + 6HCl -> 2AlCl_3 + 3H_2}$$

Calcoliamo le moli di alluminio ($MM=26.98$)

$$n_{\text{Al}}=\frac{15}{26.98}$$

Le moli di idrogeno saranno uguali alle moli di alluminio moltiplicate per $\frac{3}{2}$. Dall'equazione di stato ne ricaviamo allora il volume:

$$PV=nRT \implies V=\frac{nRT}{P}$$

$$\implies V=\frac{3 \cdot 15 \cdot 0.0821 \cdot 293.15 \cdot 760}{2 \cdot 26.98 \cdot 780}=19.53\;L$$

\textbf{1.2.} Quanti grammi di idrossido di potassio sono necessari per neutralizzare totalmente 870 ml di una soluzione 0.015 M di acido solforico? Quali sono i prodotti di reazione?

\vspace{0.2cm}\large\textbf{Svolgimento}\normalsize

\vspace{0.2cm}La reazione che avviene è

$$\ce{2KOH + H_2SO_4 -> K_2SO_4 + 2H_2O}$$

Il rapporto stechiometrico tra KOH e $\rm H_2SO_4$ è 2:1, per cui per neutralizzare le moli di acido ci serve un numero di moli di base pari al doppio.

Calcoliamo le moli di acido. Per fare ciò facciamo la proporzione

$$0.015:1000=n_{\rm H_2SO_4}:870
\implies
n_{\rm H_2SO_4}=\frac{870 \cdot 0.015}{1000}=1.305 \cdot 10^{-2}\;mol$$

$$\implies n_{\rm KOH}=2 \cdot 1.305 \cdot 10^{-2}=2.610 \cdot 10^{-2}\;mol$$

da cui

$$g=n_{\rm KOH} \cdot MM_{\rm KOH}
=2.610 \cdot 10^{-2} \cdot 56.1056
=1.4643\;grammi$$

\vspace{0.2cm}\textbf{1.2.} Ad un campione di g 0.744 di solfato di magnesio idrato, disciolto in acqua, fu aggiunto cloruro di bario in eccesso ottenendo un precipitato di solfato di bario che, dopo filtrazione ed essiccazione, pesava 0.70 g. Determinare il numero di molecole di acqua contenute nella formula del solfato di magnesio idrato.

\vspace{0.2cm}\large\textbf{Svolgimento}\normalsize

\vspace{0.2cm}

\vspace{0.2cm}\textbf{1.2.} Lo zinco in acido cloridrico produce cloruro di zinco ed idrogeno. Calcolare il volume di
idrogeno misurato a c.n. ed i grammi di cloruro di zinco che si ottengono per reazione di 2.42 g di zinco con 100 mL di una soluzione acquosa di HCl 0.5 M.

\vspace{0.2cm}\large\textbf{Svolgimento}\normalsize

\vspace{0.2cm}La reazione che avviene è

$$\ce{2Zn + 2HCl -> 2ZnCL + H_2 ^}$$

Vediamo innanzitutto le moli dei reagenti:

$$n_{\rm Zn}=\frac{g}{MM_{\rm Zn}}=\frac{2.42}{65.38}=3.7014 \cdot 10^{-2}\;mol$$

Per le moli di zinco facciamo la proporzione

$$n_{\rm HCl}:100=0.5:1000
\implies
n_{\rm HCl}=\frac{100 \cdot 0.5}{1000}=5 \cdot 10^{-2}\;mol$$

Le moli di zinco sono in difetto, quindi esso sarà il reagente limitante e dovremo fare riferimento ad esso.

Il rapporto stechiometrico tra zinco e e cloruro di zinco è 2:2 cioè 1:1, quindi otterremo $3.7014 \cdot 10^{-2}$ moli di ZnCl. I grammi saranno dati da

$$g=n_{\rm HCl} \cdot MM_{\rm HCl}=3.7014 \cdot 10^{-2} \cdot 136.286=5.0445\;grammi$$

Per quanto riguarda l'idrogeno, il rapporto stechiometrico è 2:1, per cui ne verranno prodotte $\frac{3.7014 \cdot 10^{-2}}{2}=1.8507 \cdot 10^{-2}$ moli. Per ottenere il volume dobbiamo usare l'equazione di stato dei gas:

$$PV=nRT \implies V=\frac{nRT}{P}$$

$$\implies V_{\rm H_2}
=\frac{1.8507 \cdot 10^{-2} \cdot 0.082 \cdot 273.15}{1}
=0.4145\;L$$

\vspace{0.2cm}\textbf{1.2.} Quanti grammi di ossigeno sono necessari per trasformare 1.2461 g di potassio in ossido?

\vspace{0.2cm}\large\textbf{Svolgimento}\normalsize

\vspace{0.2cm}La reazione che avviene è

$$\ce{2K + O_2 -> K_2O}$$

Il rapporto stechiometrico tra potassio e ossigeno è 2:1, quindi servono una quantità di moli di ossigeno pari alla metà di quelle di potassio. Avremo che

$$n_{\rm K}=\frac{1.2461}{39.0983}=3.1871 \cdot 10^{-2}$$

$$\implies n_{\rm O_2}
=\frac{3.1871 \cdot 10^{-2}}{2}
\approx 1.5935 \cdot 10^{-2}$$

da cui

$$g=n_{\rm O_2} \cdot MM_{\rm O_2}
=1.5935 \cdot 10^{-2} \cdot 32=0.51\;grammi$$

\vspace{0.2cm}\textbf{1.2.} Il solfuro ferroso reagisce con l'acido cloridrico liberando solfuro di idrogeno allo stato gassoso. Quale volume di una soluzione di HCl 0.1 N è necessario per la reazione completa con 5.4637 gr di solfuro ferroso?

\vspace{0.2cm}\large\textbf{Svolgimento}\normalsize

\vspace{0.2cm}La reazione che avviene è

$$\ce{FeS + 2HCl -> FeCl_2 + H_2S ^}$$

Il rapporto stechiometrico tra solfuro ferroso e acido cloridrico è 1:2, per cui il numero di moli di HCl sarà pari al doppio di quelle di FeS:

$$n_{\rm HCl}=2 \cdot n_{\rm FeS}=2 \cdot \frac{5.4637}{87.91}=6.2151 \cdot 10^{-2}\;mol$$

I litri di HCl si calcoleranno attraverso la proporzione

$$0.1:1=6.2151 \cdot 10^{-2}:V
\implies
V=\frac{6.2151 \cdot 10^{-2}}{0.1} \approx 0.6215\;L$$

\vspace{0.2cm}\textbf{1.2.} Una lamina di magnesio di 2.1234 g si scioglie completamente in una soluzione di acido cloridrico. Quanti litri di idrogeno si sviluppano se misurati a 25°C e 1140 torr di pressione?

\vspace{0.2cm}\large\textbf{Svolgimento}\normalsize

\vspace{0.2cm}La reazione che avviene è

$$\ce{2Mg + 2HCl -> 2MgCl + H_2 ^}$$

Il rapporto stechiometrico tra magnesio e idrogeno è 2:1, per cui il numero di moli di idrogeno sarà pari alla metà del numero di moli di magnesio:

$$n_{\text{Mg}}=\frac{g}{MM_{\text{Mg}}}=\frac{2.1234}{24,305}=8.7365 \cdot 10^{-2} \implies n_{\text{H}_2}=\frac{8.7365 \cdot 10^{-2}}{2}\approx 4.3682 \cdot 10^{-2}$$

A questo punto per trovare i litri basterà applicare l'equazione di stato dei gas:

$$PV=nRT \implies V=\frac{nRT}{P}$$

$$\implies V_{\text{H}_2}=\frac{4.3682 \cdot 10^{-2} \cdot 0.082 \cdot 298.15 \cdot 760}{1140}=0.7120\,L$$

\vspace{0.2cm}\textbf{1.2.} Il solfuro ferroso reagisce con l'acido cloridrico liberando solfuro di idrogeno allo stato gassoso. Quale volume di una soluzione di HCl 0.1 N è necessario per la reazione completa con 5.4637 gr di solfuro ferroso?

\vspace{0.2cm}\large\textbf{Svolgimento}\normalsize

\vspace{0.2cm}La reazione che avviene è

$$\ce{FeS + 2HCl -> H_2S ^ + FeCl_2}$$

\vspace{0.2cm}\textbf{1.2.} 3.5048 grammi di $\rm H_2SO_4$ all'85\% in massa vengono fatti reagire con 2.2375 grammi di ossido di alluminio $\rm Al_2O_3$. Determinare i grammi di tutti i composti a fine reazione.

\vspace{0.2cm}\large\textbf{Svolgimento}\normalsize

\vspace{0.2cm}La reazione che avviene è

$$\ce{H_2SO_4 + Al_2O_3 -> Al_2(SO_4)_3 + H_2O}$$

Bilanciamo. Avendo 3 ioni solfato a destra dovremo avere 3 molecole di acido solforico a sinistra, ma quindi dovremo avere 3 molecole d'acqua a destra. La reazione bilanciata, dunque, sarà:

$$\ce{3H_2SO_4 + Al_2O_3 -> Al_2(SO_4)_3 + 3H_2O}$$

A questo punto calcoliamo le moli di reagenti:

$$n_{\rm H_2SO_4}=\frac{g}{MM_{\rm H_2SO_4}}
=\frac{3.5048}{98.079}=3.5734 \cdot 10^{-2}\;mol$$

$$n_{\rm Al_2O_3}=\frac{g}{MM_{\rm Al_2O_3}}
=\frac{2.2375}{101.96}
=2.1945 \cdot 10^{-2}\;mol$$

Il rapporto stechiometrico tra acido solforico e ossido di alluminio è 3:1, per cui i due composti reagirebbero totalmente soltanto se il numero di moli di acido fossero pari al triplo di quelle dell'ossido. Non essendo così, reagiranno tutte le moli di acido (e quindi a fine reazioni non avremo grammi di questo) ma le moli di ossido che reagiscono saranno pari a

$$n_{\rm Al_2O_3}(\text{reagenti})
=\frac{n_{\rm H_2SO_4}}{3}
=\frac{3.5734 \cdot 10^{-2}}{3}
=1.1911 \cdot 10^{-2}\:mol$$

e quindi le moli restanti saranno date da

$$n_{\rm Al_2O_3}(\text{restanti})
=(2.1945 - 1.1911) \cdot 10^{-2}
=1.0034 \cdot 10^{-2}\;mol$$

a cui corrisponderanno un numero di grammi pari a

$$g_{\rm Al_2O_3}=n_{\rm Al_2O_3} \cdot MM_{\rm Al_2O_3}
=1.0034 \cdot 10^{-2} \cdot 101.96
\approx 1.0231\;grammi$$

Per quanto riguarda i prodotti dobbiamo tenere come riferimento le moli di acido solforico.

Il rapporto stechiometrico tra acido e solfato di alluminio è 3:1, quindi avremo

$$n_{\rm Al_2(SO_4)3}=\frac{n_{\rm H_2O_4}}{3}
=\frac{3.5734 \cdot 10^{-2}}{3}
=1.7867 \cdot 10^{-2}\;mol$$

da cui

$$g_{\rm Al_2(SO_4)3}=n_{\rm Al_2(SO_4)3} \cdot MM_{\rm Al_2(SO_4)3}
=1.7867 \cdot 10^{-2} \cdot 342.15
=6.1132\;grammi$$

Il rapporto stechiometrico tra acido e acqua è 1:1, quindi avremo lo stesso numero di moli.

$$g_{\rm H_2O}=n_{\rm H_2O} \cdot MM_{\rm H_2O}
=3.5734 \cdot 10^{-2} \cdot 18
=0.6432\;grammi$$

\vspace{0.2cm}\textbf{1.2.} Un grammo (1.0000) di una miscela costituita da NaCl e KCl sciolti in acqua ha richiesto, per la completa precipitazione dei cloruri sotto forma di AgCl, 51.92 mL di nitrato d'argento 0.3 M. Determinare la composizione percentuale in massa della miscela.

\vspace{0.2cm}\large\textbf{Svolgimento}\normalsize

\vspace{0.2cm}Bisogna impostare un sistema. La prima equazione riguarda i grammi dei composti, la seconda le moli:

$$\begin{cases}
    g_{\rm NaCl} + g_{\rm KCl}=1.0000\\
    n_{\rm NaCl} + _{\rm KCl}=n_{\rm Ag(NO_3)}
\end{cases}$$

Tuttavia possiamo ridurre le incognite a due, in quanto grammi e moli sono legati tra loro tramite la relazione

$$n=\frac{g}{MM}
\implies
g=n \cdot MM$$

Calcoliamo le moli di $\rm Ag(NO_3)$:

$$n_{\rm Al(NO_3)_3}=\frac{0.3 \cdot 51.92}{1000}=1.5576 \cdot 10^{-2}$$

Poniamo per semplicità $x=n_{\rm NaCl}$ e $y=n_{\rm KCl}$. Avremo

$$\begin{cases}
    58.44x + 74.55y=1\\
    x+y=1.5576 \cdot 10^{-2}
\end{cases}$$

dove i coefficienti della prima equazione sono rispettivamente i pesi molecolari dell'NaCl e dell'AgCl.

$$\implies
\begin{cases}
    x + \displaystyle \frac{74.55}{58.44}y=\displaystyle \frac{1}{58.44}\\[0.2cm]
    x+y=1.5576 \cdot 10^{-2}
\end{cases}
\implies
\begin{cases}
    x + 1.2757y=1.7111 \cdot 10^-2\\
    x+y=1.5576 \cdot 10^{-2}
\end{cases}$$

$$\implies
\begin{cases}
    x=9.99 \cdot 10^{-3}\\
    y=5.5676 \cdot 10^{-3}
\end{cases}$$

A questo punto per calcolare la composizione percentuale calcoliamo i grammi:

$$g_{\rm NaCl}=9.99 \cdot 10^{-3} \cdot 58.44
=0.5838
\quad;\quad
g_{\rm KCl}=5.5676 \cdot 10^{-3} \cdot 74.55
=0.4151$$

cioè la miscela è composta al 58\% da NaCl e al 42\% da KCl.

\vspace{0.2cm}\textbf{1.2.} Calcolare i grammi di bromuro di litio che si ottengono da 4.5678 g di carbonato di litio fatti reagire con acido bromidrico. La reazione, da bilanciare, genera inoltre anidride carbonica ed acqua.

\vspace{0.2cm}\large\textbf{Svolgimento}\normalsize

\vspace{0.2cm}La reazione che avviene è

$$\ce{Li_2CO_3 + HBr -> CO_2 + LiBr + H_2O}$$

Per bilanciarla notiamo che a sinistra abbiamo due atomi di litio mentre a destra uno solo, per cui moltiplichiamo LiBr per 2; di conseguenza avremo anche due atomi di bromo che dobbiamo avere pure a sinistra, quindi moltiplichiamo per 2 HBr e in questo modo bilanciamo anche gli idrogeni, dato che a destra ne abbiamo due nell'acqua:

$$\ce{Li_2CO_3 + 2HBr -> CO_2 + 2LiBr + H_2O}$$

A questo punto notiamo che il rapporto stechiometrico tra $\rm Li_2CO_3$ e LiBr è di 1:1, quindi le moli di bromuro prodotte saranno tante quante le moli di carbonato messe a reagire:

$$n_{\rm LiBr}=n_{\rm Li_2CO_3}
=\frac{g_{\rm Li_2CO_3}}{MM_{\rm Li_2CO_3}}
=\frac{4.5678}{73.891}=6.1818 \cdot 10^{-2}\;mol$$

I grammi di LiBr saranno

$$g_{\rm LiBr}=n \cdot MM=6.1818 \cdot 10^{-2} \cdot 86.845
=5.3686\;grammi$$