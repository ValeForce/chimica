\begin{esercizio}
    Calcolare l'intensità di corrente necessaria perché in due ore si possano ottenere 2.2345 grammi di zinco al catodo di una cella elettrolitica contenente 900 mL di soluzione di cloruro di zinco 0.5 M.
\end{esercizio}
\begin{soluzione}
    La reazione che deve avvenire è
    \begin{equation*}
        \ce{Zn^{2+} + 2e^- -> Zn^{0}}
    \end{equation*}
    Vediamo a quanti equivalenti corrispondono i grammi di zinco. Poiché nella reazione si scambiano 2 elettroni, la massa equivalente sarà uguale al peso atomico diviso 2:
    \begin{equation*}
        n_{eq}=\frac{g}{ME_{\rm Zn}}=\frac{2 \cdot g}{MA_{\rm Zn}}
        =\frac{2 \cdot 2.2345}{65.38}
        =6.8354 \cdot 10^{-2} \; \rm eq
    \end{equation*}
    Per la seconda legge di Faraday, essi corrisponderanno a $6.8354 \cdot 10^{-2}$ Faraday di corrente.\\
    Sapendo inoltre che 2 ore corrispondono a 7200 secondi, avremo che
    \begin{equation*}
        I=\frac{Q}{t}
        =\frac{6.8354 \cdot 10^{-2} \cdot 96485}{7200}=0.916 \; \rm A
    \end{equation*}
\end{soluzione}

\newpage

\begin{esercizio}
    Calcolare l'intensità di corrente necessaria perché in due ore si possano ottenere 3.5887 grammi di rame al catodo di una cella elettrolitica contenente 2 L di soluzione di cloruro rameico 0.1 M.
\end{esercizio}
\begin{soluzione}
    La reazione che deve avvenire è
    \begin{equation*}
        \ce{Cu^{2+} + 2e^- -> Cu^{0}}
    \end{equation*}
    Vediamo a quanti equivalenti corrispondono i grammi di rame. Poiché nella reazione si scambiano 2 elettroni, la massa equivalente sarà uguale al peso atomico diviso 2:
    \begin{equation*}
        n_{eq}
        =\frac{g}{ME_{\rm Cu}}
        =\frac{2 \cdot g}{MA_{\rm Cu}}
        =\frac{2 \cdot 3.5887}{63.546}
        =11.2948 \cdot 10^{-2} \; \rm eq
    \end{equation*}
    Per la seconda legge di Faraday, essi corrisponderanno a $11.2948 \cdot 10^{-2}$ Faraday di corrente.\\
    Sapendo inoltre che 2 ore corrispondono a 7200 secondi, avremo che
    \begin{equation*}
        I=\frac{Q}{t}
        =\frac{11.2948 \cdot 10^{-2} \cdot 96485}{7200}
        =1.5136 \; \rm A
    \end{equation*}
\end{soluzione}

\newpage

\begin{esercizio}
    Una corrente di 5 A passa per 30 min in un fuso di NaCl. Calcolare i grammi di sodio ed il volume di cloro a c.n. che si formano.
\end{esercizio}
\begin{soluzione}
    La carica che scorre è data da
    \begin{equation*}
        I=\frac{Q}{t}
        \implies
        Q=I \cdot t
        =5 \cdot 30 \cdot 60
        =9000 \; \rm C
    \end{equation*}
    Calcoliamo gli equivalenti che si depositano agli elettrodi:
    \begin{equation*}
        n_{eq}=\frac{Q}{96485}
        =\frac{9000}{96485}
        =9.3279 \cdot 10^{-2} \; \rm eq
    \end{equation*}
    Le reazioni che avvengono agli elettrodi sono
    \begin{equation*}
        \ce{Na^+ + e^- -> Na}
        \quad;\quad
        \ce{2Cl^- -> Cl_2 + 2e^-}
    \end{equation*}
    Nel caso del sodio, moli ed equivalenti coincidono perché viene scambiato un solo elettrone, quindi i grammi prodotti saranno dati da
    \begin{equation*}
        g_{\rm Na}=n_{\rm Na} \cdot MM_{\rm Na}
        =9.3279 \cdot 10^{-2} \cdot 22.98
        =2.1435 \; \rm grammi
    \end{equation*}
    Per quanto riguarda il cloro, visto che vengono scambiati due elettroni il numero di moli sarà pari al doppio del numero di equivalenti. Dall'equazione di stato dei gas troviamo il volume di cloro:
    \begin{equation*}
        PV=nRT
        \implies
        V_{\rm Cl_2}=\frac{nRT}{P}
        =\frac{9.3279 \cdot 10^{-2} \cdot 2 \cdot 0.082 \cdot 273.15}{1}
        =4.1786 \; \rm L
    \end{equation*}
\end{soluzione}

\newpage

\begin{esercizio}
    Determinare la f.e.m. della seguente pila:
    \begin{center}
        \begin{tabular}{|p{3.7cm}||p{3.7cm}|}
             Zn & $\rm Cr^{3+}$ \, \big[$2.5 \cdot 10^{-1}$ M\big]\\[0.5ex]
             $\rm Zn^{2+}$ \, \big[$1.5 \cdot 10^{-2}$ M\big] & Cr\\[0.5ex]
        \end{tabular}
    \end{center}
    sapendo che $E^{0}_{(\rm Zn^{2+}/Zn)}=-0.76 \; \rm V$ e $E^{0}_{(\rm Cr^{3+}/Cr)}=-0.74 \; \rm V$.
\end{esercizio}
\begin{soluzione}
    Scriviamo le reazioni di riduzione delle singole semicelle:
    \begin{equation*}
        \ce{Cr^{3+}(aq) + 3 e^- -> Cr(s)}
        \quad;\quad
        \ce{Zn^{2+}(aq) + 2 e^- -> Zn(s)}
    \end{equation*}
    Poiché il potenziale standard del cromo è maggiore di quello dello zinco, sarà esso a ridursi al catodo, per cui la reazione che avviene è la prima.\\
    La forza elettromotrice sarà data da
    \begin{equation*}
        f.e.m. = E_{\rm catodo} - E_{\rm anodo}
    \end{equation*}
    Calcoliamo singolarmente i potenziali
    \begin{gather*}
        E_{\rm catodo}
        =E^0_{(\rm Cr^{3+}/Cr)} + \frac{0.059}{3} \log \frac{a_{\rm Cr^{3+}}}{a_{\rm Cr}}
        =-0.74 + \frac{0.059}{3} \log \frac{2.5 \cdot 10^{-1}}{1}
        =-0.75 \; \rm V
        \\
        E_{\rm anodo} = E^0_{(\rm Zn^{2+}/Zn)} + \frac{0.059}{2} \log \frac{a_{\rm Zn^{2+}}}{a_{\rm Zn}}
        =-0.76 + \frac{0.059}{2} \log \frac{1.5 \cdot 10^{-2}}{1}
        =-0.81 \; \rm V
    \end{gather*}
    (Nota: le attività delle specie pure (Zn e Cr) si considerano unitarie.)\\
    da cui
    \begin{equation*}
        f.e.m.=-0.75 - (-0.81) = 0.06 \; \rm V
    \end{equation*}
\end{soluzione}

\newpage

\begin{esercizio}
    Calcolare il potenziale della semicella

    \begin{center}
        \begin{tabular}{|p{3.7cm}|}
             Ag\\[0.5ex]
             $\rm AgNO_3$ \, \big[$4 \cdot 10^{-2}$ M\big]\\[0.5ex]
        \end{tabular}
    \end{center}

    sapendo che $E^{0}_{(\rm Ag^{+}/Ag)}=+0.8 \; \rm V$.
\end{esercizio}
\begin{soluzione}
    Il potenziale sarà dato, banalmente, da
    \begin{equation*}
        E = E^0_{(\rm Ag^{+}/Ag)} + \frac{0.059}{1} \log \frac{a_{\rm Ag^{+}}}{a_{\rm Ag}}
        =0.8 + \frac{0.059}{1} \log \frac{4 \cdot 10^{-2}}{1}
        =0.72 \; \rm V
    \end{equation*}
\end{soluzione}

\newpage

\begin{esercizio}
    Una corrente di 18 A fluisce attraverso un fuso di $\rm Al_2O_3$ ed al catodo si depositano 15.45 g di alluminio. Calcolare il volume di ossigeno ($P=1.2 \; \rm atm$; $T=18\; ^{\circ} \rm C$) che si forma all'anodo e la durata del processo.
\end{esercizio}
\begin{soluzione}
    Per prima cosa calcoliamo la durata del processo. L'allumina $\rm Al_2O_3$ si dissocia secondo la reazione 
    \begin{equation*}
        \ce{Al_2O_3 -> 2Al^{3+} + 3O^{2-}}
    \end{equation*}
    La reazione che avviene al catodo è
    \begin{equation*}
        \ce{Al^{3+} + 3e^- -> Al }
    \end{equation*}
    Calcoliamo gli equivalenti di alluminio. Siccome vengono scambiati tre elettroni, essi saranno pari al numero di moli moltiplicati per 3:
    \begin{equation*}
        n_{eq}=n \cdot 3
        =\frac{3 \cdot 15.45}{26.98}
        =1.7179 \; \rm eq
    \end{equation*}
    Ad essi corrisponderanno $1.7179$ Faraday di corrente. La durata del processo sarà data da
    \begin{equation*}
        I=\frac{Q}{t}
        \implies
        t=\frac{Q}{I}=\frac{1.7179 \cdot 96485}{18}
        =9208.42 \; \rm s
    \end{equation*}
    A questo punto calcoliamo il volume di ossigeno prodotto. La reazione che avviene all'anodo è
    \begin{equation*}
        \ce{2O^{2-} -> O_2 + 4e^-}
    \end{equation*}
    Se al catodo si sono depositati $1.7179$ equivalenti di sostanza, altrettanti se ne depositeranno all'anodo. Siccome nella reazione vengono scambiati 4 elettroni, le moli saranno pari al numero di equivalenti diviso 4:
    \begin{equation*}
        n=\frac{n_{eq}}{4}
        =\frac{1.7179}{4}
        =0.4295 \; \rm mol
    \end{equation*}
    Il volume sarà dato dall'equazione di stato dei gas
    \begin{equation*}
        PV=nRT
        \implies
        V_{\rm O_2}=\frac{nRT}{P}=\frac{0.4295 \cdot 0.082 \cdot 291.15}{1.2}=8.5450 \; \rm L
    \end{equation*}
\end{soluzione}

\newpage

\begin{esercizio}
    Calcolare la f.e.m. della seguente pila:

\begin{center}
    \begin{tabular}{|p{3.7cm}||p{3.7cm}|}
         Fe(s) & Cd(s)\\[0.5ex]
         $\rm Fe^{2+}$ \, \big[0.0120 M\big] & $\rm Cd^{2+}$ \, \big[0.8988 M\big] \\[0.5ex]
    \end{tabular}
\end{center}

sapendo che $E^0_{(\rm Fe^{2+}/Fe)}=0.44 \; \rm V$ e $E^0_{(\rm Cd^{2+}/Cd)}=-0.40 \; \rm V$.
\end{esercizio}
\begin{soluzione}
    Le reazioni di riduzione delle singole semicelle sono
    \begin{equation*}
        \ce{Fe^{2+}(aq) + 2 e^- -> Fe(s)}
        \quad;\quad
        \ce{Cd^{2+}(aq) + 2 e^- -> Cd(s)}
    \end{equation*}
    Poiché il potenziale standard del ferro è maggiore di quello del cadmio, sarà esso a ridursi al catodo, per cui avviene la prima reazione. La forza elettromotrice sarà data da
    \begin{equation*}
        f.e.m. = E_{\rm catodo} - E_{\rm anodo}
    \end{equation*}
    Calcoliamo singolarmente i potenziali
    \begin{gather*}
        E_{\rm catodo}
        =E^0_{(\rm Fe^{2+}/Fe)} + \frac{0.059}{2} \log \frac{a_{\rm Fe^{2+}}}{a_{\rm Fe}}
        =0.44 + \frac{0.059}{2} \log \frac{0.0120}{1}
        =0.38 \; \rm V
        \\
        E_{\rm anodo} = E^0_{(\rm Cd^{2+}/Cd)} + \frac{0.059}{2} \log \frac{a_{\rm Cd^{2+}}}{a_{\rm Cd}}
        =-0.40 + \frac{0.059}{2} \log \frac{0.8988}{1}
        \approx -0.40 \; \rm V
    \end{gather*}
    da cui
    \begin{equation*}
        f.e.m.=0.38 - (-0.40) = 0.78 \; \rm V
    \end{equation*}
\end{soluzione}

\newpage

\begin{esercizio}
    Calcolare la f.e.m. della seguente pila:

    \begin{center}
        \begin{tabular}{p{0.6cm}|p{3cm}||p{3cm}|p{1cm}}
            Pt & $[\rm Sn^{4+}]$=1 M & $[\rm Zn^{2+}]$=1 M & Pt\\[0.5ex]
            & $[\rm Sn^{2+}]$=1 M & Zn &\\[0.5ex]
        \end{tabular}
    \end{center}

    sapendo che $E^0_{(\rm Sn^{4+}/Sn^{2+})}=0.15 \; \rm V$ e $E^0_{(\rm Zn^{2+}/Zn)}=-0.76 \; \rm V$.
\end{esercizio}
\begin{soluzione}
    Le reazioni di riduzione delle singole semicelle sono

$$\ce{Sn^{4+}(aq) + 2 e^- -> Sn^{2+}(aq)}
\quad;\quad
\ce{Zn^{2+}(aq) + 2 e^- -> Zn(s)}$$

Poiché il potenziale standard dello stagno è maggiore di quello dello zinco, sarà esso a ridursi al catodo, per cui avviene la prima reazione. La forza elettromotrice sarà data da

$$f.e.m. = E_{\rm catodo} - E_{\rm anodo}$$

Calcoliamo singolarmente i potenziali

$$E_{\rm catodo}
=E^0_{(\rm Sn^{4+}/Sn^{2+})} + \frac{0.059}{2} \log \frac{a_{\rm Sn^{4+}}}{a_{\rm Sn^{2+}}}
=0.15 + \frac{0.059}{2} \log \frac{1}{1}
=0.15 + 0 = 0.15 \; \rm V$$

$$E_{\rm anodo} = E^0_{(\rm Zn^{2+}/Zn)} + \frac{0.059}{2} \log \frac{a_{\rm Zn^{2+}}}{a_{\rm Zn}}
=-0.76 + \frac{0.059}{2} \log \frac{1}{1}
= -0.76 + 0=-0.76 \; \rm V$$

da cui

$$f.e.m.=0.15 - (-0.76) = 0.91 \; \rm V$$
\end{soluzione}

\newpage

\begin{esercizio}
    Data la pila:

\begin{center}
    \begin{tabular}{p{0.6cm}|p{3cm}||p{3cm}|p{1cm}}
        & Ag(s) & Cu(s) & \\[0.5ex]
        & $\rm Ag^{+}\;3 \cdot 10^{-3}$ M & $\rm Cu^{2+}$ 0.9 M&\\[0.5ex]
    \end{tabular}
\end{center}

stabilire quali reazioni avvengono nelle due semicelle e calcolare la f.e.m.\,, sapendo che $E^0_{(\rm Ag^{+}/Ag)}=0.80 \; \rm V$ e $E^0_{(\rm Cu^{2+}/Cu)}=0.34 \; \rm V$.
\end{esercizio}
\begin{soluzione}
    Le reazioni di riduzione delle singole semicelle sono

$$\ce{Ag^{+} + e^- -> Ag}
\quad;\quad
\ce{Cu^{2+} + 2 e^- -> Cu}$$

Poiché il potenziale standard dell'argento è maggiore di quello del rame, sarà esso a ridursi al catodo, mentre all'anodo avverrà la reazione \ce{Cu -> Cu^{2+} + 2 e^-}. La forza elettromotrice sarà data da

$$f.e.m. = E_{\rm catodo} - E_{\rm anodo}$$

Calcoliamo singolarmente i potenziali

$$E_{\rm catodo}
=E^0_{(\rm Ag^{2+}/Ag)} + \frac{0.059}{1} \log \frac{a_{\rm Ag^{2+}}}{a_{\rm Ag}}
=0.80 + \frac{0.059}{1} \log \frac{3 \cdot 10^{-3}}{1}
=0.65 \; \rm V$$

$$E_{\rm anodo} = E^0_{(\rm Cu^{2+}/Cu)} + \frac{0.059}{2} \log \frac{a_{\rm Cu^{2+}}}{a_{\rm Cu}}
=0.34 + \frac{0.059}{2} \log \frac{0.9}{1}
\approx 0.34 \; \rm V$$

da cui

$$f.e.m.=0.65 - 0.34 = 0.31 \; \rm V$$
\end{soluzione}

\newpage

\begin{esercizio}
    Calcolare la f.e.m. della seguente pila:

\begin{center}
    \begin{tabular}{p{0.6cm}|p{3.6cm}||p{3.6cm}|p{1cm}}
        Ag & $[\rm Ag^+]=1.0 \cdot 10^{-7}$ M & $[\rm Ag^+]=1.0 \cdot 10^{-2}$ M & Ag
    \end{tabular}
\end{center}

sapendo che $E^0_{(\rm Ag^{+}/Ag)}=0.7994 \; \rm V$.
\end{esercizio}
\begin{soluzione}
    Si tratta di una pila concentrazione (in particolare del tipo dell'ES.1 in §8.2.10). In questo caso la d.d.p. è data da

    \begin{equation*}
        E=E_1 - E_2 = \frac{0.059}{n} \log \frac{a_1}{a_2}
    \end{equation*}

    dove $a_1$ è la concentrazione più elevata (è importante che sia così altrimenti il potenziale risulterebbe negativo\footnote{Se la f.e.m. fosse negativa fisicamente significherebbe che la pila sta agendo come un ricevitore di energia elettrica piuttosto che come una fonte, ecco perché è un risultato non accettabile.}) e $n$ è il numero di elettroni scambiati. Poiché per passare da $\rm Ag^{+}$ ad Ag serve un solo elettrone $n=1$ e quindi:

\begin{equation*}
    E=0.059 \cdot \log \frac{1.0 \cdot 10^{-2}}{1.0 \cdot 10^{-7}}
    =0.059 \cdot \log (1.0 \cdot 10^{5})
    =0.295 \; \rm V
\end{equation*}
\end{soluzione}

\newpage

\begin{esercizio}
    Si fa passare una corrente costante di $10 \; \rm A$ per 30 minuti in una cella contenente idrossido di sodio fuso, usando elettrodi inerti. Si calcoli la massa del metallo che si è formata al catodo ed il volume del gas che si è sviluppato all'anodo (\ce{ 4OH^- -> O_2 + 2H_2O + 4e^-}), trascurando l'$\rm H_2O$, a condizioni normali.
\end{esercizio}
\begin{soluzione}
    Si tratta di un'elettrolisi di sali fusi. Infatti non è possibile fare elettrolisi della base NaOH in acqua, in quanto se mettessimo sodio metallico in acqua otterremo un'esplosione. Inoltre il potenziale standard di riduzione del sodio è uno dei più negativi, per cui non possiamo idrolizzare un sale di sodio o una base di sodio per ottenere sodio: al catodo infatti si andrà a sviluppare la specie con potenziale maggiore, cioè l'idrogeno che ha potenziale pari a 0.\\
    In altre parole, se facessimo elettrolisi di una soluzione acquosa di idrossido di sodio, al catodo otterremmo un flusso di idrogeno, non di sodio metallico. Ecco perché qualora ci serva il sodio metallico dobbiamo lavorare sul fuso, in assenza di acqua.\\
    In tale fuso di idrossido di sodio viene fatta passare una corrente di 10 A per un tempo $t=30 \cdot 60=1800 \; \rm s$, che corrispondono ad una carica pari a

\begin{equation*}
    Q=I \cdot t=10 \cdot 1800=18000 \; \rm C
\end{equation*}

che in Faraday corrispondono a

\begin{equation*}
    Q(\rm F)=\frac{18000}{96485}=1.8656 \cdot 10^{-1} \; F
\end{equation*}

Per le leggi di Faraday, tale quantità sarà pari al numero di equivalenti di sostanza depositata al catodo. Inoltre, siccome al catodo avviene la reazione di riduzione del sodio

\begin{equation*}
    \ce{Na^+ + 1e^- -> Na}
\end{equation*}

in cui si scambia un solo elettrone, equivalenti e moli coincidono. Ne segue che le moli di sodio depositate $n_{\rm Na}$ sono pari a $1.8656 \cdot 10^{-1} \; \rm mol$. La massa di sodio sarà allora

\begin{equation*}
    g_{\rm Na}=MM_{\rm Na} \cdot n_{\rm Na}
    =1.8656 \cdot 10^{-1} \cdot 22.9898
    =4.2890 \; \rm grammi
\end{equation*}

Passiamo all'anodo: qui avviene l'ossidazione dello ione $\rm OH^-$, i cui prodotti sono ossigeno e acqua. La reazione che avviene è

\begin{equation*}
    \ce{4OH^- -> O_2 + 2H_2O + 4e^-}
\end{equation*}

Noi vogliamo calcolare il volume di ossigeno prodotto, e per calcolarlo ci servono le moli. Poiché ogni molecola di $\rm O_2$ consuma 4 elettroni, le moli saranno pari ad un quarto degli equivalenti della sostanza che si ossida. Dunque:

\begin{equation*}
    n_{\rm O_2}
    =\frac{1.8656 \cdot 10^{-1}}{4}
    =0.4664 \cdot 10^{-1} \; \rm mol
\end{equation*}

A questo punto per ricavare il volume basterà applicare la legge dei gas perfetti:

\begin{equation*}
    PV=nRT
    \implies
    V=\frac{nRT}{P}
    \implies
    V_{\rm O_2}=\frac{0.4664 \cdot 10^{-1} \cdot 0.082 \cdot 273.15}{1}
    =1.0446 \; \rm L
\end{equation*}
\end{soluzione}

\newpage

\begin{esercizio}
    Una pila di concentrazione è formata da due semicelle nelle quali i terminali inerti di platino sono immersi in due soluzioni, una di acido cloridrico e l'altra di acido acetico ($K_a = 1.8 \times 10^{-5}$) entrambe 0.10 molari. La temperatura è di 298.15 K e su entrambi gli elettrodi viene fatto gorgogliare dell'idrogeno gassoso alla pressione di 1 atm. Calcolare la differenza di potenziale della pila.
\end{esercizio}
\begin{soluzione}
    Rappresentiamo innanzitutto le due semicelle:
    \begin{center}
        \begin{tabular}{l|c||c|l}
            Pt ($\rm H_2$) & $\ce{HCl} \; (0.10 \; M)$ & $\ce{CH_3COOH} \; (0.10 \; M)$ & Pt ($\rm H_2$)\\[0.5ex]
            $P_{\rm H_2}=1 \; \rm atm$ & $a_1=?$ & $a_2=?$ & $P_{\rm H_2}=1 \; \rm atm$\\[0.5ex]
        \end{tabular}
    \end{center}
    Come possiamo notare, le due semicelle sono elettrodi a idrogeno (Pt | $\rm H_2$ (1 atm) | soluzione), quindi la reazione che avviene agli elettrodi è
    \begin{equation*}
        \ce{H^+ + e^- <--> \frac{1}{2} H_2}
    \end{equation*}
    Si tratta pertanto di una pila a concentrazione. In questo caso la d.d.p. è data da
    \begin{equation*}
        E=E_1 - E_2 = \frac{0.059}{n} \log \frac{a_1}{a_2}
    \end{equation*}
    dove $a_1$ è la concentrazione più elevata e $n$ è il numero di elettroni scambiati, che in questo caso è pari a 1.\\
    Per capire quale sarà la differenza di potenziale tra le due semicelle, dobbiamo innanzitutto conoscere l'attività degli ioni $\rm H^+$ in ciascuna soluzione.\\
    Nella prima semicella, quella di HCl, l'acido è forte e si dissocia completamente, per cui la concentrazione degli ioni $\rm H^+$ sarà pari a quella dell'acido, cioè $a_{\rm H^+}(\rm HCl)=0.10 \; \rm mol/L$. Nella seconda semicella, quella di acido acetico, l'acido è debole e si dissocia parzialmente secondo la reazione
    \begin{equation*}
        \ce{CH_3COOH <--> H^+ + CH_3COO^-}
    \end{equation*}
    Possiamo calcolare la concentrazione degli ioni $\rm H^+$ in prima approssimazione utilizzando la costante di dissociazione acida come
    \begin{equation*}
        a_{\rm H^+}({\rm CH_3COOH})
        =\sqrt{K_a C_a}
        =\sqrt{1.8 \cdot 10^{-5} \cdot 0.1}
        =1.3416 \cdot 10^{-3} \; \rm mol/L
    \end{equation*}
    Abbiamo quindi trovato che $a_1=a_{\rm H^+}(\rm HCl)$. Arrivati a questo punto possiamo calcolare la differenza di potenziale
    \begin{equation*}
        E=0.059 \log \left( \frac{0.1}{1.3416 \cdot 10^{-3}} \right)
        =0.111 \; \rm V
    \end{equation*}
\end{soluzione}