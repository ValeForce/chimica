\textbf{5.1} Calcolare l'intensità di corrente necessaria perché in due ore si possano ottenere 2.2345 grammi di zinco al catodo di una cella elettrolitica contenente 900 mL di soluzione di cloruro di zinco 0.5 M.

\vspace{0.2cm}\large\textbf{Svolgimento}\normalsize

\vspace{0.2cm}La reazione che deve avvenire è

$$\ce{Zn^{2+} + 2e^- -> Zn^{0}}$$

Vediamo a quanti equivalenti corispondo i grammi di zinco. Poiché nella reazione si scambiano 2 elettroni, la massa equivalente sarà uguale al peso atomico diviso 2:

$$ME_{\rm Zn}\frac{MA_{\rm Zn}}{2}
=\frac{2.2345}{2 \cdot 65.38}
=1.7090 \cdot 10^{-2}\;eq$$

Per la seconda legge di Faraday, essi corrisponderanno a $1.7090 \cdot 10^{-2}$ Faraday di corrente.

Sapendo inoltre che 2 ore corrispondono a 7200 secondi, avremo che

$$I=\frac{Q}{t}
=\frac{1.7090 \cdot 10^{-2} \cdot 96485}{7200}=0.2290\;A$$

\vspace{0.2cm}\textbf{5.2} Calcolare l'intensità di corrente necessaria perché in due ore si possano ottenere 3.5887 grammi di rame al catodo di una cella elettrolitica contenente 2 L di soluzione di cloruro rameico 0.1M.

\vspace{0.2cm}\large\textbf{Svolgimento}\normalsize

\vspace{0.2cm}

\vspace{0.2cm}\textbf{5.3} Una corrente di 5 A passa per 30 min in un fuso di NaCl. Calcolare i grammi di sodio ed il
volume di cloro a c.n. che si formano.

\vspace{0.2cm}\large\textbf{Svolgimento}\normalsize

\vspace{0.2cm}

\vspace{0.2cm}\textbf{5.4} Determinare la f.e.m. della seguente pila:

\begin{center}
    \begin{tabular}{|p{3.7cm}||p{3.7cm}|}
         Zn & $\rm Cr^{3+}$ \, \big[$2.5 \cdot 10^{-1}$-M\big]\\[0.5ex]
         $\rm Zn^{2+}$ \, \big[$1.5 \cdot 10^{-2}$-M\big] & Cr\\[0.5ex]
    \end{tabular}
\end{center}

Sapendo che $E^{0}_{(\rm Zn^{2+}/Zn)}=-0.76V$ e $E^{0}_{(\rm Cr^{3+}/Cr)}=-0.74V$.

\vspace{0.2cm}\large\textbf{Svolgimento}\normalsize

\vspace{0.2cm}

\vspace{0.2cm}\textbf{5.5} Calcolare il potenziale della semicella 

\begin{center}
    \begin{tabular}{|p{3.7cm}|}
         Ag\\[0.5ex]
         $\rm AgNO_3$ \, \big[$4 \cdot 10^{-2}$ M\big]\\[0.5ex]
    \end{tabular}
\end{center}

Sapendo che $E^{0}_{(\rm Ag^{+}/Ag)}=+0.8V$

\vspace{0.2cm}\large\textbf{Svolgimento}\normalsize

\vspace{0.2cm}

\vspace{0.2cm}\textbf{5.6} Una corrente di 18 Ampere fluisce attraverso un fuso di $\rm Al_2O_3$ ed al catodo si depositano 15.45 g
di alluminio. Calcolare il volume di ossigeno (P= 1.2 atm; T= 18°C) che si forma all’anodo e la
durata del processo.

\vspace{0.2cm}\large\textbf{Svolgimento}\normalsize

\vspace{0.2cm}

\vspace{0.2cm}\textbf{5.7} Calcolare la f.e.m. della seguente pila:

\begin{center}
    \begin{tabular}{|p{3.7cm}||p{3.7cm}|}
         Fe(s) & Cd(s)\\[0.5ex]
         $\rm Fe^{2+}$ \, \big[0.0120-M\big] & $\rm Cd^{2+}$ \, \big[0.8988-M\big] \\[0.5ex]
    \end{tabular}
\end{center}

sapendo che $E^0_{(\rm Fe^{2+}/Fe)}=0.44V$ e $E^0_{(\rm Cd^{2+}/Cd)}=-0.40V$.

\ce{Fe(s) -> Fe^{2+}(aq) + 2 e^-}

\ce{Cd^{2+}(aq) -> Cd(s) + 2 e^-}

\vspace{0.2cm}\large\textbf{Svolgimento}\normalsize

\vspace{0.2cm}