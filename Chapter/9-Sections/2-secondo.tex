\vspace{0.2cm}\textbf{2.1} Bilanciare la  seguente reazione e calcolare i grammi di iodio che si ottengono facendo reagire 21.5 g di $\rm KIO_3$ $(MM=214.0)$ con 57.0 g di KI $(MM=166.0)$.

\vspace{0.2cm}
\begin{center}Iodato di potassio + ioduro di potassio + acido solforico \ce{->}

\ce{->}iodio +solfato di potassio +acqua. 
\end{center}

\large\textbf{Svolgimento}\normalsize

\vspace{0.2cm}La reazione che avviene è

$$\ce{KIO_3 + KI + H_2SO_4 -> I_2 + K_2SO_4 + H_2O}$$

Le dissociazioni delle varie speci sono

$$\ce{KIO_3 -> K^+ + IO_3^-}$$
$$\ce{KI -> K^+ + I^-}$$
$$\ce{H_2SO_4 -> 2H^+ + SO_4^{2-}}$$
$$\ce{I_2 -> I_2^{(0)}}\text{(non si dissocia)}$$
$$\ce{K_2SO_4 -> 2K^+ + SO_4^{2-}}$$

Lo iodio passa da n.o. +5 nello ione iodato e da n.o. -1 nello ioduro a n.o. 0 nella molecola $\rm I_2$, quindi in parte si riduce e in parte si ossida, cioè siamo davanti ad una reazione di disproporzione.

La reazione semplificata sarà

$$\ce{IO_3^- + I^- + H^+ -> I_2 + H_2O}$$

Le semireazioni redox che avvengono sono 

$$\begin{cases}
    \ce{IO_3^- + 5e^-  -> I^0}\\
    \ce{I^- -> I^0 + e^-}
\end{cases}$$

Moltiplichiamo per 5 la specie dello ioduro e e per 1 quella dello iodato. Lo iodio verrà moltiplicato per la somma, ma siccome la specie è $\rm I_2$ moltiplichiamo per 3:

$$\ce{IO_3^- + 5I^- + H^+ -> 3I_2 + H_2O}$$

Bilanciamo le cariche. A sinistra abbiamo una carica negativa dallo iodato e 5 cariche negative dallo ioduro, per un totale di 6 cariche negative; a destra non abbiamo cariche, per cui per bilanciare abbiamo bisogno di 6 ioni $\rm H^+$ e quindi di 3 molecole d'acqua:

$$\ce{IO_3^- + 5I^- + 6H^+ -> 3I_2 + 3H_2O}$$

La reazione completa allora sarà

$$\ce{KIO_3 + 5KI + 3H_2SO_4 -> 3I_2 + 3K_2SO_4 + 3H_2O}$$

dove abbiamo successivamente bilanciato gli atomi di potassio.

Dopo aver bilanciato la reazione, notiamo che il rapporto stechiometrico tra iodato di potassio è ioduro di potassio è di 1:5, quindi una mole del primo reagirà con 5 moli del secondo. Calcoliamo le moli:

$$n_{\rm KIO_3}=\frac{21.5}{214}=1.0047 \cdot 10^{-1}\;mol
\quad;\quad
n_{\rm KI}=\frac{57}{166}=3.4337 \cdot 10^{-1}\;mol$$

Le moli del KI non sono pari a 5 volte il numero di moli del $\rm KIO_3$, per cui lo ioduro sarà il reagente limitante.

A questo punto guardiamo il rapporto stechiometrico tra il reagente limitante e lo iodio: esso è 5:3, per cui le moli di iodio saranno date dalla proporzione

$$5:3=n_{\rm KI}:n_{\rm I_2}
\implies
n_{\rm I_2}=\frac{5 \cdot 3.4337 \cdot 10^{-1}}{3}=5.7229 \cdot 10^{-1}\;mol$$

I grammi di $\rm I_2$ saranno dati da

$$g_{\rm I_2}=n \cdot MM
=5.7229 \cdot 10^{-1} \cdot 253.8
=145.2472\;grammi$$


\vspace{0.2cm}\textbf{2.2} Stabilire quanti grammi di bicromato di potassio sono necessari per l'ossidazione di 1.2345 grammi di solfato ferroso nella seguente reazione:

\begin{center}
    Solfato ferroso + Bicromato di potassio + Acido solforico \ce{->}
    
    \ce{->} Solfato di cromo (III) + Solfato di ferro (III) + Solfato di potassio + Acqua.   
\end{center}

\large\textbf{Svolgimento}\normalsize

\vspace{0.2cm}La reazione che avviene è

$$\ce{FeSO_4 + K_2Cr_2O_7 + H_2SO_4 -> Cr_2(SO_4)_3 + Fe_2(SO_4)_3 + K_2SO_4 + H_2O}$$

Le molecole dissociate sono

$$\ce{FeSO_4 -> Fe^{2+} + SO_4^{2-}}$$
$$\ce{K_2Cr_2O_7 -> 2K^+ + Cr_2O_7^{2-}}$$
$$\ce{H_2SO_4 -> 2H^+ + SO_4^{2-}}$$
$$\ce{Cr_2(SO_4)_3 -> 2Cr^{3+} + 3SO_4^{2-}}$$
$$\ce{Fe_2(SO_4)_3 -> 2Fe^{3+} + 3SO_4^{2-}}$$
$$\ce{K_2SO_4 -> 2K^+ + SO_4^{2-}}$$

Il ferro passa da n.o +2 a n.o. +3, quindi si è ossidato; il cromo passa da n.o. + 6 nel bicromato a n.o. +3 nel solfato di cromo, quindi si è ridotto.

L'equazione semplificata sarà

$$\ce{2Fe^{2+} + Cr_2O_7^{2-} + H^+ -> 2Cr^{3+} + 2Fe^{3+} + H_2O}$$

Le semireazioni redox che avvengono sono

$$\begin{cases}
    \ce{Fe^{2+} -> Fe^{3+} + e^-}\\
    \ce{Cr^{6+} + 3e^- -> Cr^{3+}}
\end{cases}$$

Visto che abbiamo due atomi di ferro e due di cromo, moltiplicheremo per due le speci chimiche contenenti il cromo e per 6 quelle contenenti il ferro:

$$\ce{12Fe^{2+} + 2Cr_2O_7^{2-} + H^+ -> 4Cr^{3+} + 12Fe^{3+} + H_2O}$$

A questo punto bilanciamo le cariche. A destra abbiamo 24 cariche positive dal ferro e 4 cariche negative dallo ione bicromato, per un totale di 20 cariche positive; a destra abbiamo 12 cariche positive dal cromo e 36 dal ferro, per un totale di 48 cariche positive. La differenza è di 28, quindi avremo bisogno di 28 ioni $\rm H^+$, e quindi 14 molecole d'acqua

$$\ce{12Fe^{2+} + 2Cr_2O_7^{2-} + 28H^+ -> 4Cr^{3+} + 12Fe^{3+} + 14H_2O}$$

Torniamo alla reazione completa

$$\ce{12FeSO_4 + 2K_2Cr_2O_7 + 14H_2SO_4 -> 2Cr_2(SO_4)_3 + 6Fe_2(SO_4)_3 + K_2SO_4 + 14H_2O}$$

Il solfato di potassio non ha partecipato alla reazione, quindi deve essere bilanciato. Abbiamo 4 atomi di potassio a sinistra, 2 a destra, quindi basterà moltiplicare per 2:

$$\ce{12FeSO_4 + 2K_2Cr_2O_7 + 14H_2SO_4 -> 2Cr_2(SO_4)_3 + 6Fe_2(SO_4)_3 + 2K_2SO_4 + 14H_2O}$$

Notiamo, inoltre, che possiamo semplificare tutti i coefficienti, per cui possiamo riscrivere la reazione come

$$\ce{6FeSO_4 + K_2Cr_2O_7 + 7H_2SO_4 -> Cr_2(SO_4)_3 + 3Fe_2(SO_4)_3 + K_2SO_4 + 7H_2O}$$

A questo punto notiamo che il rapporto stechiometrico tra solfato ferroso e bicromato di potassio è di 6:1, per cui per la totale ossidazione del solfato saranno necessarie un numero di moli del bicromato pari ad un sesto delle moli del primo. Calcoliamo le moli di $\rm FeSO_4$:

$$n_{\rm FeSO_4}=\frac{1.2345}{151.908}=8.1266 \cdot 10^{-3}$$

Le moli di bicromato necessarie saranno dato della proporzione

$$6:1=n_{\rm FeSO_4}:n_{\rm K_2Cr_2O_7}
\implies
n_{\rm K_2Cr_2O_7}
=\frac{8.1266 \cdot 10^{-3}}{6}
=1.3544 \cdot 10^{-3}\;mol$$

a cui corrispondono un numero di grammi pari a

$$g_{\rm K_2Cr_2O_7}
=1.3544 \cdot 10^{-3} \cdot 294.185
=0.3984\;grammi$$

\vspace{0.2cm}\textbf{2.3} Completare e bilanciare la seguente reazione e calcolare i grammi del sale di stagno che si formano a partire da 1.000 gr di permanganato di potassio: 
permanganato di potassio + cloruro stannoso + acido cloridrico =

\vspace{0.2cm}\large\textbf{Svolgimento}\normalsize

\vspace{0.2cm}In questo caso non abbiamo trovato una risposta plausibile. Cercando tra i testi, abbiamo trovato che la reazione che avviene è

$$\ce{KMnO4 + SnCl2 + HCl -> KCl + MnCl2 + SnCl4 + H2O}$$

Scriviamo la dissociazione delle varie speci:

$$\ce{KMnO4 -> K^+ + MnO_4^-}$$
$$\ce{SnCl2 -> Sn^{2+} + 2Cl^-}$$
$$\ce{HCl -> H^+ + Cl^-}$$
$$\ce{KCl -> K^+ + Cl^-}$$
$$\ce{MnCl2 -> Mn^{2+} + 2Cl^-}$$
$$\ce{SnCl4 -> Sn^{4+} + 4Cl^-}$$

Il manganese passo da n.o. +7 nello ione permanganato a n.o. +2 nel cloruro di
manganese, quindi si riduce; lo stagno passa da n.o. +2 nel cloruro stannoso a n.o. +4 nel cloruro stannico, quindi si è ossidato.

L'equazione semplificata sarà

$$\ce{MnO_4^- + Sn^{2+} + H^+ -> Mn^{2+} + Sn^{4+} + H_2O}$$

Le semireazioni che avvengono sono

$$\begin{cases}
    \ce{Mn^{7+} + 5e^- -> Mn^{2+}}\\
    \ce{Sn^{2+} -> Sn^{4+} + 2e^-}
\end{cases}$$

Moltiplichiamo quindi per 2 le speci del manganese e per 5 quello dello stagno:

$$\ce{2MnO_4^- + 5Sn^{2+} + H^+ -> 2Mn^{2+} + 5Sn^{4+} + H_2O}$$

A questo punto bilanciamo le cariche. A sinistra abbiamo due cariche negative dal permanganato e 10 cariche positive dallo stagno, per un totale di 8 cariche positive; a destra abbiamo 4 cariche positive dal manganese e 20 dallo stagno, per un totale di 24 cariche positive. La differenza è di 16 cariche, per cui ci serviranno 16 ioni $\rm H^+$ e quindi 8 molecole d'acqua: 

$$\ce{2MnO_4^- + 5Sn^{2+} + 16H^+ -> 2Mn^{2+} + 5Sn^{4+} + 8H_2O}$$

Tornando alla reazione completa avremo

$$\ce{2KMnO4 + 5SnCl2 + 16HCl -> 2KCl + 2MnCl2 + 5SnCl4 + 8H2O}$$

dove abbiamo bilanciato in un secondo momento gli atomi di potassio.

A questo punto calcoliamo i grammi del cloruro stannico a partire dal permanganato di potassio. Calcoliamo innanzitutto le moli di quest'ultimo:

$$n_{\rm KMnO_4}=\frac{1.0000}{158.034}
\approx 6.3277 \cdot 10^{-3}\;mol$$

Il rapporto stechiometrico tra $\rm KMnO_4$ e $\rm SnCl_4$ è 2:5, per cui 

$$2:5=n_{\rm KMnO_4}:n_{\rm SnCl_4}
\implies
n_{\rm SnCl_4}=\frac{5 \cdot 6.3277 \cdot 10^{-3}}{2}
=15.8192 \cdot 10^{-3}\;mol$$

da cui

$$g_{\rm SnCl_4}=n \cdot MM=15.8192 \cdot 10^{-3} \cdot 260.5=4.1209\;grammi$$

\vspace{0.2cm}\textbf{2.4} Calcolare i grammi di solfato ferrico che si ottengono mescolando in acqua 2.3456 grammi di permanganato di potassio  ed 12.3456 grammi di solfato ferroso, in eccesso di acido solforico.

\vspace{0.2cm}\large\textbf{Svolgimento}\normalsize

\vspace{0.2cm}Il testo non ci dice tutti i prodotti, però possiamo ottenerli: notiamo infatti che sia tra reagenti che tra i prodotti abbiamo un composto contenente ferro, mentre abbiamo composti contenenti il manganese solo tra i reagenti. Notiamo inoltre che il ferro in questa reazione si ossida, per cui il manganese si deve ridurre. Siccome siamo in presenza di acido solforico, il quale è un forte ossidante, intuiamo che il manganese passerà al n.o. di ossidazione più basso, per cui la reazione che avviene è

$$\ce{KMnO_4 + FeSO_4 + H_2SO_4 -> Fe_2(SO_4)_3 + MnSO_4 + K_2SO_4 + H_2O}$$

Scriviamo le dissociazioni delle varie speci:

$$\ce{KMnO_4 -> K^+ + MnO_4^-}$$
$$\ce{FeSO_4 -> Fe^{2+} + SO_4^{2-}}$$
$$\ce{H_2SO_4 -> 2H^+ + SO_4^{2-}}$$
$$\ce{Fe_2(SO_4)_3 -> 2Fe^{3+} + 3SO_4^{2-}}$$
$$\ce{MnSO_4 -> Mn^{2+} + SO_4^{2-}}$$
$$\ce{K_2SO_4 -> 2K^+ + SO_4^{2-}}$$

Il manganese passo da n.o. +7 nello ione permanganato a n.o. +2 nel solfato di manganese, quindi si riduce; il ferro passa da n.o. +2 nel solfato ferroso a n.o. +3 nel solfato ferrico, quindi si riduce.

L'equazione semplificata sarà

$$\ce{MnO_4^- + 2Fe^{2+} + H^+ -> 2Fe^{3+} + Mn^{2+} + H_2O}$$

Le semireazioni redox che avvengono sono

$$\begin{cases}
    \ce{Fe^{2+} -> Fe^{3+} + e^-}\\
    \ce{Mn^{7+} + 5e^- -> Mn^{2+}}
\end{cases}$$

Visto che abbiamo due atomi di ferro moltiplicheremo per due le speci del manganese e per 5 quelle del ferro:

$$\ce{2MnO_4^- + 10Fe^{2+} + H^+ -> 10Fe^{3+} + 2Mn^{2+} + H_2O}$$

A questo punto bilanciamo le cariche. A sinistra abbiamo 2 cariche negative dal permanganato e 20 cariche positive dal ferro, per un totale di 18 cariche positive; a destra abbiamo 30 cariche positive dal ferro e 4 dal manganese, per un totale di 34 cariche positive. La differenza è di 16 cariche, per cui ci serviranno 16 ioni $\rm H^+$ e quindi 8 molecole di acqua:

$$\ce{2MnO_4^- + 10Fe^{2+} + 16H^+ -> 10Fe^{3+} + 2Mn^{2+} + 8H_2O}$$

Tornando alla reazione completa avremo

$$\ce{2KMnO_4 + 10FeSO_4 + 8H_2SO_4 -> 5Fe_2(SO_4)_3 + 2MnSO_4 + K_2SO_4 + 8H_2O}$$

A questo punto calcoliamo le moli di reagenti:

$$n_{\rm KMnO_4}=\frac{2.3456}{158.034}
=1.4842 \cdot 10^{-2}\;mol
\quad;\quad
n_{\rm FeSO_4}=\frac{12.3456}{151.908}
=8.1270 \cdot 10^{-2}\;mol$$

Il rapporto stechiometrico tra $\rm KMnO_4$ e $\rm FeSO_4$ è 2:10, cioè 1:5, per cui affinché avvenga i reagenti reagiscano totalmente le moli di solfato devo essere pari a 5 volte quelle del permanganato, cioè dovrebbero essere $1.4842 \cdot 10^{-2} \cdot 5=7.421 \cdot 10^{-2}$ moli. Ne abbiamo addirittura di più, per cui il permanganato reagisce totalmente e resterà una parte di solfato indissociata.

A questo punto guardiamo il rapporto stechiometrico tra il solfato ferroso è il solfato ferrico: esso è 10:5, cioè 2:1, per cui verranno prodotte un numero di moli pari alla metà:

$$n_{\rm Fe_3(SO_4)_3}=\frac{7.421 \cdot 10^{-2}}{2}=3.7105 \cdot 10^{-2}\;mol$$

e i grammi saranno pari a 

$$g=n \cdot MM=3.7105 \cdot 10^{-2} \cdot 399.88=14.8375\;grammi$$

\vspace{0.2cm}\textbf{2.5} Calcolare quanti litri di cromato di potassio 0.14 M sono necessari per ossidare 20.000 g di solfato ferroso:
\begin{center}
\ce{FeSO4(aq) + K2CrO4(aq) + H2SO4(aq) ->}

\ce{-> Fe2(SO4)3(aq) + Cr2(SO4)3(aq) + K2SO4(aq) + H2O(l)}
\end{center}

\large\textbf{Svolgimento}\normalsize

\vspace{0.2cm}Scriviamo le dissociazioni delle varie speci:

$$\ce{FeSO4 -> Fe^{2+} + SO_4^{2-}}$$
$$\ce{K2CrO4 -> 2K^+ + CrO_4^{2-}}$$
$$\ce{H2SO4 -> 2H^+ + SO_4^{2-}}$$
$$\ce{Fe2(SO4)3 -> 2Fe^{3+} + 3SO_4^{2-}}$$
$$\ce{Cr2(SO4)3 -> 2Cr^{3+} + 3SO_4^{2-}}$$
$$\ce{K2SO4 -> 2K^+ + SO4^{2-}}$$

Il ferro passa da n.o. +2 nel solfato ferroso a n.o. +3 nel solfato ferrico, quindi si è ossidato; il cromo passa da n.o. + 6 nel cromato di potassio a n.o. +3 nel solfato di cromo (III), quindi si è ridotto. La reazione semplificata sarà

$$\ce{2Fe^{2+} + 2CrO_4^{2-} + H^+ -> 2Fe^{3+} + 2Cr^{3+} + H_2O}$$

dove abbiamo moltiplicato per 2 le speci a sinistra perché a destra le speci sono nella forma $\rm Fe_2$ e $\rm Cr_2$.

Le semireazioni redox che avvengono sono

$$\begin{cases}
    \ce{Fe^{2+} -> Fe^{3+} + e^-}\\
    \ce{Cr^{6+} + 3e^- -> Cr^{3+}}
\end{cases}$$

Visto che abbiamo due atomi di ferro e due di cromo dovremmo moltiplicare per due le speci del cromo e per 6 quelle del ferro, ma possiamo semplicemente moltiplicare per 1 le speci del cromo e per 3 quelle del ferro:

$$\ce{6Fe^{2+} + 2CrO_4^{2-} + H^+ -> 6Fe^{3+} + 2Cr^{3+} + H_2O}$$

A questo punto bilanciamo le cariche: a sinistra abbiamo 12 cariche positive dal ferro e 4 cariche negative dal cromato, per cui in totale abbiamo 8 cariche positive; a destra abbiamo 18 cariche positive dal ferro e 6 cariche positive dal cromo, per un totale di 28 cariche positive. La differenza è di 16 cariche, per cui ci serviranno 16 ioni $\rm H^+$ e quindi 8 molecole d'acqua

$$\ce{6Fe^{2+} + 2CrO_4^{2-} + 16H^+ -> 6Fe^{3+} + 2Cr^{3+} + 8H_2O}$$

La reazione completa sarà

\begin{center}

\ce{6FeSO4(aq) + 2K2CrO4(aq) + 8H2SO4(aq) ->}

\ce{-> 3Fe2(SO4)3(aq) + Cr2(SO4)3(aq) + 2K2SO4(aq) + 8H2O(l)}

\end{center}

dove abbiamo bilanciato gli atomi di potassio in un secondo momento.

\vspace{0.2cm}\textbf{2.6}  Bilanciare la seguente reazione e calcolare i grammi di $\rm CO_2$ $(MM=44.01)$ liberati da 2.0000 grammi di $\rm KMnO_4$ $(MM=158.04)$:

$$\ce{KMnO4 + K2C2O4 + H2SO4 -> MnSO4 + CO2 + K2SO4 + H2O}$$

\vspace{0.2cm}\large\textbf{Svolgimento}\normalsize

\vspace{0.2cm}Scriviamo le dissociazioni delle varie speci:

$$\ce{KMnO4 -> K^+ + MnO4^-}$$
$$\ce{K2C2O4 -> 2K^+ + C2O_4^{2-}}$$
$$\ce{H2SO4 -> 2H^+ + SO_4^{2-}}$$
$$\ce{MnSO4 -> Mn^{2+} + SO_4^{2-}}$$
$$\ce{CO2 -> CO2}\;\text{(non si dissocia)}$$
$$\ce{K2SO4 -> 2K^+ + SO_4^{2-}}$$

Il manganese passo da n.o. +7 nello ione permanganato a n.o. +2 nel solfato di manganese, quindi si è ridotto; il carbonio passa da n.o. +3 nello ione ossalato $\rm C_2O_4^{2-}$ a n.o +4 nell'anidride carbonica, quindi si è ossidato.

L'equazione semplificata sarà

$$\ce{MnO4^- + C2O_4^{2-} + H^+ -> Mn^{2+} + CO_2 + H_2O}$$

Le semireazioni redox che avvengono sono

$$\begin{cases}
    \ce{C^{3+} -> C^{4+} + e^-}\\
    \ce{Mn^{7+} + 5e^- -> Mn^{2+}}
\end{cases}$$

Visto che abbiamo due atomi di carbonio moltiplicheremo per due le speci del manganese e per 5 quelle del carbonio. Inoltre poiché a sinistra la specie è $\rm C_2O_4$ e a destra $\rm CO_2$ moltiplicheremo quest'ultima per 10:

$$\ce{2MnO4^- + 5C2O_4^{2-} + H^+ -> 2Mn^{2+} + 10CO_2 + H_2O}$$

A questo punto bilanciamo le cariche. A sinistra abbiamo 2 cariche negative dal permanganato più 10 cariche negative dall'ossalato, per un totale di 12 cariche negative; a destra abbiamo solo 4 cariche positive dal manganese, quindi per bilanciare abbiamo bisogno di 16 atomi di $\rm H^+$ e quindi 8 molecole d'acqua:

$$\ce{2MnO4^- + 5C2O_4^{2-} + 16H^+ -> 2Mn^{2+} + 10CO_2 + 8H_2O}$$

Tornando alla reazione completa avremo

$$\ce{2KMnO4 + 5K2C2O4 + 8H2SO4 -> 2MnSO4 + 10CO2 + 6K2SO4 + 8H2O}$$

dove il $\rm K_2SO_4$ è stato bilanciato in un secondo momento.

A questo punto calcoliamo i grammi di $\rm CO_2$. Il rapporto stechiometrico tra questa specie il $\rm KMnO_4$ è di 2:10 cioè 1:5, quindi verranno prodotte un numero di moli di anidride carbonica pari a 5 volte quelle di permanganato di potassio:

$$n_{\rm CO_2}=5 \cdot n_{\rm KMnO_4}
=5 \cdot \frac{2.0000}{158.034}
=1.2655 \cdot 10^{-2}\;mol$$

e i grammi di $\rm CO_2$ saranno dati da

$$g=n \cdot MM=1.2655 \cdot 10^{-2} \cdot 44.01
=0.5570\;grammi$$

\vspace{0.2cm}\textbf{2.7} Bilanciare la seguente reazione e calcolare i litri di SO2 a c.n. necessari per la riduzione di 1.5
grammi di bicromato di potassio.

\begin{center}
    Bicromato di potassio + anidride solforosa + acido cloridrico \ce{->}

    \ce{->} solfato di cromo (III) + cloruro di potassio + acqua
\end{center}

\large\textbf{Svolgimento}\normalsize

\vspace{0.2cm}La reazione che avviene, in formule, è

$$\ce{K_2Cr_2O_7 + SO_2 + HCl -> Cr_2(SO_4)_3 + KCl + H_2O}$$

Scriviamo le dissociazioni delle molecole

$$\ce{K_2Cr_2O_7 -> 2K^+ + Cr_2O_7^{2-}}$$
$$\ce{SO_2 -> SO_2^{(0)}}$$
$$\ce{HCl -> H^+ + Cl^-}$$
$$\ce{Cr_2(SO_4)_3 -> 2Cr^{3+} + 3SO_4^{2-}}$$
$$\ce{KCl -> K^+ + Cl^-}$$

Il cromo passa da n.o. +6 nel bicromato a n.o. +3 nel solfato di cromo, quindi si è ridotto; lo zolfo passa da n.o. + 4 nell'anidride a n.o. +6 nel solfato, quindi si è ossidato.

L'equazione semplificata sarà

$$\ce{Cr_2O_7^{2-} + 3SO_2 + H^+ -> 2Cr^{3+} + 3SO_4^{2-} + H_2O}$$

Le semireazioni redox che avvengono sono

$$\begin{cases}
    \ce{Cr^{6+} +3e^- -> Cr^{3+}}\\
    \ce{S^{4+} -> S^{6+} +2e}
\end{cases}$$

Visto che abbiamo due atomi di cromo e tre di zolfo, dovremmo moltiplicare per 6 le speci delle zolfo e del cromo. \E chiaro che in questo caso non ha senso moltiplicare, quindi passiamo direttamente al bilancio delle cariche: a destra abbiamo solo 2 cariche negative dal bicromato; a destra abbiamo 6 cariche positive dal cromo e 6 cariche negative dal solfato, per un totale di 0 cariche. Per bilanciare abbiamo bisogno di 2 ioni $\rm H^+$ e quindi di una molecola di $\rm H_2O$:

$$\ce{Cr_2O_7^{2-} + 3SO_2 + 2H^+ -> 2Cr^{3+} + 3SO_4^{2-} + H_2O}$$

La reazione completa sarà

$$\ce{K_2Cr_2O_7 + 3SO_2 + 2HCl -> Cr_2(SO_4)_3 + 2KCl + H_2O}$$

dove il potassio è stato bilanciato a posteriori.

\vspace{0.2cm}\textbf{2.8} Bilanciare la seguente reazione e calcolare i grammi di permanganato necessari per ossidare una
mole di solfato ferroso in eccesso di acido solforico:

\begin{center}
permanganato di potassio + solfato ferroso + acido solforico \ce{->}

\ce{->}solfato di manganese(II) + solfato ferrico + solfato di potassio + acqua
\end{center}

\large\textbf{Svolgimento}\normalsize

\vspace{0.2cm}La reazione che avviene è

$$\ce{KMnO_4 + FeSO_4 + H_2SO_4 -> MnSO_4 + Fe_2(SO_4)_3 + K_2SO_4 + H_2O}$$

Scriviamo le dissociazioni delle varie speci:

$$\ce{KMnO_4 -> K^+ + MnO_4^-}$$
$$\ce{FeSO_4 -> Fe^{2+} + SO_4^{2-}}$$
$$\ce{H_2SO_4 -> 2H^+ + SO_4^{2-}}$$
$$\ce{MnSO_4 -> Mn^{2+} + SO_4^{2-}}$$
$$\ce{Fe_2(SO_4)_3 -> 2Fe^{3+} + 3SO_4^{2-}}$$
$$\ce{K_2SO_4 -> 2K^+ + SO_4^{2-}}$$

Il manganese passa da n.o. +7 nel permanganato a n.o. +2 nel solfato di manganese, quindi si è ridotto; il ferro passa da n.o. +2 nel solfato ferroso a n.o. +3 nel solfato ferrico, quindi si è ossidato.

La reazione semplificata sarà

$$\ce{MnO_4^- + 2Fe^{2+} + H^+ -> Mn^{2+} + 2Fe^{3+} + H_2O}$$

dove abbiamo moltiplicato per due le speci del ferro perché a destra la specie è del tipo $\rm Fe_2$.

Le semireazioni che avvengono sono:

$$\begin{cases}
    \ce{Mn^{7+} + 5e^- -> Mn^{2+}}\\
    \ce{Fe^{2+} -> Fe^{3+} + e^-}
\end{cases}$$

Siccome abbiamo due atomi di ferro, moltiplichiamo per 2 le speci del manganese e per 5 quelle del ferro:

$$\ce{2MnO_4^- + 10Fe^{2+} + H^+ -> 2Mn^{2+} + 10Fe^{3+} + H_2O}$$

Bilanciamo le cariche. A sinistra abbiamo 2 cariche negative dal permanganato e 20 cariche positive dal ferro, per un totale di 18 cariche positive; a destra abbiamo 4 cariche positive dal manganese e 30 cariche positive dal ferro, per un totale di 34 cariche positive. La differenza è di 16 cariche positive, per cui ci serviranno 16 atomi di idrogeno e quindi 8 molecole d'acqua:

$$\ce{2MnO_4^- + 10Fe^{2+} + 16H^+ -> 2Mn^{2+} + 10Fe^{3+} + 8H_2O}$$

La reazione completa sarà

$$\ce{2KMnO_4 + 10FeSO_4 + 8H_2SO_4 -> 5Fe_2(SO_4)_3 + 2MnSO_4 + K_2SO_4 + 8H_2O}$$

dove abbiamo bilanciato in un secondo momento gli atomi di potassio.

A questo punto vediamo quante moli di permanganato di potassio servono per la completa ossidazione di una mole di solfato ferroso. Il rapporto stechiometrico tra questi è 2:10 cioè 1:5, per cui servirà 1/5 di mole di permanganato. I grammi corrispondenti sono pari a

$$g_{\rm KMnO_4}=n \cdot MM=\frac{1}{5} \cdot 158.034
=31.6068\;grammi$$

\vspace{0.2cm}\textbf{2.9} Bilanciare la seguente reazione e calcolare il volume di NO misurato a c.n. che si sviluppa a
partire da 1.3540 g di rame:

\begin{center}
\ce{Cu(s) + HNO3(aq) \ce{->} Cu(NO3)2(aq) + NO(g) + H2O(l)}
\end{center}

\large\textbf{Svolgimento}\normalsize

\vspace{0.2cm}Scriviamo la dissociazione delle varie speci chimiche:

$$\ce{Cu -> Cu^0}$$
$$\ce{HNO_3 -> H^+ + NO_3^-}$$
$$\ce{Cu(NO_3)_2 -> Cu^2+ + 2NO_3^-}$$
$$\ce{NO -> NO^{(0)}} \text{ non si dissocia}$$

Il rame passa da n.o. 0 nella forma atomica a n.o. +2 nel nitrato di rame, quindi si è ossidato; l'azoto passa da n.o. +5 nello ione nitrato a n.o. +2 nel monossido di azoto, quindi si è ridotto.

L'equazione semplificata sarà

$$\ce{Cu + NO_3^- + H^+ -> Cu^{2+} + NO + H_2O}$$

Le semireazioni che avvengono sono

$$\begin{cases}
    \ce{Cu^0 -> Cu^{2+} + 2e^-}\\
    \ce{N^{5+} + 3e^- -> N^{2+}}
\end{cases}$$

Moltiplichiamo quindi per 3 le speci del rame e per 2 quelle dell'azoto:

$$\ce{3Cu + 2NO_3^- + H^+ -> 3Cu^{2+} + 2NO + H_2O}$$

Bilanciamo le cariche. A sinistra abbiamo due cariche negative dallo ione nitrato, mentre a destra abbiamo sei cariche positive dallo ione rameoso. Per bilanciare ci servono 8 ioni $\rm H^+$ e quindi quattro molecole d'acqua:

$$\ce{3Cu + 2NO_3^- + 8H^+ -> 3Cu^{2+} + 2NO + 4H_2O}$$

La reazione completa sarà quindi

$$\ce{3Cu + 8HNO3 -> 3Cu(NO3)2(aq) + 2NO(g) + 4H2O(l)}$$

\textbf{Chiarimento tratto dal Giomini}

\E bene notare la discrepanza tra i valori dei coefficienti di $\rm H^+$ (8) e $\rm NO_3^-$ (2) che si traduce nel coefficiente 8 per $\rm HNO_3$. Il motivo è che nello schema di reazione si tiene conto solo degli elementi che variano il loro n.o.\,: 2 ioni $\rm NO_3^-$ diventano due molecole di NO; non si può tenere conto degli ioni $\rm NO_3^-$ che rimangono tali perché devono formare il sale $\rm Cu(NO_3)_2$ (3 molecole quindi 6 ioni). Siccome $\rm HNO_3$ fornisce sia i 2 ioni che si trasformano in NO che i 6 ioni che vanno a formare il sale, il bilanciamento assume l'aspetto sopra scritto.

(Definizione pratica: quando lo ione $\rm H^+$ proviene da una molecola in cui una specie si sta ossidando/riducendo, quando torniamo alla reazione completa la specie chimica prenderà il coefficiente dell'$\rm H^+$).

\vspace{0.2cm}\textbf{2.10} Bilanciare la seguente reazione e calcolare i grammi di cloruro di zinco che si ottengono a partire da 3.1234 grammi di anidride arseniosa.

\begin{center}
\ce{As2O3 + Zn + HCl -> AsH3 + ZnCl2 + H2O}
\end{center}

\large\textbf{Svolgimento}\normalsize

\vspace{0.2cm}Scriviamo le dissociazioni delle varie speci chimiche

$$\ce{As2O3 -> 2As^{3+} + 3O^{2-}}$$
$$\ce{Zn -> Zn^0}$$
$$\ce{HCl -> H^+ + Cl^-}$$
$$\ce{AsH_3 -> As^{3+} + 3H^{-}}$$
$$\ce{ZnCl_2 -> Zn^{2+} + 2Cl^{-}}$$

Lo zinco passa da n.o. 0 a n.o. +2 nel cloruro, quindi si ossida; L'idrogeno passa da n.o. +1 nell'acido cloridrico a n.o. -1 nell'idruro arsenioso (o arsina) $\rm AsH_3$, quindi si è ridotto.

L'equazione semplificata sarà

$$\ce{Zn + 3H^+ -> Zn^{2+} + 3H^- + H_2O }$$

Le semireazioni redox che avvengono sono

$$\begin{cases}
    \ce{Zn^{0} -> Zn^{2+} + 2e^-}\\
    \ce{H^+ + 2e^- -> H^-}
\end{cases}$$

Visto che abbiamo tre atomi di idrogeno, dobbiamo moltiplicare per 6 le speci dello zinco, mentre dovremo moltiplicare per 2 quello dell'idrogeno:

$$\ce{6Zn + 6H^+ -> 6Zn^{2+} + 6H^- + H_2O}$$

Avremmo potuto semplicemente moltiplicare per 3 quelle dello zinco, tuttavia in questo caso nel bilanciamento delle masse (le cariche sono già bilanciate: a sinistra abbiamo 6 cariche positive dall'$\rm H^+$ mentre a destra abbiamo 12 cariche positive dallo zinco e 6 cariche negative dall'idrogeno, per un totale di 6 cariche positive) avremmo ottenuto dei coefficienti non interi.

La reazione completa sarà quindi 

$$\ce{As2O3 + 6Zn + 6HCl -> 2AsH3 + 6ZnCl2 + 3H2O}$$

A questo punto dobbiamo bilanciare le masse. Notiamo infatti che a destra abbiamo 12 atomi di cloro nella specie $\rm ZnCl_2$, per cui a sinistra il coefficiente dell'HCl diventa 12, in quanto è l'unica specie ad avere atomi di cloro. Così facendo bilanciamo anche gli atomi di idrogeno, perché a destra abbiamo 6 atomi di idrogeno nell'$\rm AsH_3$ e altri 6 atomi nell'acqua. La reazione finale dunque sarà

$$\ce{As2O3 + 6Zn + 12HCl -> 2AsH3 + 6ZnCl2 + 3H2O}$$

A questo punto calcoliamo i grammi di cloruro di zinco prodotti a partire dall'anidride arseniosa. Calcoliamo le moli di quest'ultima specie:

$$n_{\rm As_2O_3}=\frac{3.1234}{197.841}
=1.5787 \cdot 10^{-5}\;mol$$

Il rapporto stechiometrico tra l'anidride arseniosa e il cloruro di zinco è 1:6, per cui il numero di moli sarà pari a 6 volte quello dell'$\rm As_2O_3$:

$$1:6=n_{\rm As_2O_3}:n_{\rm ZnCl_2}
\implies
n_{\rm ZnCl_2}=6 \cdot 1.5787 \cdot 10^{-5}
=9.4722 \cdot 10^{-5}\;mol$$

I grammi saranno allora pari a 

$$g=n \cdot MM
=9.4722 \cdot 10^{-5} \cdot 136.286
=0.1291\;grammi$$

\vspace{0.2cm}\textbf{2.11} Bilanciare la reazione e calcolare il volume di cloro gassoso necessario per ottenere 10 litri di ipoclorito 2.0 M:

\begin{center}
    Cloro(g) + idrossido di sodio(aq) \ce{->} 
    
    \ce{->} ipoclorito di sodio(aq) + cloruro di sodio(aq) + acqua. 
\end{center}

\vspace{0.2cm}\large\textbf{Svolgimento}\normalsize

\vspace{0.2cm}La reazione che avviene è

$$\ce{Cl2(g) + NaOH(aq) ->  NaClO(aq) + NaCl + H2O}$$

Scriviamo le dissociazioni delle varie speci chimiche

$$\ce{Cl_2 -> 2Cl^0 }$$
$$\ce{NaOH -> Na^+ + OH^-}$$
$$\ce{NaClO -> Na^+ + ClO^-}$$
$$\ce{NaCl -> Na^+ + Cl^-}$$

Notiamo che in questa reazione il cloro ha n.o. 0 tra i reagenti, mentre tra i prodotti ha sia n.o. +1 nell'ipoclorito che n.o. -1 nel cloruro, pertanto siamo davanti ad una reazione di disproporzione.

La reazione semplificata sarà

$$\ce{Cl_2 + OH^- -> Cl^- + ClO^- + H_2O}$$

Le semireazioni redox che avvengono sono

$$\begin{cases}
    \ce{Cl^{0} + 1e^- -> Cl^+}\\
    \ce{Cl^{0} -> Cl^+ + 1e^-}
\end{cases}$$

\E chiaro che gli elettroni sono già bilanciati. Bilanciamo le cariche.

Al secondo membro abbiamo due cariche negative mentre al primo nessuna, quindi servono due gruppi OH:

$$\ce{Cl_2 + 2OH^- -> Cl^- + ClO^- + H_2O}$$

Tornando alla reazione completa avremo

$$\ce{Cl_2 + 2NaOH -> NaCl + NaClO + H_2O}$$

Il rapporto stechiometrico tra cloro e ipoclorito è 1:1, quindi il numero di moli è lo stesso. Per calcolarle basterà moltiplicare la concentrazione per il volume

$$n_{\rm Cl_2}=n_{\rm NaClO}=c \cdot V=2.0 \cdot 10=20\;mol$$

I grammi saranno allora dati da

$$n=\frac{g}{MM} \implies g=n \cdot MM=20 \cdot 70.906=1.418,12\;grammi$$

\vspace{0.2cm}\textbf{2.12} Bilanciare la seguente reazione:

\begin{center}
    Permanganato di sodio + Cloruro stannoso + acido cloridrico \ce{->}

    \ce{->} Cloruro di sodio + Cloruro di Manganese (II) + Cloruro stannico + Acqua
\end{center}

e calcolare i grammi del sale di stagno che si formano a partire da 0.5932 grammi di permanganato di sodio

\vspace{0.2cm}\large\textbf{Svolgimento}\normalsize

\vspace{0.2cm}La reazione che avviene è

$$\ce{NaMnO4 + SnCl2 + HCl -> NaCl + MnCl2 + SnCl4 + H2O}$$

Scriviamo la dissociazione delle varie speci:

$$\ce{NaMnO4 -> Na^+ + MnO_4^-}$$
$$\ce{SnCl2 -> Sn^{2+} + 2Cl^-}$$
$$\ce{HCl -> H^+ + Cl^-}$$
$$\ce{NaCl -> Na^+ + Cl^-}$$
$$\ce{MnCl2 -> Mn^{2+} + 2Cl^-}$$
$$\ce{SnCl4 -> Sn^{4+} + 4Cl^-}$$

Il manganese passo da n.o. +7 nello ione permanganato a n.o. +2 nel cloruro di
manganese, quindi si riduce; lo stagno passa da n.o. +2 nel cloruro stannoso a n.o. +4 nel cloruro stannico, quindi si è ossidato.

L'equazione semplificata sarà

$$\ce{MnO_4^- + Sn^{2+} + H^+ -> Mn^{2+} + Sn^{4+} + H_2O}$$

Le semireazioni che avvengono sono

$$\begin{cases}
    \ce{Mn^{7+} + 5e^- -> Mn^{2+}}\\
    \ce{Sn^{2+} -> Sn^{4+} + 2e^-}
\end{cases}$$

Moltiplichiamo quindi per 2 le speci del manganese e per 5 quello dello stagno:

$$\ce{2MnO_4^- + 5Sn^{2+} + H^+ -> 2Mn^{2+} + 5Sn^{4+} + H_2O}$$

A questo punto bilanciamo le cariche. A sinistra abbiamo due cariche negative dal permanganato e 10 cariche positive dallo stagno, per un totale di 8 cariche positive; a destra abbiamo 4 cariche positive dal manganese e 20 dallo stagno, per un totale di 24 cariche positive. La differenza è di 16 cariche, per cui ci serviranno 16 ioni $\rm H^+$ e quindi 8 molecole d'acqua: 

$$\ce{2MnO_4^- + 5Sn^{2+} + 16H^+ -> 2Mn^{2+} + 5Sn^{4+} + 8H_2O}$$

Tornando alla reazione completa avremo

$$\ce{2NaMnO4 + 5SnCl2 + 16HCl -> 2NaCl + 2MnCl2 + 5SnCl4 + 8H2O}$$

dove abbiamo bilanciato in un secondo momento gli atomi di sodio.

A questo punto calcoliamo i grammi del cloruro stannico a partire dal permanganato di sodio. Calcoliamo innanzitutto le moli di quest'ultimo:

$$n_{\rm NaMnO_4}=\frac{0.5932}{141.9254}
=4.1797 \cdot 10^{-3}\;mol$$

Il rapporto stechiometrico tra $\rm NaMnO_4$ e $\rm SnCl_4$ è 2:5, per cui 

$$2:5=n_{\rm NaMnO_4}:n_{\rm SnCl_4}
\implies
n_{\rm SnCl_4}=\frac{5 \cdot 4.1797 \cdot 10^{-3}}{2}
=10.44925 \cdot 10^{-3}\;mol$$

da cui

$$g_{\rm SnCl_4}=n \cdot MM=10.44925 \cdot 10^{-3} \cdot 260.5=2.7220\;grammi$$

\vspace{0.2cm}\textbf{2.13} Bilanciare la seguente reazione e calcolare il volume in millilitri di una soluzione acquosa di acido solforico 0.1 M necessari per ridurre 1.5328 grammi di permanganato di potassio.

\begin{center}
    Permanganato di potassio + Solfito di sodio + Acido solforico \ce{->}

    \ce{->} Solfato di manganese (II) + Solfato di sodio + Solfato di potassio + acqua
\end{center}

\vspace{0.2cm}\large\textbf{Svolgimento}\normalsize

\vspace{0.2cm}La reazione che avviene è

$$\ce{KMnO_4 + Na_2SO_3 + H_2SO_4 -> MnSO_4 + Na_2SO_4 + K_2SO_4 + H_2O}$$

Scriviamo le dissociazioni delle varie speci:

$$\ce{KMnO_4 -> K^+ + MnO_4^-}$$
$$\ce{Na_2SO_3 -> 2Na^+ + SO_3^{2-}}$$
$$\ce{H_2SO_4 -> 2H^+ + SO_4^{2-}}$$
$$\ce{MnSO_4 -> Mn^{2+} + SO_4^{2-}}$$
$$\ce{Na_2SO_4 -> 2Na^+ + SO_4^{2-}}$$
$$\ce{K_2SO_4 -> 2K^+ + SO_4^{2-}}$$

Il manganese passa n.o. +7 nel permanganato a n.o. +2 nel solfato, pertanto si è ridotto; lo zolfo passa da n.o. +4 nel solfito di sodio a n.o. +6 nel solfato di sodio, quindi si è ossidato.

L'equazione semplificata sarà

$$\ce{MnO_4^- + SO_3^{2-} + H^+ -> Mn^{2+} + SO_4^{2-} + H_2O}$$

Le semireazioni redox che avvengono sono

$$\begin{cases}
\ce{Mn^{7+} + 5e^- -> Mn^{2+}}\\
\ce{S^{4+} -> S^{6+} + 2e^-}
\end{cases}$$

Moltiplichiamo quindi per 2 le speci del manganese e per 5 quelle dello zolfo:

$$\ce{2MnO_4^- + 5SO_3^{2-} + H^+ -> 2Mn^{2+} + 5SO_4^{2-} + H_2O}$$

A questo punto bilanciamo le cariche. A sinistra abbiamo due cariche negative dal permanganato e 10 cariche negative dal solfito, per un totale di 12 cariche negative; a destra abbiamo 4 cariche positive dal manganese e 10 cariche negative dal solfato, per un totale di 6 cariche negative. La differenza è di 6 cariche, per cui ci serviranno 6 ioni $\rm H^+$ e quindi 3 molecole d'acqua:

$$\ce{2MnO_4^- + 5SO_3^{2-} + 6H^+ -> 2Mn^{2+} + 5SO_4^{2-} + 3H_2O}$$

Tornando alla reazione completa avremo:

$$\ce{2KMnO_4 + 5Na_2SO_3 + 3H_2SO_4 -> 2MnSO_4 + 5Na_2SO_4 + K_2SO_4 + 3H_2O}$$

\textbf{continua}

\vspace{0.2cm}\textbf{2.14} Lo zinco in acido cloridrico produce cloruro di zinco ed idrogeno. Calcolare il volume di
idrogeno misurato a c.n. ed i grammi di cloruro di zinco che si ottengono per reazione di 2.42 g di zinco con 100 mL di una soluzione acquosa di HCl 0.5 M.

\vspace{0.2cm}\large\textbf{Svolgimento}\normalsize

\vspace{0.2cm}La reazione che avviene è

$$\ce{Zn + HCl -> ZnCl_2 + H_2 ^}$$

Scriviamo le dissociazioni delle varie speci:

$$\ce{Zn -> Zn^{(0)}}$$
$$\ce{HCl -> H^+ + Cl^-}$$
$$\ce{ZnCl_2 -> Zn^{2+} + 2Cl^-}$$
$$\ce{H_2 -> H_2^{(0)}}$$

Lo zinco passa da n.o. 0 a n.o. +2, quindi si è ossidato; l'idrogeno passa da n.o. +1 a n.o. 0, quindi si è ridotto. La reazione semplificata sarà

$$\ce{Zn + 2H^+ -> Zn^{2+} + H_2}$$

dove abbiamo messo 2 davanti ad $\rm H^+$ perché a destra la specie è $\rm H_2$.

Le semireazioni che avvengono sono

$$\begin{cases}
    \ce{Zn -> Zn^{2+} + 2e^-}\\
    \ce{2H^{+} + 2e^- -> H_2}
\end{cases}$$

Siccome dovremmo moltiplicare tutte le speci per 2 non moltiplichiamo nessuna specie. Notiamo che le cariche sono già bilanciate.

La reazione bilanciata sarà

$$\ce{Zn + 2HCl -> ZnCl_2 + H_2 ^}$$

Vediamo adesso le moli dei reagenti:

$$n_{\rm Zn}=\frac{g}{MM_{\rm Zn}}=\frac{2.42}{65.38}=3.7014 \cdot 10^{-2}\;mol$$

Per le moli di acido cloridrico facciamo la proporzione

$$n_{\rm HCl}:100=0.5:1000
\implies
n_{\rm HCl}=\frac{100 \cdot 0.5}{1000}=5 \cdot 10^{-2}\;mol$$

Il rapporto stechiometrico tra zinco e acido cloridrico è 1:2, per cui per la totale reazione dei reagenti le moli di acido devono essere pari al doppio di quelle di zinco, cioè dovrebbero essere $7.4028 \cdot 10^{-2}$. Noi però ne abbiamo solo $5 \cdot 10^{-2}$, ossia sono in difetto e quindi sarà l'HCl il reagente limitante, per cui dovremo fare riferimento ad esso.

Il rapporto stechiometrico tra acido cloridrico e cloruro di zinco è 2:1, quindi otterremo $\frac{5\cdot 10^{-2}}{2}=2.5 \cdot 10^{-2}$ moli di $\rm ZnCl_2$. I grammi saranno dati da

$$g=n_{\rm ZnCl_2} \cdot MM_{\rm ZnCl_2}=2.5 \cdot 10^{-2} \cdot 136.286=3.40715\;grammi$$

Per quanto riguarda l'idrogeno, il rapporto stechiometrico tra l'acido ed esso è 2:1, per cui anche di questo ne verranno prodotte $2.5 \cdot 10^{-2}$ moli. Per ottenere il volume dobbiamo usare l'equazione di stato dei gas:

$$PV=nRT \implies V=\frac{nRT}{P}$$

$$\implies V_{\rm H_2}
=\frac{2.5 \cdot 10^{-2} \cdot 0.082 \cdot 273.15}{1}
=0.5599\;L$$

\vspace{0.2cm}\textbf{2.15} Bilanciare la seguente reazione e calcolare i grammi di permanganato necessari per ossidare 2.2414 L di acido solforico prelevati a c.n.\,:

$$\ce{KMnO_4 + H_2S + H_2SO_4 -> MnSO_4 + S + K_2SO_4 + H_2O}$$

\vspace{0.2cm}\large\textbf{Svolgimento}\normalsize

\vspace{0.2cm}Scriviamo le dissociazioni delle varie speci:

\textbf{continua}

\vspace{0.2cm}\textbf{2.16} La reazione tra solfato ferroso e acido nitrico produce solfato ferrico, nitrato ferrico, monossido di azoto e acqua. Calcolare il volume di monossido di azoto NO(g) che si sviluppa a 27° C e 645 mmHg, facendo reagire 9.1145 grammi di sale ferroso con 3.7806 di acido.

\vspace{0.2cm}\large\textbf{Svolgimento}\normalsize

\vspace{0.2cm}La reazione che avviene è

$$\ce{FeSO_4 + HNO_3 -> Fe_2(SO_4)_3 + Fe(NO_3)_3 + NO + H_2O}$$

Scriviamo le dissociazioni delle varie speci:

$$\ce{FeSO_4 -> Fe^{2+} + SO_4^{2-}}$$
$$\ce{HNO_3 -> H^+ + NO_3^-}$$
$$\ce{Fe_2(SO_4)_3 -> 2Fe^{3+} + 3SO_4^{2-}}$$
$$\ce{Fe(NO_3)_3 -> Fe^{3+} + 3NO_3^-}$$
$$\ce{NO -> NO^{(0)}}$$

Il ferro passa da n.o. +2 nel solfato ferroso a n.o. +3 sia nel nitrato ferrico che nel solfato ferrico, quindi si ossida; l'azoto passa da n.o. +5 nell'acido nitrico a +2 nel monossido di azoto, quindi si riduce.

L'equazione semplificata sarà

$$\ce{3Fe^{2+} + NO_3^- + H^+ -> 3Fe^{3+} + NO + H_2O}$$

Attenzioniamo per un momento il coefficiente del ferro. A destra tale specie compare sia nel solfato ferrico che nel nitrato ferrico, per cui il suo coefficiente deve essere pari alla somma del numero di atomi. Siccome nel solfato la specie è del tipo $\rm Fe_2$ mentre nel nitrato è solo Fe il coefficiente risulta 3, quindi moltiplicheremo per 3 il ferro anche a destra.

Le semireazioni che avvengono sono:

$$\begin{cases}
    \ce{Fe^{2+} -> Fe^{3+} + e^-}\\
    \ce{N^{5+} +3e^- -> N^{2+}}
\end{cases}$$

Siccome abbiamo 3 atomi di ferro dovremmo moltiplicare per 3 le speci dell'azoto e per 3 quelle del ferro, quindi non moltiplichiamo. A questo punto bilanciamo le cariche: a sinistra abbiamo 6 cariche positive dal ferro e una carica negativa dal nitrato, per un totale di 5 cariche positive; a destra abbiamo solo 9 cariche positive dal ferro. La differenza è di 4 cariche, per cui ci servono 4 ioni $\rm H^+$ e quindi due molecole d'acqua:

$$\ce{3Fe^{2+} + NO_3^- + 4H^+ -> 3Fe^{3+} + NO + 2H_2O}$$

Tornando alla reazione completa avremo

$$\ce{3FeSO_4 + 4HNO_3 -> Fe_2(SO_4)_3 + Fe(NO_3)_3 + NO + 2H_2O}$$

Per quanto riguarda il coefficiente dell'$\rm HNO_3$ si rimanda al chiarimento che si trova nell'EXR. 2.9; per quanto riguarda i coefficienti del solfato ferrico bisogna ricordarsi che della quantità totale di ferro due terzi provengono dal solfato e un terzo dal nitrato. In altre parole, se ad esempio avessimo moltiplicato le speci per 3 nel bilanciamento degli elettroni avremmo ottenuto $\rm 9Fe^{3+}$ per cui due terzi di questo, cioè 6 atomi, sarebbero andati al solfato, i restanti 3 al nitrato. Inoltre, siccome nel solfato si trova in forma $\rm Fe_2$ bisognerà dividere il coefficiente.

\textbf{continua}

