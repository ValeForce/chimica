\vspace{0.2cm}\textbf{2.2} Bilanciare la  seguente reazione e calcolare i grammi di iodio che si ottengono facendo reagire 21,5 g di $\rm KIO_3$ (MM 214,0) con 57,0 g di KI (MM = 166,0).

\vspace{0.2cm}
\begin{center}Iodato di potassio + ioduro di potassio + acido solforico \ce{->}

\ce{->}iodio +solfato di potassio +acqua. 
\end{center}

\large\textbf{Svolgimento}\normalsize

\vspace{0.2cm}La reazione che avviene è

$$\ce{KIO_3 + KI + H_2SO_4 -> I_2 + K_2SO_4 + H_2O}$$

\vspace{0.2cm}\textbf{2.3} Stabilire quanti grammi di bicromato di potassio sono necessari per l’ossidazione di 1.2345 grammi di solfato ferroso nella seguente reazione:

\begin{center}
    Solfato ferroso + Bicromato di potassio + Acido solforico \ce{->}
    
    \ce{->} Solfato di cromo (III) + Solfato di ferro (III), Solfato di potassio + Acqua.   
\end{center}

\large\textbf{Svolgimento}\normalsize

\vspace{0.2cm}La reazione che avviene è

$$\ce{FeSO_4 + K_2CrO_4 + H_2SO_4 -> Cr_2(SO_4)_3 Fe_2(SO_4)_3 + K_2SO_4 + H_2O}$$

Le molecole dissociate sono

$$\ce{FeSO_4 -> Fe^{2+} + SO_4^{2-}}$$
$$\ce{K_2CrO_4 -> 2K^+ + CrO_4^{2-}}$$
$$\ce{H_2SO_4 -> 2H^+ + SO_4^{2-}}$$
$$\ce{Cr_2(SO_4)_3 -> 2Cr^{3+} + 3SO_4^{3-}}$$
$$\ce{Fe_2(SO_4)_3 -> 2Fe^{3+} + 3SO_4^{2-}}$$
$$\ce{K_2SO_4 -> 2K^+ + SO_4^{2-}}$$

Riscriviamo l'equazioni con solo le speci il cui n.o. cambia:

$$\ce{Fe^{2+} + CrO_4^{2-} + H^+ -> 2Cr^{3+} + 2Fe^{3+} + H_20}$$

\vspace{0.2cm}\textbf{2.4} Completare e bilanciare la seguente reazione e calcolare i grammi del sale di stagno che si formano a partire da 1.000 gr di permanganato di potassio: 
permanganato di potassio + cloruro stannoso + acido cloridrico =

\vspace{0.2cm}\textbf{2.5} Calcolare i grammi di solfato ferrico che si ottengano mescolando in acqua 2.3456 grammi di permanganato di potassio  ed 12,3456 grammi di solfato ferroso, in eccesso di acido solforico.

\vspace{0.2cm}\textbf{2.6} Calcolare quanti litri di cromato di potassio 0.14 M sono necessari per ossidare 20.000 g di solfato ferroso:
\begin{center}
\ce{FeSO4(aq) + K2CrO4(aq) + H2SO4(aq ) ->}

\ce{-> Fe2(SO4)3(aq) + Cr2(SO4)3(aq) + K2SO4(aq) + H2O(l)}
\end{center}

\vspace{0.2cm}\textbf{2.7} Lo zinco in acido cloridrico produce cloruro di zinco ed idrogeno. Calcolare il volume di
idrogeno misurato a c.n. ed i grammi di cloruro di zinco che si ottengono per reazione di 2.42 g di
zinco con 100 mL di una soluzione acquosa di HCl 0.5 M.

\vspace{0.2cm}\textbf{2.8} Calcolare i grammi di bromuro di litio che si ottengono da 4.5678 g di carbonato di litio fatti
reagire con acido bromidrico. La reazione, da bilanciare, genera inoltre anidride carbonica ed acqua.

\vspace{0.2cm}\textbf{2.9} Bilanciare la seguente reazione e calcolare i litri di SO2 a c.n. necessari per la riduzione di 1.5
grammi di bicromato di potassio.

Bicromato di potassio + anidride solforosa + acido cloridrico \ce{->} solfato di cromo(III) + cloruro di
potassio + acqua.

\vspace{0.2cm}\textbf{2.10} Bilanciare la seguente reazione e calcolare i grammi di permanganato necessari per ossidare una
mole di solfato ferroso in eccesso di acido solforico:

\begin{center}
permanganato di potassio + solfato ferroso + acido solforico \ce{->}

\ce{->}solfato di manganese(II) + solfato ferrico + solfato di potassio + acqua.
\end{center}

\vspace{0.2cm}\textbf{2.11} Bilanciare la seguente reazione e calcolare il volume di NO misurato a c.n. che si sviluppa a
partire da 1.3540 g di rame:

\begin{center}
\ce{Cu(s) + HNO3(aq) \ce{->} Cu(NO3)2(aq) + NO(g) + H2O(l)}
\end{center}

\vspace{0.2cm}\textbf{2.12} Bilanciare la seguente reazione e calcolare i grammi di cloruro di zinco che si ottengono a partire da 3.1234 grammi di anidride arseniosa.

\begin{center}
\ce{As2O3 + Zn + HCl -> AsH3 + ZnCl2 + H2O}
\end{center}

\vspace{0.2cm}\textbf{2.13} Bilanciare la reazione e calcolare il volume di cloro gassoso necessario per ottenere 10 litri di ipoclorito 2.0 M:

\begin{center}
    Cloro gassoso + idrossido di sodio(aq) acquoso \ce{->} 
    
    \ce{->} ipoclorito di sodio(aq) + cloruro di sodio(aq) + acqua. 
\end{center}


\ce{Cl2(g) + NaOH(aq) ->  NaClO(aq) + NaCl + H2O}


\vspace{0.2cm}\textbf{2.14} Calcolare i grammi di solfato ferrico che si ottengano mescolando in acqua 2.3456 grammi di permanganato di potassio  e 12,3456 grammi di solfato ferroso, in eccesso di acido solforico.

\vspace{0.2cm}\textbf{2.15} 

\vspace{0.2cm}\textbf{2.16} 

\vspace{0.2cm}\textbf{2.} 