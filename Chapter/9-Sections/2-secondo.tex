\begin{esercizio}
    Bilanciare la seguente reazione e calcolare i grammi di iodio che si ottengono facendo reagire 21.5 g di $\rm KIO_3$ $(MM=214.0)$ con 57.0 g di KI $(MM=166.0)$.
    \begin{gather*}
        \text{Iodato di potassio}
        + \text{Ioduro di potassio}
        + \text{Acido solforico}
        \ce{->}
        \\
        \ce{->}
        \text{Iodio}
        + \text{Solfato di potassio}
        + \text{Acqua}
    \end{gather*}
\end{esercizio}
\begin{soluzione}
    La reazione che avviene è
    \begin{equation*}
        \ce{KIO_3 + KI + H_2SO_4 -> I_2 + K_2SO_4 + H_2O}
    \end{equation*}
    Le dissociazioni delle varie specie sono
    \begin{equation*}
        \begin{split}
            \ce{KIO_3} & \ce{-> K^+ + IO_3^-}
            \\
            \ce{KI} & \ce{-> K^+ + I^-}
            \\
            \ce{H_2SO_4} & \ce{-> 2H^+ + SO_4^{2-}}
            \\
            \ce{I_2} & \ce{-> I_2^{(0)}}\text{(non si dissocia)}
            \\
            \ce{K_2SO_4} & \ce{-> 2K^+ + SO_4^{2-}}
            \\
        \end{split}
    \end{equation*}
    Lo iodio passa da n.o. $+5$ nello ione iodato e da n.o. $-1$ nello ioduro a n.o. $0$ nella molecola $\rm I_2$, quindi in parte si riduce e in parte si ossida, cioè siamo davanti ad una reazione di disproporzione.\\
    La reazione semplificata sarà
    \begin{equation*}
        \ce{IO_3^- + I^- + H^+ -> I_2 + H_2O}
    \end{equation*}
    Le semireazioni redox che avvengono sono 
    \begin{equation*}
        \begin{cases}
            \ce{IO_3^- + 5e^- -> I^0}\\
            \ce{I^- -> I^0 + e^-}
        \end{cases}
    \end{equation*}
    Moltiplichiamo per 5 la specie dello ioduro e e per 1 quella dello iodato. Lo iodio verrà moltiplicato per la somma, ma siccome la specie è $\rm I_2$ moltiplichiamo per 3:
    \begin{equation*}
        \ce{IO_3^- + 5I^- + H^+ -> 3I_2 + H_2O}
    \end{equation*}
    Bilanciamo le cariche. A sinistra abbiamo una carica negativa dallo iodato e 5 cariche negative dallo ioduro, per un totale di 6 cariche negative; a destra non abbiamo cariche, per cui per bilanciare abbiamo bisogno di 6 ioni $\rm H^+$ e quindi di 3 molecole d'acqua:
    \begin{equation*}
        \ce{IO_3^- + 5I^- + 6H^+ -> 3I_2 + 3H_2O}
    \end{equation*}
    La reazione completa allora sarà
    \begin{equation*}
        \ce{KIO_3 + 5KI + 3H_2SO_4 -> 3I_2 + 3K_2SO_4 + 3H_2O}
    \end{equation*}
    dove abbiamo successivamente bilanciato gli atomi di potassio.\\
    Dopo aver bilanciato la reazione, notiamo che il rapporto stechiometrico tra iodato di potassio è ioduro di potassio è di $1:5$, quindi una mole del primo reagirà con 5 moli del secondo. Calcoliamo le moli:
   \begin{equation*}
        n_{\rm KIO_3}=\frac{21.5}{214}=1.0047 \cdot 10^{-1} \; \text{mol}
        \quad;\quad
        n_{\rm KI}=\frac{57}{166}=3.4337 \cdot 10^{-1} \; \rm mol
   \end{equation*}
    Le moli del KI non sono pari a 5 volte il numero di moli del $\rm KIO_3$, per cui lo ioduro sarà il reagente limitante.\\
    A questo punto guardiamo il rapporto stechiometrico tra il reagente limitante e lo iodio: esso è $5:3$, per cui le moli di iodio saranno date dalla proporzione
   \begin{equation*}
    5:3=n_{\rm KI}:n_{\rm I_2}
    \implies
    n_{\rm I_2}=\frac{3 \cdot 3.4337 \cdot 10^{-1}}{5}
    =2.0602 \cdot 10^{-1} \; \rm mol
   \end{equation*}
    I grammi di $\rm I_2$ saranno dati da
   \begin{equation*}
    g_{\rm I_2}=n \cdot MM
    =2.0602 \cdot 10^{-1} \cdot 253.8
    =52.2883 \; \rm grammi
   \end{equation*}
\end{soluzione}

\newpage

\begin{esercizio}
    Stabilire quanti grammi di bicromato di potassio sono necessari per l'ossidazione di 1.2345 grammi di solfato ferroso nella seguente reazione:
    \begin{gather*}
        \text{Solfato ferroso + Bicromato di potassio + Acido solforico} \ce{->}
        \\
        \ce{->} \text{Solfato di cromo (III) + Solfato di ferro (III) + Solfato di potassio + Acqua}
    \end{gather*}
\end{esercizio}
\begin{soluzione}
    La reazione che avviene è
    \begin{equation*}
        \ce{FeSO_4 + K_2Cr_2O_7 + H_2SO_4 -> Cr_2(SO_4)_3 + Fe_2(SO_4)_3 + K_2SO_4 + H_2O}
    \end{equation*}
    Le molecole dissociate sono
    \begin{equation*}
        \begin{split}
            \ce{FeSO_4} & \ce{-> Fe^{2+} + SO_4^{2-}}
            \\
            \ce{K_2Cr_2O_7} & \ce{-> 2K^+ + Cr_2O_7^{2-}}
            \\
            \ce{H_2SO_4} & \ce{-> 2H^+ + SO_4^{2-}}
            \\
            \ce{Cr_2(SO_4)_3} & \ce{-> 2Cr^{3+} + 3SO_4^{2-}}
            \\
            \ce{Fe_2(SO_4)_3} & \ce{-> 2Fe^{3+} + 3SO_4^{2-}}
            \\
            \ce{K_2SO_4} & \ce{-> 2K^+ + SO_4^{2-}}
        \end{split}
    \end{equation*}
    Il ferro passa da n.o. $+2$ a n.o. $+3$, quindi si è ossidato; il cromo passa da n.o. $+6$ nel bicromato a n.o. $+3$ nel solfato di cromo, quindi si è ridotto.\\
    L'equazione semplificata sarà
    \begin{equation*}
        \ce{2Fe^{2+} + Cr_2O_7^{2-} + H^+ -> 2Cr^{3+} + 2Fe^{3+} + H_2O}
    \end{equation*}
    Le semireazioni redox che avvengono sono
    \begin{equation*}
        \begin{cases}
            \ce{Fe^{2+} -> Fe^{3+} + e^-}\\
            \ce{Cr^{6+} + 3e^- -> Cr^{3+}}
        \end{cases}
    \end{equation*}
    Visto che abbiamo due atomi di ferro e due di cromo, moltiplicheremo per due le specie chimiche contenenti il cromo e per 6 quelle contenenti il ferro:
    \begin{equation*}
        \ce{12Fe^{2+} + 2Cr_2O_7^{2-} + H^+ -> 4Cr^{3+} + 12Fe^{3+} + H_2O}
    \end{equation*}
    A questo punto bilanciamo le cariche. A destra abbiamo 24 cariche positive dal ferro e 4 cariche negative dallo ione bicromato, per un totale di 20 cariche positive; a destra abbiamo 12 cariche positive dal cromo e 36 dal ferro, per un totale di 48 cariche positive. La differenza è di 28, quindi avremo bisogno di 28 ioni $\rm H^+$, e quindi 14 molecole d'acqua
    \begin{equation*}
        \ce{12Fe^{2+} + 2Cr_2O_7^{2-} + 28H^+ -> 4Cr^{3+} + 12Fe^{3+} + 14H_2O}
    \end{equation*}
    Torniamo alla reazione completa
    \begin{equation*}
        \ce{12FeSO_4 + 2K_2Cr_2O_7 + 14H_2SO_4 -> 2Cr_2(SO_4)_3 + 6Fe_2(SO_4)_3 + K_2SO_4 + 14H_2O}
    \end{equation*}
    Il solfato di potassio non ha partecipato alla reazione, quindi deve essere bilanciato. Abbiamo 4 atomi di potassio a sinistra, 2 a destra, quindi basterà moltiplicare per 2:
    \begin{equation*}
        \ce{12FeSO_4 + 2K_2Cr_2O_7 + 14H_2SO_4 -> 2Cr_2(SO_4)_3 + 6Fe_2(SO_4)_3 + 2K_2SO_4 + 14H_2O}
    \end{equation*}
    Notiamo, inoltre, che possiamo semplificare tutti i coefficienti, per cui possiamo riscrivere la reazione come
    \begin{equation*}
        \ce{6FeSO_4 + K_2Cr_2O_7 + 7H_2SO_4 -> Cr_2(SO_4)_3 + 3Fe_2(SO_4)_3 + K_2SO_4 + 7H_2O}
  \end{equation*}
    A questo punto notiamo che il rapporto stechiometrico tra solfato ferroso e bicromato di potassio è di $6:1$, per cui per la totale ossidazione del solfato saranno necessarie un numero di moli del bicromato pari ad un sesto delle moli del primo. Calcoliamo le moli di $\rm FeSO_4$:
    \begin{equation*}
        n_{\rm FeSO_4}=\frac{1.2345}{151.908}=8.1266 \cdot 10^{-3} \; \rm mol
    \end{equation*}
    Le moli di bicromato necessarie saranno dato della proporzione
    \begin{equation*}
        6:1=n_{\rm FeSO_4}:n_{\rm K_2Cr_2O_7}
        \implies
        n_{\rm K_2Cr_2O_7}
        =\frac{8.1266 \cdot 10^{-3}}{6}
        =1.3544 \cdot 10^{-3} \; \rm mol
    \end{equation*}
    a cui corrispondono un numero di grammi pari a
    \begin{equation*}
        g_{\rm K_2Cr_2O_7}
        =1.3544 \cdot 10^{-3} \cdot 294.185
        =0.3984 \; \rm grammi
    \end{equation*}
\end{soluzione}

\newpage

\begin{esercizio}
    Completare e bilanciare la seguente reazione e calcolare i grammi del sale di stagno che si formano a partire da 1.000 gr di permanganato di potassio:
    \begin{equation*}
        \text{Permanganato di potassio + Cloruro stannoso + Acido cloridrico}
        \ce{->}
    \end{equation*}
\end{esercizio}
\begin{soluzione}
    In questo caso non abbiamo trovato una risposta plausibile. Cercando tra i testi, abbiamo trovato che la reazione che avviene è
    \begin{equation*}
        \ce{KMnO4 + SnCl2 + HCl -> KCl + MnCl2 + SnCl4 + H2O}
    \end{equation*}
    Scriviamo la dissociazione delle varie specie:
    \begin{equation*}
        \begin{split}
            \ce{KMnO4} & \ce{-> K^+ + MnO_4^-}
            \\
            \ce{SnCl2} & \ce{-> Sn^{2+} + 2Cl^-}
            \\
            \ce{HCl} & \ce{-> H^+ + Cl^-}
            \\
            \ce{KCl} & \ce{-> K^+ + Cl^-}
            \\
            \ce{MnCl2} & \ce{-> Mn^{2+} + 2Cl^-}
            \\
            \ce{SnCl4} & \ce{-> Sn^{4+} + 4Cl^-}
        \end{split}
    \end{equation*}
    Il manganese passo da n.o. $+7$ nello ione permanganato a n.o. $+2$ nel cloruro di manganese, quindi si riduce; lo stagno passa da n.o. $+2$ nel cloruro stannoso a n.o. $+4$ nel cloruro stannico, quindi si è ossidato.\\
    L'equazione semplificata sarà
    \begin{equation*}
        \ce{MnO_4^- + Sn^{2+} + H^+ -> Mn^{2+} + Sn^{4+} + H_2O}
    \end{equation*}
    Le semireazioni che avvengono sono
    \begin{equation*}
        \begin{cases}
            \ce{Mn^{7+} + 5e^- -> Mn^{2+}}\\
            \ce{Sn^{2+} -> Sn^{4+} + 2e^-}
        \end{cases}
    \end{equation*}
    Moltiplichiamo quindi per 2 le specie del manganese e per 5 quello dello stagno:
    \begin{equation*}
        \ce{2MnO_4^- + 5Sn^{2+} + H^+ -> 2Mn^{2+} + 5Sn^{4+} + H_2O}
    \end{equation*}
    A questo punto bilanciamo le cariche. A sinistra abbiamo due cariche negative dal permanganato e 10 cariche positive dallo stagno, per un totale di 8 cariche positive; a destra abbiamo 4 cariche positive dal manganese e 20 dallo stagno, per un totale di 24 cariche positive. La differenza è di 16 cariche, per cui ci serviranno 16 ioni $\rm H^+$ e quindi 8 molecole d'acqua:
    \begin{equation*}
        \ce{2MnO_4^- + 5Sn^{2+} + 16H^+ -> 2Mn^{2+} + 5Sn^{4+} + 8H_2O}
    \end{equation*}
    Tornando alla reazione completa avremo
    \begin{equation*}
        \ce{2KMnO4 + 5SnCl2 + 16HCl -> 2KCl + 2MnCl2 + 5SnCl4 + 8H2O}
    \end{equation*}
    dove abbiamo bilanciato in un secondo momento gli atomi di potassio.\\
    Calcoliamo adesso i grammi del cloruro stannico a partire dal permanganato di potassio. Calcoliamo innanzitutto le moli di quest'ultimo:
    \begin{equation*}
        n_{\rm KMnO_4}=\frac{1.0000}{158.034}
        \approx 6.3277 \cdot 10^{-3} \; \rm mol
    \end{equation*}
    Il rapporto stechiometrico tra $\rm KMnO_4$ e $\rm SnCl_4$ è $2:5$, per cui 
    \begin{equation*}
        2:5=n_{\rm KMnO_4}:n_{\rm SnCl_4}
        \implies
        n_{\rm SnCl_4}=\frac{5 \cdot 6.3277 \cdot 10^{-3}}{2}
        =15.8192 \cdot 10^{-3} \; \rm mol
    \end{equation*}
    da cui
    \begin{equation*}
        g_{\rm SnCl_4}
        =n \cdot MM=15.8192 \cdot 10^{-3} \cdot 260.5
        =4.1209 \; \rm grammi
    \end{equation*}
\end{soluzione}

\newpage

\begin{esercizio}
    Calcolare i grammi di solfato ferrico che si ottengono mescolando in acqua 2.3456 grammi di permanganato di potassio e 12.3456 grammi di solfato ferroso, in eccesso di acido solforico.
\end{esercizio}
\begin{soluzione}
    Il testo non ci dice tutti i prodotti, però possiamo ottenerli: notiamo infatti che sia tra reagenti che tra i prodotti abbiamo un composto contenente ferro, mentre abbiamo composti contenenti il manganese solo tra i reagenti. Notiamo inoltre che il ferro in questa reazione si ossida, per cui il manganese si deve ridurre. Siccome siamo in presenza di acido solforico, il quale è un forte ossidante, intuiamo che il manganese passerà al numero di ossidazione più basso, per cui la reazione che avviene è
    \begin{equation*}
        \ce{KMnO_4 + FeSO_4 + H_2SO_4 -> Fe_2(SO_4)_3 + MnSO_4 + K_2SO_4 + H_2O}
    \end{equation*}
    Scriviamo le dissociazioni delle varie specie:
    \begin{equation*}
        \begin{split}
            \ce{KMnO_4} & \ce{-> K^+ + MnO_4^-}
            \\
            \ce{FeSO_4} & \ce{-> Fe^{2+} + SO_4^{2-}}
            \\
            \ce{H_2SO_4} & \ce{-> 2H^+ + SO_4^{2-}}
            \\
            \ce{Fe_2(SO_4)_3} & \ce{-> 2Fe^{3+} + 3SO_4^{2-}}
            \\
            \ce{MnSO_4} & \ce{-> Mn^{2+} + SO_4^{2-}}
            \\
            \ce{K_2SO_4} & \ce{-> 2K^+ + SO_4^{2-}}
        \end{split}
    \end{equation*}
    Il manganese passo da n.o. $+7$ nello ione permanganato a n.o. $+2$ nel solfato di manganese, quindi si riduce; il ferro passa da n.o. $+2$ nel solfato ferroso a n.o. $+3$ nel solfato ferrico, quindi si riduce.\\
    L'equazione semplificata sarà
    \begin{equation*}
        \ce{MnO_4^- + 2Fe^{2+} + H^+ -> 2Fe^{3+} + Mn^{2+} + H_2O}
    \end{equation*}
    Le semireazioni redox che avvengono sono
    \begin{equation*}
        \begin{cases}
            \ce{Fe^{2+} -> Fe^{3+} + e^-}\\
            \ce{Mn^{7+} + 5e^- -> Mn^{2+}}
        \end{cases}
    \end{equation*}
    Visto che abbiamo due atomi di ferro moltiplicheremo per due le specie del manganese e per 5 quelle del ferro:
    \begin{equation*}
        \ce{2MnO_4^- + 10Fe^{2+} + H^+ -> 10Fe^{3+} + 2Mn^{2+} + H_2O}
    \end{equation*}
    A questo punto bilanciamo le cariche. A sinistra abbiamo 2 cariche negative dal permanganato e 20 cariche positive dal ferro, per un totale di 18 cariche positive; a destra abbiamo 30 cariche positive dal ferro e 4 dal manganese, per un totale di 34 cariche positive. La differenza è di 16 cariche, per cui ci serviranno 16 ioni $\rm H^+$ e quindi 8 molecole di acqua:
    \begin{equation*}
        \ce{2MnO_4^- + 10Fe^{2+} + 16H^+ -> 10Fe^{3+} + 2Mn^{2+} + 8H_2O}
    \end{equation*}
    Tornando alla reazione completa avremo
    \begin{equation*}
        \ce{2KMnO_4 + 10FeSO_4 + 8H_2SO_4 -> 5Fe_2(SO_4)_3 + 2MnSO_4 + K_2SO_4 + 8H_2O}
    \end{equation*}
    A questo punto calcoliamo le moli di reagenti:
    \begin{equation*}
        n_{\rm KMnO_4}=\frac{2.3456}{158.034}
        =1.4842 \cdot 10^{-2} \; \text{mol}
        \quad;\quad
        n_{\rm FeSO_4}=\frac{12.3456}{151.908}
        =8.1270 \cdot 10^{-2} \; \rm mol
    \end{equation*}
    Il rapporto stechiometrico tra $\rm KMnO_4$ e $\rm FeSO_4$ è $2:10$, cioè $1:5$, per cui affinché avvenga i reagenti reagiscano totalmente le moli di solfato devo essere pari a 5 volte quelle del permanganato, cioè dovrebbero essere $1.4842 \cdot 10^{-2} \cdot 5=7.421 \cdot 10^{-2}$ moli. Ne abbiamo addirittura di più, per cui il permanganato reagisce totalmente e resterà una parte di solfato indissociata.\\
    A questo punto guardiamo il rapporto stechiometrico tra il solfato ferroso è il solfato ferrico: esso è $10:5$, cioè $2:1$, per cui verranno prodotte un numero di moli pari alla metà:
    \begin{equation*}
        n_{\rm Fe_3(SO_4)_3}=\frac{7.421 \cdot 10^{-2}}{2}=3.7105 \cdot 10^{-2} \; \rm mol
    \end{equation*}
    e i grammi saranno pari a
    \begin{equation*}
        g=n \cdot MM=3.7105 \cdot 10^{-2} \cdot 399.88=14.8375 \; \rm grammi
    \end{equation*}
\end{soluzione}

\newpage

\begin{esercizio}
    Calcolare quanti litri di cromato di potassio 0.14 M sono necessari per ossidare 20.000 grammi di solfato ferroso:
    \begin{gather*}
        \ce{FeSO4(aq) + K2CrO4(aq) + H2SO4(aq) ->}
        \\
        \ce{-> Fe2(SO4)3(aq) + Cr2(SO4)3(aq) + K2SO4(aq) + H2O(l)}
    \end{gather*}
\end{esercizio}
\begin{soluzione}
    Scriviamo le dissociazioni delle varie specie:
    \begin{equation*}
        \begin{split}
            \ce{FeSO4} & \ce{-> Fe^{2+} + SO_4^{2-}}
            \\
            \ce{K2CrO4} & \ce{-> 2K^+ + CrO_4^{2-}}
            \\
            \ce{H2SO4} & \ce{-> 2H^+ + SO_4^{2-}}
            \\
            \ce{Fe2(SO4)3} & \ce{-> 2Fe^{3+} + 3SO_4^{2-}}
            \\
            \ce{Cr2(SO4)3} & \ce{-> 2Cr^{3+} + 3SO_4^{2-}}
            \\
            \ce{K2SO4} & \ce{-> 2K^+ + SO4^{2-}}
        \end{split}
    \end{equation*}
    Il ferro passa da n.o. $+2$ nel solfato ferroso a n.o. $+3$ nel solfato ferrico, quindi si è ossidato; il cromo passa da n.o. $+6$ nel cromato di potassio a n.o. $+3$ nel solfato di cromo (III), quindi si è ridotto. La reazione semplificata sarà
    \begin{equation*}
        \ce{2Fe^{2+} + 2CrO_4^{2-} + H^+ -> 2Fe^{3+} + 2Cr^{3+} + H_2O}
    \end{equation*}
    dove abbiamo moltiplicato per 2 le specie a sinistra perché a destra le specie sono nella forma $\rm Fe_2$ e $\rm Cr_2$.\\
    Le semireazioni redox che avvengono sono
    \begin{equation*}
        \begin{cases}
            \ce{Fe^{2+} -> Fe^{3+} + e^-}\\
            \ce{Cr^{6+} + 3e^- -> Cr^{3+}}
        \end{cases}
    \end{equation*}
    Visto che abbiamo due atomi di ferro e due di cromo dovremmo moltiplicare per due le specie del cromo e per 6 quelle del ferro, ma possiamo semplicemente moltiplicare per 1 le specie del cromo e per 3 quelle del ferro:
    \begin{equation*}
        \ce{6Fe^{2+} + 2CrO_4^{2-} + H^+ -> 6Fe^{3+} + 2Cr^{3+} + H_2O}
    \end{equation*}
    A questo punto bilanciamo le cariche: a sinistra abbiamo 12 cariche positive dal ferro e 4 cariche negative dal cromato, per cui in totale abbiamo 8 cariche positive; a destra abbiamo 18 cariche positive dal ferro e 6 cariche positive dal cromo, per un totale di 28 cariche positive. La differenza è di 16 cariche, per cui ci serviranno 16 ioni $\rm H^+$ e quindi 8 molecole d'acqua
    \begin{equation*}
        \ce{6Fe^{2+} + 2CrO_4^{2-} + 16H^+ -> 6Fe^{3+} + 2Cr^{3+} + 8H_2O}
    \end{equation*}
    La reazione completa sarà
    \begin{gather*}
        \ce{6FeSO4(aq) + 2K2CrO4(aq) + 8H2SO4(aq) ->}
        \\
        \ce{-> 3Fe2(SO4)3(aq) + Cr2(SO4)3(aq) + 2K2SO4(aq) + 8H2O(l)}
    \end{gather*}
    dove abbiamo bilanciato gli atomi di potassio in un secondo momento.\\
    Calcoliamo ora i litri di cromato di potassio. Per prima cosa calcoliamo le moli di solfato ferroso:
    \begin{equation*}
        n_{\rm FeSO_4}=\frac{20}{151.908}
        =1.3166 \cdot 10^{-1} \; \rm mol
    \end{equation*}
    Il rapporto stechiometrico tra solfato ferroso $\rm FeSO_4$ e cromato di potassio $\rm K_2CrO_4$ è $6:2$ cioè $3:1$, per cui le moli di cromato di potassio necessarie saranno pari a un terzo di quelle del solfato ferroso:
    \begin{equation*}
        3:1=n_{\rm FeSO_4}:n_{\rm K_2CrO_4}
        \implies
        n_{\rm K_2CrO_4}
        =\frac{n_{\rm FeSO_4}}{3}
        =\frac{1.3166 \cdot 10^{-1}}{3}
        =4.3887 \cdot 10^{-2} \; \rm mol
    \end{equation*}
    Il volume sarà dato da
    \begin{equation*}
        C=\frac{n}{V}
        \implies
        V=\frac{n}{C}
        =\frac{4.3887 \cdot 10^{-2}}{0.14}=0.3135 \; \rm L
    \end{equation*}
\end{soluzione}

\newpage

\begin{esercizio}
    Bilanciare la seguente reazione e calcolare i grammi di $\rm CO_2$ $(MM=44.01)$ liberati da 2.0000 grammi di $\rm KMnO_4$ $(MM=158.04)$:
    \begin{equation*}
        \ce{KMnO4 + K2C2O4 + H2SO4 -> MnSO4 + CO2 + K2SO4 + H2O}
    \end{equation*}
\end{esercizio}
\begin{soluzione}
    Scriviamo le dissociazioni delle varie specie:
    \begin{equation*}
        \begin{split}
            \ce{KMnO4} & \ce{-> K^+ + MnO4^-}
            \\
            \ce{K2C2O4} & \ce{-> 2K^+ + C2O_4^{2-}}
            \\
            \ce{H2SO4} & \ce{-> 2H^+ + SO_4^{2-}}
            \\
            \ce{MnSO4} & \ce{-> Mn^{2+} + SO_4^{2-}}
            \\
            \ce{CO2} & \ce{-> CO2}\;\text{(non si dissocia)}
            \\
            \ce{K2SO4} & \ce{-> 2K^+ + SO_4^{2-}}
        \end{split}
    \end{equation*}
    Il manganese passo da n.o. $+7$ nello ione permanganato a n.o. $+2$ nel solfato di manganese, quindi si è ridotto; il carbonio passa da n.o. $+3$ nello ione ossalato $\rm C_2O_4^{2-}$ a n.o $+4$ nell'anidride carbonica, quindi si è ossidato.\\
    L'equazione semplificata sarà
    \begin{equation*}
        \ce{MnO4^- + C2O_4^{2-} + H^+ -> Mn^{2+} + CO2 + H2O}
    \end{equation*}
    Le semireazioni redox che avvengono sono
    \begin{equation*}
        \begin{cases}
            \ce{C^{3+} -> C^{4+} + e^-}
            \\
            \ce{Mn^{7+} + 5e^- -> Mn^{2+}}
        \end{cases}
    \end{equation*}
    Visto che abbiamo due atomi di carbonio moltiplicheremo per due le specie del manganese e per 5 quelle del carbonio. Inoltre poiché a sinistra la specie è $\rm C_2O_4$ e a destra $\rm CO_2$ moltiplicheremo quest'ultima per 10:
    \begin{equation*}
        \ce{2MnO4^- + 5C2O_4^{2-} + H^+ -> 2Mn^{2+} + 10CO2 + H2O}
    \end{equation*}
    A questo punto bilanciamo le cariche. A sinistra abbiamo 2 cariche negative dal permanganato più 10 cariche negative dall'ossalato, per un totale di 12 cariche negative; a destra abbiamo solo 4 cariche positive dal manganese, quindi per bilanciare abbiamo bisogno di 16 atomi di $\rm H^+$ e quindi 8 molecole d'acqua:
    \begin{equation*}
        \ce{2MnO4^- + 5C2O_4^{2-} + 16H^+ -> 2Mn^{2+} + 10CO_2 + 8H2O}
    \end{equation*}
    Tornando alla reazione completa avremo
    \begin{equation*}
        \ce{2KMnO4 + 5K2C2O4 + 8H2SO4 -> 2MnSO4 + 10CO2 + 6K2SO4 + 8H2O}
    \end{equation*}
    dove il $\rm K_2SO_4$ è stato bilanciato in un secondo momento.\\
    Calcoliamo adesso i grammi di $\rm CO_2$. Il rapporto stechiometrico tra questa specie il $\rm KMnO_4$ è di $2:10$ cioè $1:5$, quindi verranno prodotte un numero di moli di anidride carbonica pari a 5 volte quelle di permanganato di potassio:
    \begin{equation*}
        n_{\rm CO_2}
        =5 \cdot n_{\rm KMnO_4}
        =5 \cdot \frac{2.0000}{158.04}
        =6.3275 \cdot 10^{-2} \; \rm mol
    \end{equation*}
    e i grammi di $\rm CO_2$ saranno dati da
    \begin{equation*}
        g_{\rm CO_2}
        =n \cdot MM=6.3275 \cdot 10^{-2} \cdot 44.01
        =0.2785 \; \rm grammi
    \end{equation*}
\end{soluzione}

\newpage

\begin{esercizio}
    Bilanciare la seguente reazione e calcolare i litri di SO2 a c.n. necessari per la riduzione di 1.5 grammi di bicromato di potassio.
    \begin{gather*}
        \text{Bicromato di potassio + anidride solforosa + acido cloridrico} \ce{->}
        \\
        \ce{->} \text{Solfato di cromo (III) + Cloruro di potassio + Acqua}
    \end{gather*}
\end{esercizio}
\begin{soluzione}
    La reazione che avviene, in formule, è
    \begin{equation*}
        \ce{K_2Cr_2O_7 + SO_2 + HCl -> Cr_2(SO_4)_3 + KCl + H_2O}
    \end{equation*}
    Scriviamo le dissociazioni delle molecole
    \begin{equation*}
        \begin{split}
            \ce{K_2Cr_2O_7} & \ce{-> 2K^+ + Cr_2O_7^{2-}}
            \\
            \ce{SO_2} & \ce{-> SO_2^{(0)}}
            \\
            \ce{HCl} & \ce{-> H^+ + Cl^-}
            \\
            \ce{Cr_2(SO_4)_3} & \ce{-> 2Cr^{3+} + 3SO_4^{2-}}
            \\
            \ce{KCl} & \ce{-> K^+ + Cl^-}
        \end{split}
    \end{equation*}
    Il cromo passa da n.o. $+6$ nel bicromato a n.o. $+3$ nel solfato di cromo, quindi si è ridotto; lo zolfo passa da n.o. $+4$ nell'anidride a n.o. $+6$ nel solfato, quindi si è ossidato.\\
    L'equazione semplificata sarà
    \begin{equation*}
        \ce{Cr_2O_7^{2-} + 3SO_2 + H^+ -> 2Cr^{3+} + 3SO_4^{2-} + H_2O}
    \end{equation*}
    Le semireazioni redox che avvengono sono
    \begin{equation*}
        \begin{cases}
            \ce{Cr^{6+} +3e^- -> Cr^{3+}}\\
            \ce{S^{4+} -> S^{6+} +2e^-}
        \end{cases}
    \end{equation*}
    Visto che abbiamo due atomi di cromo e tre di zolfo, dovremmo moltiplicare per 6 le specie delle zolfo e del cromo. \E chiaro che in questo caso non ha senso moltiplicare, quindi passiamo direttamente al bilancio delle cariche: a destra abbiamo solo 2 cariche negative dal bicromato; a destra abbiamo 6 cariche positive dal cromo e 6 cariche negative dal solfato, per un totale di 0 cariche. Per bilanciare abbiamo bisogno di 2 ioni $\rm H^+$ e quindi di una molecola di $\rm H_2O$:
    \begin{equation*}
        \ce{Cr_2O_7^{2-} + 3SO_2 + 2H^+ -> 2Cr^{3+} + 3SO_4^{2-} + H_2O}
    \end{equation*}
    La reazione completa sarà
    \begin{equation*}
        \ce{K_2Cr_2O_7 + 3SO_2 + 2HCl -> Cr_2(SO_4)_3 + 2KCl + H_2O}
    \end{equation*}
    dove il potassio è stato bilanciato a posteriori.\\
    A questo punto calcoliamo i litri di $\rm SO2$ necessari per ridurre il bicromato di potassio. Calcoliamo le moli di quest'ultimo:
    \begin{equation*}
        n_{\rm K_2Cr_2O_7}
        =\frac{1.5}{294.19}
        =5.0987 \cdot 10^{-3} \; \rm mol
    \end{equation*}
    Il rapporto stechiometrico tra bicromato e anidride solforosa è $1:3$, per cui le moli di $\rm SO2$ saranno date dalla proporzione
    \begin{equation*}
        1:3=n_{\rm K_2Cr_2O_7}:n_{\rm SO_2}
        \implies
        n_{\rm SO_2}=\frac{3 \cdot 5.0987 \cdot 10^{-3}}{1}
        =1.5296 \cdot 10^{-2} \; \rm mol
    \end{equation*}
    Il volume sarà dato dall'equazione di stato dei gas:
    \begin{equation*}
        PV=nRT
        \implies
        V=\frac{nRT}{P}
        =\frac{1.5296 \cdot 10^{-2} \cdot 0.082 \cdot 273.15}{1}
        =0.3426 \; \rm L
    \end{equation*}
\end{soluzione}

\newpage

\begin{esercizio}
    Bilanciare la seguente reazione e calcolare i grammi di permanganato necessari per ossidare una mole di solfato ferroso in eccesso di acido solforico:
    \begin{gather*}
        \text{Permanganato di potassio + Solfato ferroso + Acido solforico} \ce{->}
        \\
        \ce{->} \text{Solfato di manganese(II) + Solfato ferrico + Solfato di potassio + Acqua}
    \end{gather*}
\end{esercizio}
\begin{soluzione}
    La reazione che avviene è
    \begin{equation*}
        \ce{KMnO_4 + FeSO_4 + H_2SO_4 -> MnSO_4 + Fe_2(SO_4)_3 + K_2SO_4 + H_2O}
    \end{equation*}
    Scriviamo le dissociazioni delle varie specie:
    \begin{equation*}
        \begin{split}
            \ce{KMnO_4} & \ce{-> K^+ + MnO_4^-}
            \\
            \ce{FeSO_4} & \ce{-> Fe^{2+} + SO_4^{2-}}
            \\
            \ce{H_2SO_4} & \ce{-> 2H^+ + SO_4^{2-}}
            \\
            \ce{MnSO_4} & \ce{-> Mn^{2+} + SO_4^{2-}}
            \\
            \ce{Fe_2(SO_4)_3} & \ce{-> 2Fe^{3+} + 3SO_4^{2-}}
            \\
            \ce{K_2SO_4} & \ce{-> 2K^+ + SO_4^{2-}}
            \\
        \end{split}
    \end{equation*}
    Il manganese passa da n.o. $+7$ nel permanganato a n.o. $+2$ nel solfato di manganese, quindi si è ridotto; il ferro passa da n.o. $+2$ nel solfato ferroso a n.o. $+3$ nel solfato ferrico, quindi si è ossidato.\\
    La reazione semplificata sarà
    \begin{equation*}
        \ce{MnO_4^- + 2Fe^{2+} + H^+ -> Mn^{2+} + 2Fe^{3+} + H_2O}
    \end{equation*}
    dove abbiamo moltiplicato per due le specie del ferro perché a destra la specie è del tipo $\rm Fe_2$.\\
    Le semireazioni che avvengono sono:
    \begin{equation*}
        \begin{cases}
            \ce{Mn^{7+} + 5e^- -> Mn^{2+}}\\
            \ce{Fe^{2+} -> Fe^{3+} + e^-}
        \end{cases}
    \end{equation*}
    Siccome abbiamo due atomi di ferro, moltiplichiamo per 2 le specie del manganese e per 5 quelle del ferro:
    \begin{equation*}
        \ce{2MnO_4^- + 10Fe^{2+} + H^+ -> 2Mn^{2+} + 10Fe^{3+} + H_2O}
    \end{equation*}
    Bilanciamo le cariche. A sinistra abbiamo 2 cariche negative dal permanganato e 20 cariche positive dal ferro, per un totale di 18 cariche positive; a destra abbiamo 4 cariche positive dal manganese e 30 cariche positive dal ferro, per un totale di 34 cariche positive. La differenza è di 16 cariche positive, per cui ci serviranno 16 atomi di idrogeno e quindi 8 molecole d'acqua:
    \begin{equation*}
        \ce{2MnO_4^- + 10Fe^{2+} + 16H^+ -> 2Mn^{2+} + 10Fe^{3+} + 8H_2O}
    \end{equation*}
    La reazione completa sarà
    \begin{equation*}
        \ce{2KMnO_4 + 10FeSO_4 + 8H_2SO_4 -> 5Fe_2(SO_4)_3 + 2MnSO_4 + K_2SO_4 + 8H_2O}
    \end{equation*}
    dove abbiamo bilanciato in un secondo momento gli atomi di potassio.\\
    A questo punto vediamo quante moli di permanganato di potassio servono per la completa ossidazione di una mole di solfato ferroso. Il rapporto stechiometrico tra questi è $2:10$ cioè $1:5$, per cui servirà $1/5$ di mole di permanganato. I grammi corrispondenti sono pari a
    \begin{equation*}
        g_{\rm KMnO_4}
        =n \cdot MM
        =\frac{1}{5} \cdot 158.034
        =31.6068 \; \rm grammi
    \end{equation*}
\end{soluzione}

\newpage

\begin{esercizio}\label{EXR:chiarimento_HNO3}
    Bilanciare la seguente reazione e calcolare il volume di NO misurato a c.n. che si sviluppa a partire da 1.3540 g di rame:
    \begin{equation*}
        \ce{Cu(s) + HNO3(aq) \ce{->} Cu(NO3)2(aq) + NO(g) + H2O(l)}
    \end{equation*}
\end{esercizio}
\begin{soluzione}
    Scriviamo la dissociazione delle varie specie chimiche:
    \begin{equation*}
        \begin{split}
            \ce{Cu} & \ce{-> Cu^0}
            \\
            \ce{HNO_3} & \ce{-> H^+ + NO_3^-}
            \\
            \ce{Cu(NO_3)_2} & \ce{-> Cu^{2+} + 2NO_3^-}
            \\
            \ce{NO} & \ce{-> NO^{(0)}}
        \end{split}
    \end{equation*}
    Il rame passa da n.o. $0$ nella forma atomica a n.o. $+2$ nel nitrato di rame, quindi si è ossidato; l'azoto passa da n.o. $+5$ nello ione nitrato a n.o. $+2$ nel monossido di azoto, quindi si è ridotto.\\
    L'equazione semplificata sarà
    \begin{equation*}
        \ce{Cu + NO_3^- + H^+ -> Cu^{2+} + NO + H_2O}
    \end{equation*}
    Le semireazioni che avvengono sono
    \begin{equation*}
        \begin{cases}
            \ce{Cu^0 -> Cu^{2+} + 2e^-}\\
            \ce{N^{5+} + 3e^- -> N^{2+}}
        \end{cases}
    \end{equation*}
    Moltiplichiamo quindi per 3 le specie del rame e per 2 quelle dell'azoto:
    \begin{equation*}
        \ce{3Cu + 2NO_3^- + H^+ -> 3Cu^{2+} + 2NO + H_2O}
    \end{equation*}
    Bilanciamo le cariche. A sinistra abbiamo due cariche negative dallo ione nitrato, mentre a destra abbiamo sei cariche positive dallo ione rameoso. Per bilanciare ci servono 8 ioni $\rm H^+$ e quindi quattro molecole d'acqua:
    \begin{equation*}
        \ce{3Cu + 2NO_3^- + 8H^+ -> 3Cu^{2+} + 2NO + 4H_2O}
    \end{equation*}
    La reazione completa sarà quindi
    \begin{equation*}
        \ce{3Cu + 8HNO_3 -> 3Cu(NO_3)_2(aq) + 2NO(g) + 4H_2O(l)}
    \end{equation*}
    \textbf{Chiarimento tratto dal Giomini}\\
    \E bene notare la discrepanza tra i valori dei coefficienti di $\rm H^+$ (8) e $\rm NO_3^-$ (2) che si traduce nel coefficiente 8 per $\rm HNO_3$. Il motivo è che nello schema di reazione si tiene conto solo degli elementi che variano il loro n.o.\,: 2 ioni $\rm NO_3^-$ diventano due molecole di NO; non si può tenere conto degli ioni $\rm NO_3^-$ che rimangono tali perché devono formare il sale $\rm Cu(NO_3)_2$ (3 molecole quindi 6 ioni). Siccome $\rm HNO_3$ fornisce sia i 2 ioni che si trasformano in NO che i 6 ioni che vanno a formare il sale, il bilanciamento assume l'aspetto sopra scritto.\\
    (Definizione pratica: quando lo ione $\rm H^+$ proviene da una molecola in cui una specie si sta ossidando/riducendo, quando torniamo alla reazione completa la specie chimica prenderà il coefficiente dell'$\rm H^+$. In particolare il professore ha detto che in generale si considera il coefficiente più grande tra i due, che di solito è proprio quello dell'idrogeno).\\[0.2cm]
    A questo punto calcoliamo il volume di NO che si sviluppa. Calcoliamo innanzitutto le moli di rame:
    \begin{equation*}
        n_{\rm Cu}
        =\frac{1.3540}{63.546}
        =2.1307 \cdot 10^{-2} \; \rm mol
    \end{equation*}
    Il rapporto stechiometrico tra rame e monossido di azoto è $3:2$, per cui le moli di quest'ultimo saranno date dalla proporzione
    \begin{equation*}
        3:2=n_{\rm Cu}:n_{\rm NO}
        \implies
        n_{\rm NO}
        =\frac{2 \cdot 2.1307 \cdot 10^{-2}}{3}
        =1.4205 \cdot 10^{-2} \; \rm mol
    \end{equation*}
    Il volume di NO sarà dato, tramite l'equazione di stato dei gas, da
    \begin{equation*}
        PV=nRT
        \implies
        V=\frac{nRT}{P}
        =\frac{1.4205 \cdot 10^{-2} \cdot 0.082 \cdot 273.15}{1}
        =0.3182 \; \rm L
    \end{equation*}
\end{soluzione}

\newpage

\begin{esercizio}
    Bilanciare la seguente reazione e calcolare i grammi di cloruro di zinco che si ottengono a partire da 3.1234 grammi di anidride arseniosa.
    \begin{equation*}
        \ce{As2O3 + Zn + HCl -> AsH3 + ZnCl2 + H2O}
    \end{equation*}
\end{esercizio}
\begin{soluzione}
    Scriviamo le dissociazioni delle varie specie chimiche
    \begin{equation*}
        \begin{split}
            \ce{As2O3} & \ce{-> 2As^{3+} + 3O^{2-}}
            \\
            \ce{Zn} & \ce{-> Zn^0}
            \\
            \ce{HCl} & \ce{-> H^+ + Cl^-}
            \\
            \ce{AsH3} & \ce{-> As^{3+} + 3H^-}
            \\
            \ce{ZnCl2} & \ce{-> Zn^{2+} + 2Cl^-}
        \end{split}
    \end{equation*}
    Lo zinco passa da n.o. $0$ a n.o. $+2$ nel cloruro, quindi si ossida; L'idrogeno passa da n.o. $+1$ nell'acido cloridrico a n.o. $-1$ nell'idruro arsenioso (o arsina) $\rm AsH_3$, quindi si è ridotto.\\
    L'equazione semplificata sarà
    \begin{equation*}
        \ce{Zn + 3H^+ -> Zn^{2+} + 3H^- + H_2O }
    \end{equation*}
    Le semireazioni redox che avvengono sono
    \begin{equation*}
        \begin{cases}
            \ce{Zn^{0} -> Zn^{2+} + 2e^-}\\
            \ce{H^+ + 2e^- -> H^-}
        \end{cases}
    \end{equation*}
    Visto che abbiamo tre atomi di idrogeno, dobbiamo moltiplicare per 6 le specie dello zinco, mentre dovremo moltiplicare per 2 quello dell'idrogeno:
    \begin{equation*}
        \ce{6Zn + 6H^+ -> 6Zn^{2+} + 6H^- + H_2O}
    \end{equation*}
    Avremmo potuto semplicemente moltiplicare per 3 quelle dello zinco, tuttavia in questo caso nel bilanciamento delle masse (le cariche sono già bilanciate: a sinistra abbiamo 6 cariche positive dall'$\rm H^+$ mentre a destra abbiamo 12 cariche positive dallo zinco e 6 cariche negative dall'idrogeno, per un totale di 6 cariche positive) avremmo ottenuto dei coefficienti non interi.\\
    La reazione completa sarà quindi
    \begin{equation*}
        \ce{As2O3 + 6Zn + 6HCl -> 2AsH3 + 6ZnCl2 + 3H2O}
    \end{equation*}
    A questo punto dobbiamo bilanciare le masse. Notiamo infatti che a destra abbiamo 12 atomi di cloro nella specie $\rm ZnCl_2$, per cui a sinistra il coefficiente dell'HCl diventa 12, in quanto è l'unica specie ad avere atomi di cloro. Così facendo bilanciamo anche gli atomi di idrogeno, perché a destra abbiamo 6 atomi di idrogeno nell'$\rm AsH_3$ e altri 6 atomi nell'acqua. La reazione finale dunque sarà
    \begin{equation*}
        \ce{As2O3 + 6Zn + 12HCl -> 2AsH3 + 6ZnCl2 + 3H2O}
    \end{equation*}
    A questo punto calcoliamo i grammi di cloruro di zinco prodotti a partire dall'anidride arseniosa. Calcoliamo le moli di quest'ultima specie:
    \begin{equation*}
        n_{\rm As_2O_3}
        =\frac{3.1234}{197.841}
        =1.5787 \cdot 10^{-2} \; \rm mol
    \end{equation*}
    Il rapporto stechiometrico tra l'anidride arseniosa e il cloruro di zinco è $1:6$, per cui il numero di moli sarà pari a 6 volte quello dell'$\rm As_2O_3$:
    \begin{equation*}
        1:6=n_{\rm As_2O_3}:n_{\rm ZnCl_2}
        \implies
        n_{\rm ZnCl_2}
        =6 \cdot 1.5787 \cdot 10^{-2}
        =9.4722 \cdot 10^{-2} \; \rm mol
    \end{equation*}
    I grammi saranno allora pari a
    \begin{equation*}
        g
        =n \cdot MM
        =9.4722 \cdot 10^{-2} \cdot 136.286
        =1.291 \; \rm grammi
    \end{equation*}
\end{soluzione}

\newpage

\begin{esercizio}
    Bilanciare la reazione e calcolare il volume di cloro gassoso necessario per ottenere 10 litri di ipoclorito 2.0 M:
    \begin{gather*}
        \text{Cloro(g) + Idrossido di sodio(aq)} \ce{->}
        \\
        \ce{->} \text{Ipoclorito di sodio(aq) + Cloruro di sodio(aq) + Acqua}
    \end{gather*}
\end{esercizio}
\begin{soluzione}
    La reazione che avviene è
    \begin{equation*}
        \ce{Cl2(g) + NaOH(aq) -> NaClO(aq) + NaCl + H2O}
    \end{equation*}
    Scriviamo le dissociazioni delle varie specie chimiche
    \begin{equation*}
        \begin{split}
            \ce{Cl2} & \ce{-> 2Cl^0}
            \\
            \ce{NaOH} & \ce{-> Na^+ + OH^-}
            \\
            \ce{NaClO} & \ce{-> Na^+ + ClO^-}
            \\
            \ce{NaCl} & \ce{-> Na^+ + Cl^-}
        \end{split}
    \end{equation*}
    Notiamo che in questa reazione il cloro ha n.o. $0$ tra i reagenti, mentre tra i prodotti ha sia n.o. $+1$ nell'ipoclorito che n.o. $-1$ nel cloruro, pertanto siamo davanti ad una reazione di disproporzione.\\
    La reazione semplificata sarà
    \begin{equation*}
        \ce{Cl_2 + OH^- -> Cl^- + ClO^- + H_2O}
    \end{equation*}
    Le semireazioni redox che avvengono sono
    \begin{equation*}
        \begin{cases}
            \ce{Cl^{0} + 1e^- -> Cl^+}\\
            \ce{Cl^{0} -> Cl^+ + 1e^-}
        \end{cases}
    \end{equation*}
    \E chiaro che gli elettroni sono già bilanciati. Bilanciamo le cariche.\\
    Al secondo membro abbiamo due cariche negative mentre al primo nessuna, quindi servono due gruppi OH:
    \begin{equation*}
        \ce{Cl_2 + 2OH^- -> Cl^- + ClO^- + H_2O}
    \end{equation*}
    Tornando alla reazione completa avremo
    \begin{equation*}
        \ce{Cl_2 + 2NaOH -> NaCl + NaClO + H_2O}
    \end{equation*}
    Il rapporto stechiometrico tra cloro e ipoclorito è $1:1$, quindi il numero di moli è lo stesso. Per calcolarle basterà moltiplicare la concentrazione per il volume
    \begin{equation*}
        n_{\rm Cl_2}=n_{\rm NaClO}=C \cdot V=2.0 \cdot 10=20 \; \rm mol
    \end{equation*}
    I grammi saranno allora dati da
    \begin{equation*}
        n=\frac{g}{MM}
        \implies
        g=n \cdot MM=20 \cdot 70.906=141.812 \; \rm grammi
    \end{equation*}
\end{soluzione}

\newpage

\begin{esercizio}
    Bilanciare la seguente reazione:

\begin{center}
    Permanganato di sodio + Cloruro stannoso + acido cloridrico \ce{->}

    \ce{->} Cloruro di sodio + Cloruro di Manganese (II) + Cloruro stannico + Acqua
\end{center}

e calcolare i grammi del sale di stagno che si formano a partire da 0.5932 grammi di permanganato di sodio.
\end{esercizio}
\begin{soluzione}
    La reazione che avviene è
    \begin{equation*}
        \ce{NaMnO4 + SnCl2 + HCl -> NaCl + MnCl2 + SnCl4 + H2O}
    \end{equation*}
    Scriviamo la dissociazione delle varie specie:
    \begin{equation*}
        \begin{split}
            \ce{NaMnO4} & \ce{-> Na^+ + MnO_4^-}
            \\
            \ce{SnCl2} & \ce{-> Sn^{2+} + 2Cl^-}
            \\
            \ce{HCl} & \ce{-> H^+ + Cl^-}
            \\
            \ce{NaCl} & \ce{-> Na^+ + Cl^-}
            \\
            \ce{MnCl2} & \ce{-> Mn^{2+} + 2Cl^-}
            \\
            \ce{SnCl4} & \ce{-> Sn^{4+} + 4Cl^-}
        \end{split}
    \end{equation*}
    Il manganese passo da n.o. $+7$ nello ione permanganato a n.o. $+2$ nel cloruro di manganese, quindi si riduce; lo stagno passa da n.o. $+2$ nel cloruro stannoso a n.o. $+4$ nel cloruro stannico, quindi si è ossidato.\\
    L'equazione semplificata sarà
    \begin{equation*}
        \ce{MnO_4^- + Sn^{2+} + H^+ -> Mn^{2+} + Sn^{4+} + H_2O}
    \end{equation*}
    Le semireazioni che avvengono sono
    \begin{equation*}
        \begin{cases}
            \ce{Mn^{7+} + 5e^- -> Mn^{2+}}\\
            \ce{Sn^{2+} -> Sn^{4+} + 2e^-}
        \end{cases}
    \end{equation*}
    Moltiplichiamo quindi per 2 le specie del manganese e per 5 quello dello stagno:
    \begin{equation*}
        \ce{2MnO_4^- + 5Sn^{2+} + H^+ -> 2Mn^{2+} + 5Sn^{4+} + H_2O}
    \end{equation*}
    A questo punto bilanciamo le cariche. A sinistra abbiamo due cariche negative dal permanganato e 10 cariche positive dallo stagno, per un totale di 8 cariche positive; a destra abbiamo 4 cariche positive dal manganese e 20 dallo stagno, per un totale di 24 cariche positive. La differenza è di 16 cariche, per cui ci serviranno 16 ioni $\rm H^+$ e quindi 8 molecole d'acqua:
    \begin{equation*}
        \ce{2MnO_4^- + 5Sn^{2+} + 16H^+ -> 2Mn^{2+} + 5Sn^{4+} + 8H_2O}
    \end{equation*}
    Tornando alla reazione completa avremo
    \begin{equation*}
        \ce{2NaMnO4 + 5SnCl2 + 16HCl -> 2NaCl + 2MnCl2 + 5SnCl4 + 8H2O}
    \end{equation*}
    dove abbiamo bilanciato in un secondo momento gli atomi di sodio.\\
    A questo punto calcoliamo i grammi del cloruro stannico a partire dal permanganato di sodio. Calcoliamo innanzitutto le moli di quest'ultimo:
    \begin{equation*}
        n_{\rm NaMnO_4}
        =\frac{0.5932}{141.9254}
        =4.1797 \cdot 10^{-3} \; \rm mol
    \end{equation*}
    Il rapporto stechiometrico tra $\rm NaMnO_4$ e $\rm SnCl_4$ è $2:5$, per cui
    \begin{equation*}
        2:5=n_{\rm NaMnO_4}:n_{\rm SnCl_4}
        \implies
        n_{\rm SnCl_4}
        =\frac{5 \cdot 4.1797 \cdot 10^{-3}}{2}
        =10.44925 \cdot 10^{-3} \; \rm mol
    \end{equation*}
    da cui
    \begin{equation*}
        g_{\rm SnCl_4}
        =n \cdot MM
        =10.44925 \cdot 10^{-3} \cdot 260.5
        =2.7220 \; \rm grammi
    \end{equation*}
\end{soluzione}

\newpage

\begin{esercizio}
    Bilanciare la seguente reazione e calcolare il volume in millilitri di una soluzione acquosa di acido solforico 0.1 M necessari per ridurre 1.5328 grammi di permanganato di potassio.
    \begin{gather*}
        \text{Permanganato di potassio + Solfito di sodio + Acido solforico} \ce{->}
        \\
        \ce{->} \text{Solfato di manganese (II) + Solfato di sodio + Solfato di potassio + Acqua}
    \end{gather*}
\end{esercizio}
\begin{soluzione}
    La reazione che avviene è
    \begin{equation*}
        \ce{KMnO_4 + Na_2SO_3 + H_2SO_4 -> MnSO_4 + Na_2SO_4 + K_2SO_4 + H_2O}
    \end{equation*}
    Scriviamo le dissociazioni delle varie specie:
    \begin{equation*}
        \begin{split}
            \ce{KMnO_4} & \ce{-> K^+ + MnO_4^-}
            \\
            \ce{Na_2SO_3} & \ce{-> 2Na^+ + SO_3^{2-}}
            \\
            \ce{H_2SO_4} & \ce{-> 2H^+ + SO_4^{2-}}
            \\
            \ce{MnSO_4} & \ce{-> Mn^{2+} + SO_4^{2-}}
            \\
            \ce{Na_2SO_4} & \ce{-> 2Na^+ + SO_4^{2-}}
            \\
            \ce{K_2SO_4} & \ce{-> 2K^+ + SO_4^{2-}}
            \\
        \end{split}
    \end{equation*}
    Il manganese passa n.o. $+7$ nel permanganato a n.o. $+2$ nel solfato, pertanto si è ridotto; lo zolfo passa da n.o. $+4$ nel solfito di sodio a n.o. $+6$ nel solfato di sodio, quindi si è ossidato.\\
    L'equazione semplificata sarà
    \begin{equation*}
        \ce{MnO_4^- + SO_3^{2-} + H^+ -> Mn^{2+} + SO_4^{2-} + H_2O}
    \end{equation*}
    Le semireazioni redox che avvengono sono
    \begin{equation*}
        \begin{cases}
            \ce{Mn^{7+} + 5e^- -> Mn^{2+}}\\
            \ce{S^{4+} -> S^{6+} + 2e^-}
            \end{cases}
    \end{equation*}
    Moltiplichiamo quindi per 2 le specie del manganese e per 5 quelle dello zolfo:
    \begin{equation*}
        \ce{2MnO_4^- + 5SO_3^{2-} + H^+ -> 2Mn^{2+} + 5SO_4^{2-} + H_2O}
    \end{equation*}
    A questo punto bilanciamo le cariche. A sinistra abbiamo due cariche negative dal permanganato e 10 cariche negative dal solfito, per un totale di 12 cariche negative; a destra abbiamo 4 cariche positive dal manganese e 10 cariche negative dal solfato, per un totale di 6 cariche negative. La differenza è di 6 cariche, per cui ci serviranno 6 ioni $\rm H^+$ e quindi 3 molecole d'acqua:
    \begin{equation*}
        \ce{2MnO_4^- + 5SO_3^{2-} + 6H^+ -> 2Mn^{2+} + 5SO_4^{2-} + 3H_2O}
    \end{equation*}
    Tornando alla reazione completa avremo:
    \begin{equation*}
        \ce{2KMnO_4 + 5Na_2SO_3 + 3H_2SO_4 -> 2MnSO_4 + 5Na_2SO_4 + K_2SO_4 + 3H_2O}
    \end{equation*}
    A questo punto calcoliamo gli mL di acido solforico necessari ad ossidare il permanganato. Per prima cosa calcoliamo le moli di quest'ultimo:
    \begin{equation*}
        n_{\rm KMnO_4}=\frac{1.5328}{158.034}
        =9.6992 \cdot 10^{-3} \; \rm mol
    \end{equation*}
    Il rapporto stechiometrico tra permanganato di potassio e acido solforico è $2:3$, per cui le moli di acido saranno date dalla proporzione
    \begin{equation*}
        2:3=n_{\rm H_2SO_4}:n_{\rm KMnO_4}
        \implies
        n_{\rm H_2SO_4}=\frac{2 \cdot 9.6992 \cdot 10^{-3}}{3}
        =6.4661 \cdot 10^{-3} \; \rm mol
    \end{equation*}
    Il volume sarà dato da
    \begin{equation*}
        C=\frac{n}{V}
        \implies
        V=\frac{n}{C}=\frac{6.4661 \cdot 10^{-3}}{0.1} \cdot 1000
        =64.6612 \; \rm mL
    \end{equation*}
    dove abbiamo moltiplicato per 1000 perché il volume è richiesto in mL.
\end{soluzione}

\newpage

\begin{esercizio}
    Lo zinco in acido cloridrico produce cloruro di zinco ed idrogeno. Calcolare il volume di idrogeno misurato a c.n. ed i grammi di cloruro di zinco che si ottengono per reazione di 2.42 g di zinco con 100 mL di una soluzione acquosa di HCl $0.5$ M.
\end{esercizio}
\begin{soluzione}
    La reazione che avviene è
    \begin{equation*}
        \ce{Zn + HCl -> ZnCl_2 + H_2 ^}
    \end{equation*}
    Scriviamo le dissociazioni delle varie specie:
    \begin{equation*}
        \begin{split}
            \ce{Zn} & \ce{-> Zn^{(0)}}
            \\
            \ce{HCl} & \ce{-> H^+ + Cl^-}
            \\
            \ce{ZnCl_2} & \ce{-> Zn^{2+} + 2Cl^-}
            \\
            \ce{H_2} & \ce{-> H_2^{(0)}}
        \end{split}
    \end{equation*}
    Lo zinco passa da n.o. $0$ a n.o. $+2$, quindi si è ossidato; l'idrogeno passa da n.o. $+1$ a n.o. $0$, quindi si è ridotto. La reazione semplificata sarà
    \begin{equation*}
        \ce{Zn + 2H^+ -> Zn^{2+} + H_2}
    \end{equation*}
    dove abbiamo messo 2 davanti ad $\rm H^+$ perché a destra la specie è $\rm H_2$.\\
    Le semireazioni che avvengono sono
    \begin{equation*}
        \begin{cases}
            \ce{Zn^{(0)} -> Zn^{2+} + 2e^-}\\
            \ce{2H^+ + 2e^- -> H_2^{(0)}}
        \end{cases}
    \end{equation*}
    Siccome dovremmo moltiplicare tutte le specie per 2 non moltiplichiamo nessuna specie. Notiamo che le cariche sono già bilanciate.\\
    La reazione bilanciata sarà
    \begin{equation*}
        \ce{Zn + 2HCl -> ZnCl_2 + H_2}
    \end{equation*}
    Vediamo adesso le moli dei reagenti:
    \begin{equation*}
        n_{\rm Zn}
        =\frac{g}{MM_{\rm Zn}}
        =\frac{2.42}{65.38}
        =3.7014 \cdot 10^{-2} \; \rm mol
    \end{equation*}
    Per le moli di acido cloridrico facciamo la proporzione
    \begin{equation*}
        n_{\rm HCl}:100=0.5:1000
        \implies
        n_{\rm HCl}=\frac{100 \cdot 0.5}{1000}
        =5 \cdot 10^{-2} \; \rm mol
    \end{equation*}
    Il rapporto stechiometrico tra zinco e acido cloridrico è $1:2$, per cui per la totale reazione dei reagenti le moli di acido devono essere pari al doppio di quelle di zinco, cioè dovrebbero essere $7.4028 \cdot 10^{-2}$. Noi però ne abbiamo solo $5 \cdot 10^{-2}$, ossia sono in difetto e quindi sarà l'HCl il reagente limitante, per cui dovremo fare riferimento ad esso.\\
    Il rapporto stechiometrico tra acido cloridrico e cloruro di zinco è $2:1$, quindi otterremo $\frac{5\cdot 10^{-2}}{2}=2.5 \cdot 10^{-2}$ moli di $\rm ZnCl_2$. I grammi saranno dati da
    \begin{equation*}
        g=n_{\rm ZnCl_2} \cdot MM_{\rm ZnCl_2}
        =2.5 \cdot 10^{-2} \cdot 136.286
        =3.40715 \; \rm grammi
    \end{equation*}
    Per quanto riguarda l'idrogeno, il rapporto stechiometrico tra l'acido ed esso è $2:1$, per cui anche di questo ne verranno prodotte $2.5 \cdot 10^{-2}$ moli. Per ottenere il volume dobbiamo usare l'equazione di stato dei gas:
    \begin{equation*}
        PV=nRT \implies V=\frac{nRT}{P}
    \end{equation*}
    Dunque:
    \begin{equation*}
        \implies V_{\rm H_2}
        =\frac{2.5 \cdot 10^{-2} \cdot 0.082 \cdot 273.15}{1}
        =0.5599 \; \rm L
    \end{equation*}
\end{soluzione}

\newpage

\begin{esercizio}
    Bilanciare la seguente reazione e calcolare i grammi di permanganato di potassio necessari per ossidare 2.2414 L di acido solforico prelevati a c.n.\,:
    $$\ce{KMnO_4 + H_2S + H_2SO_4 -> MnSO_4 + S + K_2SO_4 + H_2O}$$
\end{esercizio}
\begin{soluzione}
    Scriviamo le dissociazioni delle varie specie:
    \begin{equation*}
        \begin{split}
            \ce{KMnO_4} & \ce{-> K^+ + MnO_4^-}
            \\
            \ce{H_2S} & \ce{-> 2H^+ + S^{2-}}
            \\
            \ce{H_2SO_4} & \ce{-> 2H^+ + SO_4^{2-}}
            \\
            \ce{MnSO_4} & \ce{-> Mn^{2+}}
            \\
            \ce{S} & \ce{-> S^{(0)}}
            \\
            \ce{K_2SO_4} & \ce{-> 2K^+ + SO_4^{2-}}
            \\
        \end{split}
    \end{equation*}
    Il manganese passa n.o. $+7$ nel permanganato a n.o. $+2$ nel solfato e dunque si è ridotto; lo zolfo passa da n.o. $-2$ nell'acido solfidrico a n.o. $0$ nello zolfo puro, quindi si è ossidato.\\
    L'equazione semplificata sarà
    \begin{equation*}
        \ce{MnO_4^- + S^{2-} + H^+ -> Mn^{2+} + S + H_2O}
    \end{equation*}
    Le semireazioni redox che avvengono sono
    \begin{equation*}
        \begin{cases}
            \ce{Mn^{7+} + 5e^- -> Mn^{2+}}\\
            \ce{S^{2-} -> S^{(0)} + 2e^-}
        \end{cases}
    \end{equation*}
    Moltiplichiamo quindi per 5 le specie dello zolfo e per 2 quelle del manganese:
    \begin{equation*}
        \ce{2MnO_4^- + 5S^{2-} + H^+ -> 2Mn^{2+} + 5S + H_2O}
    \end{equation*}
    A questo punto bilanciamo le cariche. A sinistra abbiamo 2 cariche negative dal permanganato e 10 cariche negative dallo zolfo, per un totale di 12 cariche negative; a destra abbiamo solo 4 cariche positive dal manganese. La differenza è di 16 cariche, per cui ci serviranno 16 ioni $\rm H^+$ e quindi 8 molecole d'acqua:
    \begin{equation*}
        \ce{2MnO_4^- + 5S^{2-} + 16H^+ -> 2Mn^{2+} + 5S + 8H_2O}
    \end{equation*}
    A questo punto torniamo alla reazione completa. Attenzione però: tra i reagenti c'è la specie $\rm H_2S$, la quale contiene idrogeno. Ne segue che dei 16 ioni $\rm H^+$ che sono necessari per bilanciare le cariche non tutti nella reazione completa andranno nell'acido solforico. In particolare, siccome il coefficiente dello ione solfuro $\rm S^{2-}$ è 5 e per ogni molecola servono due atomi di idrogeno, 10 ioni $\rm H^+$ andranno in questa specie e solo i restanti 6 andranno nell'acido solforico. Quindi si avrà
    \begin{equation*}
        \ce{2KMnO_4 + 5H_2S + 3H_2SO_4 -> 2MnSO_4 + 5S + K_2SO_4 + 8H_2O}
    \end{equation*}
    Come al solito notiamo che siccome nell'$\rm H_2S$ e nell'$\rm H_2SO_4$ la specie è $\rm H_2$ abbiamo dovuto dimezzare i coefficienti.\\
    A questo punto calcoliamo i grammi di permanganato necessari per ossidare 2.2414 L di acido solforico prelevati a c.n.\,. Innanzitutto calcoliamo le moli di acido solforico attraverso l'equazione di stato dei gas:
    \begin{equation*}
        PV=nRT
    \implies
    n=\frac{PV}{RT}
    =\frac{1 \cdot 2.2414}{0.082 \cdot 298.15}
    \approx 0.1 \; \rm mol
    \end{equation*}
    Notiamo come saremmo potuti arrivare allo stesso risultato attraverso attraverso la legge di avogadro: trovandoci in condizioni normali, 1 mole di sostanza occuperà 22.414 litri, per cui conoscendo il volume potevamo ottenere le moli tramite la proporzione
    \begin{equation*}
        1:22.414=n:2.2414
        \implies
        n=\frac{1 \cdot 2.2414}{22.414}=0.1 \; \rm mol
    \end{equation*}
    Il rapporto stechiometrico tra $\rm KMnO_4$ e $\rm H_2SO_4$ è $2:3$, per cui le moli di permanganato necessarie possono essere trovate tramite la proporzione
    \begin{equation*}
        2:3=n_{\rm KMnO_4}:n_{\rm H_2SO_4}
        \implies
        n_{\rm KMnO_4}=\frac{2 \cdot 0.1}{3}=6.6667 \cdot 10^{-2} \; \rm mol
    \end{equation*}
    e i grammi corrispondenti saranno
    \begin{equation*}
        g
        =n \cdot MM=6.6667 \cdot 10^{-2} \cdot 158.034
        =10.5356 \; \rm grammi
    \end{equation*}
\end{soluzione}

\newpage

\begin{esercizio}
    La reazione tra solfato ferroso e acido nitrico produce solfato ferrico, nitrato ferrico, monossido di azoto e acqua. Calcolare il volume di monossido di azoto $\rm NO(g)$ che si sviluppa a $27 \; ^{\circ} \rm C$ e $645 \; \rm mmHg$, facendo reagire 9.1145 grammi di sale ferroso con 3.7806 grammi di acido.
\end{esercizio}
\begin{soluzione}
    La reazione che avviene è
    \begin{equation*}
        \ce{FeSO_4 + HNO_3 -> Fe_2(SO_4)_3 + Fe(NO_3)_3 + NO + H_2O}
    \end{equation*}
    Scriviamo le dissociazioni delle varie specie:
    \begin{equation*}
        \begin{split}
            \ce{FeSO_4} & \ce{-> Fe^{2+} + SO_4^{2-}}
            \\
            \ce{HNO_3} & \ce{-> H^+ + NO_3^-}
            \\
            \ce{Fe_2(SO_4)_3} & \ce{-> 2Fe^{3+} + 3SO_4^{2-}}
            \\
            \ce{Fe(NO_3)_3} & \ce{-> Fe^{3+} + 3NO_3^-}
            \\
            \ce{NO} & \ce{-> NO^{(0)}}
            \\
        \end{split}
    \end{equation*}
    Il ferro passa da n.o. $+2$ nel solfato ferroso a n.o. $+3$ sia nel nitrato ferrico che nel solfato ferrico, quindi si ossida; l'azoto passa da n.o. $+5$ nell'acido nitrico a $+2$ nel monossido di azoto, quindi si riduce.\\
    L'equazione semplificata sarà
    \begin{equation*}
        \ce{3Fe^{2+} + NO_3^- + H^+ -> 3Fe^{3+} + NO + H_2O}
    \end{equation*}
    Prestiamo attenzione per un momento al coefficiente del ferro. A destra tale specie compare sia nel solfato ferrico che nel nitrato ferrico, per cui il suo coefficiente deve essere pari alla somma del numero di atomi. Siccome nel solfato la specie è del tipo $\rm Fe_2$ mentre nel nitrato è solo Fe il coefficiente risulta 3, quindi moltiplicheremo per 3 il ferro anche a destra.\\
    Le semireazioni che avvengono sono:
    \begin{equation*}
        \begin{cases}
            \ce{Fe^{2+} -> Fe^{3+} + e^-}\\
            \ce{N^{5+} +3e^- -> N^{2+}}
        \end{cases}
    \end{equation*}
    Siccome abbiamo 3 atomi di ferro dovremmo moltiplicare per 3 le specie dell'azoto e per 3 quelle del ferro, quindi non moltiplichiamo. A questo punto bilanciamo le cariche: a sinistra abbiamo 6 cariche positive dal ferro e una carica negativa dal nitrato, per un totale di 5 cariche positive; a destra abbiamo solo 9 cariche positive dal ferro. La differenza è di 4 cariche, per cui ci servono 4 ioni $\rm H^+$ e quindi due molecole d'acqua:
    \begin{equation*}
        \ce{3Fe^{2+} + NO_3^- + 4H^+ -> 3Fe^{3+} + NO + 2H_2O}
    \end{equation*}
    Tornando alla reazione completa avremo
    \begin{equation*}
        \ce{3FeSO_4 + 4HNO_3 -> Fe_2(SO_4)_3 + Fe(NO_3)_3 + NO + 2H_2O}
    \end{equation*}
    Per quanto riguarda il coefficiente dell'$\rm HNO_3$ si rimanda al chiarimento che si trova nell'esercizio \ref{EXR:chiarimento_HNO3}; per quanto riguarda i coefficienti del solfato ferrico bisogna ricordarsi che della quantità totale di ferro due terzi provengono dal solfato e un terzo dal nitrato. In altre parole, se ad esempio avessimo moltiplicato le specie per 3 nel bilanciamento degli elettroni avremmo ottenuto $\rm 9Fe^{3+}$ per cui due terzi di questo, cioè 6 atomi, sarebbero andati al solfato, i restanti 3 al nitrato. Inoltre, siccome nel solfato si trova in forma $\rm Fe_2$ bisognerà dividere il coefficiente.\\
    A questo punto calcoliamo il volume di $\rm NO$ che si sviluppa. Per prima cosa dobbiamo vedere chi è il reagente limitante tra l'$\rm FeSO_4$ e l'$\rm HNO_3$, quindi calcoliamone le moli:
    \begin{equation*}
        n_{\rm FeSO_4}=\frac{9.1145}{151.908}
        =6 \cdot 10^{-2} \; \text{mol}
        \quad;\quad
        n_{\rm HNO_3}=\frac{3.7806}{63.01}
        =6 \cdot 10^{-2} \; \rm mol
    \end{equation*}
    Il rapporto stechiometrico è $3:4$, per cui significa che per la totale reazione di 3 moli di solfato servirebbero 4 moli di acido. Nel nostro caso abbiamo $6 \cdot 10^{-2}$ moli di solfato, per cui affinché avvenga la totale reazione servirebbero un numero di moli di acido date dalla proporzione
    \begin{equation*}
        3:4=6 \cdot 10^{-2}:n_{\rm HNO_3}
        \implies
        n_{\rm HNO_3}=\frac{6 \cdot 10^{-2} \cdot 4}{3}
        =8 \cdot 10^{-2} \; \rm mol
    \end{equation*}
    e avendone noi solo $6 \cdot 10^{-2}$, questo sarà il reagente limitante. Guardiamo quindi il rapporto stechiometrico tra l'acido e l'NO. Esso è $4:1$, per cui se ne formeranno un numero di moli pari a un quarto, cioè $1.5 \cdot 10^{-2}$ moli, e il volume corrispondente sarà, usando l'equazione di stato dei gas, pari a
    \begin{equation*}
        PV=nRT
        \implies
        V=\frac{nRT}{P}
        =\frac{1.5 \cdot 10^{-2} \cdot 0.082 \cdot 300.15 \cdot 670}{645}
        =0.3835 \; \rm L
    \end{equation*}
\end{soluzione}

\newpage

\begin{esercizio}
    Bilanciare la seguente reazione, tenendo conto della dissociazione ionica.
    \begin{equation*}
        \ce{Cl2(g) + KIO3(aq) + KOH(aq) -> KIO4(aq) + KCl(aq) + H2O(l)}
    \end{equation*}
\end{esercizio}
\begin{soluzione}
    Scriviamo le dissociazioni delle varie specie:
    \begin{equation*}
        \begin{split}
            \ce{Cl2} & \ce{-> Cl2^0} \text{ (non si dissocia)}
            \\
            \ce{KIO3} & \ce{-> K^+ + IO_3^-}
            \\
            \ce{KOH} & \ce{-> K^+ + OH^-}
            \\
            \ce{KIO4} & \ce{-> K^+ + IO_4^-}
            \\
            \ce{KCl} & \ce{-> K^+ + Cl^-}
            \\
        \end{split}
    \end{equation*}
    Il cloro passa da n.o. $0$ nel $\rm Cl_2$ a n.o. $-1$ nel KCl, quindi si è ridotto; lo iodio passa da n.o. $+5$ nello iodato di potassio $\rm KIO_3$ a n.o. $+7$ nel periodato di potassio $\rm KIO_4$, quindi si è ossidato.\\
    L'equazione semplificata sarà
    \begin{equation*}
        \ce{Cl2 + IO3^- + OH^- -> IO4^- + 2Cl^- + H2O}
    \end{equation*}
    Nota: abbiamo moltiplicato $\rm Cl^-$ per 2 perché a sinistra la specie è $\rm Cl_2$.\\
    Le semireazioni redox che avvengono sono
    \begin{equation*}
        \begin{cases}
            \ce{Cl_2 + 2e^- -> 2Cl^-}\\
            \ce{I^{5+} -> I^{7+} + 2e^-}
        \end{cases}
    \end{equation*}
    Siccome dovremmo moltiplicare sia le specie dello iodio che quelle del cloro per 2 non moltiplichiamo nulla e passiamo direttamente a bilanciare le cariche e quindi le masse. A sinistra abbiamo solo una carica negativa dello ione iodato, a destra ne abbiamo una proveniente dallo ione periodato e 2 provenienti dallo ione cloruro, per un totale di 3 cariche negative. La differenza è di due cariche negative, per cui ci serviranno due ioni $\rm OH^-$ e dunque una molecola di $\rm H_2O$:
    \begin{equation*}
        \ce{Cl2 + IO3^- + 2OH^- -> IO4^- + 2Cl^- + H2O}
    \end{equation*}
    La reazione bilanciata allora sarà
    \begin{equation*}
        \ce{Cl2 + KIO3 + 2KOH(aq) -> KIO4 + 2KCl(aq) + H2O(l)}
    \end{equation*}
\end{soluzione}

\newpage

\begin{esercizio}
    Bilanciare la seguente reazione, tenendo conto della dissociazione ionica.
    \begin{gather*}
        \ce{FeSO4(aq) + K2CrO4(aq) + H2SO4(aq) ->}
        \\
        \ce{-> Fe2(SO4)3(aq) + Cr2(SO4)3(aq) + K2SO4(aq) + H2O(l)}
    \end{gather*}
\end{esercizio}
\begin{soluzione}
    Scriviamo le dissociazioni delle varie specie:
    \begin{equation*}
        \begin{split}
            \ce{FeSO4} & \ce{-> Fe^{2+} + SO_4^{2-}}
            \\
            \ce{K2CrO4} & \ce{-> 2K^+ + CrO_4^{2-}}
            \\
            \ce{H2SO4} & \ce{-> 2H^+ + SO_4^{2-}}
            \\
            \ce{Fe2(SO4)3} & \ce{-> 2Fe^{3+} + 3SO_4^{2-}}
            \\
            \ce{Cr2(SO4)3} & \ce{-> 2Cr^{3+} + 3SO_4^{2-}}
            \\
            \ce{K2SO4} & \ce{-> K^+ + SO_4^{2-}}
            \\
        \end{split}
    \end{equation*}
    Il ferro passa da n.o. $+2$ nel solfato ferroso a n.o. $+3$ nel solfato ferrico, quindi si è ossidato; il cromo passa da n.o. $+6$ nel cromato di potassio a n.o. $+3$ nel solfato di cromo (III), dunque si è ridotto.\\
    La reazione semplificata sarà
    \begin{equation*}
        \ce{2Fe^{2+} + 2CrO_4^{2-} + H^+ -> 2Fe^{3+} + 2Cr^{3+} + H_2O}
    \end{equation*}
    dove abbiamo moltiplicato le specie del ferro e del cromo a sinistra per 2 perché a destra le specie sono del tipo $\rm Fe_2$ e $\rm Cr_2$.\\
    Le semireazioni redox che avvengono sono:
    \begin{equation*}
        \begin{cases}
            \ce{2Fe^{2+} -> 2Fe^{3+} + 2e^-} \quad (\text{1 per atomo})\\
            \ce{2Cr^{6+} + 6e^- -> 2Cr^{3+}} \quad (\text{3 per atomo})
        \end{cases}
    \end{equation*}
    Dovremmo quindi moltiplicare per 2 le specie del cromo e per 6 quelle del ferro, ma possiamo semplificare e moltiplicarle rispettivamente per 1 e per 3:
    \begin{equation*}
        \ce{6Fe^{2+} + 2CrO_4^{2-} + H^+ -> 6Fe^{3+} + 2Cr^{3+} + H_2O}
    \end{equation*}
    A questo punto bilanciamo le cariche. A sinistra abbiamo 12 cariche positive dallo ione ferroso e 4 cariche negative dallo ione cromato, per un totale di 8 cariche positive; a destra abbiamo 18 cariche positive dallo ione ferrico e altre 6 cariche positive dallo ione $\rm Cr^{3+}$, per un totale di 24 cariche positive. La differenza è di 16 cariche, per cui ci serviranno 16 ioni $\rm H^+$ e quindi 8 molecole di $\rm H_2O$:
    \begin{equation*}
        \ce{6Fe^{2+} + 2CrO_4^{2-} + 16H^+ -> 6Fe^{3+} + 2Cr^{3+} + 8H_2O}
    \end{equation*}
    La reazione bilanciata allora sarà
    \begin{equation*}
        \ce{6FeSO4 + 2K2CrO4 + 8H2SO4 -> 3Fe2(SO4)3 + Cr2(SO4)3 + 2K2SO4 + 8H2O}
    \end{equation*}
    Da notare che i pedici a destra sono dimezzati perché le specie sono del tipo $\rm Fe_2$ e $\rm Cr_2$. Inoltre sono stati successivamente bilanciati gli atomi di potassio.
\end{soluzione}

\newpage

\begin{esercizio}
    Calcolare quanti litri di NO si formano a c.n. facendo reagire 0.15 moli di Cu con 0.35 moli di acido nitrico:
    \begin{equation*}
        \text{Rame + Acido nitrico} \ce{->} \text{Nitrato di rame + NO(g) + Acqua}
    \end{equation*}
\end{esercizio}
\begin{soluzione}
    La reazione che avviene è
    \begin{equation*}
        \ce{Cu + HNO_3 -> Cu(NO_3)_2 + NO + H_2O}
    \end{equation*}
    Scriviamo le dissociazioni delle varie specie:
    \begin{equation*}
        \begin{split}
            \ce{Cu} & \ce{-> Cu^0}
            \\
            \ce{HNO_3} & \ce{-> H^+ + NO_3^-}
            \\
            \ce{Cu(NO_3)_2} & \ce{-> Cu^{2+} + 2NO_3^-}
            \\
            \ce{NO} & \ce{-> NO^0} \text{ (non si dissocia)}
        \end{split}
    \end{equation*}
    Il rame passa da n.o. $0$ nel Cu a n.o. $+2$ nel nitrato di rame, quindi si è ossidato; l'azoto passa da n.o. $+5$ nell'acido nitrico a n.o. $+2$ nell'NO, quindi si è ridotto.\\
    La reazione semplificata sarà
    \begin{equation*}
        \ce{Cu + NO_3^- + H^+ -> Cu^{2+} + NO + H_2O}
    \end{equation*}
    Le semireazioni che avvengono sono
    \begin{equation*}
        \begin{cases}
            \ce{Cu^0 -> Cu^{2+} + 2e^-}\\
            \ce{N^{5+} + 3e^- -> N^{2+}}
        \end{cases}
    \end{equation*}
    Moltiplichiamo per 2 le specie dell'azoto e per 3 quelle del rame:
    \begin{equation*}
        \ce{3Cu + 2NO_3^- + H^+ -> 3Cu^{2+} + 2NO + H_2O}
    \end{equation*}
    A questo punto bilanciamo le cariche. A sinistra abbiamo solo 2 cariche negative dallo ione nitrato, a destra abbiamo 6 cariche positive dallo ione $\rm Cu^{2+}$. La differenza è di 8 cariche positive, per cui ci servono 8 ioni $\rm H^+$ e quindi 4 molecole d'acqua:
    \begin{equation*}
        \ce{3Cu + 2NO_3^- + 8H^+ -> 3Cu^{2+} + 2NO + 4H_2O}
    \end{equation*}
    La reazione completa sarà
    \begin{equation*}
        \ce{3Cu + 8HNO_3 -> 3Cu(NO_3)_2 + 2NO + 4H_2O}
    \end{equation*}
    \textbf{Chiarimento tratto dal Silvestroni-Rallo}: Nella reazione l'$\rm HNO_3$ non ha soltanto funzione ossidante, ma anche funzione salificante, estranea alla vera reazione redox. Infatti nell'equazione di reazione $\rm Cu^{2+}$ compare come $\rm Cu(NO_3)_2$ e poiché i $\rm Cu^{2+}$ sono 3, occorrono 6 molecole $\rm HNO_3$ per formare $\rm 3Cu(NO_3)_2$. Pertanto il coefficiente di $\rm HNO_3$ vale 8: due molecole con funzione ossidante e sei molecole con funzione salificante.\\
    A questo punto calcoliamo i litri di NO. Il rapporto stechiometrico tra Cu e $\rm HNO_3$ è $3:8$, per cui le moli di acido nitrico saranno in difetto. Infatti affinché tutte le moli di rame reagiscano dovremmo avere
    \begin{equation*}
        3:8=0.15:n_{\rm HNO_3}
        \implies
        n_{\rm HNO_3}
        =\frac{8 \cdot 0.15}{3}
        =0.4 \; \rm mol
    \end{equation*}
    ma ne abbiamo solo 0.35, per cui sarà l'acido il reagente limitante. Il rapporto stechiometrico tra questo e l'NO è $8:2$ cioè $4:1$, per cui le moli di NO prodotte saranno date dalla relazione
    \begin{equation*}
        4:1=n_{\rm HNO_3}:n_{\rm NO}
        \implies
        n_{\rm NO}
        =\frac{0.35}{4}
        =8.75 \cdot 10^{-2} \; \rm mol
    \end{equation*}
    I litri si calcoleranno tramite l'equazione di stato dei gas
    \begin{equation*}
        PV=nRT
        \implies
        V_{\rm NO}
        =\frac{8.75 \cdot 10^{-2} \cdot 0.082 \cdot 273.15}{1}
        =1.9598 \; \rm L
    \end{equation*}
\end{soluzione}

\newpage

\begin{esercizio}
    Bilanciare la seguente reazione:
    \begin{gather*}
        \ce{K2Cr2O7(aq) + KI(aq) + H2SO4(aq) ->}
        \\
        \ce{-> Cr2(SO4)3(aq) + I2(s) + K2SO4(aq) + H2O(l)}
    \end{gather*}
\end{esercizio}
\begin{soluzione}
    Scriviamo le dissociazioni delle varie specie:
    \begin{equation*}
        \begin{split}
            \ce{K2Cr2O7} & \ce{-> 2K^+ + Cr2O7^{2-}}
            \\
            \ce{KI} & \ce{-> K^+ + I^-}
            \\
            \ce{H2SO4} & \ce{-> 2H^+ + SO_4^{2-}}
            \\
            \ce{Cr2(SO4)3} & \ce{-> 2Cr^{3+} + 3SO_4^{2-}}
            \\
            \ce{I2} & \ce{-> I2^{0}} \text{ (non si dissocia)}
            \\
            \ce{K2SO4} & \ce{-> 2K^+ + SO_4^{2-}}
            \\
        \end{split}
    \end{equation*}
    Il cromo passa da n.o. $+6$ nello ione bicromato a n.o. $+3$ nel solfato di cromo (III), quindi si è ridotto; lo iodio passa da n.o. $-1$ nello ioduro di potassio a n.o. $0$ nell'$\rm I_2$, quindi si è ossidato.\\
    La reazione semplificata sarà
    \begin{equation*}
        \ce{Cr2O7^{2-} + 2I^- + H^+ -> 2Cr^{3+} + I_2 + H_2O}
    \end{equation*}
    Nota: abbiamo moltiplicato $\rm I^-$ per due perché la specie a destra è $\rm I_2$.\\
    Le semireazioni redox che avvengono sono
    \begin{equation*}
        \begin{cases}
            \ce{2Cr^{6+} + 6e^- -> 2Cr^{3+}} \quad (\text{3 per atomo})\\
            \ce{2I^- + 2e^- -> I_2} \quad (\text{1 per atomo})
        \end{cases}
    \end{equation*}
    Moltiplichiamo per 3 le specie dello iodio e per 1 quelle del cromo:
    \begin{equation*}
        \ce{Cr2O7^{2-} + 6I^- + H^+ -> 2Cr^{3+} + 3I_2 + H_2O}
    \end{equation*}
    A questo punto bilanciamo le cariche. A sinistra abbiamo 2 cariche negative dallo ione bicromato e 6 cariche negative dallo ione ioduro, per un totale di 8 cariche negative; a destra abbiamo solo 6 cariche positive dallo ione $\rm Cr^{3+}$. La differenza è di 14 cariche positive, per cui ci serviranno 14 ioni $\rm H^+$ e quindi 7 molecole d'acqua:
    \begin{equation*}
        \ce{Cr2O7^{2-} + 6I^- + 14H^+ -> 2Cr^{3+} + 3I_2 + 7H_2O}
    \end{equation*}
    Tornando alla reazione completa avremo
    \begin{equation*}
        \ce{K2Cr2O7 + 6KI + 7H2SO4 -> Cr2(SO4)3 + 3I2 + 4K2SO4 + 7H2O}
    \end{equation*}
    Da notare che il coefficiente dell'acido solforico è dimezzato perché la specie è del tipo $\rm H_2$. Inoltre abbiamo bilanciato in un secondo momento gli atomi di potassio.
\end{soluzione}

\newpage

\begin{esercizio}
    Bilanciare la seguente reazione
    \begin{equation*}
        \ce{Cr2O3(s) + KNO3(aq) + NaOH(aq) -> Na2CrO4(aq) + KNO2(aq) + H2O(l)}
    \end{equation*}
\end{esercizio}
\begin{soluzione}
    Scriviamo le dissociazioni delle varie specie:
    \begin{equation*}
        \begin{split}
            \ce{Cr2O_3} & \ce{-> Cr2O_3^{(0)}} \text{ (non si dissocia)}
            \\
            \ce{KNO3} & \ce{-> K^+ + NO_3^-}
            \\
            \ce{NaOH} & \ce{-> Na^+ + OH^-}
            \\
            \ce{Na2CrO4} & \ce{-> 2Na^+ + CrO_4^{2-}}
            \\
            \ce{KNO_2} & \ce{-> K^+ + NO_2^-}
            \\
        \end{split}
    \end{equation*}
    Il cromo passa da n.o. $+3$ nell'ossido di cromo (III) a n.o. $+6$ nel cromato di sodio, quindi si ossida; l'azoto passa da n.o. $+5$ nel nitrato di potassio a n.o. $+3$ nel nitrito di potassio, quindi si riduce.\\
    La reazione semplificata sarà
    \begin{equation*}
        \ce{Cr_2O_3 + NO_3^- + OH^- -> 2CrO_4^{2-} + NO_2^- + H_2O}
    \end{equation*}
    dove la specie $\rm CrO_4^{2-}$ è moltiplicata per 2 perché a sinistra la specie è del tipo $\rm Cr_2$.\\
    Le semireazioni redox che avvengono sono
    \begin{equation*}
        \begin{cases}
            \ce{Cr^{3+} -> Cr^{6+} + 3e^-}\\
            \ce{N^{5+} + 2e^- -> N^{3+}}
        \end{cases}
    \end{equation*}
    Dovremmo quindi moltiplicare per 6 le specie dell'azoto (perché gli atomi di cromo sono 2) e per 2 quelle del cromo. Per semplicità moltiplichiamo rispettivamente per 3 e per 1:
    \begin{equation*}
        \ce{Cr_2O_3 + 3NO_3^- + OH^- -> 2CrO_4^{2-} + 3NO_2^- + H_2O}
    \end{equation*}
    A questo punto bilanciamo le cariche. A sinistra abbiamo solo 3 cariche negative dallo ione nitrato; a destra abbiamo 4 cariche negative dallo ione cromato e 3 cariche negative dallo ione nitrito, per un totale di 7 cariche negative. La differenza è di 4 cariche negative, per cui ci servono 4 ioni $\rm OH^-$ e quindi 2 molecole di $\rm H_2O$:
    \begin{equation*}
        \ce{2Cr_2O_3 + 3NO_3^- + 4OH^- -> 2CrO_4^{2-} + 3NO_2^- + 2H_2O}
    \end{equation*}
    La reazione completa sarà
    \begin{equation*}
        \ce{Cr2O3 + 3KNO3 + 4NaOH -> 2Na2CrO4 + 3KNO2 + 2H2O}
    \end{equation*}
\end{soluzione}

\newpage

\begin{esercizio}
    Bilanciare la seguente reazione e calcolare il volume di monossido di azoto a c.n. che si ottiene a partire da 4.7523 grammi di solfuro di rame:
    \begin{gather*}
        \text{Acido nitrico + Solfuro di rame} \ce{->}
        \\
        \ce{->} \text{Nitrato di rame + Zolfo + Monossido di azoto + Acqua}
    \end{gather*}
\end{esercizio}
\begin{soluzione}
    La reazione che avviene è
    \begin{equation*}
        \ce{CuS + HNO_3 -> Cu(NO_3)_2 + S + NO + H_2O }
    \end{equation*}
    Scriviamo le dissociazioni delle varie specie:
    \begin{equation*}
        \begin{split}
            \ce{CuS} & \ce{-> Cu^{2+} + S^{2-}}
            \\
            \ce{HNO_3} & \ce{-> H^+ + NO_3^-}
            \\
            \ce{Cu(NO_3)_2} & \ce{-> Cu^{2+} + 2NO_3^-}
            \\
            \ce{S} & \ce{-> S^0}
            \\
            \ce{NO} & \ce{-> NO} \; \text{(non si dissocia)}
        \end{split}
    \end{equation*}
    L'azoto passa da n.o. $+5$ nello ione nitrato a n.o. $+2$, quindi si è ridotto; lo zolfo passa da n.o. $-2$ nello ione solfuro a n.o. $0$ nello zolfo, quindi si è ossidato.\\
    La reazione semplificata sarà
    \begin{equation*}
        \ce{S^{2-} + H^+ + NO_3^- -> S + NO + H_2O}
    \end{equation*}
    Le semireazioni che avvengono sono
    \begin{equation*}
        \begin{cases}
            \ce{S^{2-} -> S^0 + 2e^-}\\
            \ce{N^{5+} + 3e^- -> N^{2+}}
        \end{cases}
    \end{equation*}
    Moltiplichiamo quindi per 2 le specie dell'azoto e per 3 quelle dello zolfo:
    \begin{equation*}
        \ce{3S^{2-} + H^+ + 2NO_3^- -> 3S + 2NO + H_2O}
    \end{equation*}
    A questo punto bilanciamo le cariche. A sinistra abbiamo 6 cariche negative dallo ione solfuro e 2 dallo ione nitrato, per un totale di 8 cariche negative; a destra non abbiamo cariche. La differenza è di 8 cariche negative, quindi ci servono 8 ioni $\rm H^+$ e quindi 4 molecole di acqua:
    \begin{equation*}
        \ce{3S^{2-} + 8H^+ + 2NO_3^- -> 3S + 2NO + 4H_2O}
    \end{equation*}
    La reazione completa sarà
    \begin{equation*}
        \ce{3CuS + 8HNO_3 -> 3Cu(NO_3)_2 + 3S + 2NO + 4H_2O }
    \end{equation*}
    dove il coefficiente del $\rm Cu(NO_3)_2$ è dovuto al fatto che degli 8 atomi di azoto solo due si riducono, mentre gli altri 6 servono per la formazione del sale.\\
    Passiamo adesso a calcolare il volume di monossido di azoto a partire dai grammi di solfuro di rame. Il rapporto stechiometrico tra $\rm CuS$ e $\rm NO$ è $3:2$, per cui le moli di $\rm NO$ saranno date dalla proporzione
    \begin{equation*}
        3:2=n_{\rm CuS}:n_{\rm NO}
        \implies
        n_{\rm NO}=\frac{2 \cdot n_{\rm CuS}}{3}
        =\frac{2}{3} \cdot \frac{4.7523}{95.611}
        =3.3136 \cdot 10^{-2} \; \rm mol
    \end{equation*}
    Dall'equazione di stato dei gas perfetti avremo che
    \begin{equation*}
        PV=nRT
        \implies
        V=\frac{nRT}{P}
        =\frac{3.3136 \cdot 10^{-2} \cdot 0.082 \cdot 273.15}{1}
        =7.4220 \cdot 10^{-3} \; \rm L
    \end{equation*}
\end{soluzione}