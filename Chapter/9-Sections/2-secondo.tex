\vspace{0.2cm}\textbf{2.2} Bilanciare la  seguente reazione e calcolare i grammi di iodio che si ottengono facendo reagire 21,5 g di $\rm KIO_3$ (MM 214,0) con 57,0 g di KI (MM = 166,0).

\vspace{0.2cm}
\begin{center}Iodato di potassio + ioduro di potassio + acido solforico \ce{->}

\ce{->}iodio +solfato di potassio +acqua. 
\end{center}

\large\textbf{Svolgimento}\normalsize

\vspace{0.2cm}La reazione che avviene è

$$\ce{KIO_3 + KI + H_2SO_4 -> I_2 + K_2SO_4 + H_2O}$$

Le dissociazioni delle varie speci sono

$$\ce{KIO_3 -> K^+ + IO_3^-}$$
$$\ce{KI -> K^+ + I^-}$$
$$\ce{H_2SO_4 -> 2H^+ + SO_4^-}$$
$$\ce{I_2 -> I_2^{(0)}}\text{(non si dissocia)}$$
$$\ce{K_2SO_4 -> 2K^+ + SO_4^-}$$

Lo iodio passa da n.o. +5 nello ione iodato e da n.o. -1 nello ioduro a n.o. 0 nella molecola $\rm I_2$, quindi in parte si riduce e in parte si ossida, cioè siamo davanti ad una reazione di disproporzione.

La reazione semplificata sarà

$$\ce{IO_3^- + I^- + H^+ -> I_2 + H_2O}$$

Le semireazioni redox che avvengono sono 

$$\begin{cases}
    \ce{IO_3^- + 5e^-  -> I^0}\\
    \ce{I^- -> I^0 + e^-}
\end{cases}$$

Moltiplichiamo per 5 la specie dello ioduro e e per 1 quella dello iodato. Lo iodio verrà moltiplicato per la somma, ma siccome la specie è $\rm I_2$ moltiplichiamo per 3:

$$\ce{IO_3^- + 5I^- + H^+ -> 3I_2 + H_2O}$$

Bilanciamo le cariche. A sinistra abbiamo una carica negativa dallo iodato e 5 cariche negative dallo ioduro, per un totale di 6 cariche negative; a destra non abbiamo cariche, per cui per bilanciare abbiamo bisogno di 6 ioni $\rm H^+$ e quindi di 3 molecole d'acqua:

$$\ce{IO_3^- + 5I^- + 6H^+ -> 3I_2 + 3H_2O}$$

La reazione completa allora sarà

$$\ce{KIO_3 + 5KI + 3H_2SO_4 -> 3I_2 + 3K_2SO_4 + 3H_2O}$$

dove abbiamo successivamente bilanciato gli atomi di potassio.

\vspace{0.2cm}\textbf{2.3} Stabilire quanti grammi di bicromato di potassio sono necessari per l’ossidazione di 1.2345 grammi di solfato ferroso nella seguente reazione:

\begin{center}
    Solfato ferroso + Bicromato di potassio + Acido solforico \ce{->}
    
    \ce{->} Solfato di cromo (III) + Solfato di ferro (III), Solfato di potassio + Acqua.   
\end{center}

\large\textbf{Svolgimento}\normalsize

\vspace{0.2cm}La reazione che avviene è

$$\ce{FeSO_4 + K_2Cr_2O_7 + H_2SO_4 -> Cr_2(SO_4)_3 + Fe_2(SO_4)_3 + K_2SO_4 + H_2O}$$

Le molecole dissociate sono

$$\ce{FeSO_4 -> Fe^{2+} + SO_4^{2-}}$$
$$\ce{K_2Cr_2O_7 -> 2K^+ + Cr_2O_7^{2-}}$$
$$\ce{H_2SO_4 -> 2H^+ + SO_4^{2-}}$$
$$\ce{Cr_2(SO_4)_3 -> 2Cr^{3+} + 3SO_4^{3-}}$$
$$\ce{Fe_2(SO_4)_3 -> 2Fe^{3+} + 3SO_4^{2-}}$$
$$\ce{K_2SO_4 -> 2K^+ + SO_4^{2-}}$$

Il ferro passa da n.o +2 a n.o. +3, quindi si è ossidato; il cromo passa da n.o. + 6 nel bicromato a n.o. +3 nel solfato di cromo, quindi si è ridotto.

L'equazione semplificata sarà

$$\ce{2Fe^{2+} + Cr_2O_7^{2-} + H^+ -> 2Cr^{3+} + 2Fe^{3+} + H_2O}$$

Le semireazioni redox che avvengono sono

$$\begin{cases}
    \ce{Fe^{2+} -> Fe^{3+} + e^-}\\
    \ce{Cr^{6+} + 3e^- -> Cr^{3+}}
\end{cases}$$

Visto che abbiamo due atomi di ferro e due di cromo, moltiplicheremo per due le speci chimiche contenenti il cromo e per 6 quelle contenenti il ferro:

$$\ce{12Fe^{2+} + 2Cr_2O_7^{2-} + H^+ -> 4Cr^{3+} + 12Fe^{3+} + H_2O}$$

A questo punto bilanciamo le cariche. A destra abbiamo 24 cariche positive dal ferro e 4 cariche negative dallo ione bicromato, per un totale di 20 cariche positive; a destra abbiamo 12 cariche positive dal cromo e 36 dal ferro, per un totale di 48 cariche positive. La differenza è di 28, quindi avremo bisogno di 28 ioni $\rm H^+$, e quindi 14 molecole d'acqua

$$\ce{12Fe^{2+} + 2Cr_2O_7^{2-} + 28H^+ -> 4Cr^{3+} + 12Fe^{3+} + 14H_2O}$$

Torniamo alla reazione completa

$$\ce{12FeSO_4 + 2K_2Cr_2O_7 + 14H_2SO_4 -> 2Cr_2(SO_4)_3 + 6Fe_2(SO_4)_3 + K_2SO_4 + 14H_2O}$$

Il solfato di potassio non ha partecipato alla reazione, quindi deve essere bilanciato. Abbiamo 4 atomi di potassio a sinistra, 2 a destra, quindi basterà moltiplicare per 2:

$$\ce{12FeSO_4 + 2K_2Cr_2O_7 + 14H_2SO_4 -> 2Cr_2(SO_4)_3 + 6Fe_2(SO_4)_3 + 2K_2SO_4 + 14H_2O}$$

Notiamo, inoltre, che possiamo semplificare tutti i coefficienti, per cui possiamo riscrivere la reazione come

$$\ce{6FeSO_4 + K_2Cr_2O_7 + 7H_2SO_4 -> Cr_2(SO_4)_3 + 3Fe_2(SO_4)_3 + K_2SO_4 + 7H_2O}$$

\vspace{0.2cm}\textbf{2.4} Completare e bilanciare la seguente reazione e calcolare i grammi del sale di stagno che si formano a partire da 1.000 gr di permanganato di potassio: 
permanganato di potassio + cloruro stannoso + acido cloridrico =

\vspace{0.2cm}\large\textbf{Svolgimento}\normalsize

\vspace{0.2cm}

\vspace{0.2cm}\textbf{2.5} Calcolare i grammi di solfato ferrico che si ottengano mescolando in acqua 2.3456 grammi di permanganato di potassio  ed 12,3456 grammi di solfato ferroso, in eccesso di acido solforico.

\vspace{0.2cm}\large\textbf{Svolgimento}\normalsize

\vspace{0.2cm}\textbf{Non lo so vita mi sembra scritta male sta frase}Il testo non ci dice tutti i prodotti, però possiamo ottenerli: notiamo infatti che sia tra reagenti che tra i prodotti abbiamo un composto contenente ferro, mentre abbiamo composti composti contenenti il manganese solo tra i reagenti. Dobbiamo trovare un prodotto contenente il manganese

Pertanto, la reazione che avviene è

$$\ce{KMnO_4 + FeSO_4 + H_2SO_4 -> Fe_2(SO_4)_3 + MnSO_4 + K_2SO_4 + H_2O}$$

Scriviamo le dissociazioni delle varie speci:

$$\ce{KMnO_4 -> K^+ + MnO_4^-}$$
$$\ce{FeSO_4 -> Fe^{2+} + SO_4^{2-}}$$
$$\ce{H_2SO_4 -> 2H^+ + SO_4^{2-}}$$
$$\ce{Fe_2(SO_4)_3 -> 2Fe^{3+} + 3SO_4^{2-}}$$
$$\ce{MnSO_4 -> Mn^{2+} + SO_4^{2-}}$$
$$\ce{K_2SO_4 -> 2K^+ + SO_4^{2-}}$$

Il manganese passo da n.o. +7 nello ione permanganato a n.o. +2 nel solfato di manganese, quindi si riduce; il ferro passa da n.o. +2 nel solfato ferroso a n.o. +3 nel solfato ferrico, quindi si riduce.

L'equazione semplificata sarà

$$\ce{MnO_4^- + 2Fe^{2+} + H^+ -> 2Fe^{3+} + Mn^{2+} + H_2O}$$

Le semireazioni redox che avvengono sono

$$\begin{cases}
    \ce{Fe^{2+} -> Fe^{3+} + e^-}\\
    \ce{Mn^{7+} + 5e^- -> Mn^{2+}}
\end{cases}$$

Visto che abbiamo due atomi di ferro moltiplicheremo per due le speci del manganese e per 5 quelle del ferro:

$$\ce{2MnO_4^- + 10Fe^{2+} + H^+ -> 10Fe^{3+} + 2Mn^{2+} + H_2O}$$

A questo punto bilanciamo le cariche. A sinistra abbiamo 2 cariche negative dal permanganato e 20 cariche positive dal ferro, per un totale di 18 cariche positive; a destra abbiamo 30 cariche positive dal ferro e 4 dal manganese, per un totale di 34 cariche positive. La differenza è di 16 cariche, per cui ci serviranno 16 ioni $\rm H^+$ e quindi 8 molecole di acqua:

$$\ce{2MnO_4^- + 10Fe^{2+} + 16H^+ -> 10Fe^{3+} + 2Mn^{2+} + 8H_2O}$$

Tornando alla reazione completa avremo

$$\ce{2KMnO_4 + 10FeSO_4 + 8H_2SO_4 -> 5Fe_2(SO_4)_3 + 2MnSO_4 + K_2SO_4 + 8H_2O}$$

\vspace{0.2cm}\textbf{2.6} Calcolare quanti litri di cromato di potassio 0.14 M sono necessari per ossidare 20.000 g di solfato ferroso:
\begin{center}
\ce{FeSO4(aq) + K2CrO4(aq) + H2SO4(aq) ->}

\ce{-> Fe2(SO4)3(aq) + Cr2(SO4)3(aq) + K2SO4(aq) + H2O(l)}
\end{center}

\large\textbf{Svolgimento}\normalsize

\vspace{0.2cm}

\vspace{0.2cm}\textbf{2.7} Lo zinco in acido cloridrico produce cloruro di zinco ed idrogeno. Calcolare il volume di
idrogeno misurato a c.n. ed i grammi di cloruro di zinco che si ottengono per reazione di 2.42 g di
zinco con 100 mL di una soluzione acquosa di HCl 0.5 M.

\vspace{0.2cm}\large\textbf{Svolgimento}\normalsize

\vspace{0.2cm}

\vspace{0.2cm}\textbf{2.8} Calcolare i grammi di bromuro di litio che si ottengono da 4.5678 g di carbonato di litio fatti reagire con acido bromidrico. La reazione, da bilanciare, genera inoltre anidride carbonica ed acqua.

\vspace{0.2cm}\large\textbf{Svolgimento}\normalsize

\vspace{0.2cm}La reazione che avviene è

$$\ce{Li_2CO_3 + HBr -> CO_2 + LiBr + H_2O}$$

Scriviamo le dissociazioni delle varie speci:

$$\ce{Li_2CO_3 -> 2Li^+ + CO_3^{2-}}$$
$$\ce{HBr -> H^+ + Br^-}$$

\textbf{niente sono esploso, controlla valenza idrogeno negli idruri}

\vspace{0.2cm}\textbf{2.9} Bilanciare la seguente reazione e calcolare i litri di SO2 a c.n. necessari per la riduzione di 1.5
grammi di bicromato di potassio.

\begin{center}
    Bicromato di potassio + anidride solforosa + acido cloridrico \ce{->}

    \ce{->} solfato di cromo(III) + cloruro di potassio + acqua
\end{center}

\large\textbf{Svolgimento}\normalsize

\vspace{0.2cm}La reazione che avviene, in formule, è

$$\ce{K_2Cr_2O_7 + SO_2 + HCl -> Cr_2(SO_4)_3 + KCl + H_2O}$$

Scriviamo le dissociazioni delle molecole

$$\ce{K_2Cr_2O_7 -> 2K^+ + Cr_2O_7^{2-}}$$
$$\ce{SO_2 -> S^{4+} 2O^{2-}}$$
$$\ce{HCl -> H^+ + Cl^-}$$
$$\ce{Cr_2(SO_4)_3 -> 2Cr^{3+} + 3SO_4^{2-}}$$
$$\ce{KCl -> K^+ + Cl^-}$$

Il cromo passa da n.o. +6 nel bicromato a n.o. +3 nel solfato di cromo, quindi si è ridotto; lo zolfo passa da n.o. + 4 nell'anidride a n.o. +6 nel solfato, quindi si è ossidato.

L'equazione semplificata sarà

$$\ce{Cr_2O_7^{2-} + 3SO_2 + H^+ -> 2Cr^{3+} 3SO_4^{2-} + H_2O}$$

Notiamo che tra i reagenti abbiamo messo $\rm SO_2$ e non $\rm S^{4+}$. Si potrebbe infatti pensare di scrivere quest'ultimo nella reazione semplificata, similmente a quanto avviene nel caso del passaggio da ione permanganato $\rm MNO_4^{+}$ al manganese $\rm Mn^{2+}$. Il motivo è che l'ossigeno influenza il n.o. del composto dello zolfo, per cui o lo consideriamo sia tra reagenti che prodotti o non lo consideriamo.

Le semireazioni redox che avvengono sono

$$\begin{cases}
    \ce{Cr^{6+} +3e^- -> Cr^{3+}}\\
    \ce{S^{4+} -> S^{6+} +2e}
\end{cases}$$

Visto che abbiamo due atomi di cromo e tre di zolfo, dovremmo moltiplicare per 6 le speci delle zolfo e del cromo. \E chiaro che in questo caso non ha senso moltiplicare, quindi passiamo direttamente al bilancio delle cariche: a destra abbiamo solo 2 cariche negative dal bicromato; a destra abbiamo 6 cariche positive. Per bilanciare abbiamo bisogno di 4 ioni $\rm H^+$ e quindi due molecole di $\rm H_2O$:

$$\ce{Cr_2O_7^{2-} + 3SO_2 + 4H^+ -> 2Cr^{3+} 3SO_4^{2-} + 2H_2O}$$

La reazione completa sarà

$$\ce{K_2Cr_2O_7 + 3SO_2 + 4HCl -> Cr_2(SO_4)_3 + 2KCl + 2H_2O}$$

dove il potassio è stato bilanciato a posteriori.

\vspace{0.2cm}\textbf{2.10} Bilanciare la seguente reazione e calcolare i grammi di permanganato necessari per ossidare una
mole di solfato ferroso in eccesso di acido solforico:

\begin{center}
permanganato di potassio + solfato ferroso + acido solforico \ce{->}

\ce{->}solfato di manganese(II) + solfato ferrico + solfato di potassio + acqua
\end{center}

\large\textbf{Svolgimento}\normalsize

\vspace{0.2cm}La reazione che avviene è

$$\ce{KMnO_4 + FeSO_4 -> MnSO_4 + Fe_2(SO_4)_3 + K_2SO_4 + H_2O}$$

\vspace{0.2cm}\textbf{2.11} Bilanciare la seguente reazione e calcolare il volume di NO misurato a c.n. che si sviluppa a
partire da 1.3540 g di rame:

\begin{center}
\ce{Cu(s) + HNO3(aq) \ce{->} Cu(NO3)2(aq) + NO(g) + H2O(l)}
\end{center}

\large\textbf{Svolgimento}\normalsize

\vspace{0.2cm}

\vspace{0.2cm}\textbf{2.12} Bilanciare la seguente reazione e calcolare i grammi di cloruro di zinco che si ottengono a partire da 3.1234 grammi di anidride arseniosa.

\begin{center}
\ce{As2O3 + Zn + HCl -> AsH3 + ZnCl2 + H2O}
\end{center}

\large\textbf{Svolgimento}\normalsize

\vspace{0.2cm}Scriviamo le dissociazioni delle varie speci chimiche

$$\ce{As2O3 -> 2As^{3+} + 3O^{2-}}$$
$$\ce{Zn -> Zn^0}$$
$$\ce{HCl -> H^+ + Cl^-}$$
$$\ce{AsH_3 -> As^{3+} + 3H^{-}}$$
$$\ce{ZnCl_2 -> Zn^{2+} + 2Cl^{-}}$$

Lo zinco passa da n.o. 0 a n.o. +2 nel cloruro, quindi si ossida; L'idrogeno passa da n.o. +1 nell'acido cloridrico a n.o. -1 nell'idruro arsenioso (o arsina) $\rm AsH_3$, quindi si è ridotto.

L'equazione semplificata sarà

$$\ce{Zn + 3H^+ -> Zn^{2+} + 3H^- + H_2O }$$

Le semireazioni redox che avvengono sono

$$\begin{cases}
    \ce{Zn^{0} -> Zn^{2+} + 2e^-}\\
    \ce{H^+ 2e^- -> H^-}
\end{cases}$$

Visto che abbiamo tre atomi di idrogeno, dobbiamo moltiplicare per 6 le speci dello zinco, mentre dovremo moltiplicare per 2 quello dell'idrogeno:

$$\ce{6Zn + 6H^+ -> 6Zn^{2+} + 6H^- + H_2O}$$

Avremmo potuto semplicemente moltiplicare per 3 quelle dello zinco, tuttavia in questo caso nel bilanciamento delle masse (le cariche sono già bilanciate) avremmo ottenuto dei coefficienti non interi. Notiamo inoltre che, per bilanciare le masse, dobbiamo guardare anche le speci che non partecipano all'ossidoriduzione: per avere una molecola di $\rm As_2O_3$ servono 1.5 molecole d'acqua, o per meglio dire per due molecole di anidride arseniosa servono 3 molecole d'acqua. Tuttavia 3 molecole d'acqua implicano 6 ulteriori atomi di idrogeno, che dobbiamo aggiungere al conteggio.

La reazione completa sarà quindi 

$$\ce{2As2O3 + 6Zn + 12HCl -> 2AsH3 + 6ZnCl2 + 3H2O}$$

\vspace{0.2cm}\textbf{2.13} Bilanciare la reazione e calcolare il volume di cloro gassoso necessario per ottenere 10 litri di ipoclorito 2.0 M:

\begin{center}
    Cloro(g) + idrossido di sodio(aq) \ce{->} 
    
    \ce{->} ipoclorito di sodio(aq) + cloruro di sodio(aq) + acqua. 
\end{center}

\vspace{0.2cm}\large\textbf{Svolgimento}\normalsize

\vspace{0.2cm}La reazione che avviene è

$$\ce{Cl2(g) + NaOH(aq) ->  NaClO(aq) + NaCl + H2O}$$

Scriviamo le dissociazioni delle varie speci chimiche

$$\ce{Cl_2 -> 2Cl^0 }$$
$$\ce{NaOH -> Na^+ + OH^-}$$
$$\ce{NaClO -> Na^+ + ClO^-}$$
$$\ce{NaCl -> Na^+ + Cl^-}$$

Notiamo che in questa reazione il cloro ha n.o. 0 tra i reagenti, mentre tra i prodotti ha sia n.o. +1 nell'ipoclorito che n.o. -1 nel cloruro, pertanto siamo davanti ad una reazione di disproporzione.

La reazione semplificata sarà

$$\ce{Cl_2 + OH^- -> Cl^- + ClO^- + H_2O}$$

Le semireazioni redox che avvengono sono

$$\begin{cases}
    \ce{Cl^{0} + 1e^- -> Cl^+}\\
    \ce{Cl^{0} -> Cl^+ + 1e^-}
\end{cases}$$

\E chiaro che gli elettroni sono già bilanciati. Bilanciamo le cariche.

Al secondo membro abbiamo due cariche negative mentre al primo nessuna, quindi servono due gruppi OH:

$$\ce{Cl_2 + 2OH^- -> Cl^- + ClO^- + H_2O}$$

Tornando alla reazione completa avremo

$$\ce{Cl_2 + 2NaOH -> NaCl + NaClO + H_2O}$$

Il rapporto stechiometrico tra cloro e ipoclorito è 1:1, quindi il numero di moli è lo stesso. Per calcolarle basterà moltiplicare la concentrazione per il volume

$$n_{\rm Cl_2}=n_{\rm NaClO}=c \cdot V=2.0 \cdot 10=20\;mol$$

I grammi saranno allora dati da

$$n=\frac{g}{MM} \implies g=n \cdot MM=20 \cdot 70.906=1.418,12\;grammi$$

\vspace{0.2cm}\textbf{2.14} Calcolare i grammi di solfato ferrico che si ottengano mescolando in acqua 2.3456 grammi di permanganato di potassio  e 12,3456 grammi di solfato ferroso, in eccesso di acido solforico.

\vspace{0.2cm}\large\textbf{Svolgimento}\normalsize

\vspace{0.2cm}La reazione che avviene è

$$\ce{FeSO_4 + KMnO_4 + H_2SO_4 -> Fe_2(SO_4)_3 + MnSO_4 + K_2SO_4 + H_2O}$$

Scriviamo la dissociazione delle varie speci chimiche

$$\ce{FeSO_4 -> Fe^{2+} + SO_4^{2-}}$$
$$\ce{KMnO_4}$$
$$\ce{H_2SO_4}$$
$$\ce{Fe_2(SO_4)_3}$$
$$\ce{MnSO_4}$$
$$\ce{K_2SO_4}$$

\vspace{0.2cm}\textbf{2.15} 

\vspace{0.2cm}\large\textbf{Svolgimento}\normalsize

\vspace{0.2cm}

\vspace{0.2cm}\textbf{2.16} 

\vspace{0.2cm}\large\textbf{Svolgimento}\normalsize

\vspace{0.2cm}

\vspace{0.2cm}\textbf{2.} 