\textbf{4.1} Calcolare il pH di una soluzione ottenuta miscelando 40 mL di $\rm NH_3$ 0.1 M ($k_b = 1.8 \cdot 10^{-5}$) con 5.0 mL di HI 0.3 M.

\vspace{0.2cm}\large\textbf{Svolgimento}\normalsize

\vspace{0.2cm}Calcoliamo innanzitutto le moli:

$$n_{\text{NH}_3}=\frac{0.1 \cdot 40}{1000}=4 \cdot 10^{-3}
\quad ; quad
n_{\text{HI}}=\frac{0.3 \cdot 5}{1000}=1.5 \cdot 10^{-3}$$

Le moli di HI sono in difetto, quindi questo sarà il reagente limitante. Scriviamo la reazione che avviene, mettendo sopra le moli iniziali dei vari composti e sotto quelle finali:

\begin{center}
    \begin{tabular}{ccccccc}
        $4 \cdot 10^{-3}$ &  & $1.5 \cdot 10^{-3}$ & & /\\
        $\rm NH_3$ & + & HI & \ce{->} & $\rm NH_4I$ \\
        $2.5 \cdot 10^{-3}$ &  &  / & & $1.5 \cdot 10^{-3}$\\
    \end{tabular}
\end{center}

Le speci che restano dopo la reazione sono una base debole e un suo sale con acido forte, pertanto ci troviamo davanti ad una soluzione tampone del secondo tipo. Calcoliamo quindi la concentrazione degli ioni $\rm OH^-$, per poi calcolare il pOH e quindi il pH

$$[\text{OH}^-]=k_b\frac{c_b}{c_s}
=1.8 \cdot 10^{-5}\,\frac{2.5 \cdot 10^{-3}}{1.5 \cdot 10^{-3}}= 3 \cdot 10^{-5}$$

$$\rm pOH=\log\left(\frac{1}{3 \cdot 10^{-5}}\right)=4.52$$

$$\implies \rm pH=14-pOH=14-4.52=9.48$$

\vspace{0.2cm}\textbf{4.2} Trovare il pH di un litro di soluzione che contenga 0,2 moli di $\rm CH_3COOH$ e 10 grammi di $\rm CH_3COONa$. $k_a = 1,8 \cdot 10^{-5}$. Come varia il pH aggiungendo 1 g di $\rm H_2SO_4$?

\vspace{0.2cm}\textbf{4.3} A 250 mL di una soluzione 0.4-M di HCN ($k_a = 2.1 \cdot 10^{-9}$) vengono aggiunti 4.000 grammi di NaOH. Calcolare il pH della soluzione.

\vspace{0.2cm}\textbf{4.4} Calcolare il pH della soluzione ottenuta mescolando 400 ml di $\rm NH_4OH$ 0.2 N e 200 ml di HCl 0.15-N ($\rm k_b = 1.8 \cdot 10^{-5}$).

\vspace{0.2cm}\textbf{4.5} Calcolare il pH di una soluzione di 2800 mL ottenuta sciogliendo 15 grammi di acido acetico e 11 grammi di idrossido di sodio.

\vspace{0.2cm}\textbf{4.6} Calcolare il pH della soluzione ottenuta mescolando 100 ml di cloruro di ammonio 0.1-N e 100
ml di acido cloridrico 0.1-N. $k_{b_{\rm NH_3}}=1.8 \cdot 10^{-5}$.

\vspace{0.2cm}\textbf{4.7} Calcolare il pH della soluzione ottenuta sciogliendo in acqua 3.5 g di acido acetico e 2.8 g di idrossido di sodio. Il volume finale della soluzione è di 650 mL.

\vspace{0.2cm}\textbf{4.8} Calcolare il pH di una soluzione $1.000 \cdot 10^{-2}$-M di un acido debole HA monoprotico con costante di dissociazione di $4.25 \rm  10^{-5}$.

\vspace{0.2cm}\textbf{4.9} Quanti grammi di acido acetico sono necessari per preparare mezzo litro di una soluzione a pH 3.50?

\vspace{0.2cm}\textbf{4.10} Calcolare il pH di una soluzione ottenuta diluendo con acqua fino al volume di 500 mL, 2.5 mL
di una soluzione di acido perclorico al 70\% in peso ed avente densità d=1.67 g/mL.

\vspace{0.2cm}\textbf{4.11} Calcolare il pH di una soluzione ottenuta sciogliendo 0,04 moli di morfina ($k_b = 8,0 \cdot 10^{-7}$) in 600 ml di acqua.

\vspace{0.2cm}\textbf{4.12} Calcolare la costante di dissociazione dell’acido ipocloroso sapendo che il pH di una soluzione 0.1-N è 4.26.

\vspace{0.2cm}\textbf{4.13} Calcolare il pH di una soluzione di 900 mL ottenuta sciogliendo 15 grammi di acido acetico e 10 grammi di idrossido di sodio in acqua ($k_a = 1.8 \cdot 10^{-5}$).

\vspace{0.2cm}\textbf{4.14} 2.5 mL di una soluzione di $\rm HClO_4$ (70\%, d=1.67 g/mL) vengono diluiti con acqua fino al
volume di 500 mL. Poi 50 mL di questa soluzione vengono mescolati con 25 mL di una soluzione di KOH 0.1-M. Calcolare il pH di quest’ultima soluzione.

\vspace{0.2cm}\textbf{4.15} A 750 ml di una soluzione di $\rm NH_4OH$ 0.15-N vengono fatti gorgogliare 1.3079 litri di HCl, prelevati a c.n.\,. Calcolare il pH della soluzione.

\vspace{0.2cm}\textbf{4.16} Calcolare il pH di una soluzione 0.35-M di NaCN ($k{a_{\rm HCN}} = 2.1 \cdot 10^{-9}$).

\vspace{0.2cm}\textbf{4.17} Calcolare il pH di una soluzione ottenuta mescolando 100 mL di soluzione di acido acetico 0,1-M ($k_a = 1,8 \cdot 10^{-5}$) con 1 mL di HCl 1-M.

\vspace{0.2cm}\textbf{4.18}  In una soluzione 1,1-M di HCN, l’acido è dissociato per lo 0,0019\% a 25 °C. Trovare la concentrazione degli ioni $\rm H_3O^+$, il pH della soluzione e la costante di ionizzazione dell’acido. 

\vspace{0.2cm}\textbf{4.19}  Calcolare quanti mL di una soluzione di NaOH 0,2-M, devono essere mescolati con
100 mL di una soluzione di $\rm CH_3COOH$ 0,2-M, affinché il pH della soluzione risultante sia 4,05. $k_{a_{\rm CH3COOH}}= 1,8 \cdot 10^{-5}$.

\vspace{0.2cm}\textbf{4.20} 3.504 gr di ammoniaca acquosa vengono sciolti in 870 mL di soluzione finale e di questa ne vengono prelevati 250 mL. A questa aliquota di soluzione vengono aggiunti 2.8868 grammi di acido perclorico. Calcolare il pH sapendo che $k_{b_{\rm NH_3}}=1.8 \cdot 10^{-5}$.

\vspace{0.2cm}\textbf{4.21} Calcolare il pH di una soluzione di 987 mL contenente 6.5432 gr di $\rm NH_4Cl$ e 3.9997 gr di NaOH.

\vspace{0.2cm}\textbf{4.22} 0.4 L di HCl gassoso prelevato a c.n. vengono fatti gorgogliare in 325 mL di una soluzione acquosa di ammoniaca 0.1-N senza notare variazione del volume finale. Calcolare il pH della soluzione e come esso varia quando 10 mL di NaOH 0,1 M vengono mescolati ad essa.