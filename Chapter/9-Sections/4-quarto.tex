\begin{esercizio}
    Calcolare il pH di una soluzione ottenuta miscelando 40 mL di $\rm NH_3$ 0.1 M ($K_b = 1.8 \cdot 10^{-5}$) con 5.0 mL di HI 0.3 M.
\end{esercizio}
\begin{soluzione}
    Calcoliamo innanzitutto le moli:
    \begin{equation*}
        n_{\text{NH}_3}=\frac{0.1 \cdot 40}{1000}
        =4 \cdot 10^{-3} \; \text{mol}
        \quad ; \quad
        n_{\text{HI}}=\frac{0.3 \cdot 5}{1000}
        =1.5 \cdot 10^{-3} \;\text{mol}
    \end{equation*}
    Le moli di HI sono in difetto, quindi questo sarà il reagente limitante. Scriviamo la reazione che avviene, mettendo sopra le moli iniziali dei vari composti e sotto quelle finali:
    \begin{center}
        \begin{tabular}{ccccccc}
            $4 \cdot 10^{-3}$ & & $1.5 \cdot 10^{-3}$ & & /\\
            $\rm NH_3$ & + & HI & \ce{->} & $\rm NH_4I$ \\
            $2.5 \cdot 10^{-3}$ & & / & & $1.5 \cdot 10^{-3}$\\
        \end{tabular}
    \end{center}
    Le specie che restano dopo la reazione sono una base debole e un suo sale con acido forte, pertanto ci troviamo davanti ad una soluzione tampone del secondo tipo. Calcoliamo quindi la concentrazione degli ioni $\rm OH^-$, per poi calcolare il pOH e quindi il pH
    \begin{gather*}
        [\text{OH}^-]=K_b\frac{C_b}{C_s}
        =1.8 \cdot 10^{-5}\,\frac{2.5 \cdot 10^{-3}}{1.5 \cdot 10^{-3}}
        = 3 \cdot 10^{-5} \; \rm mol/L
        \\[0.1cm]
        \rm pOH=\log\left(\frac{1}{3 \cdot 10^{-5}}\right)=4.52
        \\[0.1cm]
        \implies \rm pH=14-pOH=14-4.52=9.48
    \end{gather*}
\end{soluzione}

\newpage

\begin{esercizio}
    Trovare il pH di un litro di soluzione che contenga 0.2 moli di $\rm CH_3COOH$ e 10 grammi di $\rm CH_3COONa$. $K_a = 1.8 \cdot 10^{-5}$. Come varia il pH aggiungendo 1 g di $\rm H_2SO_4$?
\end{esercizio}
\begin{soluzione}
    Calcoliamo innanzitutto le moli di sale:

$$n_{\rm CH_3COONa}
=\frac{g}{MM_{\rm CH_3COONa}}
=\frac{10}{82.0343}
=0.1219 \; \rm mol$$

Poiché in soluzione abbiamo un acido debole ed un suo sale con base forte ci troviamo davanti ad una soluzione tampone di primo tipo. Il pH sarà allora dato da

$$[\text{H}_3\text{O}^+]
=K_a \frac{C_a}{C_s}
=1,8 \cdot 10^{-5} \frac{0.2}{0.1219}
=2.9532 \cdot 10^{-5} \; \rm mol/L$$

$$\implies \rm pH
=\log{\left( \frac{1}{2.9532 \cdot 10^{-5}} \right)}
=4.53$$

Vediamo ora come varia il pH dopo l'aggiunta di acido solforico. Calcoliamone le moli:

$$n_{\rm H_2SO_4}
=\frac{1}{98.079}
=1.0196 \cdot 10^{-2} \; \rm mol$$

Avendo 1 litro di soluzione, la concentrazione dell'acido solforico è pari a $1.0196 \cdot 10^{-2} \; \rm mol/L$

Quando aggiungiamo un acido ad una soluzione tampone del primo tipo il pH è dato da

$${[\text{H}_3\text{O}^+]}=\textit{K}_\textit{a} \frac{C_a + \rm{[\text{H}_3\text{O}^+]}}{C_s - \rm{[\text{H}_3\text{O}^+]}}
=1,8 \cdot 10^{-5} \frac{0.2 + 1.0196 \cdot 10^{-2}}{0.1219 - 1.0196 \cdot 10^{-2}}
=3.3871 \cdot 10^{-5} \; \rm mol/L$$

$$\implies \rm pH=\log{\left( \frac{1}{3.3871 \cdot 10^{-5}} \right)}=4.47$$

Il pH è variato solo di 0.06, il che è coerente col fatto che si tratta di una soluzione tampone.
\end{soluzione}

\newpage

\begin{esercizio}
    A 250 mL di una soluzione 0.4 M di HCN ($K_a = 2.1 \cdot 10^{-9}$) vengono aggiunti 4.000 grammi di NaOH. Calcolare il pH della soluzione.
\end{esercizio}
\begin{soluzione}

Calcoliamo innanzitutto le moli.

Per le moli di HCN facciamo una proporzione (si suppone che non ci sia variazione di volume dopo l'aggiunta di NaOH):

$$n_{\rm HCN}:250=0.4:1000
\implies
n_{\rm HCN}=\frac{250 \cdot 0.4}{1000}=0.1 \; \rm mol$$

Per le moli di NaOH invece abbiamo

$$n_{\rm NaOH}=\frac{g}{MM_{\rm NaOH}}
=\frac{4.000}{39.997} \approx 0.1 \; \rm mol$$

Scriviamo la reazione con le concentrazioni iniziali sopra e quelle finali sotto:

\begin{center}
    \begin{tabular}{ccccccc}
        $0.1$ &  & $0.1$ & & / &&\\
        $\rm NaOH$ & + & HCN & \ce{->} & $\rm NaCN$ & + & $\rm H_2O$\\
        / &  &  / & & $0.1$ &&\\
    \end{tabular}
\end{center}

Poiché le moli dei reagenti sono uguali, i due composti si neutralizzano totalmente dando 0.1 moli di cianato di sodio. Quest'ultimo è un sale, che in acqua si trova dissociato in ione $\rm Na^+$ e ione $\rm CN^-$:

$$\ce{NaCN -> Na^+ + CN^-}$$

Mentre lo ione $\rm Na^+$ non dà luogo ad ulteriori reazioni, lo ione $\rm CN^-$ reagisce con una molecola d'acqua e dà un equilibrio:

$$\ce{CN^- + H_2O <--> HCN + OH^-}$$

Siamo quindi davanti ad una reazione di idrolisi basica. La concentrazione degli ioni $\rm OH^-$ sarà data da

$$[\text{OH}^-]
=\sqrt{\frac{K_w}{K_a}\cdot C_s}
=\sqrt{\frac{10^{-14}}{2.1 \cdot 10^{-9}} \cdot 0.4}
=1.3801 \cdot 10^{-3} \; \rm mol/L$$

$$\implies \rm pOH
=\log{\left(\frac{1}{1.3801 \cdot 10^{-3}}\right)}
=2.86$$

$$\implies \rm pH=14-pOH=14-2.86=11.14$$
\end{soluzione}

\newpage

\begin{esercizio}
    Calcolare il pH della soluzione ottenuta mescolando 400 ml di $\rm NH_4OH$ 0.2 N e 200 ml di HCl 0.15 N ($K_b = 1.8 \cdot 10^{-5}$).
\end{esercizio}
\begin{soluzione}
    Calcoliamo innanzitutto le moli. Per entrambe le specie normalità e molarità coincidono, in quanto l'$\rm NH_4OH$ cede uno ione $\rm OH^-$ e l'HCl cede uno ione $\rm H^+$, per cui:

$$n_{\text{NH}_4\text{OH}}=\frac{0.2 \cdot 400}{1000}=8 \cdot 10^{-2} \; \text{mol}
\quad ; \quad
n_{\text{HCl}}=\frac{0.15 \cdot 200}{1000}=3 \cdot 10^{-2} \; \text{mol}$$

Nota: il composto $\rm NH_4OH$ non è altro che l'addotto $\rm NH_3 \cdot H_2O$.

Scriviamo la reazione con le concentrazioni iniziali sopra e quelle finali sotto:

\begin{center}
    \begin{tabular}{ccccccc}
        $8 \cdot 10^{-2}$ &  & $3 \cdot 10^{-2}$ & & / &&\\
        $\rm NH_4OH$ & + & HCl & \ce{->} & $\rm NH_4Cl$ & + & $\rm H_2O$\\
        $5 \cdot 10^{-2}$ &  &  / & & $3 \cdot 10^{-2}$&&\\
    \end{tabular}
\end{center}

A reazione finita abbiamo una base debole più un suo sale con acido forte, cioè una soluzione tampone del secondo tipo. La concentrazione degli ioni $\rm OH^-$ sarà data da

$$[\text{OH}^-]=K_b\frac{C_b}{C_s}
=1.8 \cdot 10^{-5}\,\frac{5 \cdot 10^{-2}}{3 \cdot 10^{-2}}
= 3 \cdot 10^{-5} \; \rm mol/L$$

$$\rm pOH=\log\left(\frac{1}{3 \cdot 10^{-5}}\right)=4.52$$

$$\implies \rm pH=14-pOH=14-4.52=9.48$$
\end{soluzione}

\newpage

\begin{esercizio}
    Calcolare il pH della soluzione ottenuta mescolando 100 mL di cloruro di ammonio 0.1 N e 100 ml di acido cloridrico 0.1 N. $K_{b_{\rm NH_3}}=1.8 \cdot 10^{-5}$.
\end{esercizio}
\begin{soluzione}
    Analizziamo le singole specie.

L'HCl è un acido forte, per cui in acqua si dissocia totalmente in ione $\rm Cl^-$ e ione $\rm H_3O^+$:

$$\ce{HCl + H_2O -> H_3O^+ + Cl^-}$$

Per il cloruro di ammonio abbiamo invece

$$\ce{NH_4Cl -> NH_4^+ + Cl^-}$$

Mentre lo ione $\rm Cl^-$ non produce ulteriori reazioni, lo ione $\rm NH_4^+$ instaura un equilibrio di idrolisi, producendo ammoniaca e ioni $\rm H_3O^+$:

$$\ce{NH_4^+ + H_2O <--> NH_3 + H_3O^+}$$

In soluzione sono presenti già parecchi ioni $\rm H_3O^+$ provenienti dalla dissociazione dell'acido forte, per cui tale equilibrio è spostato totalmente a sinistra e di fatto la reazione non avviene.

Calcoliamo adesso le moli di HCl.

L'acido cloridrico è una specie monoprotica, per cui molarità e normalità coincidono. Facciamo quindi la proporzione

$$n_{\rm HCl}:100=0.1:1000
\implies
n_{\rm HCl}=\frac{100 \cdot 0.1}{1000}=1 \cdot 10^{-2} \; \rm mol$$

Avendo le moli possiamo calcolare la concentrazione degli ioni $\rm H_3O^+$:

$$[\rm H_3O^+]=\frac{1 \cdot 10^{-2} \cdot 1000}{200}=5 \cdot 10^{-2} \; mol/L$$

In quanto si producono un numeri di moli di $\rm H_3O^+$ pari al numero di moli di acido.

Il pH sarà dato da

$$\rm pH=\log\left(\frac{1}{5 \cdot 10^{-2}}\right)=1.30$$
\end{soluzione}

\newpage

\begin{esercizio}
    Calcolare il pH della soluzione ottenuta sciogliendo in acqua 3.5 g di acido acetico e 2.8 g di idrossido di sodio. Il volume finale della soluzione è di 650 mL.
\end{esercizio}
\begin{soluzione}
    Calcoliamo innanzitutto le moli:

$$n_{\rm CH_3COOH}=\frac{g}{MM_{\rm CH_3COOH}}=\frac{3.5}{60.052}=5.8283 \cdot 10^{-2} \; \rm mol$$

$$n_{\rm NaOH}=\frac{2.8}{39.997}=7.0005 \cdot 10^{-2} \; \rm mol$$

Le moli di acido acetico sono in difetto, quindi sarà il reagente limitante.

Scriviamo la reazione con le concentrazioni iniziali sopra e quelle finali sotto:

\begin{center}
    \begin{tabular}{ccccccc}
        $5.8283 \cdot 10^{-2}$ &  & $7.0005 \cdot 10^{-2}$ & & / &&\\
        $\rm CH_3COOH$ & + & NaOH & \ce{->} & $\rm CH_3COONa$ & + & $\rm H_2O$\\
        / & & $1.1722 \cdot 10^{-2}$ & & $5.8283 \cdot 10^{-2}$&&\\
    \end{tabular}
\end{center}

A fine reazione abbiamo una base forte che è l'idrossido di sodio e un sale che è l'acetato di sodio, ottenuto a partire da un acido debole.

Analizziamo le singole specie.

L'idrossido di sodio è una base forte, per cui si dissocia totalmente in $\rm Na^+$ e $\rm OH^-$:

$$\ce{NaOH -> Na^+ + OH^-}$$

La concentrazione degli ioni $\rm OH^-$ derivanti da tale specie è pari a

$$[\rm OH^-]=\frac{1.1722 \cdot 10^{-2} \cdot 1000}{650}=1.8034 \cdot 10^{-2} \; \rm mol/L$$

L'acetato di sodio in acqua si dissocia in ione $\rm Na^+$ e ione acetato $\rm CH_3COO^-$; quest'ultimo dà luogo a una reazione di equilibrio di idrolisi basica:

$$\ce{CH_3COO^- + H_2O <--> CH_3COOH + OH^-}$$

Tuttavia, in soluzione sono già presenti molti ioni $\rm OH^-$ provenienti dalla dissociazione della base forte, per cui l'equilibrio sarà totalmente spostato a sinistra e di fatto tale reazione non avviene.

Il pOH sarà allora dato solo dalla dissociazione della base forte:

$$\rm pOH=\log{\left(\frac{1}{1.8034 \cdot 10^{-2}}\right)}=5.15$$

$$\implies \rm pH=14-pOH=14-5.15=8.85$$

Nota: il testo non fornisce la costante di dissociazione dell'acido acetico in quanto è superfluo calcolare la concentrazioni degli ioni $\rm OH^-$ provenienti dal sale, poiché ogni volta che in soluzione abbiamo una specie forte e una debole (e questo include anche i sali che danno luogo a reazioni di idrolisi), prevale la specie forte e quindi il pH sarà determinato unicamente da questa.
\end{soluzione}

\newpage

\begin{esercizio}
    Calcolare il pH di una soluzione di 2800 mL ottenuta sciogliendo 15 grammi di acido acetico e 11 grammi di idrossido di sodio.
\end{esercizio}
\begin{soluzione}
    Calcoliamo innanzitutto le moli
    \begin{gather*}
        n_{\text{acido}}=\frac{g}{MM_{\text{acido}}}=\frac{15}{60.052}=2.4978 \cdot 10^{-1} \; \rm mol
        \\
        n_{\text{NaOH}}=\frac{g}{MM_{\text{NaOH}}}=\frac{11}{39.997}=2.7502 \cdot 10^{-1} \; \rm mol
    \end{gather*}
    Scriviamo la reazione con le concentrazioni iniziali sopra e quelle finali sotto:
    \begin{center}
        \begin{tabular}{ccccccc}
            $2.4978 \cdot 10^{-1}$ &  & $2.7502 \cdot 10^{-1}$ & & / &&\\
            $\rm CH_3COOH$ & + & NaOH & \ce{->} & $\rm CH_3COONa$ & + & $\rm H_2O$\\
            / &  &  $2.524 \cdot 10^{-2}$ & & $2.4978 \cdot 10^{-1}$&&\\
        \end{tabular}
    \end{center}
    A fine reazione, come nell'esercizio precedente, abbiamo una base forte e e sale ottenuto a partire da un acido debole. Per calcolare il pH sarà quindi sufficiente considerare la concentrazione degli ioni $\rm OH^-$ prodotti dalla base forte:
    \begin{gather*}
        [\rm OH^-]=\frac{2.524 \cdot 10^{-2} \cdot 1000}{2800}
        =9.0143 \cdot 10^{-4} \; \rm mol/L
        \\
        \rm pOH
        =\log{\left(\frac{1}{9.0143 \cdot 10^{-3}}\right)}
        =2.04
        \\
        \implies \rm pH=14-pOH=14-2.04=11.96
    \end{gather*}
\end{soluzione}

\newpage

\begin{esercizio}
    Calcolare il pH di una soluzione $1.000 \cdot 10^{-2}$ M di un acido debole HA monoprotico con costante di dissociazione di $4.25 \cdot 10^{-5}$.
\end{esercizio}
\begin{soluzione}
    Nel caso di acidi deboli, la concentrazione degli ioni $\text{H}_3\text{O}^+$ è data da
    \begin{equation*}
        [\text{H}_3\text{O}^+]
        =\sqrt{K_a \cdot C_a}
        =\sqrt{4.25 \cdot 10^{-5} \cdot 1.000 \cdot 10^{-2}}
        =6.5192 \cdot 10^{-4} \; \rm mol/L
    \end{equation*}
    da cui segue che il pH sarà
    \begin{equation*}
        \rm pH=\log{\left(\frac{1}{6.5192 \cdot 10^{-4}}\right)}\approx 3.19
    \end{equation*}
\end{soluzione}

\newpage

\begin{esercizio}
    Quanti grammi di acido acetico sono necessari per preparare mezzo litro di una soluzione a pH 3.50?
\end{esercizio}
\begin{soluzione}
    L'acido acetico è un acido debole monoprotico, cioè dissociandosi cede un solo ione $\rm H^+$. Sappiamo che
    \begin{equation*}
        [\text{H}_3\text{O}^+] = \sqrt{K_a \cdot C_a}
    \implies
    [\text{H}_3\text{O}^+]^2 = K_a \cdot C_a
    \implies
    C_a=\frac{[\text{H}_3\text{O}^+]^2}{K_a}
    \end{equation*}
    d'altro canto, sappiamo che
    \begin{equation*}
        \rm pH=\log{\left(\frac{1}{[H_3O^+]}\right)}
        \implies
        10^{pH}=\frac{1}{[H_3O^+]}
        \implies
        [H_3O^+]=\frac{1}{10^{pH}}
    \end{equation*}
    Da cui avremo
    \begin{gather*}
        \rm [H_3O^+]=\frac{1}{10^{3.5}}=3.1623 \cdot 10^{-4} \; \rm mol/L
        \\
        \implies
        C_a=\frac{(3.1623 \cdot 10^{-4})^2}{1.8 \cdot 10^{-5}}=5.6921 \cdot 10^{-2}
        =5.5556 \cdot 10^{-3} \; \rm mol/L
    \end{gather*}
    Siccome abbiamo mezzo litro di soluzione, avremo $2.7778 \cdot 10^{-3}$ moli. I grammi si calcoleranno come
    \begin{equation*}
        g=n_{\text{acido}} \cdot MM_{\text{acido}}
        =2.7778 \cdot 10^{-3} \cdot 60.052
        =0.1668 \; \rm grammi
    \end{equation*}
\end{soluzione}

\newpage

\begin{esercizio}
    Calcolare il pH di una soluzione ottenuta diluendo con acqua fino al volume di 500 mL, 2.5 mL di una soluzione di acido perclorico al 70\% in peso ed avente densità $d=1.67 \; \rm g/mL$.
\end{esercizio}
\begin{soluzione}
    Per prima cosa calcoliamo i grammi di acido perclorico. I $2.5 \; \rm mL$ iniziali di soluzione, corrispondono alla massa

$$g=V \cdot d=2.5 \cdot 1.67=4.175 \; \rm grammi$$

Di questi, solo il 70\% è costituito da acido perclorico, per cui

$$g_{\rm HClO_4}
=\frac{70 \cdot 4.175}{100}
=2.9225 \; \rm grammi$$

A questo punto calcoliamo le moli:

$$n_{\rm HClO_4}
=\frac{g}{MM_{\rm HClO_4}}
=\frac{2.9225}{100.46}=2.9091 \cdot 10^{-2} \; \rm mol$$

L'acido perclorico è una specie forte che in acqua si dissocia in ione $\rm H_3O^+$ e ione perclorato $\rm ClO_4^-$:

$$\ce{HClO_4 + H_2O -> ClO_4^- + H_3O^+}$$

Essendo totalmente dissociato, la concentrazione degli ioni $\rm H_3O^+$ sarà data da

$$[\text{H}_3\text{O}^+]=\frac{n}{V(L)}=\frac{2.9091 \cdot 10^{-2} \cdot 10^{3}}{500}=5.8182 \cdot 10^{-2} \; \rm mol/L$$

$$\implies \rm pH=\log{\left( \frac{1}{5.8182 \cdot 10^{-2}} \right)}=1.23$$
\end{soluzione}

\newpage

\begin{esercizio}
    Calcolare il pH di una soluzione ottenuta sciogliendo 0.04 moli di morfina ($K_b = 8.0 \cdot 10^{-7}$) in 600 ml di acqua.
\end{esercizio}
\begin{soluzione}
    Dal valore della costante di equilibrio deduciamo che si tratta di una base debole ($K<1$), per cui la concentrazione degli ioni $\rm OH^-$ sarà data da

$$[\text{OH}^-]=\sqrt{K_b \cdot C_b}$$

Calcoliamo la concentrazione della base:

$$C_b=\frac{n}{V(L)}=
\frac{0.04 \cdot 1000}{600}=6.6667 \cdot 10^{-2} \; \rm mol/L$$

da cui

$$\rm [OH^-]=\sqrt{8.0 \cdot 10^{-7} \cdot 6.6667 \cdot 10^{-2}}
=2.3094 \cdot 10^{-4} \; \rm mol/L$$

$$\implies \rm pOH=\log{\left(\frac{1}{2.3094 \cdot 10^{-4}}\right)}
=3.64$$

$$\implies \rm pH=14 - pOH
=14-3.64=10.36$$

\end{soluzione}

\newpage

\begin{esercizio}
    Calcolare la costante di dissociazione dell'acido ipocloroso sapendo che il pH di una soluzione 0.1 N è 4.26.
\end{esercizio}
\begin{soluzione}
    Dal valore di pH della soluzione deduciamo che si tratta di un acido debole, per cui la concentrazione degli ioni $\rm H_3O^+$ sarà data da:

$$[\text{H}_3\text{O}^+]=\sqrt{K_a \cdot C_a}
\implies
K_a=\frac{[\rm H_3O^+]^2}{C_a}$$

D'altra parte si ha che

$$\rm pH=\log{\left( \frac{1}{[H_3O^+]} \right)}
\implies
10^{pH}=\frac{1}{[H_3O^+]}
\implies
[H_3O^+]=\frac{1}{10^{pH}}$$

$$\rm [H_3O^+]=\frac{1}{10^{4.26}}=5.4954 \cdot 10^{-5} \; \rm mol/L$$

L'HClO è un acido monoprotico, per cui normalità e molarità coincidono. Avremo quindi che

$$K_a=\frac{(5.4954 \cdot 10^{-5})^2}{0.1}
=3.0199 \cdot 10^{-8}$$
\end{soluzione}

\newpage

\begin{esercizio}
    Calcolare il pH di una soluzione di 900 mL ottenuta sciogliendo 15 grammi di acido acetico e 10 grammi di idrossido di sodio in acqua ($K_a = 1.8 \cdot 10^{-5}$).
\end{esercizio}
\begin{soluzione}
    \textbf{Disclaimer}: Siamo fortemente dubbiosi circa la seguente risoluzione. Suggeriamo di chiederlo ad un ricevimento ed in caso comunicarci la soluzione.

\vspace{0.2cm}Calcoliamo innanzitutto le moli

$$n_{\rm CH_3COOH}=\frac{15}{60.052}=0.2498 \; \text{mol}
\quad;\quad
n_{\rm NaOH}=\frac{10}{39.997}=0.2500 \; \text{mol}$$

Le moli di acido sono in difetto, quindi esso sarà il reagente limitante. Scriviamo la reazione che avviene, mettendo sopra le quantità iniziali e sotto quelle finali:

\begin{center}
    \begin{tabular}{ccccccc}
        $0.2498$ &  & $0.2500$ & & / &&\\
        $\rm CH_3COOH$ & + & $\rm NaOH$ & \ce{->} & $\rm CH_3COONa$ & + & $\rm H_2O$\\
        / &  & $2 \cdot 10^{-4}$ & & $0.2498$ &&\\
    \end{tabular}
\end{center}

Analizziamo i composti che abbiamo a fine reazione: abbiamo acetato di sodio, che si dissocia in ione $\rm Na^+$ che non dà luogo ad ulteriori reazioni e in ione $\rm CH_3COO^-$ che invece instaura un equilibrio di idrolisi basica, la cui concentrazione degli ioni $\rm OH^-$ è data da

$$[\text{OH}^-]
=\sqrt{\frac{K_w}{K_a}\cdot C_s}
=\sqrt{\frac{10^{-14}}{1.8 \cdot 10^{-5}} \cdot \frac{0.2498 \cdot 1000}{900}}
=1.2417 \cdot 10^{-5} \; \rm mol/L$$

Abbiamo poi dell'NaOH in eccesso che è una base forte, per cui la concentrazione degli ioni $\rm OH^-$ provenienti da esso sarà pari a

$$[\text{OH}^-]=\frac{n_{\rm NaOH}}{V(\text{L})}
=\frac{2 \cdot 10^{-4} \cdot 1000}{900}=2.2222 \cdot 10^{-4} \; \rm mol/L$$

La concentrazione totale sarà dato dalla somma delle concentrazioni:

$$\rm [OH^-]_{tot}=2.2222 \cdot 10^{-4} + 1.2417 \cdot 10^{-5}
=2.3462 \cdot 10^{-4} \; mol/L$$

da cui

$$\rm pOH=\log{\left(\frac{1}{[OH^-]}\right)}
=\log{\left(\frac{1}{2.3462 \cdot 10^{-4}}\right)}=3.63$$

$$\implies \rm pH=14-pOH=14-3.63=10.37$$

Nota: negli esercizi precedenti abbiamo detto che quando abbiamo una specie forte e una debole possiamo trascurare la concentrazione della seconda specie. In questo caso però non è stato possibile perché abbiamo pochissime moli della specie forte, e in conseguenza a ciò le concentrazioni degli ioni $\rm OH^-$ sono confrontabili.

\end{soluzione}

\newpage

\begin{esercizio}
    2.5 mL di una soluzione di $\rm HClO_4$ (70\%, d=1.67 g/mL) vengono diluiti con acqua fino al volume di 500 mL. Poi 50 mL di questa soluzione vengono mescolati con 25 mL di una soluzione di KOH 0.1 M. Calcolare il pH di quest'ultima soluzione.
\end{esercizio}
\begin{soluzione}
    Per prima cosa calcoliamo i grammi di acido perclorico. I $2.5$ mL iniziali di soluzione, corrispondono alla massa

$$g=V \cdot d=2.5 \cdot 1.67=4.175 \; \rm grammi$$

Di questi, solo il 70\% è costituito da acido perclorico, per cui

$$g_{\rm HClO_4}=\frac{70 \cdot 4.175}{100}=2.9225 \; \rm grammi$$

A questo punto calcoliamo le moli:

$$n_{\rm HClO_4}
=\frac{g}{MM_{\rm HClO_4}}
=\frac{2.9225}{100.46}=2.9091 \cdot 10^{-2} \; \rm mol$$

L'acido perclorico è una specie forte che in acqua si dissocia in ione $\rm H_3O^+$ e ione perclorato $\rm ClO_4^-$:

$$\ce{HClO_4 + H_2O -> ClO_4^- + H_3O^+}$$

Essendo totalmente dissociato, la concentrazione degli ioni $\rm H_3O^+$ sarà data da

$$[\text{H}_3\text{O}^+]=\frac{n}{V(L)}=\frac{2.9091 \cdot 10^{-2} \cdot 10^{3}}{500}=5.8182 \cdot 10^{-2} \; \rm mol/L$$

Tale quantità sarà la molarità $M$ della soluzione. Tramite questa vediamo quante moli abbiamo in 50 mL:

$$n:50=5.8182 \cdot 10^{-2}:1000
\implies
n=\frac{5.8182 \cdot 10^{-2} \cdot 50}{1000}
=2.9091 \cdot 10^{-3} \; \rm mol$$

Calcoliamo poi le moli di KOH tramite la proporzione

$$n:25=0.1:1000
\implies
n=\frac{0.1 \cdot 25}{1000}
=2.5 \cdot 10^{-3} \; \rm mol$$

Scriviamo la reazione che avviene, mettendo sopra le quantità iniziali e sotto quelle finali:

\begin{center}
    \begin{tabular}{ccccccc}
        $2.9091 \cdot 10^{-3}$ &  & $2.5 \cdot 10^{-3}$ & & / &&\\
        $\rm HClO_4$ & + & $\rm KOH$ & \ce{->} & $\rm KClO_4$ & + & $\rm H_2O$\\
        $4.091 \cdot 10^{-4}$ &  & / & & $2.5 \cdot 10^{-3}$ &&\\
    \end{tabular}
\end{center}

A fine reazione abbiamo perclorato di potassio e acido perclorico. Il perclorato di potassio deriva da un acido e una base forti e quindi non dà luogo ad altre reazioni, pertanto il pH sarà determinato solo dall'acido perclorico in eccesso. Essendo un acido forte, la concentrazione degli ioni $\rm H_3O^+$ è data dal

$$[\text{H}_3\text{O}^+]=\frac{n_{\rm HClO_4}}{V_{tot}(L)}=\frac{4.091 \cdot 10^{-4} \cdot 1000}{75}=5.4547 \cdot 10^{-3} \; \rm mol/L$$

$$\implies \rm pH=\log{\left(\frac{1}{5.4547 \cdot 10^{-3}}\right)}=2.26$$
\end{soluzione}

\newpage

\begin{esercizio}
    A 750 ml di una soluzione di $\rm NH_4OH$ 0.15 N vengono fatti gorgogliare 1.3079 litri di HCl, prelevati a c.n.\,. Calcolare il pH della soluzione.
\end{esercizio}
\begin{soluzione}
    Calcoliamo le moli.

Le moli di $\rm NH_4OH$ si calcolano dalla proporzione

$$n:750=0.15:1000
\implies
n=\frac{750 \cdot 0.15}{1000}
=0.1125 \; \rm mol$$

Le moli di HCl invece saranno date dall'equazione di stato dei gas:

$$PV=nRT \implies n=\frac{PV}{RT}
=\frac{1 \cdot 1.3079}{0.082 \cdot 273.15}
=0.0584 \; \rm mol$$

Le moli di HCl sono in difetto, quindi esso sarà il reagente limitante.

Scriviamo la reazione che avviene, mettendo sopra le quantità iniziali e sotto quelle finali:

\begin{center}
    \begin{tabular}{ccccccc}
        $0.1125$ &  & $0.0584$ & & / &&\\
        $\rm NH_4OH$ & + & $\rm HCl$ & \ce{->} & $\rm NH_4Cl$ & + & $\rm H_2O$\\
        $0.0541$ &  & / & & $0.0584$ &&\\
    \end{tabular}
\end{center}

A fine reazione abbiamo una base debole e un suo sale con acido forte, quindi ci troviamo davanti ad una soluzione tampone di secondo tipo. La concentrazione degli ioni $\rm OH^-$ sarà data da

$$[\text{OH}^-]=K_b\,\frac{C_b}{C_s}
=1.8 \cdot 10^{-5}\,\frac{0.0541}{0.0584}
=1.6675 \cdot 10^{-5} \; \rm mol/L$$

$$\rm pOH=\log \left( \frac{1}{1.6675 \cdot 10^{-5}} \right)=4.78$$

$$\implies \rm pH=14-pOH=14-4.78=9.22$$
\end{soluzione}

\newpage

\begin{esercizio}
    Calcolare il pH di una soluzione 0.35 M di NaCN ($K_{a_{\rm HCN}} = 2.1 \cdot 10^{-9}$).
\end{esercizio}
\begin{soluzione}
    Supponiamo di avere 1 litro di soluzione, quindi avremo 0.35 moli. Il cianato di sodio in acqua si dissocia in ione $\rm Na^+$ ione cianato $\rm CN^-$. Mentre il primo non dà luogo ad ulteriori reazioni, il secondo, essendo la base coniugata di un acido debole, dà luogo all'equilibrio

$$\ce{CN^- + H_2O <--> HCN + OH^-}$$

il pH sarà quindi basico, in quanto siamo davanti ad una reazione di idrolisi. La concentrazione degli ioni $\rm OH^-$ sarà data da


$$[\text{OH}^-]
=\sqrt{\frac{K_w}{K_a}\cdot C_s}
=\sqrt{\frac{10^{-14}}{2.1 \cdot 10^{-9}} \cdot 0.35}
=1.2910 \cdot 10^{-3} \; \rm mol/L$$

$$\implies \rm pOH
=\log{\left(\frac{1}{1.2910 \cdot 10^{-3}}\right)}
=2.89$$

$$\implies \rm pH=14-pOH=14-2.89=11.11$$
\end{soluzione}

\newpage

\begin{esercizio}
    Calcolare il pH di una soluzione ottenuta mescolando 100 mL di soluzione di acido acetico 0.1 M ($K_a=1.8 \cdot 10^{-5}$) con 1 mL di HCl 1 M.
\end{esercizio}
\begin{soluzione}
    Stiamo mescolando un acido forte e un acido debole per cui, dato che entrambi in acqua danno luogo ad ioni $\rm H_3O^+$, per calcolare il pH dobbiamo sommare le concentrazioni.

Calcoliamo innanzitutto le moli:

$$n_{\rm CH_3COOH}=\frac{0.1 \cdot 100}{1000}=1 \cdot 10^{-2} \; \text{mol}
\quad;\quad
n_{\rm HCl}=\frac{1 \cdot 1}{1000}=1 \cdot 10^{-3} \; \rm mol$$

La concentrazione degli ioni $\rm H_3O^+$ provenienti dall'HCl saranno dati, nel volume finale di $101 \; \rm mL$ da

$$[\text{H}_3\text{O}^+]=\frac{n_{\rm HCl}}{V}
=\frac{1 \cdot 10^{-3} \cdot 1000}{101}
=9.9010\cdot 10^{-3} \; \rm mol/L$$

La concentrazioni di quelli derivanti dal $\rm CH_3COOH$ invece saranno dati da

$$[\text{H}_3\text{O}^+]=\sqrt{K_a \cdot C_a}
=\sqrt{1.8 \cdot 10^{-5}\cdot \frac{1 \cdot 10^{-2} \cdot 1000}{101}}
=1.3350 \cdot 10^{-3} \; \rm mol/L$$

La concentrazione totale sarà data da

$$[\rm H_3O^+]=1.3350 \cdot 10^{-3} + 9.9010\cdot 10^{-3}
=1.1236 \cdot 10^{-2} \; \rm mol/L$$

$$\implies
\rm pH=\log{\left(\frac{1}{1.1236 \cdot 10^{-2}}\right)}
=1.95$$
\end{soluzione}

\newpage

\begin{esercizio}
    In una soluzione 1.1 M di HCN, l'acido è dissociato per lo 0.0019\% a $25 \; ^{\circ} \rm C$. Trovare la concentrazione degli ioni $\rm H_3O^+$, il pH della soluzione e la costante di ionizzazione dell'acido.
\end{esercizio}
\begin{soluzione}
    Innanzitutto calcoliamo la concentrazione degli ioni $\rm H_3O^+$. L'acido cianidrico si dissocia in ione cianato e ione $\rm H_3O^+$, ma essendo un acido debole non si dissocerà del tutto. Per calcolarla partiamo dal grado di dissociazione, il quale è definito come il numero di moli dissociate fratto il numero di moli totali, da cui

$$\alpha=\frac{n_{\rm dissociate}}{n_{\rm totali}}
\implies
n_{\rm dissociate}=\alpha \cdot n_{\rm totali}$$

Supponiamo di partire da un litro di soluzione, allora avremo 1.1 moli di acido; inoltre si ha che $\alpha=1.9 \cdot 10^{-5}$, quindi

$$n_{\rm dissociate}=1.9 \cdot 10^{-5} \cdot 1.1
=2.09 \cdot 10^{-5} \; \rm mol$$

ed avendo supposto di avere un litro di soluzione questa sarà anche la concentrazione degli ioni $\rm H_3O^+$.

Avendo tale quantità possiamo calcolare direttamente il pH:

$$\rm pH=\log{\left(\frac{1}{H_3O^+}\right)}
=\log{\left(\frac{1}{2.09 \cdot 10^{-5}}\right)}=4.68$$

Infine calcoliamo la costante di dissociazione. Ricordiamo che in acqua l'acido cianidrico si dissocia come

$$\ce{HCN + H_2O -> CN^- + H_3O^+}$$

e quindi la costante di dissociazione sarà data da

$$K_a=\rm \frac{[CN^-]\cdot[H_3O^+]}{[HCN]}$$

Tuttavia le moli di ione cianato e ione $\rm H_3O^+$ provengono dalle stesse moli di acido che si dissociano, quindi possiamo considerare la concentrazione di una sola specie elevata al quadrato. Dunque:

$$K_a=\rm \frac{[H_3O^+]^2}{[HCN]}=\frac{(2.09 \cdot 10^{-5})^2}{1.1}=3.971 \cdot 10^{-10}$$

dove come concentrazione dell'acido cianidrico abbiamo presa quella iniziale perché il numero di moli dissociate è piccolo e quindi trascurabile ($1.1 - 2.9 \cdot 10^-5 \sim 1.1$).
\end{soluzione}

\newpage

\begin{esercizio}
    Calcolare quanti mL di una soluzione di NaOH 0.2 M devono essere mescolati con 100 mL di una soluzione di $\rm CH_3COOH$ 0.2 M affinché il pH della soluzione risultante sia 4.05. $K_{a_{\rm CH3COOH}}=1.8 \cdot 10^{-5}$.
\end{esercizio}
\begin{soluzione}
    Il pH della soluzione finale è tipico di una soluzione tampone del primo tipo. Deduciamo quindi che ci troveremo davanti ad una soluzione contenente un acido debole (l'acido acetico) e un suo sale con base forte (l'acetato di sodio). Calcoliamo le moli di acido:

$$n_{\rm CH_3COOH}=\frac{0.2 \cdot 100}{1000}=2 \cdot 10^{-2} \; \rm mol$$

Chiamate $x$ le moli iniziali di base, scrivendo sopra la reazione le quantità iniziali e sotto quelle finali avremo che

\begin{center}
    \begin{tabular}{ccccccc}
        $x$ & & $2 \cdot 10^{-2}$ & & /\\
        $\rm NaOH$ & + & $\rm CH_3COOH$ & \ce{<-->} & $\rm CH_3COONa$\\
        / & & $2 \cdot 10^{-2} - x$ & & $x$\\
    \end{tabular}
\end{center}

Pertanto da una parte vale

$$\rm pH=\log{\left(\frac{1}{[H_3O^+]}\right)}
\implies
[H_3O^+]=\frac{1}{10^{pH}} \; \rm mol/L$$

$$\implies
[\text{H}_3\text{O}^+]=\frac{1}{10^{4.05}}=8.9125 \cdot 10^{-5} \; \rm mol/L$$

Dall'altra deve valere

$$[\text{H}_3\text{O}^+]=K_a\frac{C_a}{C_s}
\implies
C_s=K_a\frac{C_a}{[\text{H}_3\text{O}^+]}$$

$$\implies
x=1.8 \cdot 10^{-5}\,\frac{2 \cdot 10^{-2} - x}{8.9125 \cdot 10^{-5}}
\implies
x=0.2020(2 \cdot 10^{-2} - x)$$

$$\implies
1.2020 x = 0.4040 \cdot 10^{-2}
\implies x=3.3611 \cdot 10^{-3} \; \rm mol$$

I ml di base saranno banalmente dati da

$$C_s=\frac{n}{V}
\implies
V(\text{mL})=\frac{n}{C_s}
=\frac{3.3611 \cdot 10^{-3}}{0.2} \cdot 1000
=16.8053 \; \rm mL$$

\end{soluzione}

\newpage

\begin{esercizio}
    3.504 gr di ammoniaca acquosa vengono sciolti in 870 mL di soluzione finale e di questa ne vengono prelevati 250 mL. A questa aliquota di soluzione vengono aggiunti 2.8868 grammi di acido perclorico. Calcolare il pH sapendo che $K_{b_{\rm NH_3}}=1.8 \cdot 10^{-5}$.
\end{esercizio}
\begin{soluzione}
    Per prima cosa calcoliamo le moli di ammoniaca

$$n_{\rm NH_3}
=\frac{g}{MM_{\rm NH_3}}
=\frac{3.504}{17.031}
=0.2057 \; \rm mol$$

Queste moli saranno disciolte in 870 mL di soluzione. Per trovare il numero di moli $n'$ che ci sono nei 250 mL che vengono prelevati, facciamo la proporzione

$$n:870=n':250
\implies
n'=\frac{0.2057 \cdot 250}{870}
=5.9109 \cdot 10^{-2} \; \rm mol$$

A questo punto calcoliamo le moli di acido perclorico $\rm HClO_4$

$$n_{\rm HClO_4}=\frac{2.8868}{100.46}=2.8736 \cdot 10^{-2} \; \rm mol$$

Scriviamo la reazione che avviene, mettendo sopra le quantità iniziali e sotto quelle finali:

\begin{center}
    \begin{tabular}{ccccc}
        $5.9109 \cdot 10^{-2}$ &  & $2.8736 \cdot 10^{-2}$ & & /\\
        $\rm NH_3$ & + & $\rm HClO_4$ & \ce{->} & $\rm NH_4ClO_4$\\
        $3.6789 \cdot 10^{-2}$ &  & / & & $2.8736 \cdot 10^{-2}$\\
    \end{tabular}
\end{center}

A fine reazione abbiamo una base debole (l'ammoniaca) e un suo sale con acido forte (il perclorato d'ammonio), dunque ci troviamo davanti ad una soluzione tampone del secondo tipo. La concentrazione degli ioni $\rm OH^-$ sarà data da

$$[\text{OH}^-]=K_b\,\frac{C_b}{C_s}
=1.8 \cdot 10^{-5}\,\frac{3.6789 \cdot 10^{-2}}{2.8736 \cdot 10^{-2}}
=2.3044 \cdot 10^{-5} \; \rm mol/L$$

$$\rm pOH=\log \left( \frac{1}{2.3044 \cdot 10^{-5}} \right)=4.64$$

$$\implies \rm pH=14-pOH=14-4.64=9.36$$

\end{soluzione}

\newpage

\begin{esercizio}
    Calcolare il pH di una soluzione di 987 mL contenente 6.5432 gr di $\rm NH_4Cl$ e 3.9997 gr di NaOH.
\end{esercizio}
\begin{soluzione}
    Calcoliamo innanzitutto le moli

$$n_{\text{NH}_4\text{Cl}}=\frac{g}{MM_{\text{NH}_4\text{Cl}}}=\frac{6.5432}{53.491}=1.2232 \cdot 10^{-1} \; \rm mol$$

$$n_{\text{NaOH}}=\frac{g}{MM_{\text{NaOH}}}=\frac{3.9997}{39.997}=1.0000 \cdot 10^{-1} \; \rm mol$$

Le moli di NaOH sono in difetto, quindi esso sarà il reagente limitante. Scriviamo la reazione che avviene, mettendo sopra le quantità iniziali e sotto quelle finali:

\begin{center}
    \begin{tabular}{ccccccc}
        $1.2232 \cdot 10^{-1}$ &  & $1.0000 \cdot 10^{-1}$ & & / & & /\\
        $\rm NH_4Cl$ & + & NaOH & \ce{->} & $\rm NH_4OH$ & + & NaCl\\
        $2.2320 \cdot 10^{-2}$ &  & / & & $1.0000 \cdot 10^{-1}$ & &$1.0000 \cdot 10^{-1}$\\
    \end{tabular}
\end{center}

Guardando le specie che restano a fine reazione notiamo che abbiamo una base debole e un suo sale con acido forte\footnote{Non consideriamo il cloruro di sodio perché non dà luogo ad ulteriori reazioni, non influenzando quindi il valore del pH.}, cioè abbiamo una soluzione tampone del secondo tipo. Si avrà che

$$[\text{OH}^-]=K_b\,\frac{C_b}{C_s}
=1.8 \cdot 10^{-5}\,\frac{1.0000 \cdot 10^{-1}}{2.2320 \cdot 10^{-2}}
=8.0645 \cdot 10^{-5} \; \rm mol/L$$

$$\rm pOH=\log \left( \frac{1}{8.0645 \cdot 10^{-5}} \right)=4.09$$

$$\implies \rm pH=14-pOH=14-4.09=9.91$$

\end{soluzione}

\newpage

\begin{esercizio}
    0.4 L di HCl gassoso prelevato a c.n. vengono fatti gorgogliare in 325 mL di una soluzione acquosa di ammoniaca 0.1 N senza notare variazione del volume finale. Calcolare il pH della soluzione e come esso varia quando 10 mL di NaOH 0.1 M vengono mescolati ad essa.
\end{esercizio}
\begin{soluzione}
    Calcoliamo innanzitutto le moli di HCl tramite l'equazione di stato dei gas:

$$n_{\text{HCl}}
=\frac{PV}{RT}
=\frac{1 \cdot 0.4}{0.082 \cdot 273.15}
=1.7858 \cdot 10^{-2} \; \rm mol$$

Le moli di $\rm NH_3$ saranno invece date da

$$n_{\text{NH}_3}
=\frac{0.1 \cdot 325}{1000}
=3.25 \cdot 10^{-2} \; \rm mol$$

(Per l'ammoniaca normalità e molarità coincidono, in quanto essa è una base che può acquistare un protone)

Le moli di HCl sono in difetto, quindi esso sarà il reagente limitante. Scriviamo la reazione che avviene, mettendo sopra le quantità iniziali e sotto quelle finali:

\begin{center}
    \begin{tabular}{ccccccc}
        $1.7858 \cdot 10^{-2}$ &  & $3.25 \cdot 10^{-2}$ & & / & &\\
        HCl & + & $\rm NH_3$ & \ce{->} & $\rm NH_4Cl$ & + & $\rm H_2O$\\
        / &  &  $1.4642 \cdot 10^{-2}$ & & $1.7858 \cdot 10^{-2}$ & &\\
    \end{tabular}
\end{center}

Guardando le moli dopo la reazione, notiamo che abbiamo una base debole più un suo sale con acido forte: ci troviamo davanti ad una soluzione tampone del secondo tipo. Calcoliamo la concentrazione degli ioni $\rm OH^-$

$$[\text{OH}]^-=K_b\,\frac{C_b}{C_s}
=1.8 \cdot 10^{-5}\,\frac{1.4642 \cdot 10^{-2}}{1.7858 \cdot 10^{-2}}
=1.4758 \cdot 10^{-5} \; \rm mol/L$$

$$\rm pOH=\log \left( \frac{1}{1.4758 \cdot 10^{-5}} \right)=4.83$$

$$\implies \rm pH=14-pOH=14-4.83=9.17$$

Vediamo adesso come varia il pH dopo l'aggiunta di NaOH. Calcoliamone innanzitutto le moli:

$$n_{\rm NaOH}=\frac{0.1 \cdot 10}{1000}=1 \cdot 10^{-3} \; \rm mol$$

La concentrazione degli ioni $\rm OH^-$ dopo l'aggiunta sarà data da

$${[\text{OH}^-]}=K_b \frac{C_b + {[\text{OH}^-]}}{C_s - \rm{[\text{OH}^-]}}
=1.8 \cdot 10^{-5} \cdot \frac{1.4642 \cdot 10^{-2} + 1 \cdot 10^{-3}}{1.7858 \cdot 10^{-2} - 1 \cdot 10^{-3}}
=1.6702 \cdot 10^{-5} \; \rm mol/L$$

$$\rm pOH=\log \left( \frac{1}{1.6702 \cdot 10^{-5}} \right)=4.78$$

$$\implies \rm pH=14-pOH=14-4.78=9.22$$

\end{soluzione}

\newpage

\begin{esercizio}
    Calcolare il pH della soluzione acquosa di un acido debole mono-protico il quale, alla concentrazione 0.02 M risulta dissociato per il 10\%.
\end{esercizio}
\begin{soluzione}
    Chiamiamo HA il generico acido. Se di questo ne abbiamo una mole, il testo ci dice che solo 0.1 moli di questo si dissoceranno in ione $\rm H^+$ e ione $\rm A^-$. Scriviamo le quantità iniziali sopra la reazione che avviene e quelle finali sotto.

\begin{center}
    \begin{tabular}{ccccccc}
        1 & & / & & /\\
        $\rm HA$ & \ce{<-->} & $\rm H^+$ & + & $\rm A^-$\\
        $0.9$ & & $0.1$ & & $0.1$\\
    \end{tabular}
\end{center}

La costante di equilibrio\footnote{Una spiegazione più dettagliata si trova in §\ref{par:forza_degli_acidi_e_delle_basi} e in §\ref{par:acidi_deboli}.} allora sarà data, partendo da una concentrazione 0.02 M, da

$$K=0.02 \cdot \frac{0.1 \cdot 0.1}{0.9}=2.2222 \cdot 10^{-4}$$

Essendo da un acido debole, la concentrazione degli ioni $\rm H_3O^+$ sarà data da

$$[\text{H}_3\text{O}^+] = \sqrt{K_a \cdot C_a}
=\sqrt{2.2222 \cdot 10^{-4} \cdot 0.02}=2.1082 \cdot 10^{-3}\; \rm mol/L$$

$$\implies \rm pH=\log{\left(\frac{1}{2.1082 \cdot 10^{-3}}\right)}=2.68$$
\end{soluzione}

\newpage

\begin{esercizio}
    Il grado di dissociazione dell'acido monocloro acetico in una soluzione $1.0 \cdot 10^{-2}$ M è 0.309. Trovare quanti grammi del suo sale di sodio devono essere aggiunti ad un litro della precedente soluzione per avere un pH pari a 2.90.
\end{esercizio}
\begin{soluzione}
    Visto che si parla di grado di dissociazione sappiamo già che tale acido è una specie debole. Chiamato questo HA per semplicità, esso in acqua dà luogo all'equilibrio

$$\ce{HA <--> H^+ + A^-}$$

Supponiamo di partire da 1 mole di acido e scriviamo sopra la reazione le quantità iniziali e sotto quelle finali:

\begin{center}
    \begin{tabular}{ccccccc}
        1 & & / & & /\\
        $\rm HA$ & \ce{<-->} & $\rm H^+$ & + & $\rm A^-$\\
        $1- \alpha$ & & $\alpha$ & & $\alpha$\\
    \end{tabular}
\end{center}

dove $\alpha$ è il grado di dissociazione.

La costante di equilibrio sarà data da

$$K_a=\frac{\alpha^2}{1-\alpha^2}$$

ma siccome stiamo partendo da una concentrazione $C=1.0 \cdot 10^{-2} \; \rm mol/L$, la costante si calcolerà come

$$K_a=C\,\frac{\alpha^2}{1-\alpha^2}
=1.0 \cdot 10^{-2} \cdot \frac{0.309^2}{1-0.309}
=1.3818 \cdot 10^{-3}$$

A questo punto consideriamo la soluzione da realizzare: dobbiamo mischiare un acido debole con un suo sale con base forte, per cui ci troviamo di fronte ad una soluzione tampone del primo tipo. La concentrazione degli ioni $\rm H_3O^+$ sarà data da

$$[\text{H}_3\text{O}^+] = K_a \frac{C_a}{C_s}$$

Ovviamene la concentrazione del sale è incognita. D'altra parte però sappiamo che il pH deve essere pari a 2.90, per cui 

$$\rm pH=\log{\left(\frac{1}{[H_3O^+]}\right)}
\implies
[H_3O^+]=\frac{1}{10^{pH}}$$

$$\implies
\rm [H_3O^+]=\frac{1}{10^{2.90}}
=1.2589 \cdot 10^{-3} \; \rm mol/L$$

Ne segue che

$$C_s=K_a\frac{C_a}{\rm [H_3O^+]}
=1.3818 \cdot 10^{-3} \cdot \frac{1.0 \cdot 10^{-2}}{1.2589 \cdot 10^{-3}}
=1.0976 \cdot 10^{-2} \; \rm mol/L$$

E siccome abbiamo un litro di soluzione, questo sarà anche il numero di moli.

A questo punto per calcolare il peso molecolare dovremmo conoscere l'espressione del monocloro acetato di sodio. Non è un problema: l'espressione dell'acido monocloro acetico si trova in §7.3.3, \textit{"Forza degli acidi e delle basi"}, ed essa è $\rm CH_2ClCOOH$ da cui segue che il suo sale di sodio è $\rm CH_2ClCOONa$, il cui peso molecolare è 116.48. Quindi:

$$g=n \cdot MM=1.0976 \cdot 10^{-2} \cdot 94.5=1.0372 \; \rm grammi$$

\end{soluzione}

\newpage

\begin{esercizio}
    Calcolare il pH di una soluzione ottenuta mescolando 650 mL di una soluzione di idrossido di potassio 0.14 M con 910 mL di una soluzione di acido acetico 0.10 M. $K_a = 1.78 \cdot 10^{-5}$ (HAc).
\end{esercizio}
\begin{soluzione}
    Calcoliamo le moli di idrossido di potassio e di acido acetico.
$$n_{\rm KOH}=0.14\cdot 0.650 = 9.1000\cdot 10^{-2} \; \rm mol$$
$$n_{\rm CH_3COOH}=0.10\cdot 0.910 = 9.1000\cdot 10^{-2} \; \rm mol$$
Scriviamo la reazione con le concentrazioni iniziali sopra e quelle finali sotto:

\begin{center}
    \begin{tabular}{ccccccc}
        $9.1\cdot 10^{-2}$ &  & $9.1\cdot 10^{-2}$ & & / &&\\
        $\rm KOH$ & + & $\rm CH_3COOH$ & \ce{->} & $\rm CH_3COOK$ & + & $\rm H_2O$\\
        / &  &  / & & $9.1\cdot 10^{-2}$ &&\\
    \end{tabular}
\end{center}

Poichè le moli dei reagenti sono uguali, i composti si neutralizzano totalmente dando solo acetato di potassio. Dunque avviene idrolisi basica, infatti:
$$\ce{CH_3COOK \ce{->} CH_3COO^- + K^+}$$
Lo ione K$^+$ non dà luogo ad altre reazioni, invece lo ione acetato dà la seguente reazione di equilibrio:
$$\ce{CH_3COO^- + H_2O <--> CH_3COOH + OH^-}$$
La concentrazione degli ioni OH$^-$ sarà data da 

$$[\text{OH}^-]
=\sqrt{\frac{K_w}{K_a}\cdot C_s}
=\sqrt{\frac{10^{-14}}{1.78 \cdot 10^{-5}} \cdot \frac{9.1\cdot 10^{-2}}{1.56}}
=5.7246 \cdot 10^{-6} \; \rm mol/L$$

$$\implies \rm pOH
=\log{\left(\frac{1}{5.7246 \cdot 10^{-6}}\right)}
=5.24$$

$$\implies \rm pH=14-pOH=14-5.24=8.76$$
\end{soluzione}

\newpage

\begin{esercizio}
    Calcolare il pH di una soluzione formata mescolando 50 mL di una soluzione 0.1 M di acetato di sodio ($\rm CH_3COONa$) con 100 mL di una soluzione 0.1 M di acido acetico ($\rm CH_3COOH$; $K_a=1.8 \cdot 10^{-5}$). Calcolare anche come varia il pH della soluzione quando 10 mL di HCl 0.1 M vengono mescolati ad essa.
\end{esercizio}
\begin{soluzione}
    Si tratta di una soluzione tampone del primo tipo. Calcoliamo innanzitutto le moli

$$n_{\rm AcNa}=\frac{0.1 \cdot 50}{1000}= 5 \cdot 10^{-3} \; \text{mol}
\quad;\quad
n_{\rm HAc}=\frac{0.1 \cdot 100}{1000}=1 \cdot 10^{-2} \; \rm mol$$

La concentrazione degli ioni $\rm H_3O^+$ è pari a

$$[\text{H}_3\text{O}^+]=K_a\frac{C_a}{C_s}
=1.8 \cdot 10^{-5}\,\frac{\frac{1 \cdot 10^{-2}}{150}}{\frac{5 \cdot 10^{-3}}{150}}=3.6 \cdot 10^{-5} \; \rm mol/L$$

Il pH sarà dato da

$$\rm pH=\log{\left(\frac{1}{3.6 \cdot 10^{-5}}\right)}=4.44$$

Vediamo adesso come varia il pH dopo l'aggiunta di 10 mL di HCl 0.1 M. Le moli di questo sono

$$n_{\rm HCl}=\frac{10 \cdot 0.1}{1000}=1 \cdot 10^{-3} \; \rm mol$$

per cui la nuova concentrazione degli ioni $\rm H_3O^+$ sarà

$${[\text{H}_3\text{O}^+]}=\textit{K}_\textit{a} \frac{C_a + \rm{[\text{H}_3\text{O}^+]}}{C_s - \rm{[\text{H}_3\text{O}^+]}}
=1.8 \cdot 10^{-5} \, \frac{1 \cdot 10^{-2} + 1 \cdot 10^{-3}}{5 \cdot 10^{-3} - 1 \cdot 10^{-3}}
=3.636 \cdot 10^{-5} \; \rm mol/L$$

pertanto il nuovo pH sarà pari a

$$\rm pH=\log{\left(\frac{1}{3.636 \cdot 10^{-5}}\right)}=4.43$$

Notiamo che il pH è variato solo dello 0.01. Il motivo è che le moli del sale sono esattamente la metà di quelle dell'acido, per cui la soluzione ha il massimo potere tamponante e di conseguenza minimizza molto bene le variazioni del pH.
\end{soluzione}

\newpage

\begin{esercizio}
    A 100 mL di una soluzione 0.2 M di HCN ($K_a=2.1 \cdot 10^{-9}$) vengono aggiunti 40 mL di idrossido di sodio 0.5 M. Calcolare la concentrazione di $\rm H_3O^+$ della soluzione.
\end{esercizio}
\begin{soluzione}
    Calcoliamo innanzitutto le moli.

$$n_{\rm HCN}:100=0.2:1000
\implies
n_{\rm HCN}=\frac{100 \cdot 0.2}{1000}=2 \cdot 10^{-2} \; \rm mol$$

$$n_{\rm NaOH}:40=0.5:1000
\implies
n_{\rm NaOH}=\frac{40 \cdot 0.5}{1000}=2 \cdot 10^{-2} \; \rm mol$$

Scriviamo la reazione con le concentrazioni iniziali sopra e quelle finali sotto:

\begin{center}
    \begin{tabular}{ccccccc}
        $2 \cdot 10^{-2}$ &  & $2 \cdot 10^{-2}$ & & / &&\\
        $\rm NaOH$ & + & HCN & \ce{->} & $\rm NaCN$ & + & $\rm H_2O$\\
        / &  &  / & & $2 \cdot 10^{-2}$ &&\\
    \end{tabular}
\end{center}

Poiché le moli dei reagenti sono uguali, i due composti si neutralizzano totalmente dando $2 \cdot 10^{-2}$ moli di cianato di sodio. Quest'ultimo è un sale, che in acqua si trova dissociato in ione $\rm Na^+$ e ione $\rm CN^-$:

$$\ce{NaCN -> Na^+ + CN^-}$$

Mentre lo ione $\rm Na^+$ non dà luogo ad ulteriori reazioni, lo ione $\rm CN^-$ reagisce con una molecola d'acqua e dà un equilibrio:

$$\ce{CN^- + H_2O <--> HCN + OH^-}$$

Siamo quindi davanti ad una reazione di idrolisi basica. La concentrazione degli ioni $\rm OH^-$ sarà data da

$$[\text{OH}^-]
=\sqrt{\frac{K_w}{K_a}\cdot C_s}
=\sqrt{\frac{10^{-14}}{2.1 \cdot 10^{-9}} \cdot \frac{2 \cdot 10^{-2}}{140 \cdot 10^{-2}}}
=8.2479 \cdot 10^{-4} \; \rm mol/L$$

A noi però interessa la concentrazione degli ioni $\rm H_3O^+$. Per ottenerla basta ricordare che

$$\rm [H_3O^+][OH^-]=\textit{K}_{\textit{w}}=10^{-14}$$

da cui

$$\rm [H_3O^+]
=\frac{10^{-14}}{[OH^-]}
=\frac{10^{-14}}{8.2479 \cdot 10^{-4}}=1.2121 \cdot 10^{-11} \; \rm mol/L$$
\end{soluzione}

\newpage

\begin{esercizio}
    $3.5$ grammi di un acido debole monoprotico ($MM=120$) sono stati sciolti in acqua fino ad un volume totale di $250 \; \rm mL$. In queste condizioni il grado di dissociazione dell'acido è pari al $2\%$. A questa soluzione sono stati aggiunti $291.67 \; \rm mL$ di idrossido di sodio 0.1 M. Calcolare il pH della soluzione finale.
\end{esercizio}
\begin{soluzione}
    Detto $\rm HA$ il generico acido debole monoprotico, di questo abbiamo un numero di moli pari a

$$n_{\rm HA}
=\frac{3.5}{120}=2.9167 \cdot 10^{-2} \; \rm mol$$

Calcoliamo adesso le moli di NaOH

$$n_{\rm NaOH}
=C \cdot V(\text{L})=0.1 \cdot 291.67 \cdot 10^{-3}
=2.9167 \cdot 10^{-2} \; \rm mol$$

Scriviamo allora la reazione che avviene, mettendo sopra le moli iniziali e sotto quelle finali:

\begin{center}
    \begin{tabular}{ccccccccc}
        $2.9167 \cdot 10^{-2}$ & & $2.9167 \cdot 10^{-2}$ & & / & &\\
        $\rm NaOH$ & + & $\rm HA$ & \ce{<-->} & $\rm NaA$ & + & $\rm H_2O$\\
        / & & / & & $2.9167 \cdot 10^{-2}$ & &\\
    \end{tabular}
\end{center}

Notiamo che poiché le moli iniziali di acido e base sono uguali esse reagiscono totalmente, dunque alla fine avremo solo un sale dell'acido con base forte. Essendo un sale, esso in acqua si dissocia come

$$\ce{NaA -> Na^+ + A^-}$$

Mentre lo ione $\rm Na^+$ non dà luogo ad ulteriori reazioni, lo ione $\rm A^-$, che è la base coniugata di un acido debole, in acqua instaura l'equilibrio

$$\ce{A^- + H_2O <--> HA + OH^-}$$

Siamo quindi davanti ad una reazione di idrolisi basica. La concentrazione degli ioni $\rm OH^-$ sarà data da

$$[\text{OH}^-]
=\sqrt{\frac{K_w}{K_a}\cdot C_s}$$

Per procedere con il calcolo dobbiamo però conoscere sia la costante acida $K_a$ che la concentrazione del sale $C_s$.

Per calcolare $K_a$ consideriamo di nuovo l'acido debole $\rm HA$. Esso in soluzione instaura l'equilibrio

$$\ce{HA <--> H^+ + A^-}$$

per cui la costante di equilibrio sarà data da

$$K_a=\rm \frac{[H^+] \cdot [A^-]}{[HA]}$$

Per calcolare la concentrazione dei prodotti ricordiamo che il numero di moli dissociate è dato da

$$n_{\rm dissociate}=\alpha \cdot n_{\rm totali}$$

dove $\alpha$ è il grado di dissociazione. Svolgendo il calcolo si ha:

$$n_{\rm dissociate}
=\frac{2}{100} \cdot \frac{3.5}{120}
=5.8333 \cdot 10^{-4} \; \rm mol$$

Avremo dunque che

$$\rm [HA]=\frac{2.9167 \cdot 10^{-2}}{250 \cdot 10^{-3}}=8.7868 \cdot 10^{-2} \; mol/L$$
$$\rm [H^+]=[A^-]=\frac{5.8333 \cdot 10^{-4}}{250 \cdot 10^{-3}}=2.3333 \cdot 10^{-3} \; mol/L$$

In definitiva

$$K_a=\frac{\big( 2.3333 \cdot 10^{-3} \big)^2}{8.7868 \cdot 10^{-2}}=6.1947 \cdot 10^{-5}$$

Per quanto riguarda $C_s$, dobbiamo per prima cosa fare attenzione al fatto che il volume da considerare sarà quello dato dalla somma del volume di acido più quello dell'idrossido, che è pari a $V_{\rm tot}=291.67 + 250=541.67 \; \rm mL$. Dunque:

$$C_s=\frac{2.9167 \cdot 10^{-2}}{541.67 \cdot 10^{-3}}=5.3846 \cdot 10^{-2} \; \rm mol/L$$

Possiamo finalmente calcolare la concentrazione degli ioni $\rm OH^-$

$$[\text{OH}^-]
=\sqrt{\frac{10^{-14}}{6.1947 \cdot 10^{-5}} \cdot 5.3846 \cdot 10^{-2}}
=2.9483 \cdot 10^{-6} \; \rm mol/L$$

e dunque

$$\rm pOH=\log\left(\frac{1}{[OH^-]}\right)=\log\left(\frac{1}{2.9483 \cdot 10^{-6}}\right)=5.53$$

$$\implies \rm pH=14 - pOH=14 - 5.43=8.47$$
\end{soluzione}

\newpage

\begin{esercizio}
    Calcolare il pH di una soluzione di 780 mL ottenuta sciogliendo 6.0000 grammi di acido acetico e 4.3000 grammi di idrossido di sodio.
\end{esercizio}
\begin{soluzione}
    Calcoliamo innanzitutto le moli:

$$n_{\rm CH_3COOH}=\frac{g}{MM_{\rm CH_3COOH}}=\frac{6}{60.052}=9.9913 \cdot 10^{-2} \; \rm mol$$

$$n_{\rm NaOH}=\frac{4.3}{39.997}=10.7508 \cdot 10^{-2} \; \rm mol$$

Le moli di acido acetico sono in difetto, quindi sarà il reagente limitante.

Scriviamo la reazione con le concentrazioni iniziali sopra e quelle finali sotto:

\begin{center}
    \begin{tabular}{ccccccc}
        $9.9913 \cdot 10^{-2}$ & & $10.7508 \cdot 10^{-2}$ & & / &&\\
        $\rm CH_3COOH$ & + & NaOH & \ce{->} & $\rm CH_3COONa$ & + & $\rm H_2O$\\
        / & & $0.7595 \cdot 10^{-2}$ & & $9.9913 \cdot 10^{-2}$&&\\
    \end{tabular}
\end{center}

A fine reazione abbiamo una base forte che è l'idrossido di sodio e un sua sale che è l'acetato di sodio, ottenuto a partire da un acido debole.

Analizziamo le singole specie.

L'idrossido di sodio è una base forte, per cui si dissocia totalmente in $\rm Na^+$ e $\rm OH^-$:

\begin{equation*}
    \ce{NaOH -> Na^+ + OH^-}
\end{equation*}

L'acetato di sodio in acqua si dissocia in ione $\rm Na^+$ e ione acetato $\rm CH_3COO^-$; quest'ultimo dà luogo a una reazione di equilibrio di idrolisi basica:

\begin{equation*}
    \ce{CH_3COO^- + H_2O <--> CH_3COOH + OH^-}
\end{equation*}

Tuttavia, in soluzione sono già presenti parecchi ioni $\rm OH^-$ provenienti dalla dissociazione della base forte, per cui l'equilibrio sarà totalmente spostato a sinistra e di fatto tale reazione non avviene. Ne segue che il pOH sarà determinato soltanto dalla dissociazione della base forte.

La concentrazione degli ioni $\rm OH^-$ proveniente dall'$\rm NaOH$ è pari a

\begin{equation*}
    [\rm OH^-]=\frac{0.7595 \cdot 10^{-2} \cdot 10^{-2} \cdot 1000}{780}=1.2820 \cdot 10^{-2} \; \rm mol/L
\end{equation*}

e quindi

\begin{equation*}
    \rm pOH=\log{\left(\frac{1}{1.2820 \cdot 10^{-2}}\right)}=1.89
\end{equation*}

\begin{equation*}
    \implies \rm pH=14-pOH=14-1.89=12.11
\end{equation*}
\end{soluzione}

\newpage

\begin{esercizio}
    Calcolare la concentrazione di una soluzione di ipoclorito di sodio avente pH pari a 9.03, sapendo che l'acido ipocloroso ha una costante acida pari a $K_a=2.95 \cdot 10^{-8}$.
\end{esercizio}
\begin{soluzione}
    La soluzione è composta da NaClO il quale è un sale, dunque come tutti i sali è un elettrolita forte e in acqua si dissocerà totalmente secondo la reazione

\begin{equation*}
    \ce{NaClO -> Na^+ + ClO^-}
\end{equation*}

Mentre lo ione $\rm Na^+$ non subisce ulteriori reazioni, lo ione ipoclorito $\rm ClO^-$, che è la base coniugata dell'acido ipocloroso il quale è un acido debole (ciò lo deduciamo dal valore della costante $K_a$), reagirà con l'acqua strappandole un protone per formare $\rm HClO$ e lasciando un eccesso di ioni $\rm OH^-$:

\begin{equation*}
    \ce{ClO^- + H_2O <--> HClO + OH^-}
\end{equation*}

Ci troviamo dunque in presenza di un'idrolisi basica, tant'è che il testo ci dice che il pH ha valore $9.03$. La concentrazione degli ioni $\rm OH^-$ sarà allora data da

\begin{equation*}
    [\text{OH}^-]
    =\sqrt{\frac{K_w}{K_a}\cdot C_s}
\end{equation*}

da cui ricaviamo la formula per il calcolo della concentrazione del sale, che è quella che ci interessa. Infatti si ha

\begin{equation*}
    [\text{OH}^-]^2
    =\frac{K_w}{K_a}\cdot C_s
    \implies
    C_s
    =[\text{OH}^-]^2 \cdot \frac{K_a}{K_w}
\end{equation*}

La quantità $\rm [OH^-]$ la ricaviamo dalla formula del calcolo del pOH:

\begin{equation*}
    \rm pOH=\log \left( \frac{1}{[OH^-]} \right)
    \implies
    [OH^-]=10^{pOH}
\end{equation*}

Ricordiamo che il testo ci dice che la soluzione ha pH pari a 9.03, ciò significa che

\begin{equation*}
    \rm pOH=14-9.03=4.97
\end{equation*}

per cui in definitiva avremo

\begin{equation*}
    C_s
    ={\big( 10^{4.97} \big)}^2 \cdot \frac{2.95 \cdot 10^{-8}}{1 \cdot 10^{-14}}
    =3.39 \cdot 10^{-4} \; \rm mol/L
\end{equation*}
\end{soluzione}