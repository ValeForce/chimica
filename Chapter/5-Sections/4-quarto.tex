Le proprietà colligative ci dicono che più che la concentrazione delle singole speci è importante il numero di particelle che si ha in soluzione.
\subsection{Temperatura di ebollizione (ebullioscopia)}
Finora abbiamo fatto esempi di soluzioni nelle quali tutti i componenti fossero volatili. Cosa accade se mettiamo in soluzione un dato solvente più un solvente non volatile?

Esempi tipici sono zucchero e cloruro di sodio in acqua.

La legge di Raoult ci dice che la tensione di vapore della soluzione sarà uguale alle tensioni di vapore delle speci pure per le loro frazioni molari per fissate temperature:

$$P=\rchi_AP^0_A + \rchi_BP^0_B$$

dove supponiamo che A sia il solvente e B sia il soluto.

Se per esempio B non fosse volatile, la tensione di vapore della specie B pura sarà zero, cioè $P_B^0 \approxeq 0$ e l'equazione di Raoult diventa

$$P=\rchi_A P_A^0$$

cioè la tensione di vapore di una soluzione di un soluto non volatile è uguale alla tensione di vapore del solvente puro per la frazione molare di questo. Se quindi ad esempio abbiamo una soluzione di acqua e cloruro di sodio, dovremo calcolare la tensione di vapore dell'acqua, perché l'NaCl non contribuisce alla tensione di vapore totale.

Tuttavia per definizione qualunque frazione molare è minore di 1, quindi $\rchi_A<1$. Ne segue che $P < P_A^0$, cioè la tensione di vapore di una soluzione di un soluto non volatile sarà sempre minore della tensione di vapore del solvente alla stessa temperatura.

Ad esempio sappiamo che la temperatura di ebollizione di un liquido è la temperatura alla quale la sua tensione di vapore eguaglia la pressione atmosferica. Se quindi abbiamo acqua pura in riva al mare la tensione di vapore dell'acqua raggiungerà i 760 torr (cioè la pressione atmosferica) a 100° C, se invece fossimo in montagna dove la pressione esterna è più bassa, bollirà prima. Supponiamo di essere in riva al mase. Nell'istante in cui aggiungiamo del cloruro di sodio all'acqua osserviamo un abbassamento della tensione di vapore. Ne segue che anche se adesso raggiungessimo i 100° C l'acqua non bollirà, perché a parità di temperatura la soluzione avrà una tensione di vapore più bassa di 760 torr. Quindi l'acqua con il sale bolle a temperatura più alta.

Per definizione inoltre $\rchi_A + \rchi_B = 1 \implies \rchi_A = 1 - \rchi_B$. Possiamo allora riscrivere la legge di Raoult come 

$$P=\rchi_AP^0_A=(1- \rchi_B)P^0_A=P^0_A - \rchi_BP^0_A$$

$$\implies P=P^0_A - \rchi_BP^0_A \implies P^0_A - P = \rchi_BP^0_A$$

$$\implies \rchi_B=\frac{P^0_A - P}{P^0_A} \implies \rchi_B=\frac{\Delta P}{P^0_A}$$
dove $\Delta P$ è la variazione della tensione di vapore nel pasaggio da solvente puro ($P_A^0$) a soluzione ($P$).

Il rapporto $\Delta P/P_A^0$ rappresenta una variazione relativa delle tensione di vapore del solvente e che è uguale alla frazione molare del soluto $\rchi_B$.

Possiamo usare allora entrambe le espressioni per calcolare P, sia quella con $\rchi_A$ che quella con $\rchi_B$. Inoltre abbiamo che

$$\frac{\Delta P}{P_A^0}= \rchi_B = \frac{n_B}{n_A + n_B}$$

Abbiamo visto che le tensioni di vapore delle soluzioni sono più basse di quelle del solvente puro, e in conseguenza a ciò la temperatura di ebollizione aumenta rispetto a quella del solvente puro. Lo studio dell'aumento del punto di ebollizione di un solvente per aggiunta di un soluto prende il nome di \textit{ebullioscopia}

Come facciamo a calcolare il $\Delta t_{eb}$ tra soluzione e solvente puro?

Si ha che $\Delta t$ è pari al prodotto di una costante $k_{eb}$ specifixa del solvente moltiplicata per la molalità e per un \textbf{coefficiente di Van't Hoff} $i$:

$$\Delta t_{eb} = k_{eb} \cdot m \cdot i$$

Attenzione! il coefficiente $i$ dipende dal numero di particelle, ovvero dipende dal numero di particelle in cui si dissocia la specie messa in soluzione. Vediamo degli esempi.

\begin{itemize}
    \item Se abbiamo del saccarosio messo in acqua, questa specie non si dissocia in alcun modo perché non è un elettrolita (cioè non è una sostanza capace di dissociarsi in ioni quando viene disciolta in acqua). Resta quindi sciolto ma come molecola. In questo caso $i$ vale 1;
    \item Se mettiamo in acqua del cloruro di sodio, questo si dissocia in ioni Na$^+$ e Cl$^-$, quindi per ogni unità formula NaCl escono fuori due particelle. In questo caso $i$ vale 2;
    \item Se avessimo idrossido di calcio, questo si dissocerebbe in uno ione Ca$^{2+}$ e due ioni OH$^-$, quindi ogni unità formula Ca(OH)$_2$ dà luogo a tre particelle. In questo caso $i$ vale 3.
\end{itemize}

In generale quindi $i$ vale 1 per i non elettroliti e più di 1 a seconda di quale elettrolita abbiamo.

\vspace{0.2cm}La variazione della temperatura di ebollizione non dipende dalla concentrazione della specie, ma dal numero di particelle totali che avremo in soluzione. Il $\Delta t$ ebollioscopico quindi sarà

$$\Delta t_{eb}=k_{eb} \frac{b \cdot 1000}{MM_b \cdot a}$$

con $a$ grammi di solvente e $b$ grammi di soluto.

La costante ebullioscopica $k_{eb}$ è tabulata. Questa può essere calcolata subito: mettiamoci nel caso di una specie che non si dissocia, quindi $i=1$, allora sarà

$$\Delta t_{eb}= k_{eb} \cdot m
\implies
k_{eb} = \frac{\Delta t_{eb}}{m}$$

Quindi sperimentalmente $k_{eb}$ è pari al $\Delta t_{eb}$ di una soluzione 1-molale (come unità di misura).

Se esplicitata, $k_{eb}$ sarà

$$k_{eb}=\frac{R \cdot T_{eb}^2 \cdot MM_{solvente}}{1000 \cdot \Delta H^0_{eb}}$$

\subsubsection{Excursus: a che serve la pentola a pressione?}
Quando cucinimao abbiamo una soluzione. Con la pentola a pressione chiudiamo il sistema in modo tale che tutto il vapore che si genera resti dentro. In questo modo la pressione all'interno aumenta rispetto alla pressione esterna, quindi all'interno della pentola ci sarà una pressione che sarà somma delle pressione atmosferica pià quella che si sta realizzando all'interno. Ciò significa che la soluzione bollirà a temperatura più elevata, in quanto la tensione di vapore dovrà raggiungere valori più alti.

Ecco perché un dato cibo dentro la pentola a pressione si cucina prima, perché tale pentola ci permette di raggiungere temperature interne più elevate. Infatti una pentola comunque col coperchio appoggiato se contiene acqua la farà bollire a 100° C, perché la pressione all'interno sarà quella esterna, in quanto se si realizza un minimo di extrapressione il vapore esce; se invece abbiamo una pentola a pressione al suo interno ci sarà una pressione molto più elevata di quella atmosferica, cosa che permette alla soluzione di raggiungere temperature superiori ai 100° C e che ha come conseguenza una cottura più veloce.
\subsection{Temperatura di congelamento (crioscopia)}
Quando nevica per evitare incidenti si sparge sale sulla neve, la quale così facendo si scioglie.

Nei radiatori viene messo il glicole etilenico, detto liquido anti-gelo, il quale in zone non troppo fredde può essere mischiato con acqua (in quelle troppo fredde non si può mischiare perché l'acqua è l'unica specie che congelando aumenta di volume, per cui il radiatore si spaccherebbe).

In altre parole, le soluzioni avranno un punto di gelo più basso del solvente puro. Quindi ad esempio la neve col sale si scioglie perché avrà un punto di congelamento più basso dell'acqua pura che la temperatura esterna non riesce a raggiungere.

Per \textit{temperatura di congelamento} di un liquido si intende la temperatura alla quale il liquido e il suo solido mostrano la stessa tensione di vapore. Ad esempio nell'acqua si ha il congelamento quando quel primo cristallo di ghiaccio che inizia a formarsi ha la stessa tensione di vapore dell'acqua ancora liquida. Va da notare che il primo ghiaccio che si forma per abbassamento di temperatura di soluzioni non contiene soluto, ma solo solvente.
\subsection{Pressione osmotica}
\subsubsection{L'osmosi}
\subsection{Tensione superficiale}