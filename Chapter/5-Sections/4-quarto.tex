Finora abbiamo fatto esempi di soluzioni nelle quali tutti i componenti fossero volatili. Cosa accade se mettiamo in soluzione un dato solvente più un solvente non volatile?

Esempi tipici sono zucchero e cloruro di sodio in acqua.

La legge di Raoult ci dice che la tensione di vapore della soluzione sarà uguale alle tensioni di vapore delle speci pure per le loro frazioni molari per fissate temperature:

$$P=\rchi_AP^0_A + \rchi_BP^0_B$$

dove supponiamo che A sia il solvente e B sia il soluto.

$$P=\rchi_AP^0_A=(1- \rchi_B)P^0_A=P^0_A - \rchi_BP^0_A$$
$$\implies P=P^0_A - \rchi_BP^0_A$$
$$\implies P^0_A - P = \rchi_BP^0_A$$
$$\implies \rchi_B=\frac{P^0_A - P}{P^0_A} \implies \rchi_B=\frac{\Delta P}{P^0_A}$$
\subsection{Temperatura di ebollizione (ebullioscopia)}
\subsection{Temperatura di congelamento (crioscopia)}
\subsection{Pressione osmotica}
\subsubsection{L'osmosi}
\subsection{Tensione superficiale}