\subsection{La legge di Raoult}
\subsubsection{DEF. Soluzione ideale}
Supponiamo di avere due liquidi in soluzione, ad esempre alcol etilico e acqua. Diciamo che questa soluzione è \textit{ideale} se per i due liquidi, i quali devono essere totalmente miscibili cioè A si scioglie totalmente in B o vicevecersa B si scioglie totalmente in A qualunque siano le proporzioni, il $\Delta$H di mescolamento è pari a zero.
$$p_AV=n_ART
\qquad
p_BV=n_BRT$$

$$(p_A + p_B)V= (n_A + n_B)RT$$

$$pV=(n_A + n_B)RT$$

$$\frac{p_AV}{pV}= \frac{n_ART}{(n_A + n_B)RT}
\implies
\frac{p_A}{p} = \frac{n_A}{n_A + n_B}\equiv \rchi '_A
\implies
p_A=p \cdot \rchi '_A$$

$$\rchi '_A=\frac{p_A}{p}=
\frac{p_A}{\rchi_A p_A^0 + \rchi_B p_B^0}=
\frac{\rchi_A p_A^0}{\rchi_A p_A^0 + \rchi_B p_B^0}$$

$$\rchi '_B=\frac{p_B}{p}=
\frac{p_B}{\rchi_A p_A^0 + \rchi_B p_B^0}=
\frac{\rchi_B p_B^0}{\rchi_A p_A^0 + \rchi_B p_B^0}$$

$$\rchi'_A=\frac{\rchi_A}{\rchi_A + \rchi_B\frac{p^0_B}{p^0_A}}
\qquad
\rchi'_B=\frac{\rchi_B}{\rchi_A\frac{p^0_B}{p^0_A} + \rchi_B}$$