Si definisce \textit{solubilità} la massima quantità di soluto che può essere disciolta in una data quantità di
solvente. \E quindi la concentrazione di una soluzione satura.

Ci concentreremo sul caso di gas disciolti in liquidi.
\subsection{La legge di Henry}
Tale legge regola la solubilità dei gas nei liquidi.

La solubilità si misura in moli su litro ed aumenta all'aumentare della pressione del gas esercitata sul liquido.
\begin{figure}[htp]
    \centering
    \includegraphics[width=10cm]{immagini/solubilità.png}
\end{figure}

C'è però un limite di solubilità (chiamato spesso prodotto di solubilità) oltre al quale, nelle condizioni in cui ci troviamo, non è possibile fare assorbire ulteriore gas alla soluzione.
L'equazione di tale legge è:

$$S=k_HP$$

$k_H$ è la costante di Henry, la quale è espressa dall'equazione di Van't Hoff:

$$k_H=k_{298 \, K} \, e^{\big[\frac{\Delta H_{diss}}{R} \big( \frac{1}{T} - \frac{1}{298} \big) \big]}$$

dove

$k_{298 \, K}$ è la costate di equilibrio del processo di solubilizzazione;

$\Delta H_{diss}$ è l'entalpia molare di solubilizzazione delle molecole di gas nel liquido (in J/mol);

$R$ è la costante dei gas ideali;

$T$ è la temperatura assoluta espressa in Kelvin.

Da tale relazione deduciamo che la temperatura è determinante per la solubilità, in particolare i gas si sciolgono meglio a temperature più basse. Se invece riscaldiamo, il gas tende a essere eliminato dalle soluzioni già sature di quel gas.

\begin{minipage}{0.5 \textwidth}
    \begin{figure}[H]
        \centering
        \includegraphics[width=5cm]{immagini/solubilità_temperatura.png}
    \end{figure}
\end{minipage}
\begin{minipage}{0.4 \textwidth}
    \vspace{0.8cm}Il grafico riporta la solubilità dei gas in acqua al variare della pressione. Notiamo che c'è una marcata diversa solubilità al variare dei gas: l'elio si scioglie poco, l'azoto mediamente e l'ossigeno tanto.
\end{minipage}

\vspace{0.4cm}Ad esempio i sub hanno delle bombole che posso essere riempite con aria ad elevata pressione, cioè riempite con azoto all'80\% e ossigeno al 20\% circa. L'ossigeno viene consumato con la respirazione, l'azoto no. Quest'ultimo è un gas inerte, cioè reagisce poco, ma si scioglie nel sangue.

Essendoci parecchio azoto, i subacquei devono eseguire un processo di decompressione, ossia devono risalire lentamente in base a quanto siano andati in profondità. Ciò si fa perché sebbene l'azoto sia entrato in soluzione non viene consumato, e se man mano che la pressione diminuisce con la risalita si libera velocemente può creare delle bolle che arrivate al cervello causano embolia. In questi casi serve una camera iperbarica per portare velocemente la pressione a condizioni tali che l'azoto si disciolga nel sangue per farlo liberare lentamente.

Oggi si usano miscele di elio e ossigeno, perché l'elio si scioglie molto meno nel sangue e quindi è necessario un tempo di decompressione minore.

\subsubsection{Excursus: come si preparano le bevande gassate?}

Si prepara la bevanda normalmente, senza essere gassata. Dopodiché si mette nella bottiglia e si inserisce anidride carbonica con pressioni elevate di 8-10 atm. Quest'ultima riesce a penetrare nella fase liquida e a sciogliersi in una certa percentuale in funzione della temperatura, cioè ad una data temperatura se ne scioglieranno un certo numero di grammi per litro. 

Per quanto attiene ai gas, più freddo è il liquido e maggiore è il gas che riesce a sciogliersi. Infatti se riscaldiamo una bevanda gassosa perdiamo l'effervescenza.

Le bottiglie, tenute laddove c'è un'elevata pressione del gas, quando poi vengono chiuse mantengono l'effervescenza.

Questo processo avviene sia con l'anidride carbonica che con l'ossigeno: per l'anidride carbonica ciò avviene perché essa all'interno ha legami polari, mentre nel caso dell'ossigeno, che formalmente è una molecola apolare, esso interagisce con gli idrogeni dell'acqua perché questi sono caricati positivamente, e quindi "generano" una sorta di carica indotta sull'ossigeno quando questo si scioglie in acqua, ossia delle cariche possono pre-esistere oppure possono venire indotte quando si mescolano queste sostanze in acqua, e questo fa si che ci sia una piccola solubilità. Diciamo piccola perché non appena apriamo la bottiglia l'effervescenza va a diminuire, cioè il gas tende a liberarsi, per cui è stato il nostro aver costretto il liquido con una pressione elevata di questi gas che ne ha permesso il discioglimento, altrimenti l'ossigeno che si scioglie in acqua è pochissimo, sebbene riesca a far respirare i pesci in acqua.

\vspace{0.2cm}Quindi, ricapitolando, i gas si sciolgono nei liquidi. La loro solubilità è data dalle legge di Henry, la quale dice che la solubilità è direttamente proporzionale alla pressione esercitata dal gas. Inoltre la costante di proporzionalità contiene la temperatura e ci si accorge che al diminuire di quest'ultima la solubilità aumenta, al contrario di molti solidi in acqua che sono più solubili all'aumentare della temperatura.