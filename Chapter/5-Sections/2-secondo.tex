La concentrazione di una soluzione è la grandezza che esprime il rapporto tra la quantità di soluto e la quantità di solvente
\subsection{Modi di esprimere la concentrazione}
La concentrazione può essere espressa in diversi modi:
\subsubsection{Percentuale in massa}
Con questo metodo si ottiene un numero adimensionale definito come

$$ \frac{\text{grammi di soluto}}{100\text{ grammi di soluzione}} \cdot 100$$

Va da notare che nei 100g di soluzione sono sommati solvente e soluto. Ciò nella pratica corrispondere a prendere un certo numero di grammi di soluto e aggiungere grammi di solvente fino ad arrivare a 100g
\subsubsection{Percentuale in volume}
\E definita come

$$ \frac{\text{mL di soluto}}{100\text{ mL di soluzione}} \cdot 100$$

\subsubsection{Frazione molare}
\E un rapporto tra le moli di un qualunque componente della soluzione diviso la somma delle moli totali. Nel caso di una soluzione di due sole componenti A e B la frazione molare, indicata con $\rchi$, sarà

$$\rchi_A=\frac{n_A}{n_A + n_B}
\qquad
\rchi_B=\frac{n_B}{n_A + n_B}
\qquad
\rchi_A + \rchi_B = 1$$

\E insito nella definizione che la somma delle frazioni molari sarà sempre uguale a 1.
\subsubsection{Molarità (M)}
Si definisce come le moli di soluto in un litro di soluzione. Va da notare che avremo così moli su volume. Ciò nella pratica corrisponde, ad esempio ad avere un contenitore di cui conosciamo il volume, mettiamo un certo numero di grammi di NaCl (che è il soluto) che corrisponderanno ad un certo numero di moli e poi aggiungiamo H$_2$O (che è il solvente) fin quando il volume totale (dato sia da solvente che soluto) equivale a 1L (quindi NO certe moli + 1L di acqua).
\subsubsection{Molalità (m)}
Si definisce come le moli di soluto in 1000 grammi di solvente puro (e non di soluzione!). Ciò significa che in questo caso se ad esempio aggiungiamo una certa quantità di grammi di NaCl ne dovremo poi aggiungere 1000 di H$_2$O,per cui il volume finale sarà più di 1L e la massa della soluzione sarà più di 1000 grammi.

Molarità e molalità sono quindi molto diverse.
\subsubsection{Normalità (N)}
Si definisce come gli equivalenti di soluto in 1L di soluzione.

La definizione è identica a quella della molarità tranne per il fatto che al posto delle moli abbiamo gli equivalenti. Cosa sono questi?

Le moli sono definite come grammi su peso molecolare (g/MM), gli equivalenti come grammi su peso equivalente (g/ME). Il peso equivalente è il peso molecolare diviso un numero. Di volta in volta bisogna vedere a quale reazione stia partecipando quel composto e cosa succede. In particolare:
\begin{itemize}
    \item Negli acidi e nelle basi il numero per cui bisogna dividere corrisponde al numero di ioni H$^+$ o OH$^-$ che liberano a seguito della dissociazione.
    \item In una reazione redox è il numero di elettroni scambiati da quella specie.
\end{itemize}

Per capire meglio facciamo degli esempi:

\begin{enumerate}
    \item \textbf{Acido cloridrico HCl}
    
    Esso in acqua si dissocia in H$^+$ e Cl$^-$. Quindi si produce uno ione da una molecola di HCl. In questo caso peso molecolare e equivalente coincidono, per cui molarità e normalità saranno la stessa cosa;
    \item \textbf{Idrossido di sodio NaOH}
    
    Esso in acqua si dissocia in Na$^+$ e OH$^-$. Quindi da una unità formula (per le speci ioniche non esistono le molecole) di NaOH in soluzione si ottiene uno ione, per cui il rapporto è 1:1. Pertanto anche in questo caso peso molecolare e equivalente coincidono, cioè molarità e normalità sono uguali.
    \item \textbf{Acido solforico H$_2$SO$_4$}
    
    Esso in acqua si dissocia liberando due ioni H$^+$ ed uno ione SO$_4^{2-}$. Quindi per ogni molecola di H$_2$SO$_4$ in soluzione otterremo due ioni. Allora in questo caso il peso equivalente sarà dato dal peso molecolare diviso due, cioè il peso equivalente è la metà. Quindi se il peso molecolare dell'H$_2$SO$_4$ è circa 98 il peso equivalente è circa 49, per cui se avessimo 98 grammi di acido solforico diremmo di averne una mole (98/98=1), mentre per ottenere gli equivalenti dovremmo dividere per 49 e quindi avremmo 2 equivalenti. Si può anche dire che una mole di H$_2$SO$_4$ messa in acqua dà una soluzione 1-molare o 2-normale.

    Quindi, siccome l'acido solforico dà due protoni in soluzione, la normalità è il doppia della molarità
    \item \textbf{Idrossido di calcio Ca(OH)$_2$}
    
    Esso in acqua si dissocia in Ca$^{2+}$ più due ioni OH$^-$. Ne segue che in questo caso il peso equivalente è la metà del peso molecolare, quindi la normalità del Ca(OH)$_2$ è il doppio della sua molarità (in un caso divido per 74, nell'altro per 37).
    \item \textbf{Permanganato di potassio KMnO$_4$}
    
    Esso nasce da un'ossidoriduzione, la quale in ambiente fortemente acido fa si che il manganese del permanganato passi da stato di ossidazione +7 a stato di ossidazione +2 con l'acquisto di 5 elettroni ($\ce{Mn^{7+} + 5e -> Mn^{2+}}$). Il peso equivalente sarà uguale al peso molecolare diviso 5, cioè avremo M.M.=158 e M.E.=31.6.

    Quindi una soluzione 1-molare di KMnO$_4$ sarà 5-normale, nel caso in cui stiamo operando una redox ed il manganese va da +7 a +2. Se invece fossimo in una reazione di scambio tra due sali, il permanganato di sodio si scinderebbe in K$^+$ e MnO$_4^-$, cioè avremo una carica positiva sullo ione potassio e una negativa sullo ione permanganato, quindi il numero per cui dovremo divididere il peso molecolare diventa pari 1, per cui peso molecolare ed equivalente coincidono, nonché anche molarità e normalità.

    \E dunque fondamentale capire in quale reazione si trova il composto e cosa sta dando luogo
\end{enumerate}
