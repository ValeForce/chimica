\normalsize \subsection{Definizioni di acido e base}
Le definizioni di acidi e basi sono variate negli anni.

Inizialmente abbiamo classificato come acido o base un dato composto in funzione dell'elettronegatività dell'atomo X nella seguente reazione

$$\ce{XOH(aq) + H_2O -> XO^-(aq) + H_3O^+(aq)}, \quad \text{X elettronegativo}$$

$$\ce{XOH(aq) + H_2O -> X^+(aq) + OH^-(aq)}, \quad \text{X elettropositivo}$$

Immaginiamo quindi di avere un generico composto formato da un atomo X con almeno un gruppo OH.

Alcuni composti liberano il loro idrogeno sotto forma di H$^+$, cioè in essi si rompe il legame ossigeno-idrogeno, altri invece liberano il gruppo O-H sotto forma di OH$^-$, cioè si rompe il legame X-OH. Quindi in funzione dell'elettronegatività dell'atomo X si può vedere quale sia il legame più facile da rompere in acqua:

\begin{itemize}
    \item Se il composto libera ioni H$^+$ diciamo che è \textbf{acido};
    \item Se il composto libera ioni OH$^-$ diciamo che è una \textbf{base}.
\end{itemize}

\subsubsection{Definizione di Arrhenius (1887)}
La definizione finora adoperata è stata enunciata da Arrhenius, il quale etichettò come acido una qualunque specie chimica che in acqua libera ioni H$^+$ e come base una qualunque specie chimica che in acqua libera ioni OH$^-$.

\vspace{0.2cm}Esempi di acidi:

\vspace{0.2cm}\ce{HCl(aq) + H_2O -> H^+(aq) + Cl^-(aq)}

\vspace{0.2cm}\ce{H_2SO_4(aq) + H_2O -> H^+(aq) + H_2SO_4^-}

\vspace{0.2cm}\ce{H_2SO_4^- + H_2O -> H^+(aq) + SO_4^{2-}}

\vspace{0.2cm}Esempi di basi:

\vspace{0.2cm}\ce{NaOH(aq) -> Na^+(aq) + OH^-(aq)}

\vspace{0.2cm}\ce{Ba(OH)_2(aq) -> Ba(OH)^+(aq) + OH^-(aq)}

\vspace{0.2cm}\ce{Ba(OH)^+ -> Ba^{2+}(aq) + OH^-(aq)}

\vspace{0.2cm}Questa definzione però pone due limiti:

\begin{enumerate}
    \item Si considera solo l'acqua come solvente, per cui se questi composti fossero in solventi diversi non avremmo modo di etichettarli.
    \item Ci sono delle incongruenze. Ad esempio l'ammoniaca NH$_3$ che è una base non ha ioni OH$^-$ da liberare, pertanto secondo questa definizione non potrebbe essere una base.
\end{enumerate}
\subsubsection{Definizione di Brönsted-Lowry (1923)}
Si giunse quindi alla teoria acido-base di Brönsted-Lowry. Con essa si etichetta acido ogni specie chimica che è in grado di donare protoni.

\vspace{0.2cm}esempi di acidi di Brönsted-Lowry

\vspace{0.2cm}\ce{CH_3COOH(aq) + H_2O <--> CH_3COO^-(aq) + H_3O^+}

acido acetico \hspace{2.55cm} ione acetato 

\vspace{0.2cm}\ce{NH_4^+(aq) + H_2O <--> NH_3(aq) + H_3O^+(aq}

ione ammonio \hspace{1.35cm} ammoniaca

\vspace{0.2cm}\ce{H_2PO_4^-(aq) + H_2O <--> H_2PO_4^{2-}(aq) + H_3O^+(aq)}

ione fosfato \hspace{2.25cm} ione fosfato

acido \hspace{3.4cm} bi-acido

\vspace{0.2cm}Lo ione H$^+$ è scritto come H$_3$O$^+$, ossia lo ione H$^+$ si addiziona ad una molecola di acqua per dar luogo a H$_3$O$^+$. Scrivere così ci indica che il ruolo del solvente diventa determinante. Dunque in acqua lo ione H$^+$ non esiste come tale, ma si somma all'acqua.

Va poi da ricordare che se scrivessimo la molecola dell'acqua secondo la teoria di Lewis, sull'ossigeno resterebbero due doppietti non coinvolti nel legame chimico. Uno di questi due è molto interno, cioè ad energia molto negativa, pertanto non è disponibile per formare legame chimico; l'altro invece è esterno in energia, quindi è disponibile a essere coinvolto nel formare alrri legami. Questo doppietto permette infatti la formazione di un \textbf{legame dativo}, ossia un legame dovuto non ad una messa in comune di elettroni da parte dei due partners, ma al fatto che uno dei due fornisce entrambi gli elettroni necessari. In questo caso i due elettroni necessari sono costituiti dal doppietto dell'ossigeno.

Inoltre in tutti e tre i casi stiamo parlando di equilibri (c'è la doppia freccia), quindi non di reazioni spostate verso destra.

\vspace{0.2cm}Secondo questa teoria poi, si definisce base ogni specie chimica che è in gradi di accettare protoni.

\vspace{0.2cm}Esempi di basi di Brönsted-Lowry

\vspace{0.2cm}\ce{NH_3(aq) + H_2O <--> NH_4^+(aq) + OH^-(aq)}

\vspace{0.2cm}Se l'ammoniaca accetta un protone dall'acqua diventa ione ammonio e l'aqua che ha perso un protone diventa ione OH$^-$. Stavolta allora il protone sarà ceduto dall'acqua. Ne segue che, siccome l'NH$_3$ ha accettato un protone, è una base, mentre l'acqua si è comportata da acido perché sta cedendo un protone. Al contrario, nelle reazioni precedenti l'acqua stava acquistando il protone delle speci chimiche.

L'acqua allora divents determinante: non è più solo il solvente, ma si può comportare da base o da acido a seconda della reazione. Tale è fatto è ciò che sfuggiva ad Arrhenius.

\vspace{0.2cm}\ce{CO_3^{2-}(aq) + H_2O <--> HCO_3^- + OH^-(aq)}

\vspace{0.2cm}Lo ione carbonato CO$_3^{2-}$ è una base perché accetta un protone strappandolo all'acqua. Accetta quindi uno ione H$^+$ diventando ione bicarbonato HCO$_3^-$ e lasciando OH$^-$. Secondo la teoria di Arrhenius il carbonato non poteva essere etichettato base, perché non ha ioni OH$^-$ da mandare in soluzione, analogamente all'ammoniaca.

Spesso, quando dopo mangiato abbiamo una sensazione di acidità, usiamo il bicarbonato perché esso consuma l'eccesso di acido che abbiamo nello stomaco formando H$_2$CO$_3$, cioè quindi consumando gli ioni H$^+$. Dunque anche la specie HCO$_3^-$ è una base.

\subsubsection{Definizione di Lewis (1923)}
In larga misura useremo la definizione di Brönsted-Lowry per acidi e basi. Tuttavia ne esiste una ancora più ampia, che è la definizione di Lewis. Con essa si definisce acido una specie in grado di accettare coppie di elettroni, che più o meno è la stessa cosa: se si libera uno one H$^+$ restano elettroni sulla specie che libera questi protoni, e quindi è come se stesse accettando una coppia di elettroni. Si definisce invece base una specie che può cedere coppie di elettroni.

\vspace{0.2cm}Esempi:

\vspace{0.2cm}\chemfig{\charge{[circle]0=\:}{H_3N}} \ce{+ H^+ <--> NH_4^+}

\vspace{0.2cm}L'ammoniaca ha un doppietto sull'azoto chd usa per legare lo ione H$^+$ che sta strappando all'acqua per formare la specie NH$_4^+$.

Chiaramente per legare lo ione H$^+$ cede questi due elettroni, mettendoli in comune, quindi l'ammoniaca continua ad essere base anche secondo la definizione di Lewis

\vspace{0.2cm}\chemfig{\charge{[circle]0=\:}{H_3N}} \ce{+ BF_3 <-->}\chemfig{\charge{[circle]0=\:}{H_3N}} \ce{\bond{->}BF_3}

\vspace{0.2cm}In questa reazione nei prodotti c'è una freccia che sta a indicare la formazione di un legame dativo: entrambi gli elettroni vengono messi in comune dall'azoto.

Prodotti di questo tipo si chiamano \textbf{addotti}. In questo caso abbiamo un addotto tra NH$_3$ e BF$_3$.

In questa reazione abbiamo una specie in grado di cedere elettroni, che è la base, e la specie BF$_3$ che è in grado di acquistare cioè accettare coppie di elettroni. Secondo la teoria dell'ibridizzazione, il BF$_3$ ha ibridi $sp^2$ utilizzati tutti e tre dal boro per formare i tre legami col fluoro. Restava però un terzo orbitale $p$ non ibridizzato perpendicolare al piano della molecola a disposizione.

Il boro ha 3 elettroni, 2 nell'orbitale $2s$ e solo uno negli orbitali $p$. Parlavamo quindi di stato di promozione così da avere un elettrone nell'orbitale $s$ e 2 elettroni in due diversi orbitali $p$. Avevamo quindi 3 orbitali occupati singolarmente e uno no. Con la reotira dell'ibridizzazione immaginavamo poi che i tre orbitali occupati si mescolassero tra loro per dare luogo a tre ibridi $sp^2$, ognuno dei quali con 1 elettrone. Formavamo quindi i 3 legami del BF$_3$, ma sul boro resta totalmente vuoto un orbitale $p$ non ibridizzato perpendicolare al piano della molecola. Esso sarà quello in grado di accettare la coppia di elettroni, cioè il doppietto dell'azoto, interagirà con questo orbitale, formando un nuovo legame chimico. Ciò che allora succede è che l'ammoniaca sta cedendo degli elettroni e il BF$_3$ li sta acquistando, pertanto secondo questa teoria quest'ultimo è un acido, cosa che nei fatti è.

\vspace{0.2cm}Etichettiamo quindi con \textit{reazione acido-base} la formazione di un legame chimico tra NH$_3$ e BF$_3$, tenendo presente la teoria di Lewis.

\vspace{0.2cm}Abbiamo visto che con la teoria di  Br\"{o}nsted-Lowry l'acqua partecipa attivamente alla reazione, comportandosi a volte come acido e a volte come base in funzione del partner. Secondo la teoria di Lewis l'acqua è etichettata come \textit{base di Lewis} e il protone (cioè lo ione H$^+$) come \textit{acido di Lewis}. La loro unione ci dà un addotto.

Quindi tutte le volte in cui la coppia di elettroni di legame viene messa in comune solo da uno dei due partner si forma un legame dativo e il composto ottenuto viene chiamato addotto.

Anche l'ammoniaca è una base di Lewis, e se ad essa sommiamo un acido di Lewis dà luogo allo ione ammonio:

\begin{figure}[htp]
    \centering
    \includegraphics[width=8cm]{immagini/acidi_e_basi_di_lewis.png}
\end{figure}

\subsection{Ioni complessi in soluzione acquosa}
Entriamo ora più nel dettaglio per quella che la teoria etichetta \textit{coppia coniugata acido-base}. Va da precisare infatti che, similmente a quanto dicevamo per le redox dove non può esistere un ossidante se non c'è constestualmente il riducente, nelle reazioni acido base se una specie perde elettroni è indispensabile che questi vengano acquistati da un'altra specie. Lo stesso vale per gli acidi e le basi: affinché una specie assuma il comportamente di acido deve essere in presenza di una base.

Secondo Br\"{o}nsted-Lowry se c'è una specie che cede un protone ci vuole una specie che acquisti questo protone. Quindi non c'è un acido o una base, ma abbiamo sempre coppie coniugate acido-base.

\vspace{0.2cm}Se mettiamo sei sali neutri in acqua, questi si dissociano e il catione viene coordinato, cioè circondato in maniera ordinata, da molecole di acqua, formando un complesso ottaedrico

$$\ce{Be^{2+}(aq) + 4H_2O(l) <--> [Be(H_2O)_4]^{2+}(aq)}$$

$$\ce{H^+(g) + H_2O(l) <--> H_3O^+(aq)}$$

$$\ce{HA + H_2O <--> H_3O^+ + A^-}$$

\subsubsection{acidi e basi}

\vspace{0.2cm}\ce{HNO_3(aq) + H_2O(l) -> H_3O^+(aq) + NO_3^-(aq)}

\vspace{0.2cm}L'acido nitrico in acqua si dissocia, liberando un protone che si somma all'acqua che diventa ione $\rm H_3O^+$. Resta lo ione $\rm NO_3^-$.

Etichettiamo come acido la specie $\rm HNO_3$ e come base coniugata la specie $\rm NO_3^-$. Chiaramente però se $\rm HNO_3$ è l'acido vuol dire che in soluzione c'era una base, che è l'$\rm H_2O$. Quindi in questa reazione l'acqua si comporta da base e $\rm H_3O^+$ è il suo acido coniugato.

Abbiamo quindi coppie coniugate, in modo da avere un acido e una base sia a sinistra che a destra. Diciamo coniugate perché le otteniamo a partire dagli acidi e dalle basi iniziali.

\vspace{0.2cm}\ce{NH_4^+(aq) + H_2O(l) <--> H_3O^+(aq) + NH_3(aq)}

\vspace{0.2cm}Lo ione ammonio $\rm NH_4^+$ libera un protone e diventa ammoniaca $\rm NH_3$, quindi lo ione ammonio è la specie acida e l'ammoniaca la sua base coniugata. Se lo ione ammonio libera un protone, è indispensabile che ci sia una specie che la acquisti. Tale specie è l'acqua, che pertanto risulta essere una base perché acquista il protone liberato. Essa inoltre genera la specie $\rm H_3O^+$ che è il suo acido coniugato.

Anche in questa reazione abbiamo sia a sinistra che a destra una coppia acido-base.

\vspace{0.2cm}\ce{HSO_4^-(aq) + H_2O(l) <--> H_3O^+(aq) + SO_4^{2-}(aq)}

\vspace{0.2cm}Il solfato acido può cedere un protone e diventare $\rm SO_4^-$, quindi esso è la base coniugata dell'$\rm HSO_4^-$

\vspace{0.2cm}\ce{NH_3(aq) + H_2O(l) <--> NH_4^+(aq) + OH^-}


\subsection{Anfoliti}

\ce{HCl(aq) + H_2O(l) -> H_3O^+(aq) + Cl^-(aq)}

\vspace{0.2cm}In questa reazione l'acqua acquista un protone, quindi è una base.

\vspace{0.2cm}\ce{NH_3(aq) + H_2O(l) <--> NH_4^+(aq) + OH^-(aq)}

\vspace{0.2cm}\ce{HCO_3^-(aq) + H_2O(l) <--> H_3O^+(aq) + CO_3^{2-}}

\vspace{0.2cm}\ce{HCO_3^-(aq) + H_2O(l) <--> H_2CO_3(aq) + OH^-}

\begin{figure}[htp]
    \centering
    \includegraphics[width=12cm]{immagini/anfoliti.png}
\end{figure}

\subsection{Reazioni acido-base}
\subsection{Forza degli acidi e delle basi}
\subsection{Autodissociazione dell'acqua}