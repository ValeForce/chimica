\subsection{Definizioni di acido e base}
Le definizioni di acidi e basi sono variate negli anni.

Inizialmente abbiamo classificato come acido o base un dato composto in funzione dell'elettronegatività dell'atomo X nella seguente reazione

$$\ce{XOH(aq) + H_2O -> XO^-(aq) + H_3O^+(aq)}, \quad \text{X elettronegativo}$$

$$\ce{XOH(aq) + H_2O -> X^+(aq) + OH^-(aq)}, \quad \text{X elettropositivo}$$

Immaginiamo quindi di avere un generico composto formato da un atomo X con almeno un gruppo OH.

Alcuni composti liberano il loro idrogeno sotto forma di H$^+$, cioè in essi si rompe il legame ossigeno-idrogeno, altri invece liberano il gruppo O-H sotto forma di OH$^-$, cioè si rompe il legame X-OH. Quindi in funzione dell'elettronegatività dell'atomo X si può vedere quale sia il legame più facile da rompere in acqua:

\begin{itemize}
    \item Se il composto libera ioni H$^+$ diciamo che è \textbf{acido};
    \item Se il composto libera ioni OH$^-$ diciamo che è una \textbf{base}.
\end{itemize}

\subsubsection{Definizione di Arrhenius (1887)}
La definizione finora adoperata è stata enunciata da Arrhenius, il quale etichettò come acido una qualunque specie chimica che in acqua libera ioni H$^+$ e come base una qualunque specie chimica che in acqua libera ioni OH$^-$.

\vspace{0.2cm}Esempi di acidi:

\vspace{0.2cm}\ce{HCl(aq) + H_2O -> H^+(aq) + Cl^-(aq)}

\vspace{0.2cm}\ce{H_2SO_4(aq) + H_2O -> H^+(aq) + H_2SO_4^-}

\vspace{0.2cm}\ce{H_2SO_4^- + H_2O -> H^+(aq) + SO_4^{2-}}

\vspace{0.2cm}Esempi di basi:

\vspace{0.2cm}\ce{NaOH(aq) -> Na^+(aq) + OH^-(aq)}

\vspace{0.2cm}\ce{Ba(OH)_2(aq) -> Ba(OH)^+(aq) + OH^-(aq)}

\vspace{0.2cm}\ce{Ba(OH)^+ -> Ba^{2+}(aq) + OH^-(aq)}

\vspace{0.2cm}Questa definzione però pone due limiti:

\begin{enumerate}
    \item Si considera solo l'acqua come solvente, per cui se questi composti fossero in solventi diversi non avremmo modo di etichettarli.
    \item Ci sono delle incongruenze. Ad esempio l'ammoniaca NH$_3$ che è una base non ha ioni OH$^-$ da liberare, pertanto secondo questa definizione non potrebbe essere una base.
\end{enumerate}
\subsubsection{Definizione di Brönsted-Lowry (1923)}
Si giunse quindi alla teoria acido-base di Brönsted-Lowry. Con essa si etichetta acido ogni specie chimica che è in grado di donare protoni.

\vspace{0.2cm}esempi di acidi di Brönsted-Lowry

\vspace{0.2cm}\ce{CH_3COOH(aq) + H_2O <--> CH_3COO^-(aq) + H_3O^+}

acido acetico \hspace{2.25cm} ione acetato 

\vspace{0.2cm}\ce{NH_4^+(aq) + H_2O <--> NH_3(aq) + H_3O^+(aq}

ione ammonio \hspace{1.2cm} ammoniaca

\vspace{0.2cm}\ce{H_2PO_4^-(aq) + H_2O <--> H_2PO_4^{2-}(aq) + H_3O^+(aq)}

ione fosfato \hspace{2cm} ione fosfato

acido \hspace{2.95cm} bi-acido

\vspace{0.2cm}Lo ione H$^+$ è scritto come H$_3$O$^+$, ossia lo ione H$^+$ si addiziona ad una molecola di acqua per dar luogo a H$_3$O$^+$. Scrivere così ci indica che il ruolo del solvente diventa determinante. Dunque in acqua lo ione H$^+$ non esiste come tale, ma si somma all'acqua.

Va poi da ricordare che se scrivessimo la molecola dell'acqua secondo la teoria di Lewis, sull'ossigeno resterebbero due doppietti non coinvolti nel legame chimico. Uno di questi due è molto interno, cioè ad energia molto negativa, pertanto non è disponibile per formare legame chimico; l'altro invece è esterno in energia, quindi è disponibile a essere coinvolto nel formare alrri legami. Questo doppietto permette infatti la formazione di un legame dativo, ossia un legame dovuto non ad una messa in comune di elettroni da parte dei due partners, ma al fatto che uno dei due fornisce entrambi gli elettroni necessari. In questo caso i due elettroni necessari sono costituiti dal doppietto dell'ossigeno.

Inoltre in tutti e tre i casi stiamo parlando di equilibri (c'è la doppia freccia), quindi non di reazioni spostate verso destra.

\vspace{0.2cm}Secondo questa teoria poi, si definisce base ogni specie chimica che è in gradi di accettare protoni.

\vspace{0.2cm}Esempi di basi di Brönsted-Lowry

\vspace{0.2cm}\ce{NH_3(aq) + H_2O <--> NH_4^+(aq) + OH^-(aq)}

\vspace{0.2cm}Se l'ammoniaca accetta un protone dall'acqua diventa ione ammonio e l'aqua che ha perso un protone diventa ione OH$^-$. Stavolta allora il protone sarà ceduto dall'acqua. Ne segue che, siccome l'NH$_3$ ha accettato un protone, è una base, mentre l'acqua si è comportata da acido perché sta cedendo un protone. Al contrario, nelle reazioni precedenti l'acqua stava acquistando il protone delle speci chimiche.

L'acqua allora divents determinante: non è più solo il solvente, ma si può comportare da base o da acido a seconda della reazione. Tale è fatto è ciò che sfuggiva ad Arrhenius.

\vspace{0.2cm}\ce{CO_3^{2-}(aq) + H_2O <--> HCO_3^- + OH^-(aq)}

\vspace{0.2cm}Lo ione carbonato CO$_3^{2-}$ è una base perché accetta un protone strappandolo all'acqua. Accetta quindi uno ione H$^+$ diventando ione bicarbonato HCO$_3^-$ e lasciando OH$^-$. Secondo la teoria di Arrhenius il carbonato non poteva essere etichettato base, perché non ha ioni OH$^-$ da mandare in soluzione, analogamente all'ammoniaca.

Spesso, quando dopo mangiato abbiamo una sensazione di acidità, usiamo il bicarbonato perché esso consuma l'eccesso di acido che abbiamo nello stomaco formando H$_2$CO$_3$, cioè quindi consumando gli ioni H$^+$. Dunque anche la specie HCO$_3^-$ è una base.

\subsubsection{Definizione di Lewis (1923)}
In larga misura useremo la definizione di Brönsted-Lowry per acidi e basi. Tuttavia ne esiste una ancora più ampia, che è la definizione di Lewis. Con essa si definisce acido una specie in grado di accettare coppie di elettroni, che più o meno è la stessa cosa: se si libera uno one H$^+$ restano elettroni sulla specie che libera questi protoni, e quindi è come se stesse accettando una coppia di elettroni. Si definisce invece base una specie che può cedere coppie di elettroni.

\vspace{0.2cm}Esempi:

\vspace{0.2cm}\chemfig{\charge{[circle]0=\:}{H_3N}} \ce{+ H^+ <--> NH_4^+}

\vspace{0.2cm}\chemfig{\charge{[circle]0=\:}{H_3N}} \ce{+ BF_3 <-->}\chemfig{\charge{[circle]0=\:}{H_3N}} \ce{\bond{->}BF_3}

\begin{figure}[htp]
    \centering
    \includegraphics[width=8cm]{immagini/acidi_e_basi_di_lewis.png}
\end{figure}

\subsection{Ioni complessi in soluzione acquosa}
\subsection{Anfoliti}
\subsection{Reazioni acido-base}
\subsection{Forza degli acidi e forza delle basi}
\subsection{Autodissociazione dell'acqua}